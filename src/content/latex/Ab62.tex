\uuid{Ab62}
\titre{Développement d'une fonction en série de Fourier}
\theme{Série}
\auteur{}
\organisation{AMSCC}	

\contenu{


\texte{ 	Soit $f \colon \R \to \R$ la fonction $2\pi$-périodique définie par :
$$f(x) =\left\{
\begin{array}{cl}
	1&\textrm{si $x \in ]-\pi;0[$}\\
	0& \textrm{si $x =0$}\\
	-1&\textrm{si $x \in ]0;\pi[$}\end{array}\right.$$ }

\begin{enumerate}
	\item \question{ Représenter graphiquement l'allure de la fonction $f$ et vérifier que $f$ est une fonction impaire. }
	\item \question{  Déterminer les coefficients de Fourier trigonométriques de la fonction $f$ et en déduire sa série de Fourier. La série de Fourier est-elle égale à $f$ en tout $x \in \R$ ?}
	\reponse{C'est une fonction impaire donc $a_n = 0$ pour tout $n$. Il reste à calculer :
		$$b_n=\frac{1}{\pi}\int_{-\pi}^{0}\,(1)\sin(nx)dx+\frac{1}{\pi}\int_{0}^{\pi}\,(-1)\sin(nx)dx=\frac{2}{n\pi}(\cos(n\pi)-1).$$ soit $$b_n=\left\{
		\begin{array}{cl}
			0&\textrm{pour $n$ pair}\\
			\frac{4}{n\pi}&\textrm{pour $n$ impair}\end{array}\right.$$
	}
\end{enumerate}
}
