\uuid{bXCC}
\titre{\'Etude des zéros d'une fonction non linéaire}
\theme{}
\auteur{}
\organisation{AMSCC}
\contenu{

\texte{ Soit 
$$
f:\;
\begin{array}{rcl}
	[0,1]&\to& \mathbb{R}\\
	x&\to& x(1+e^x)-e^x.
\end{array}
$$
On recherche les z\'eros $\ell$ de $f$ dans $[0,1]$:
$$
f(\ell)=0.
$$ }
\begin{enumerate}
	\item \question{ Démontrer qu'il existe un unique z\'ero $\ell$ de la fonction $f$ sur $[0,1]$ }.
	\reponse{
		La fonction $f$ est $C^\infty$. Sa dérivée
		$$
		f'(x)=1+xe^x,
		$$
		étant strictement positive, la fonction $f$ est strictement croissante. De plus
		$$
		f(0)=-1,\quad f(1)=1.
		$$
		La continuité et la stricte croissance de $f$ assure donc l'existence et l'unicité d'un zéro sur $[0,1]$.
	}
	
	
	\item  On définit la fonction $\varphi_1 \colon x \mapsto x-\frac{x(1+e^x)-e^x}{1+xe^x}$ et pour un $x_0$ quelconque dans $]0,1[$, on définit une suite $(x_n)$ par la relation de récurrence suivante : 
	$$\forall n \in \N\,, \quad x_{n+1} = \varphi_1(x_n)$$ 
	\begin{enumerate}
		\item \question{ Montrer que $\ell$ est un point fixe de $\varphi_1$. }
		
		\item \question{ Montrer que la suite $(x_n)$ converge localement vers $\ell$ avec un ordre de convergence supérieur ou égal à $2$. }
		\reponse{ On remarque d'abord que $\phi(x)=x$ équivaut à $f(x)=0$. Donc $l$ est l'unique point fixe de $\phi$.
			
			Une méthode itérative donnée par la relation de récurrence 
			$$
			x_{n+1}=\phi(x_n),
			$$
			converge vers le point fixe $l$ de $\phi$ pour $x_0$ pris dans un voisinage de $l$ si $|\phi'(l)|<1$. Or
			$$
			\phi'(l)=-\frac{e^l(1+l)(l(1+e^l)-e^l)}{(1+le^l)^2}.
			$$
			Or par définition
			$$
			l(1+e^l)-e^l=0,
			$$
			ce qui assure
			$$
			\phi'(l)=0.
			$$}
		
%		\item Proposer un critère d'arrêt pour cette méthode.%On considère le critère d'arrêt sur le résidu :
%		%$$\vert f(x_n)\vert < \epsilon$$
%		%Comment choisir $\epsilon>0$ pour avoir $x_n$ une valeur approchée de $\ell$  avec une pr\'ecision de $0,01$, c'est-à-dire tel que $\vert x_n -\ell\vert < 0,01$ ?
%		\rep{ Nous proposons le critère d'arrêt par majoration de l'incrément: 
%			$$
%			|x_{n+1}-x_n|<0.01.
%			$$}
	\end{enumerate}
	\item On consid\`ere
	$$
	\varphi_2:\;
	\begin{array}{rcl}
		\mathbb{R}&\to& \mathbb{R}\\
		x&\to& \frac{e^x}{1+e^x},
	\end{array}
	$$
	et la suite
	$$
	x_{n+1}=\varphi_2(x_n),\quad x_0\in [0,1].
	$$
	\begin{enumerate}
		\item \question{ Montrer que $\ell$ est l'unique point fixe de $\varphi_2$ dans $[0,1]$. }
		
		\reponse{
			Par définition
			$$
			l(1+e^l)-e^l=0,
			$$
			ce qui donne
			$$
			l(1+e^l)=e^l,
			$$
			et donc
			$$
			g(l)=\frac{e^l}{1+e^l}=l.
			$$
		} 
		
		
		\item \question{ Montrer que l'intervalle $[0,1]$ est stable par $\varphi_2$ : $\varphi_2([0,1])\subset [0,1]$. }
		
		\reponse{ La fonction $\varphi_2$ est $C^\infty$ et 
			$$
			\varphi_2'(x)=\frac{e^x}{(1+e^x)^2}>0.
			$$
			La fonction $\varphi_2$ est donc continue croissante. Il suit
			$$
			\varphi_2([0,1])= [\varphi_2(0),\varphi_2(1)].
			$$
			Or 
			$$
			\varphi_2(0)=\frac{1}{2}>0,\quad \varphi_2(1)=\frac{e}{1+e}<1.
			$$
		}
		
		\item \question{ Déterminer $L$ tel que $0<L<1$ que $\varphi_2$ est $L$-Lipschitzienne sur $[0,1]$. }
		\reponse{
			On remarque que
			$$
			\varphi_2'(x)\leq \frac{1+e^x}{(1+e^x)^2}\leq \frac{1}{1+e^x},
			$$
			et donc
			$$
			0\leq \varphi_2'(x)\leq \frac{1}{1+e}<1.
			$$
			La fonction $\varphi_2$ est donc $L$-Lipschitz sur $[0,1]$ avec 
			$$
			L=\frac{1}{1+e}.
			$$
		}
		
		\item \question{ En d\'eduire que la suite $(x_n)$ tend vers $\ell$.}
		\reponse{ Les propriétés (b) et (c) assure que $\varphi_2$ admet un unique point fixe et que la suite $(x_n)_n$ tend vers cet unique point fixe quelque soit $x_0$. La propriété (a) assure que cet unique point fixe est $l$.} 
		\item \question{ Ecrire un algorithme qui donne une approximation de $\ell$ à $2^{-20}$ près. Combien faut-il d'itérations pour atteindre cette précision ?}
		
	\end{enumerate}
\end{enumerate}

}