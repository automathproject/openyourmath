\uuid{LXkF}
\titre{ Puissance d'un test }
\theme{tests d'hypothèses}
\auteur{}
\datecreate{2023-11-27}
\organisation{AMSCC}
\contenu{

\texte{ 	On veut contrôler la durée d'utilisation d'un lot d'ampoules électriques. Soit $p$ la proportion d'ampoules défectueuses. 

Afin de choisir entre les deux hypothèses : $\begin{cases} 
	H_0 \colon p = 0.05 \\
	H_1 \colon p > 0.05
\end{cases}$, 
on prélève un échantillon aléatoire de taille $n = 144$. }

\begin{enumerate}
	\item \question{ Déterminer le seuil critique pour une erreur de première espèce $\alpha = 5\%$. }
	\item\question{  On considère que la proportion d'ampoules défectueuses est inacceptable si $p=0.10$. Pour cela, on construit le test selon les hypothèses :
	$$\begin{cases} 
		H_0 \colon p = 0.05 \\
		H_1 \colon p = 0.10
	\end{cases}$$
	Quelle est la puissance de ce test ? }
	\item \question{ Le test est-il plus efficace si on l'effectue avec une hypothèse alternative $H_1 \colon p = 0.12$ ? }
\end{enumerate}

\reponse{ \href{https://stcyrterrenetdefensegouvf-my.sharepoint.com/:x:/g/personal/maxime_nguyen_st-cyr_terre-net_defense_gouv_fr/EYlCPJY62BdMpTYscg1T6zABc9_WCIR_pzt6XSOajALYhw?e=W1oBgi&nav=MTVfezAwMDAwMDAwLTAwMDEtMDAwMC0wODAwLTAwMDAwMDAwMDAwMH0}{lien vers tableur} }
}