\uuid{Sclr}
\titre{Inverse de matrice et application}
\theme{calcul matriciel}
\auteur{}
\datecreate{2023-01-03}
\organisation{AMSCC}
\contenu{

\texte{ Soient les matrices $A=\begin{pmatrix}1 & 4 & -4 \\ -6 & -13 & 12 \\ -6 & -12 & 11\end{pmatrix}$ et $P=\begin{pmatrix}1 & -2 & 0 \\ -3 & 4 & 1 \\ -3 & 3 & 1\end{pmatrix}$. }

\begin{enumerate}
	\item \question{ Parmi les trois matrices suivantes, dire laquelle est l'inverse de $P$ en justifiant.
		$$
		P_1=\begin{pmatrix}
			1 & 2 & -2 \\
			0 & 1 & -1 \\
			2 & 0 & -2
		\end{pmatrix} \quad P_2=\begin{pmatrix}
			1 & 2 & -2 \\
			0 & 1 & -1 \\
			3 & 3 & -2
		\end{pmatrix} \quad P_3=\begin{pmatrix}
			1 & 2 & 0 \\
			0 & 1 & 1 \\
			2 & 1 & 2
		\end{pmatrix}
		$$
		}
	\reponse{ On a :
		$$
		P \cdot B=\left(\begin{array}{ccc}
			1 & -2 & 0 \\
			-3 & 4 & 1 \\
			-3 & 3 & 1
		\end{array}\right) \cdot\left(\begin{array}{ccc}
			1 & 2 & -2 \\
			0 & 1 & -1 \\
			2 & 0 & -2
		\end{array}\right)=\left(\begin{array}{ccc}
			1 & 0 & 0 \\
			-1 & -2 & 0 \\
			-1 & -3 & 1
		\end{array}\right) \neq I d
		$$
		donc $P^{-1} \neq B$
		$$
		P \cdot C=\left(\begin{array}{ccc}
			1 & -2 & 0 \\
			-3 & 4 & 1 \\
			-3 & 3 & 1
		\end{array}\right) \cdot\left(\begin{array}{ccc}
			1 & 2 & -2 \\
			0 & 1 & -1 \\
			3 & 3 & -2
		\end{array}\right)=\left(\begin{array}{lll}
			1 & 0 & 0 \\
			0 & 1 & 0 \\
			0 & 0 & 1
		\end{array}\right)=I d
		$$
		et $P^{-1}=C$
		Par unicité de la matrice inverse, sans calcul, on peut affirmer $P^{-1} \neq D$. }
	\item \texte{  Calculer $D = P^{-1}  A  P$. }
	\reponse{ $$
		P^{-1} \cdot A \cdot P=\left(\begin{array}{ccc}
			1 & 2 & -2 \\
			0 & 1 & -1 \\
			3 & 3 & -2
		\end{array}\right) \cdot\left(\begin{array}{ccc}
			1 & 4 & -4 \\
			-6 & -13 & 12 \\
			-6 & -12 & 11
		\end{array}\right) \cdot\left(\begin{array}{ccc}
			1 & -2 & 0 \\
			-3 & 4 & 1 \\
			-3 & 3 & 1
		\end{array}\right)=\left(\begin{array}{ccc}
			1 & 0 & 0 \\
			0 & -1 & 0 \\
			0 & 0 & -1
		\end{array}\right)=D
		$$ }
	\item \question{ En déduire $A^{-1}, A^{2 n}$ et $A^{2 n+1}$, pour $n \in \mathbb{N}$. }
	\indication{ Calculer d'abord $D^{-1}, D^{2 n}$ et $D^{2 n+1}$, pour $n \in \mathbb{N}$ et exprimer $A$ en fonction de $D$. }
	\reponse{ On a :
		$$
		\begin{gathered}
			D^2=D \cdot D=I \Rightarrow F^{-1}=F \\
			\left(P^{-1} \cdot A \cdot P\right)^{-1}=P^{-1} \cdot A^{-1} \cdot P=P^{-1} \cdot A \cdot P \\
			\Rightarrow A^{-1} \cdot P=P \cdot P^{-1} \cdot A \cdot P=A \cdot P \\
			\Rightarrow A^{-1}=A \cdot P \cdot P^{-1}=A
		\end{gathered}
		$$
		Donc $A^{-1}=A$ d'où $A^2 = I$ et pour tout $n \in \N$, on déduit par récurrence que $A^{2n} = I$ et $A^{2n+1} = A$. }
\end{enumerate}
}
