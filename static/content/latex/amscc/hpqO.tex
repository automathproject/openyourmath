\uuid{hpqO}
\titre{Limite d'une suite dans $\R^2$}
\theme{topologie}
\auteur{}
\datecreate{2023-03-01}
\organisation{AMSCC}
\contenu{


\texte{ Pour tout $k \in \N$, on pose 
$$X_k = \left(\frac{2}{1+k} , 1+e^{-k} \right) \in \R^2$$ }

\begin{enumerate}
	\item \question{ Soit $\ell = (0,1)$. Calculer $||X_k-\ell||_{2}$ où $||.||_{2}$ est la norme euclidienne.  }
	\reponse{$||X_k-\ell||_{2} = \sqrt{\left(\frac{2}{1+k}\right)^2 + \left(e^{-k}\right)^2} = \sqrt{\frac{4}{(1+k)^2} + e^{-2k} }$}
	\item \question{ En déduire la convergence de la suite $(X_k)$ dans $\R^2$. }
	\reponse{$||X_k-\ell||_{2} \xrightarrow[k \to +\infty]{}0$ donc $X_k \xrightarrow[k \to +\infty]{}\ell$.   }
\end{enumerate}}
