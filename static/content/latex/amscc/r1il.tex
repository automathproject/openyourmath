\uuid{r1il}
\titre{Inverse de matrice}
\theme{calcul matriciel}
\auteur{}
\datecreate{2023-01-03}
\organisation{AMSCC}
\contenu{

\texte{ Soient les matrices $A=\begin{pmatrix}1 & 4 & -3 \\ 2 & 1 & 2 \\ 1 & 3 & -2\end{pmatrix}$ et $B=\begin{pmatrix}1 & -a & 0 & 0 \\ 0 & 1 & -a & 0 \\ 0 & 0 & 1 & -a \\ 0 & 0 & 0 & 1\end{pmatrix}$.  }

\question{ Calculer $A^{-1}$ et $B^{-1}$. }
\indication{ Utiliser le pivot de Gauss. }

\reponse{ Soient :
	$$
	A=\left(\begin{array}{ccc}
		1 & 4 & -3 \\
		2 & 1 & 2 \\
		1 & 3 & -2
	\end{array}\right) \quad I_3=\left(\begin{array}{ccc}
		1 & 0 & 0 \\
		0 & 1 & 0 \\
		0 & 0 & 1
	\end{array}\right)
	$$
	
	
	Retranchons à la deuxième ligne 2 fois la première et à la troisième ligne 1 fois la première :
	$$
	\left(\begin{array}{ccc}
		1 & 4 & -3 \\
		0 & -7 & 8 \\
		0 & -1 & 1
	\end{array}\right) \quad\left(\begin{array}{ccc}
		1 & 0 & 0 \\
		-2 & 1 & 0 \\
		-1 & 0 & 1
	\end{array}\right)
	$$
	Changeons le signe de la deuxième et de la troisième ligne et permutons les lignes 2 et 3 :
	$$
	\left(\begin{array}{ccc}
		1 & 4 & -3 \\
		0 & 1 & -1 \\
		0 & 7 & -8
	\end{array}\right) \quad\left(\begin{array}{ccc}
		1 & 0 & 0 \\
		1 & 0 & -1 \\
		2 & -1 & 0
	\end{array}\right)
	$$
	Retranchons à la troisième ligne 7 fois la première :
	$$
	\left(\begin{array}{ccc}
		1 & 4 & -3 \\
		0 & 1 & -1 \\
		0 & 0 & -1
	\end{array}\right) \quad\left(\begin{array}{ccc}
		1 & 0 & 0 \\
		1 & 0 & -1 \\
		-5 & -1 & 7
	\end{array}\right)
	$$
	Changeons le signe de la troisième ligne :
	$$
	\left(\begin{array}{ccc}
		1 & 4 & -3 \\
		0 & 1 & -1 \\
		0 & 0 & 1
	\end{array}\right) \quad\left(\begin{array}{ccc}
		1 & 0 & 0 \\
		1 & 0 & -1 \\
		5 & 1 & -7
	\end{array}\right)
	$$
	Ajoutons à la première ligne 3 fois la troisième ligne et ajoutons à la deuxième ligne la troisième :
	$$
	\left(\begin{array}{lll}
		1 & 4 & 0 \\
		0 & 1 & 0 \\
		0 & 0 & 1
	\end{array}\right) \quad\left(\begin{array}{ccc}
		16 & 3 & -21 \\
		6 & 1 & -8 \\
		5 & 1 & -7
	\end{array}\right)
	$$
	Retranchons à la première ligne 4 fois la deuxième ligne :
	$$
	I_3=\left(\begin{array}{lll}
		1 & 0 & 0 \\
		0 & 1 & 0 \\
		0 & 0 & 1
	\end{array}\right)
	$$
	$$
	A^{-1}=\left(\begin{array}{ccc}
		-8 & -1 & 11 \\
		6 & 1 & -8 \\
		5 & 1 & -7
	\end{array}\right)
	$$
	
	Soient :
	$$
	B=\left(\begin{array}{cccc}
		1 & -a & 0 & 0 \\
		0 & 1 & -a & 0 \\
		0 & 0 & 1 & -a \\
		0 & 0 & 0 & 1
	\end{array}\right) \quad I_4=\left(\begin{array}{cccc}
		1 & 0 & 0 & 0 \\
		0 & 1 & 0 & 0 \\
		0 & 0 & 1 & 0 \\
		0 & 0 & 0 & 1
	\end{array}\right)
	$$
	Ajoutons à la troisième ligne $a$ fois la quatrième :
	$$
	\left(\begin{array}{cccc}
		1 & -a & 0 & 0 \\
		0 & 1 & -a & 0 \\
		0 & 0 & 1 & 0 \\
		0 & 0 & 0 & 1
	\end{array}\right) \quad\left(\begin{array}{llll}
		1 & 0 & 0 & 0 \\
		0 & 1 & 0 & 0 \\
		0 & 0 & 1 & a \\
		0 & 0 & 0 & 1
	\end{array}\right)
	$$
	Ajoutons à la deuxième ligne $a$ fois la troisième :
	$$
	\left(\begin{array}{cccc}
		1 & -a & 0 & 0 \\
		0 & 1 & 0 & 0 \\
		0 & 0 & 1 & 0 \\
		0 & 0 & 0 & 1
	\end{array}\right) \quad\left(\begin{array}{cccc}
		1 & 0 & 0 & 0 \\
		0 & 1 & a & a^2 \\
		0 & 0 & 1 & a \\
		0 & 0 & 0 & 1
	\end{array}\right)
	$$
	Ajoutons à la première ligne $a$ fois la deuxième :
	$$
	I_4=\left(\begin{array}{cccc}
		1 & 0 & 0 & 0 \\
		0 & 1 & 0 & 0 \\
		0 & 0 & 1 & 0 \\
		0 & 0 & 0 & 1
	\end{array}\right) \quad B^{-1}=\left(\begin{array}{cccc}
		1 & a & a^2 & a^3 \\
		0 & 1 & a & a^2 \\
		0 & 0 & 1 & a \\
		0 & 0 & 0 & 1
	\end{array}\right)
	$$
 }}
