\uuid{8mM7}
\titre{Lancers de dé}
\theme{variables aléatoires, loi binomiale, loi normale}
\auteur{}
\datecreate{2022-09-24}
\organisation{AMSCC}
\contenu{

\texte{ On lance $720$ fois un dé équilibré. }
\reponse{ 	Soit $X$ le nombre de fois où le nombre $6$ est apparu sur les $720$ lancers. La \va $X$ suit donc la loi $\mathcal{B}(720,\frac{1}{6})$. En particulier, $\E(X)=720\times \frac{1}{6}=120$ et  $\V(X)=720\times \frac{1}{6}\times (1-\frac{1}{6})=100$. }
\begin{enumerate}
	\item \begin{enumerate}
		\item \question{ En utilisant l'inégalité de Bienaymé-Tchebytchev, que peut-on dire de la probabilité que le nombre $6$ soit apparu entre $100$ et $140$ fois ? }
		\reponse{ En utilisant l'inégalité de Bienaymé-Tchebytchev, on a
			\[ \p(|X-\E(X)|\geq 20)\leq \frac{\V(X)}{20^2}=\frac{100}{400}=\frac{1}{4}.\]
			On en conclut
			\[ \p(100\leq X\leq 140)=\p(|X-120|\leq 20)=1-\p(|X-120|>20)\geq 1-\frac{1}{4}=\frac{3}{4}.\] }
		\item \question{ Que peut-on dire de cette même probabilité en utilisant une approximation par la loi Normale ? }
		\reponse{ On approxime la loi de $X$ par la loi $\mathcal{N}(120,\sigma^2=10^2)$. Alors
			\begin{align*}
				\p(100\leq X\leq 140)&=\p(-2\leq \frac{X-120}{10}\leq 2) \\
				&\simeq \p(-2\leq Z\leq 2), \quad \text{où} \ Z\sim \mathcal{N}(0,1) \\
				&\simeq 2\p(Z\leq 2)-1 \\
				&\simeq 2* 0.9772-1, \quad \text{par lecture de la table de loi } \mathcal{N}(0,1)  \\
				& \simeq 0.9544.
		\end{align*} }
	\end{enumerate}
	\item \question{ Déterminer le plus petit entier $n$ tel que $\mathbb{P}(|X-120|\leq n)\geq 0.9$, où $X$ est la variable aléatoire égale au nombre d'apparition du $6$ sur $720$ lancers, que l'on puisse obtenir:
		\begin{enumerate}
			\item par l'inégalité de Bienaymé-Tchebytchev,
			\item par le théorème central-limite.
	\end{enumerate} }
	\reponse{ \begin{enumerate}
			\item Par l'inégalité de Bienaymé-Tchebytchev, on a  
			\[\p(|X-120|\geq n) \leq \frac{100}{n^2}.\]
			Si on impose $\frac{100}{n^2}<0.1$, alors on a bien l'inégalité recherchée, à savoir $ \p(|X-120|\leq n) \geq 0.9$.
			Or 
			\[ \frac{100}{n^2}<0.1 \Leftrightarrow n>\sqrt{1000}\simeq 31.62.\]
			La valeur minimale de $n$ obtenue est donc $n=32$.
			\item $\p(|X-120|\leq n)=\p(|\frac{X-120}{10}|\leq \frac{n}{10})\simeq \p(Z\leq \frac{n}{10})-1$, où $Z\sim \mathcal{N}(0,1)$.
			Ainsi la condition $ \p(|X-120|\leq n) \geq 0.9$ devient $\p(Z\leq \frac{n}{10})\leq 0.95$, c'est-à-dire 
			$\frac{n}{10}\geq 1.64$. La valeur minimale de $n$ obtenue est donc $n=17$.
	\end{enumerate} }
	\item \question{ Commenter les résultats. }
	\reponse{ Le théorème central-limite est plus efficace que l'inégalité de Bienaymé-Tchebytchev. }
\end{enumerate}}
