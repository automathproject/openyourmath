\uuid{T8Ws}
\titre{Lois marginales}
\theme{variables aléatoires à densité, loi conjointe}
\auteur{}
\datecreate{2023-09-06}
\organisation{AMSCC}
\contenu{

\texte{ 	Soit $(X,Y)$ un couple de variables aléatoires suivant une loi uniforme sur le disque unité $D$ centré en $(0,0)$. }

\begin{enumerate}
	\item \question{ Déterminer le réel $\alpha$ tel que $f \colon (x,y) \mapsto \alpha \textbf{1}_D(x,y)$ soit une densité caractérisant la loi du couple $(X,Y)$. }
	\reponse{ Le disque $D$ a pour aire $\pi$ donc $(X,Y)$ admet pour densité $f(x,y)=\frac{1}{\pi} \textbf{1}_D(x,y)$ }
	\item \question{  Déterminer la loi de $X$. }
	\reponse{ D'après le cours, $X$ admet pour densité $f_X$ définie par  $f_X(x) = \int_{\R}^{} f(x,y)dy = \frac{1}{\pi} \int_{\{y^2<1-x^2\}}^{} dy$.
		
		Or $ y^2<1-x^2 \iff \begin{cases}
		-\sqrt{1-x^2} < y < \sqrt{1-x^2} \\
		-1<x<1
		\end{cases}$ donc $$f_X(x) = \frac{1}{\pi} \textbf{1}_{]-1;1[}(x)\int_{-\sqrt{1-x^2}}^{+\sqrt{1-x^2}} dy =  \frac{2}{\pi} \sqrt{1-x^2} \textbf{1}_{]-1;1[}(x)$$ }
\end{enumerate}
}
