\uuid{aOYg}
\titre{Etude d'une matrice pentadiagonale}
\theme{analyse numérique}
\auteur{}
\datecreate{2023-03-02}
\organisation{AMSCC}
\contenu{


\texte{ Soit $n \geq 3$, $\varepsilon >0$ et $A_\varepsilon$ la matrice pentadiagonale définie par 

$$A_\varepsilon = \begin{pmatrix}
	1 & \varepsilon & \varepsilon^2  &  &  & \\
	\varepsilon & 1 & \varepsilon & \varepsilon^2     & &  \\
	\varepsilon^2 & \ddots & \ddots & \ddots & & \\
	&   \ddots &  &  & \varepsilon  & \varepsilon^2 \\
	&  &  & \varepsilon &1  &\varepsilon \\
	& & & \varepsilon^2 & \varepsilon & 1 \\
\end{pmatrix}
$$
et on s'intéresse au système $A_\varepsilon x = b$ où $x,b \in \mathbb{R}^n$. }
\begin{enumerate}
	\item \question{ Ecrire une fonction Python qui génère la matrice $A_\varepsilon$ pour tout $n$ et $\varepsilon$ : }
	
	\begin{python}
		def epsmatrice(n,epsilon):
		...
		return ...
	\end{python}
	
	%\reppython{
	%def epsmatrice(n,epsilon):
	%  v = ones(n-1)*epsilon
	%  A=eye(n)+diag(v,-1)+diag(v,1)
	%  v = ones(n-2)*epsilon**2
	%  A = A + diag(v,-2) + diag(v,2)
	%  return A
	%}
	\reponse{ \insertnotebook{aOYg}}
	\item \question{ Donner un intervalle de valeurs de $\varepsilon$ pour lesquelles $A_\varepsilon$ est à diagonale strictement dominante.%, et calculer numériquement le conditionnement de $A_\varepsilon$ pour $n=10$ et $\varepsilon = 0.2$. 
	}
	\reponse{La matrice $A_\varepsilon$ est à diagonale strictement dominante si et seulement si $1 > 2\varepsilon + 2\varepsilon^2 \iff \varepsilon \in \left[0 ; \frac{\sqrt{3}-1}{2} \right]$. }
	\item \question{ Que permet de calculer la fonction suivante où \texttt{matrice} est une matrice carrée de taille $n$ quelconque ? }
		
	\begin{python}
		def rs(matrice):
		return max(abs(eigvals(matrice)))
	\end{python}

	\reponse{On reconnaît le calcul du rayon spectral de la matrice.}
	\item \question{ Ecrire une fonction Python qui génère la matrice $b_{\varepsilon} = A_\varepsilon\overline{x}$ où $\overline{x} = (1,...,1) \in \mathbb{R}^n$.}
	\begin{python}
		def epsb(n,epsilon):
		...
		return b
	\end{python}
	\item \question{ La méthode de Jacobi est-elle convergente pour $n=10$ et $\varepsilon = 0.2$ ? Si oui, résoudre le système $A_\varepsilon x = b_{\varepsilon}$ par cette méthode et donner le nombre d'itérations nécessaire pour une erreur de $10^{-8}$ et un vecteur initial $x_0 = 0$.  }
	\reponse{La matrice est à diagonale strictement dominante pour cette valeur de $\varepsilon$. La résolution demande $26$ itérations.}
	\item \question{ Soit $B$ la matrice d'itération associée à la méthode de Jacobi pour la matrice $A_\varepsilon$. Pour $n=20$ fixé, représenter graphiquement le rayon spectral de $B$ en fonction de $\varepsilon \in [0;1]$.  }
	\item \question{ Faire de même pour la méthode de Gauss-Seidel. }
\end{enumerate}}
