\uuid{TORD}
\titre{Produit matriciel}
\theme{calcul matriciel}
\auteur{}
\datecreate{2022-12-15}
\organisation{AMSCC}
\contenu{

\texte{ On considère la matrice $A=\begin{pmatrix}1 & 2 & 1 & 3 \\ 1 & 0 & 2 & 1\end{pmatrix}$. }

\question{  Calculer $A^\top\times A$ et $A \times A^\top$ où $A^\top$ désigne la matrice transposée de $A$. }


\reponse{ $$
\begin{aligned}
	& A=\left(\begin{array}{llll}
		1 & 2 & 1 & 3 \\
		1 & 0 & 2 & 1
	\end{array}\right) \Rightarrow{ }^t A=\left(\begin{array}{ll}
		1 & 1 \\
		2 & 0 \\
		1 & 2 \\
		3 & 1
	\end{array}\right) \\
	& \left({ }^t A\right) \cdot A=\left(\begin{array}{ll}
		1 & 1 \\
		2 & 0 \\
		1 & 2 \\
		3 & 1
	\end{array}\right)\left(\begin{array}{llll}
		1 & 2 & 1 & 3 \\
		1 & 0 & 2 & 1
	\end{array}\right)=\left(\begin{array}{cccc}
		2 & 2 & 3 & 4 \\
		2 & 4 & 2 & 6 \\
		3 & 2 & 5 & 5 \\
		4 & 6 & 5 & 10
	\end{array}\right) \\
	& \text { A. }\left({ }^t A\right)=\left(\begin{array}{llll}
		1 & 2 & 1 & 3 \\
		1 & 0 & 2 & 1
	\end{array}\right)\left(\begin{array}{ll}
		1 & 1 \\
		2 & 0 \\
		1 & 2 \\
		3 & 1
	\end{array}\right)=\left(\begin{array}{cc}
		15 & 6 \\
		6 & 6
	\end{array}\right) \\
	&
\end{aligned}
$$ }}
