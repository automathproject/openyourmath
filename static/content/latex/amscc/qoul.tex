\uuid{qoul}
\titre{Loi normale, calculs numériques}
\theme{loi normale}
\auteur{Maxime NGUYEN}
\datecreate{2022-10-17}
\organisation{AMSCC}
\contenu{

\begin{enumerate}
  \item \question{ A l'aide des tables de valeurs, calculer $\PP(-1.2 \leq Z \leq 1.1)$ où $Z$ suit une loi $\mathcal{N}(0,1)$. }
\reponse{On exprime la probabilité à l'aide de la fonction de répartition de la loi normale centrée réduite notée $\Phi$ : 
	\begin{align*}
		\PP(-1.2 \leq Z \leq 1.1) &= \PP(Z \leq 1.1) - \PP(Z < -1.2) \\
		&= \PP(Z \leq 1.1) - \PP(Z > 1.2) &\text{par symétrie}\\
		&= \PP(Z \leq 1.1) - (1-\PP(Z \leq 1.2)) \\
		&= \Phi(1.1)+\Phi(1.2)-1 \\
		&\approx 0{,}75
	\end{align*}  
	
}
\item \question{ A l'aide des tables de valeurs, calculer $\PP(70 \leq QI \leq 130)$ où $QI$ suit une loi $\mathcal{N}(100,\sigma=15)$. }
\reponse{On pose $Z = \frac{QI-100}{15}$ de telle sorte que $Z$ suit une loi normale centrée réduite. On exprime alors
	\begin{align*}
		\PP(70 \leq QI \leq 130) &= \PP\left(\frac{70-100}{15} \leq \frac{QI-100}{15} \leq \frac{130-100}{15}\right) \\
		&= \PP(-2 \leq Z \leq 2) \\
		&= 2\times \Phi(2) -1 \\
		&\approx 0{,}95
	\end{align*}
}
\item \question{ A l'aide des tables de valeurs, déterminer un réel $t$ tel que $\PP(|X-5|<t) = 95\%$ où 
$X$ suit une loi $\mathcal{N}(5,\sigma=1)$. }
\reponse{On pose $Z = {X-5}$ de telle sorte que $Z$ suit une loi normale centrée réduite. On exprime alors
	\begin{align*}
		\PP(|X-5| <t) &= \PP\left( -t<X-5<t\right) \\
		&= \PP\left( -t \leq Z \leq t\right) \\
		&= 2\times \Phi(t) -1 \\
	\end{align*} 
	On cherche $t \in \R$ tel que 
	\begin{align*}
		\PP(|X-5| <t) = 0.95 &\iff 2\times \Phi(t) -1 = 0.95 \\
		&\iff \Phi(t) = 0.975 \\
		&\iff t \approx 1{,}96 &\text{par lecture inverse de la table}
	\end{align*}
}
\end{enumerate}}
