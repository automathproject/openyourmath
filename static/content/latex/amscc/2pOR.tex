\uuid{2pOR}
\titre{Inversion d'une matrice}
\theme{calcul déterminant, calcul matriciel}
\auteur{Maxime NGUYEN}
\datecreate{20-12-2024}
\organisation{AMSCC}

\contenu{

\texte{ 
    Soit $A = \begin{pmatrix} 1 & 0 & 1 \\ -1 & 0 & 2 \\ 0 & 1 & -1 \end{pmatrix} \in \mathcal{M}_{3}(\mathbb{R})$. 
 }

 \begin{enumerate}
	\item \question{Calculer le déterminant de $A$.}
%	\indication{}
    \reponse{On développe $A$ suivant la 2ème colonne pour obtenir:\\
 $$\det(A)=-1 \times (2\times 1-1 \times (-1))=-3.$$   
    }
    \item \question{Montrer que $A^3-3A +3I_{3}$ est la matrice nulle.}
    
   % \indication{}
    \reponse{$A^2=\begin{pmatrix} 1 & 0 & 1 \\ -1 & 2 & -3 \\ -1 & -1 & 3 \end{pmatrix}$
    $A^3=\begin{pmatrix} 0 & 0 & 3 \\ -3 & -3 & 6 \\ 0 & 3 & -6 \end{pmatrix}$
    d'o\`u $A^3-3A=-3 I_{3} \Longleftrightarrow A^3-3A+3 I_{3}=0$.}
    \item \question{Déduire d'une des questions précédentes que $A$ est inversible et déterminer son inverse.}
\reponse{On a donc $A \times \frac{1}{-3}(A^2-3 I_{3})=I_{3}$ et la matrice inverse est donc
$A^{-1}= -\frac{1}{3}(A^2-3 I_{3})$
}
\end{enumerate}

}
