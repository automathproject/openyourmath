\uuid{yExC}
\exo7id{6283}
\auteur{mayer}
\datecreate{2011-10-16}
\isIndication{false}
\isCorrection{false}
\chapitre{Sous-variété}
\sousChapitre{Sous-variété}

\contenu{
\texte{
Soit $F=(F_1,...,F_k)$ une application $C^1$ d'un
ouvert $U$ de $\Rr^m$ dans $\Rr^k$. Notons $M=\{x\in U\; ; \;
F(x) =0 \}$ et soit $a\in M$.
}
\begin{enumerate}
    \item \question{\'Etablir l'équivalence des propriétés suivantes:
\begin{itemize}}
    \item \question{$DF(a)$ est surjective.}
    \item \question{Les formes linéaires $DF_1(a), ..., DF_k(a)$ sont
linéairement indépendantes.}
    \item \question{$\mathrm{Ker}\, DF(a) = \bigcap_{i=1}^k \mathrm{Ker}\, DF_i(a)$ est de dimension
$m-k$.
\end{itemize}}
    \item \question{Un point $a\in M$ est dit {\it point régulier}  si
$DF(a)$ est surjective. Montrer que l'ensemble des points
réguliers de $M$ est un ouvert de $M$.}
\end{enumerate}
}
