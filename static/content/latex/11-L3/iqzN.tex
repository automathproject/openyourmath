\uuid{iqzN}
\exo7id{7678}
\auteur{mourougane}
\datecreate{2021-08-11}
\isIndication{false}
\isCorrection{false}
\chapitre{Sous-variété}
\sousChapitre{Sous-variété}

\contenu{
\texte{
On considère l'application $f$ de la sphère unité $\mathcal{S}$ de $\Rr^3$ 
à valeurs dans le cylindre 
$\mathcal{C}$ de $\Rr^3$ d'équation $x^2+y^2=1$ donnée en coordonnées cartésiennes par
$$f~: \begin{pmatrix}x\\y\\z\end{pmatrix}\mapsto \begin{pmatrix}\frac{x}{\sqrt{x^2+y^2}}\\ \frac{y}{\sqrt{x^2+y^2}}\\ 
z\end{pmatrix}.$$
}
\begin{enumerate}
    \item \question{Faire une figure pour décrire géométriquement cette application.}
    \item \question{On considère un paramétrage local de la sphère unité $\mathcal{S}$ 
en coordonnées polaires
$$\begin{array}{ccc}
 F : (\theta,\phi)\in ]0,2\pi[\times]0,\pi[&\mapsto&\begin{pmatrix}\cos\theta\sin\phi\\ 
\sin\theta\sin\phi\\ \cos\phi\end{pmatrix}.
\end{array}$$
Déterminer la première forme fondamentale de la sphère dans cette paramétrisation.}
    \item \question{On considère l'application $G=f\circ F$. Montrer que c'est un paramétrage local 
du cylindre $\mathcal{C}$
et calculer la première forme fondamentale du cylindre dans cette paramétrisation.}
    \item \question{L'application $f$ est-elle un difféomorphisme local ? Un difféomorphisme de $\mathcal{S}$
sur $\mathcal{C}$ ?}
    \item \question{L'application $f$ est-elle une isométrie locale ? Conserve-t-elle les aires ?}
    \item \question{En déduire l'aire de la portion de sphère entre deux grands cercles passant 
par les poles nord
et sud et faisant entre eux un angle de mesure $\alpha$. 
Vérifier en déterminant l'aire de la sphère.}
\end{enumerate}
}
