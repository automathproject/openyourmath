\uuid{MkHI}
\exo7id{6717}
\auteur{queffelec}
\organisation{exo7}
\datecreate{2011-10-16}
\isIndication{false}
\isCorrection{false}
\chapitre{Formule de Cauchy}
\sousChapitre{Formule de Cauchy}

\contenu{
\texte{
Soit $D$ le disque unité ouvert. On dira qu'une fonction $E$ est
unitaire si elle est holomorphe dans $D$, continue sur $\overline{D}$ et
si $\vert f(z)\vert =1$ si $\vert z\vert=1$.
}
\begin{enumerate}
    \item \question{Montrer qu'une fonction unitaire dans $D$ n'a qu'un nombre fini de
zéros.

Montrer qu'une fonction unitaire sans zéro est une constante.

Montrer qu'une fonction unitaire ayant les points $a_1,a_2,\dots,a_n$
pour zéros (chacun étant compté avec son ordre de multiplicité) s'écrit
$$E(z)=c\prod_{j=1}^n{z-a_j\over 1-\overline{a_j}z}.$$}
    \item \question{Soit $f$ holomorphe sur $D$ et non identiquement nulle et supposons
qu'il existe $M>0$ tel que $\vert f(z)\vert \le M$ sur $D$. Soit $E$ une
fonction unitaire dans $D$ et telle que $f(z)/E(z)$ soit holomorphe dans
$D$. Montrer que l'on a
$$\forall z\in D,\ \vert f(z)\vert \le M\vert E(z)\vert .$$
Soit $a_1,a_2,\dots, a_n,\dots $ la suite des zéros de $f$ dans $D$,
chacun étant compté avec son ordre de multiplicité. Montrer que
$$\forall n\ge 1,\ \vert f(0)\vert \le M\vert a_1\vert\vert
a_2\vert\cdots\vert a_n\vert .$$
En déduire que si $f(0)\ne 0$, la série $\sum_{n\ge 1}(1-\vert
a_n\vert)$ converge.}
\end{enumerate}
}
