\uuid{nLfk}
\exo7id{6669}
\auteur{queffelec}
\organisation{exo7}
\datecreate{2011-10-16}
\isIndication{false}
\isCorrection{false}
\chapitre{Formule de Cauchy}
\sousChapitre{Formule de Cauchy}

\contenu{
\texte{
Soit $f(z)=\sum_{n\geq0} a_n z^n$ la somme d'une série entière de rayon infini.
Montrer que pour
$r>0$ et $n\geq 0$  
$$a_n={1\over 2\pi r^n}\int_0^{2\pi}f(re^{it})\ e^{-int}\ dt$$.

En déduire que si $|f(z)|\leq A+B|z|^k$ pour tout $z$ de module $\geq R$, $f$
est un polyn\^ome.
}
}
