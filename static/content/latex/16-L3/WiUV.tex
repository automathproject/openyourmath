\uuid{WiUV}
\exo7id{2676}
\auteur{matexo1}
\datecreate{2002-02-01}
\isIndication{false}
\isCorrection{true}
\chapitre{Théorème des résidus}
\sousChapitre{Théorème des résidus}

\contenu{
\texte{
R{\'e}soudre l'{\'e}quation $\cos z = a$, o{\`u} $a$ est un r{\'e}el $>1$. Donner le sinus des
solutions. En d{\'e}duire la valeur de
$$ I_a = \int_0^{+\infty } {dx\over (1+x^2) (a-\cos x)}.$$
}
\reponse{
On utilise 
$$ \cos(x+iy) = \cos x \cosh y -i \sin x \sinh y$$
Ici on doit avoir $\cos x \cosh y = a$ et $\sin x \sinh y = 0$. La derni{\`e}re {\'e}quation
implique $x= 0$ ou~$\pi $, ou ${y=0}$. La solution $y=0$ ne convient pas, car aucun r{\'e}el
n'a pour cosinus $a>1$. Donc il faut $x=0$, et $\cosh y = a$ ou bien $x=\pi $ et $\cosh
y=-a$. Cette derni{\`e}re {\'e}quation est sans solutions ($\cosh y\geq 1$ pour tout $y$). Les
solutions sont donc $z=\pm i\arg\cosh a$. Leur sinus est $\sin x \cosh y+i\cos x \sinh y = i\sinh
y$ ici, soit $\pm i\sqrt {a^2-1}$ (car $\sinh^2 y = \cosh^2 y -1$).

On a donc, en posant $f(z) = 1/[(1+z^2)(2-\cos z)]$ (fonction paire):
$$ I_a = i\pi  \mbox{\rm Res}(f, i) + i\pi  \mbox{\rm Res}(f, i\arg\cosh a) $$
car $i$ et $i\arg\cosh a$ sont les deux seuls p{\^o}les de $f$ de partie imaginaire $>0$.
Puis
$$\begin{array}{ccc}
 \mbox{\rm Res}(f,i) &= \left(1\over 2z(a-\cos z)\right)_{z=i} = {1\over 2i (a-\cosh 1)}\cr
\mbox{\rm Res}(f, i\arg\cosh a) &= \left(1\over (1+z^2) \sin z\right)_{z=i\arg\cosh a} =
 {1\over i(1 -\arg\cosh^2 a)\sqrt{a^2-1}}\cr\end{array}$$
Donc
$$ I_a = {\pi \over (1 -\arg\cosh^2 a)\sqrt{a^2-1}} + {\pi \over 2 (a-\cosh 1)}.$$
}
}
