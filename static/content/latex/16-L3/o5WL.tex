\uuid{o5WL}
\exo7id{7633}
\auteur{mourougane}
\organisation{exo7}
\datecreate{2021-08-10}
\isIndication{false}
\isCorrection{true}
\chapitre{Autre}
\sousChapitre{Autre}

\contenu{
\texte{
Existe-t-il une application $f :\Cc\to\Cc$ holomorphe non constante et bi-périodique de périodes $1$ et $i$ i.e. 
$$\forall z\in\Cc,\ \ f(z+1)=f(z+i)=f(z).$$
Si oui, donner un exemple. Si non, démontrer la non-existence d'une telle application.
}
\reponse{
Une telle application serait majorée en module par le maximum du module de ses valeurs sur le carré compact délimité par les points d'affixes $0,1,1+i,i$.
Elle serait donc entière et bornée donc constante, par le théorème de Liouville. Il n'y a donc pas de telle application.
}
}
