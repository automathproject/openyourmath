\uuid{e6iS}
\exo7id{2869}
\auteur{burnol}
\datecreate{2009-12-15}
\isIndication{false}
\isCorrection{false}
\chapitre{Théorème des résidus}
\sousChapitre{Théorème des résidus}

\contenu{
\texte{

}
\begin{enumerate}
    \item \question{Déterminer \[\int_0^{+\infty} \frac{dx}{1 + x^3}\] Pour ce
calcul, on considérera le contour allant le long de l'axe
réel de $0$ à $R$ puis de $R$ à $j R$ le long d'un cercle
puis de $jR$ à $0$ par un segment ($j =
\exp(i\frac{2\pi}3)$). On écrira d'une part chacune des trois
contributions à l'intégrale de contour, en faisant attention
au sens de parcours, et l'on utilisera  d'autre part le
théorème des résidus.}
    \item \question{On note, pour $|w| = 1$ et certaines
valeurs spéciales de $w$ (que l'on précisera) étant exclues,
$J(w)$ l'intégrale $\int \frac{dz}{1+z^3}$ le long du segment
infini $w\Rr^+$. Déterminer $J(w)$ en fonction de $w$.}
\end{enumerate}
}
