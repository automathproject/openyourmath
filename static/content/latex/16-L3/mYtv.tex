\uuid{mYtv}
\exo7id{2833}
\auteur{burnol}
\organisation{exo7}
\datecreate{2009-12-15}
\isIndication{false}
\isCorrection{true}
\chapitre{Théorème des résidus}
\sousChapitre{Théorème des résidus}

\contenu{
\texte{
Déterminer la série de Laurent à l'origine de
la fonction analytique $\exp(\frac 1z)$, et son résidu à
l'origine. En $z_0\neq0$ quel est le résidu de cette fonction?
}
\reponse{
Comme $e^w =\sum_{n\geq 0} \frac{w^n}{n!}$,
$$\exp (\frac{1}{z}) =\sum_{n\geq 0} \frac{1}{n!}\frac{1}{z^n} = \sum_{n=-\infty}^0 \frac{1}{(-n)!} z^n.$$
Par cons\'equent, $\mathrm{Res}(f,0) = 1$. Sinon, si $z_0\neq 0$, alors $\mathrm{Res}(f,z_0) = 0$ par holomorphie de $f$ dans  $\C\setminus \{0\}$.
}
}
