\uuid{iqMB}
\exo7id{6836}
\auteur{gijs}
\organisation{exo7}
\datecreate{2011-10-16}
\isIndication{false}
\isCorrection{false}
\chapitre{Autre}
\sousChapitre{Autre}

\contenu{
\texte{
Soit $P$ et $Q$ deux polyn\^omes à coefficients complexes
sans zéro commun et soit $z_1, \dots, z_k \in \Cc$
les zéros de $Q$ (ce qui implique que le degré de $Q$
est supérieur ou égal à $k$). On définit la
fonction $f:\Cc \setminus \{z_1, \dots, z_k\} \to \Cc$
par $f(z) = P(z)/Q(z)$.
}
\begin{enumerate}
    \item \question{Démontrer qu'il existe
une fonction continue $g: \Cc_\infty \to \Cc_\infty$ telle
que $\forall z\in \Cc\setminus\{z_1, \dots, z_k\}$~: $g(z)
= f(z)$. Quelle est la valeur de $g$ en un point $z_i$~?
Quelle est la valeur de $g$ en $\infty$~? (N'oubliez pas
de démontrer que la fonction $g$ que vous définissez
est continue.)}
    \item \question{Le résultat reste-t-il vrai si $P$ et
$Q$ ont des zéros  communs? Si non, donner un contre
exemple~; si oui, esquisser votre raisonnement.}
\end{enumerate}
}
