\uuid{WEOA}
\exo7id{6734}
\auteur{queffelec}
\datecreate{2011-10-16}
\isIndication{false}
\isCorrection{false}
\chapitre{Autre}
\sousChapitre{Autre}

\contenu{
\texte{
Soit $P(x,y)$ et $Q(x,y)$ des fonctions continues à valeurs réelles
et à dérivées partielles continues sur un ouvert connexe $\Omega $ et
sur sa frontière $C$. La formule de Green établit que
$$\int_CPdx+Qdy= \int\!\int_\Omega \left({\partial Q\over \partial
x}-{\partial P\over \partial y}\right) dxdy$$
}
\begin{enumerate}
    \item \question{Montrer la formule de Green pour une courbe fermée simple $C$ ayant
la propriété d'être rencontrée par des parallèles aux axes de
coordonnées en deux points au plus.}
    \item \question{Si $f(z,\overline{z})=u(x,y)+iv(x,y)$ est continue et possède des
dérivées partielles continues dans un ouvert connexe $\Omega $ et sur
sa frontière $C$, montrer que la formule de Green peut s'écrire sous la
forme complexe suivante
$$\int_Cf(z,\overline{z})dz=2i\int\!\int_\Omega {\partial f\over
\partial \overline{z}}dxdy$$}
    \item \question{Si $C$ est une courbe fermée simple délimitant un ouvert d'aire $A$,
montrer que $A={1\over 2i}\int_C\overline{z}dz$.}
    \item \question{Calculer $\int_C\overline{z}dz$ le long
\begin{enumerate}}
    \item \question{du cercle $\vert z-2\vert =3$}
    \item \question{du carré de sommets $z=0$, $z=2$, $z=2i$ et $z=2+2i$}
    \item \question{de l'ellipse $\vert z-3\vert +\vert z+3\vert =10$}
\end{enumerate}
}
