\uuid{QCdm}
\exo7id{2821}
\auteur{burnol}
\organisation{exo7}
\datecreate{2009-12-15}
\isIndication{false}
\isCorrection{true}
\chapitre{Théorème des résidus}
\sousChapitre{Théorème des résidus}

\contenu{
\texte{
On considère la fonction analytique $f(z) =
\frac1{\sin(z)}$ sur l'ouvert $U$ complémentaire de
$\pi\Zz$. Vérifier que la fonction $\sin(z)$ ne s'annule
jamais sur $U$. Déterminer  en tout $z_0\in U$ donné le
rayon de convergence du développement en série de Taylor de
$f$. 
\emph{Remarque :} il est déconseillé de chercher à résoudre ce
problème en déterminant explicitement les coefficients des
séries de  Taylor.
}
\reponse{
On a $\sin (z) = \frac{1}{2i}(e^{iz}-e^{-iz})=0$ si et seulement si $e^{2i\pi z}=1$ ce qui est le cas si et seulement si $z\in \pi \Z$.
Soit $z_0\in U=\C\setminus \pi \Z$. Alors, le rayon de convergence de la s\'eries de Taylor de $f$ est
$$R= \mathrm{dist} (z_0 , \pi \Z ) = \min_{n\in \Z} |z_0 -\pi n| .$$
}
}
