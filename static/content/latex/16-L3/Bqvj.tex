\uuid{Bqvj}
\exo7id{2852}
\auteur{burnol}
\datecreate{2009-12-15}
\isIndication{false}
\isCorrection{false}
\chapitre{Fonction logarithme et fonction puissance}
\sousChapitre{Fonction logarithme et fonction puissance}

\contenu{
\texte{

}
\begin{enumerate}
    \item \question{On considère la fonction analytique $\phi(a) = \mathrm{Log}(a-1) -
  \mathrm{Log}(a+1)$ dans le demi-plan supérieur et la fonction
  analytique $\psi(a) =  \mathrm{Log}(a-1) -
  \mathrm{Log}(a+1)$ dans le demi-plan inférieur. Montrer que $\phi$
  et $\psi$ sont la restriction à leurs demi-plans
  respectifs d'une fonction analytique sur
  $\Cc\setminus[-1,+1]$. 
  \emph{Indication :} il y a plusieurs
  raisonnements possibles et plusieurs indications
  possibles. Donc, débrouillez vous.}
    \item \question{On considère la fonction $a\mapsto
\frac{a-1}{a+1}$. Quelle est l'image par cette fonction de
l'intervalle $]-1,1[$?  Quelle est l'image par cette
fonction de $\Cc\setminus[-1,+1]$? En déduire que la
fonction composée $\Phi(a) = \mathrm{Log}\frac{a-1}{a+1}$ existe et
est analytique sur $\Cc\setminus[-1,+1]$. Retrouver le
résultat de la question précédente (et montrer que $\phi$,
$\psi$ et $\Phi$ coïncident dans les intersections
deux-à-deux de leurs ouverts de définitions).}
    \item \question{Quel est le développement en série de Laurent de la
  fonction analytique $\Phi$ dans la couronne
  $1<|a|<\infty$? Que vaut par exemple $\int_{|a|=2}
  \Phi(a)a^{18} da$?}
\end{enumerate}
}
