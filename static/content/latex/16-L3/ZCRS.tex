\uuid{ZCRS}
\exo7id{2802}
\auteur{burnol}
\datecreate{2009-12-15}
\isIndication{false}
\isCorrection{true}
\chapitre{Fonction holomorphe}
\sousChapitre{Fonction holomorphe}

\contenu{
\texte{
Lorsque $z$ est complexe les fonctions $\sin(z)$, $\cos(z)$, $\sh(z)$ et
   $\ch(z)$ sont définies par les formules:
\begin{align*}
\sin(z) &= \frac{e^{iz} - e^{-iz}}{2i}&\qquad \sh(z) &=
   \frac{e^{z} - e^{-z}}{2}  \\
 \cos(z) &=
   \frac{e^{iz} + e^{-iz}}{2}&\qquad \ch(z) &=
   \frac{e^{z} + e^{-z}}{2}
\end{align*}
}
\begin{enumerate}
    \item \question{Montrer que $\cos$ et $\ch$ sont des fonctions paires et
   $\sin$ et $\sh$ des fonctions impaires et donner leurs
   représentations comme
   séries entières. Prouver $e^{iz} = \cos(z) + i \sin(z)$,
   $\sin(iz) = i\sh(z)$, $\cos(iz) = \ch(z)$, $\sh(iz) =
   i\sin(z)$, $\ch(iz) = \cos(z)$.}
    \item \question{Établir les formules:
\[ \cos(z+w) = \cos(z)\cos(w) - \sin(z)\sin(w)\]
\[ \sin(z+w) = \sin(z)\cos(w) + \cos(z)\sin(w)\]
en écrivant de deux manières différentes $e^{\pm
   i(z+w)}$. Donner une autre preuve en utilisant le
   principe du prolongement analytique et la validité
   (admise) des formules pour $z$ et $w$ réels.}
    \item \question{Prouver pour tout $z$ complexe $\cos(\pi + z) = -\cos(z)$,
  $\sin(\pi+ z) =
     -\sin(z)$. Prouver $\cos(\frac\pi2 - z) = \sin(z)$.}
    \item \question{Prouver les formules $\cos^2 z + \sin^2 z = 1$ et $\ch^2 z
   - \sh^2 z = 1$ pour tout $z\in\Cc$.}
\reponse{
Il s'agit de formules bien connues lorsque les arguments $z,w$
sont r\'eels. La v\'erification \`a partir des d\'efinitions des fonctions trigonom\'etriques donn\'ees
dans l'\'enonc\'e de l'exercice est laiss\'ee au lecteur.
Voici comment obtenir la formule :
\begin{equation}\label{2} \cos(z+w) = \cos(z)\cos(w)-\sin(z)\sin(w) \qquad pour \quad z,w\in \C \, .
\end{equation}
Fixons $w\in \R$. Soit $f_w(z)=\cos (z+w)-\big( \cos(z)\cos(w)-\sin(z)\sin(w)\big)$.
La formule \eqref{2} \'etant vraie pour $z,w\in \R$, $f_w(z)=0$ pour tout $z\in \R$. Il r\'esulte du principe des z\'eros
isol\'es que $f_w$ est identiquement nulle. Autrement dit, on vient d'\'etablir la formule \eqref{2}
pour $(z,w)\in \C\times \R$. Il suffit maintenant de refaire le m\^eme argument en fixant d'abord $z\in \C$
arbitrairement et en observant que la fonction holomorphe
$$g_z(w)= \cos(z+w) - ( \cos(z)\cos(w)-\sin(z)\sin(w))$$
est nulle pour tout $w\in\R$. De nouveau $g_z\equiv 0$ par le principe des z\'eros isol\'es, d'o\`u la formule \eqref{2}
pour tout $z,w\in \C$.
}
\end{enumerate}
}
