\uuid{tRMg}
\titre{Résolution d'équation différentielle}
\theme{séries entières, équations différentielles}
\auteur{}
\datecreate{2023-06-05}
\organisation{AMSCC}
\contenu{


	\begin{enumerate}
		\item \question{ Exprimer sous forme d'une série entière la solution de l'équation différentielle $(E)$:
		\[ x^2y''(x)+x(x+1)y'(x)-y(x)=0 \qquad y'(0)=1.\] }
		\reponse{
			Soit $y$ une solution développable en série entière, de rayon de convergence $R$: on note $\displaystyle y(x)=\sum_{n=0}^{+\infty} a_n x^n$ et on a
			\[ \forall x \in ]-R;R[, \qquad y'(x)=\sum_{n=1}^{+\infty} na_n x^{n-1}\quad \text{ et } \quad y''(x)=\sum_{n=2}^{+\infty} n(n-1)a_n x^{n-2}.\]
			On a les équivalences suivantes:
			\begin{align*}
			y \text{ solution de }(E) \ & \Leftrightarrow \ 
			\forall x \in ]-R;R[, \ -a_0 + \sum_{n=2}^{+\infty} [(n-1)(n+1)a_n - (n-1)a_{n-1}] x^n =0  \\
			\ & \Leftrightarrow \
			a_0=0 \quad \text{ et } \quad  \forall n \geq 2, \ a_n=\frac{-1}{n+1} a_{n-1}
			\end{align*}
			La condition initiale donne: $y'(0)=a_1=1$.
			On en conclut que $a_0=0$, $a_1=1$ et pour tout $n\geq 2$, $a_n=2\frac{(-1)^{n+1}}{(n+1)!}$ et donc 
			\[ y(x)=2 \sum_{n=1}^{+\infty} \frac{(-1)^{n+1}}{(n+1)!} x^n,\] 
			définie sur l'intervalle $\R$. On peut également déterminer la fonction somme, ce qui donne
			\[ \forall x \in \R, \qquad y(x)=\sum_{n=1}^{+\infty} \frac{(-1)^{n+1}}{(n+1)!}x^n.\]
		}
		
		\item \question{ Déterminer l'intervalle de convergence $I$ de la série entière solution de l'équation différentielle ci-dessus. }
		\reponse{Fait ci-dessus: $D=\R$.}
		
		\item \question{ Déterminer la fonction somme de cette série entière. }
		\reponse{ On a:
			\[\forall x \in \R, \qquad xy(x)=2 \sum_{n=1}^{+\infty} \frac{(-1)^{n+1}}{(n+1)!}x^{n+1}=2\sum_{n=2}^{+\infty} \frac{(-x)^{n}}{n!}
			=2\left[\sum_{n=0}^{+\infty} \frac{(-x)^{n}}{n!}-(1-x)\right]
			\]
			d'où
			\[\forall x \in \R, \qquad y(x)=\frac{2}{x}(e^{-x}-1+x).\]
		}
		
	\end{enumerate}
}
