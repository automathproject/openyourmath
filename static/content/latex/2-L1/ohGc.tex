\uuid{ohGc}
\exo7id{5241}
\auteur{rouget}
\organisation{exo7}
\datecreate{2010-06-30}
\isIndication{false}
\isCorrection{true}
\chapitre{Suite}
\sousChapitre{Convergence}

\contenu{
\texte{
Montrer que, pour $n\geq2$, 

\begin{center}
$\cos\left(\frac{\pi}{2^{n}}\right)=\frac{1}{2}\sqrt{2+\sqrt{2+...+\sqrt{2}}}$ ($n-1$ radicaux) et $\sin\left(\frac{\pi}{2^{n}}\right)=\frac{1}{2}\sqrt{2-\sqrt{2+...+\sqrt{2}}}$ ($n-1$ radicaux).
\end{center}
En déduire $\lim_{n\rightarrow +\infty}2^n\sqrt{2-\sqrt{2+...+\sqrt{2}}}$ ($n$ radicaux).
}
\reponse{
L'égalité proposée est vraie pour $n=2$ car $\cos\frac{\pi}{2^2}=\cos\frac{\pi}{4}=\frac{\sqrt{2}}{2}$.

Soit $n\geq2$. Supposons que $\cos(\frac{\pi}{2^{n}})=\frac{1}{2}\sqrt{2+\sqrt{2+...\sqrt{2}}}$ ($n-1$ radicaux).

Alors, puisque $\cos(\frac{\pi}{2^{n+1}})>0$ (car $\frac{\pi}{2^{n+1}}$ est dans $]0,\frac{\pi}{2}[$), 

$$\cos(\frac{\pi}{2^{n+1}})=\sqrt{\frac{1+\cos(\frac{\pi}{2^{n}})}{2}}=\sqrt{\frac{1}{2}(1+\frac{1}{2}\sqrt{2+\sqrt{2+...\sqrt{2}}})}=\frac{1}{2}\sqrt{2+\sqrt{2+...\sqrt{2}}},\;(n\;\mbox{radicaux}).$$

On a montré par récurrence que, pour $n\geq2$, $\cos(\frac{\pi}{2^{n}})=\frac{1}{2}\sqrt{2+\sqrt{2+...\sqrt{2}}}$ ($n-1$ radicaux).
 
Ensuite, pour $n\geq2$, 

$$\sin(\frac{\pi}{2^{n}})=\sqrt{\frac{1}{2}(1-\cos(\frac{\pi}{2^{n-1}})}=\frac{1}{2}\sqrt{2-\sqrt{2+...\sqrt{2}}}\;(n-1\;\mbox{radicaux})$$

Enfin, 

$$2^n\sqrt{2-\sqrt{2+...\sqrt{2}}}=2^n.2\sin\frac{\pi}{2^{n+1}}\sim2^{n+1}\frac{\pi}{2^{n+1}}=\pi.$$

Donc, $\lim_{n\rightarrow +\infty}2^n\sqrt{2-\sqrt{2+...\sqrt{2}}}=\pi$.
}
}
