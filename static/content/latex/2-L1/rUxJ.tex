\uuid{rUxJ}
\exo7id{2083}
\auteur{bodin}
\organisation{exo7}
\datecreate{2008-02-04}
\isIndication{true}
\isCorrection{true}
\chapitre{Calcul d'intégrales}
\sousChapitre{Théorie}

\contenu{
\texte{
Calculer l'int\'egrale de $f:[a,b]\rightarrow \R$ comme
limite de sommes de Riemann-Darboux dans les cas suivants:
}
\begin{enumerate}
    \item \question{$f(x)=\sin x$ et $f(x)=\cos x$ sur $[0,\frac \pi 2]$ et $x_k=\frac{k\pi }{2n}$, $k=0,1,...,n$,}
\reponse{On calcul d'abord  $\int_0^{\frac \pi 2}  e^{it}\, dt$. Par le th\'eor\`eme de  Riemann-Darboux
c'est la limite de 
$$S_n = \sum_{k=0}^{n-1} (x_{k+1}-x_k) \cdot f(x_k).$$
Pour $x_k = \frac{k\pi }{2n}$ (on obtient en fait un somme de Riemann) :
$$S_n = \frac{\pi}{2n}\sum_{k=0}^{n-1}e^{\frac{ik\pi}{2n}}= \frac{\pi}{2n}\sum_{k=0}^{n-1}(e^{\frac{i\pi}{2n}})^k.$$
Ce qui est une somme g\'eom\'etrique de somme $S_n = (1-i) \frac{\frac {\pi}{2n}}{1-e^{i\frac {\pi}{2n}}}$. La limite de ce taux d'accroissement est $1+i$ (en posant $u=\frac {\pi}{2n}$ et en remarquant que $\frac{e^{iu}-1}{u} \to i$ quand $u\to 0$).
Donc $\int_0^{\frac \pi 2}  e^{it}\, dt=1+i$. Mais $e^{it} = \cos t+ i \sin t$ donc 
$\int_0^{\frac \pi 2} \cos t \, dt + \int_0^{\frac \pi 2}\sin t \, dt = 1+i$.
Par identification des parties r\'eelles et imaginaires on trouve : 
$\int_0^{\frac \pi 2} \cos t \, dt = 1$ et $\int_0^{\frac \pi 2}\sin t \, dt = 1$.}
    \item \question{$g(x)=\frac 1x$ sur $\left[ a,b\right] \subset \R_{+}^{*}$ et $%
x_k=aq^k$ , $k=0,1,...,n$ ($q$ \'etant \`a d\'eterminer),}
\reponse{On veut $x_k = aq^k$ ce qui donne bien $x_0=a$, mais il faut aussi $x_n = b$ donc
$aq^n=b$, donc $q^n = \frac ba$ soit $q = (\frac ba)^{\frac 1 n}$.
Nous cherchons la limite de $S'_n =  \sum_{k=0}^{n-1} (x_{k+1}-x_k) \cdot g(x_k).$
Il est n'est pas trop dur de montrer que $S'_n = n(q-1)$.
Pour trouver la limite quand $n\to +\infty$ c'est plus d\'elicat car $q$ d\'epend de $n$ :
$S'_n= n(q-1)= n((\frac ba)^{\frac 1 n}-1) = n(e^{\frac1n \ln\frac ba}-1)$.
En posant $u = \frac 1n$ et en remarquant que l'on obtient un taux d'accroissement on calcule :
$S'_n = \frac 1 u(e^{ u \ln\frac ba}-1) \to  \ln \frac ba = \ln b - \ln a$.
Donc $\int_a^b \frac {dt}{t} = \ln b - \ln a$.}
    \item \question{$h(x)=\alpha ^x$ sur $[a,b]$ , $\alpha >0$, et $x_k=a+(b-a).\frac kn$, $%
k=0,1,...,n$.}
\reponse{\`A l'aide des sommes g\'eom\'etrique est des taux d'accroissement on trouve 
$$\int_a^b \alpha^t \, dt = \frac{e^{\alpha b}- e^{\alpha a}}{\alpha}.$$}
\indication{\begin{enumerate}
  \item On pourra penser que le cosinus et le sinus sont les parties r\'eelles et imaginaires de la fonction $t \mapsto e^{it}$. On chercha donc d'abord \`a calculer $\int_0^{\frac \pi 2}  e^{it}\, dt$.
  \item On choisira $q$ tel que $q^n = \frac ba$.
  \end{enumerate}}
\end{enumerate}
}
