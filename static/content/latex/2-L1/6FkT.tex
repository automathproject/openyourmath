\uuid{6FkT}
\exo7id{1217}
\auteur{ridde}
\datecreate{1999-11-01}
\isIndication{false}
\isCorrection{true}
\chapitre{Continuité, limite et étude de fonctions réelles}
\sousChapitre{Fonctions équivalentes, fonctions négligeables}

\contenu{
\texte{
Calculer les limites de
}
\begin{enumerate}
    \item \question{$\dfrac{\sin x \ln (1 + x^2)}{x \tan x} \text{ en } 0$.}
\reponse{$\displaystyle{\lim _{x\rightarrow 0}{\frac
{\sin(x)\ln (1+{x}^{2})}{x\tan(x)}}=0 }$.}
    \item \question{$\dfrac{\ln (1 + \sin x)}{\tan (6x)} \text{ en } 0$.}
\reponse{$\displaystyle{\lim _{x\rightarrow 0}{\frac {\ln (1+\sin(x))}{\tan(6\,x)}}=1/6 }$.}
    \item \question{$ (\ln (e + x))^{\frac 1x} \text{ en } 0$.}
\reponse{$\displaystyle{\lim _{x\rightarrow 0}\left (\ln ({e}+x)\right )^{{x}^{-1}}={e^{{e^{-1}}}} }$.}
    \item \question{$(\ln (1 + e^{-x}))^{\frac 1x} \text{ en }  + \infty$.}
\reponse{$\displaystyle{\lim _{x\rightarrow \infty }\left (\ln (1+{e^{-x}})\right )^{{x}^{-1}}={e^{-1}} }$.}
\end{enumerate}
}
