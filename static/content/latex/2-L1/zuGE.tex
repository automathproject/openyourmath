\uuid{zuGE}
\exo7id{4430}
\auteur{quercia}
\organisation{exo7}
\datecreate{2010-03-14}
\isIndication{false}
\isCorrection{true}
\chapitre{Série numérique}
\sousChapitre{Autre}

\contenu{
\texte{
Calculer les sommes des séries suivantes :
}
\begin{enumerate}
    \item \question{$\sum_{k=2}^\infty \frac 1{k^2-1}$.}
\reponse{$\frac 34$.}
    \item \question{$\sum_{k=1}^\infty \frac 1{k(k+1)(k+2)}$.}
\reponse{$\frac 14$.}
    \item \question{$\sum_{k=1}^\infty \frac 1{k(k+1)\dots(k+p)}$.}
\reponse{$S_p - (p+1)S_{p+1} = S_p - \frac 1{(p+1)!} \Rightarrow 
             S_p=\frac 1{pp!}$.}
    \item \question{$\sum_{k=0}^\infty \frac 1{k^3+8k^2+17k+10}$.}
\reponse{$\frac{23}{144}$.}
    \item \question{$\sum_{k=1}^\infty \ln\left(1+\frac2{k(k+3)}\right)$.}
\reponse{$\ln 3$.}
    \item \question{$\sum_{k=2}^\infty \ln\left(1-\frac1{k^2}\right)$.}
\reponse{$-\ln 2$.}
    \item \question{$\sum_{k=0}^\infty \ln\left(\cos\frac\alpha{2^k}\right)$.}
\reponse{$\ln\left(\frac{\sin2\alpha}{2\alpha}\right)$.}
    \item \question{$\sum_{k=0}^\infty 2^{-k}\tan(2^{-k}\alpha)$.}
\reponse{$\frac1\alpha-2\mathrm{cotan}(2\alpha)$.}
    \item \question{$\sum_{k=0}^\infty \frac{2k^3-3k^2+1}{(k+3)!}$.}
\reponse{$109 - 40e$.}
    \item \question{$\sum_{n=p}^\infty C_n^p x^n$.}
\reponse{$\frac{x^p}{(1-x)^{p+1}}$ pour $|x|<1$ par récurrence.}
    \item \question{$\sum_{k=1}^\infty \frac{x^k}{(1-x^k)(1-x^{k+1})}$.}
\reponse{$\frac x{(1-x)^2}$ si $|x| <1$,
             $\frac 1{(1-x)^2}$ si $|x| > 1$.}
    \item \question{$\sum_{k=1}^\infty \frac{k-n[k/n]}{k(k+1)}$.}
\reponse{$S_n = \sum_{q=0}^\infty\sum_{r=1}^{n-1}
                       \frac r{(qn+r)(qn+r+1)}
                  = \sum_{q=0}^\infty\sum_{r=1}^{n-1}
                       \frac r{qn+r} - \frac r{qn+r+1}$.
                  \par
             $S_n = \sum_{q=0}^\infty \left(\frac 1{qn+1} + \frac 1{qn+2}
                       + \dots + \frac 1{qn+n} - \frac 1{q+1}\right)
                  = \lim_{N\to\infty}\left(\sum_{k=1}^{(N+1)n} \frac 1k
                       - \sum_{k=1}^{N+1} \frac 1k \right)
                  = \ln n$.}
\end{enumerate}
}
