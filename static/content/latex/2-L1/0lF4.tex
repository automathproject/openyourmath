\uuid{0lF4}
\exo7id{749}
\auteur{bodin}
\datecreate{1998-09-01}
\isIndication{true}
\isCorrection{true}
\chapitre{Fonctions circulaires et hyperboliques inverses}
\sousChapitre{Fonctions circulaires inverses}

\contenu{
\texte{
Résoudre les équations suivantes:
}
\begin{enumerate}
    \item \question{$\Arccos x = 2\Arccos \frac{3}{4}$.}
\reponse{On vérifie d'abord que $2\Arccos \frac{3}{4}\in[0,\pi]$ (sinon, l'équation 
n'aurait aucune solution). En effet, par définition, la fonction $\Arccos$ est décroissante 
sur $[-1,1]$ à valeurs dans $[0,\pi]$, donc puisque $\frac{1}{2}\le\frac{3}{4}\le 1$ 
on a $\frac{\pi}{3}\ge\cos\left(\frac{3}{4}\right)\ge 0$.
Puisque par définition $\Arccos x\in[0,\pi]$, on obtient en prenant le cosinus:
$$\Arccos x = 2\Arccos\left(\frac{3}{4}\right)\Longleftrightarrow x 
= \cos\left(2\Arccos \frac{3}{4}\right)$$
En appliquant la formule $\cos 2u = 2\cos^2u-1$, on arrive  donc \`a une unique solution
$x = 2(\frac{3}{4})^2-1 = \frac{1}{8}$.}
    \item \question{$\Arcsin x = \Arcsin \frac{2}{5} + \Arcsin \frac{3}{5}$.}
\reponse{Vérifions d'abord que $-\frac{\pi}{2}\le\Arcsin \frac{2}{5} + \Arcsin \frac{3}{5}\le\frac{\pi}{2}$. 
En effet, la fonction $\Arcsin$ est strictement croissante et 
$0<\frac{2}{5}<\frac{1}{2}<\frac{3}{5}<\frac{\sqrt{2}}{2}$, ce qui donne
$0<\Arcsin\left(\frac{2}{5}\right)<\frac{\pi}{6}<\Arcsin\left(\frac{3}{5}\right)<\frac{\pi}{4}$, 
d'où l'encadrement
$0+\frac{\pi}{6}<\Arcsin \frac{2}{5} + \Arcsin \frac{3}{5}\le\frac{\pi}{6}+\frac{\pi}{4}$.

Puisque par définition on aussi $\Arcsin x\in [-\frac{\pi}{2},\frac{\pi}{2}]$, 
il vient en prenant le sinus:
\begin{eqnarray*}
\lefteqn{\Arcsin x=\Arcsin \frac{2}{5} + \Arcsin \frac{3}{5}}\\
 &\Longleftrightarrow& x=\sin\left(\Arcsin \frac{2}{5} + \Arcsin \frac{3}{5}\right)\\
 &\Longleftrightarrow& x = \frac{3}{5}\cos\left(\Arcsin \frac{2}{5}\right)+\frac{2}{5} \cos\left(\Arcsin \frac{3}{5}\right)
\end{eqnarray*}
La dernière équivalence vient de la formule de $\sin(a+b)= \cos a \sin b + \cos b \sin a$
et de l'identité $\sin\big( \Arcsin u\big) = u$.

En utilisant la formule $\cos\left(\Arcsin x\right) = \sqrt{1-x^2}$, 
on obtient une unique solution: $x = \frac35\sqrt{\frac{21}{25}}+\frac25\frac 45 
=\frac{3\sqrt{21}+8}{25}$.}
    \item \question{$\Arctan {2x}+\Arctan x=\frac{\pi}{4}$.}
\reponse{Supposons d'abord que $x$ est solution. Remarquons d'abord que $x$ est 
nécessairement positif, puisque $\Arctan x$ a le même signe que $x$. 
Alors, en prenant la tangente des deux membres, on obtient $\tan\big(\Arctan(2x)+\Arctan(x)\big)=1$.

En utilisant la formule donnant la tangente d'une somme :
$\tan(a+b)=\frac{\tan a +\tan b}{1-\tan a \tan b}$, on obtient $\frac{2x+x}{1-2x\cdot x}=1$, 
et finalement $2x^2+3x-1=0$ qui admet une unique solution positive $x_0=\frac{-3+\sqrt{17}}{4}$. 
Ainsi, {\it si} l'équation de départ admet une solution, c'est nécessairement $x_0$. 

Or, en posant $f(x)=\Arctan(2x)+\Arctan(x)$, la fonction $f$ est continue sur $\Rr$. 
Comme $f(x)\xrightarrow[x\to -\infty]{}-\pi$ et $f(x)\xrightarrow[x\to +\infty]{}+\pi$, 
on sait d'après le théorème des valeurs intermédiaires que $f$ prend la valeur 
$\frac{\pi}{4}$ au moins une fois (et en fait une seule fois, puisque $f$ est 
strictement croissante comme somme de deux fonctions strictement croissantes). 
Ainsi l'équation de départ admet bien une solution, qui est $x_0$.}
\indication{On compose les équations par la bonne fonction (sur le bon domaine de définition), 
par exemple cosinus pour la premi\`ere. Pour la dernière, commencer par 
étudier la fonction pour montrer qu'il existe une unique solution.}
\end{enumerate}
}
