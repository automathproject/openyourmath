\uuid{3UuC}
\exo7id{3870}
\auteur{quercia}
\organisation{exo7}
\datecreate{2010-03-11}
\isIndication{false}
\isCorrection{true}
\chapitre{Continuité, limite et étude de fonctions réelles}
\sousChapitre{Continuité : théorie}

\contenu{
\texte{
Soit $f$ continue sur~$[a,b]$, à valeurs dans~$\R$, et $\delta$ un
réel positif. On note $\omega(\delta) = \sup\{|f(x)-f(y)|\text{ tq } |x-y|\le\delta\}$.
Montrer que $\omega(\delta)$ tend vers~$0$ quand $\delta$ tend vers~$0$,
puis que $\omega$ est continue.
}
\reponse{
$\omega(\delta) \to 0 \Leftrightarrow f$ (lorsque $\delta\to0^+$) est uniformément continue.

Continuité en $\delta>0$~: on remarque que $\omega$ est croissante
donc $\omega(\delta^-)$ et $\omega(\delta^+)$ existent et
encadrent $\omega(\delta)$.
Si $\delta_n \to \delta^+$ (lorsque $n\to\infty$), soient $x_n,y_n$ tels que
$\omega(\delta_n) = |f(x_n)-f(y_n)|$ et $|x_n-y_n|\le\delta_n$.
On extrait de $(x_n,y_n)$ une suite convergente vers~$(x,y)$ avec
$|x-y|\le\delta$ et $|f(x)-f(y)| = \omega(\delta^+)$ d'où
$\omega(\delta^+) \le \omega(\delta)$ puis
$\omega(\delta^+) = \omega(\delta)$.

On a aussi $\omega(\delta)= \sup\{|f(x)-f(y)|\text{ tq } |x-y|<\delta\}
\le \omega(\delta^-)$ d'où $\omega(\delta^-) = \omega(\delta)$.
}
}
