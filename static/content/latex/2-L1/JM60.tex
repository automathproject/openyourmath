\uuid{JM60}
\exo7id{5088}
\auteur{rouget}
\datecreate{2010-06-30}
\isIndication{false}
\isCorrection{true}
\chapitre{Fonctions circulaires et hyperboliques inverses}
\sousChapitre{Fonctions circulaires inverses}

\contenu{
\texte{
Simplifier les expressions suivantes :
}
\begin{enumerate}
    \item \question{$f_1(x)=\Arcsin\left(\frac{x}{\sqrt{1+x^2}}\right)$.}
\reponse{\textbf{1ère solution.} Pour tout réel $x$, $\sqrt{x^2+1}>\sqrt{x^2}=|x|$ et donc $-1<\frac{x}{\sqrt{x^2+1}}<
1$. Ainsi $f_1$ est définie et dérivable sur $\Rr$, impaire, et pour tout réel $x$,

$$f_1'(x)=\left(\frac{1}{\sqrt{x^2+1}}-\frac{1}{2}x\frac{2x}{(x^2+1)\sqrt{x^2+1}}\right)\frac{1}{\sqrt{1-\frac{x^2}{1+x^2}}}
=\frac{1}{1+x^2}=\Arctan'(x).$$
Donc il existe une constante réelle $C$ telle que pour tout réel $x$, $f_1(x)=\Arctan x+C$.
$x=0$ fournit
$C= 0$ et donc,

\begin{center}
\shadowbox{
$\forall x\in\Rr,\;\Arcsin\left(\frac{x}{\sqrt{x^2+1}}\right)=\Arctan x.$
}
\end{center}
\textbf{2ème solution.} Pour $x$ réel donné, posons $\theta=\Arctan x$. $\theta$ est dans
$\left]-\frac{\pi}{2},\frac{\pi}{2}\right[$ et $x=\tan\theta$.

\begin{align*}
\frac{x}{\sqrt{x^2+1}}&=\frac{\tan\theta}{\sqrt{1+\tan^2\theta}}=\sqrt{\cos^2\theta}\tan\theta=\cos\theta\tan\theta\;
(\mbox{car}\;\cos\theta>0)\\
 &=\sin\theta
\end{align*}
et donc

\begin{align*}
f_1(x)&=\Arcsin(\sin\theta)=\theta\;(\mbox{car}\;\theta\;\mbox{est dans}\;\left]-\frac{\pi}{2},\frac{\pi}{2}\right[)\\
 &=\Arctan x.
\end{align*}}
    \item \question{$f_2(x)=\Arccos\left(\frac{1-x^2}{1+x^2}\right)$.}
\reponse{\textbf{1ère solution.} Pour tout réel $x$, $-1<-1+\frac{2}{1+x^2}=\frac{1-x^2}{1+x^2}\leq-1+2=1$ (avec égalité
si et seulement si $x=0$). $f_2$ est donc définie et continue sur $\Rr$, dérivable sur $\Rr^*$. Pour tout réel $x$ non nul,

$$f_2'(x)=\frac{-2x(1+x^2)-2x(1-x^2)}{(1+x^2)^2}\frac{-1}{\sqrt{1-\left(\frac{1-x^2}{1+x^2}\right)^2}}=\frac{4x}{1+x
^2}\frac{1}{\sqrt{4x^2}}=\frac{2\varepsilon}{1+x^2}$$
où $\varepsilon$ est le signe de $x$. Donc il existe une
constante réelle $C$ telle que pour tout réel positif $x$, $f_2(x)=2\Arctan x+C$ (y compris $x=0$ puisque $f$ est continue en $0$).\\
$x= 0$ fournit $C=0$ et donc, pour
tout réel positif $x$, $f_2(x)=2\Arctan x$. Par parité,

\begin{center}
\shadowbox{
$\forall x\in\Rr,\;\Arccos\left(\frac{1-x^2}{1+x^2}\right)=2\Arctan|x|.$
}
\end{center}
\textbf{2ème solution.} Soit $x\in\Rr$ puis $\theta=\Arctan x$. $\theta$ est dans
$\left]-\frac{\pi}{2},\frac{\pi}{2}\right[$ et $x=\tan\theta$.

$$\frac{1-x^2}{1+x^2}=\frac{1-\tan^2\theta}{1+\tan^2\theta}=\cos^2\theta(1-\tan^2\theta)=\cos^2\theta-\sin^2\theta=
\cos(2\theta).$$
Donc

$$f_2(x)=\Arccos(\cos(2\theta))=\left\{
\begin{array}{l}
2\theta\;\mbox{si}\;\theta\in\left[0,\frac{\pi}{2}\right[\\
-2\theta\;\mbox{si}\;\theta\in\left]-\frac{\pi}{2},0\right]
\end{array}
\right.
=\left\{
\begin{array}{l}
2\Arctan x\;\mbox{si}\;x\geq0\\
-2\Arctan x\;\mbox{si}\;x\leq0
\end{array}
\right.=2\Arctan|x|.$$}
    \item \question{$f_3(x)=\Arcsin\sqrt{1-x^2}-\Arctan\left(\sqrt{\frac{1-x}{1+x}}\right)$.}
\reponse{La fonction $x\mapsto\Arcsin\sqrt{1-x^2}$ est définie et continue sur $[-1,1]$, dérivable sur $[-1,1]\setminus\{0\}$ car
pour $x$ élément de $[-1,1]$, $1-x^2$ est élément de $[0,1]$ et vaut $1$ si et seulement si $x$ vaut $0$.
$\frac{1-x}{1+x}$ est défini et positif si et seulement si $x$ est dans $]-1,1]$, et nul si et seulement si $x=1$.
$f_3$ est donc définie et continue sur $]-1,1]$, dérivable sur $]-1,0[\cup]0,1[$. Pour $x$ dans $]-1,0[\cup]0,1[$, on
note $\varepsilon$ le signe de $x$ et on
a~:~

$$f_3'(x)=-\frac{x}{\sqrt{1-x^2}}\frac{1}{\sqrt{1-(1-x^2)}}-\frac{-(1+x)-(1-x)}{(1+x)^2}
\frac{1}{2\sqrt{\frac{1-x}{1+x}}}\frac{1}{1+\frac{1-x}{1+x}}=-\frac{\varepsilon}{\sqrt{1-x^2}}+\frac{1}{2}\frac{1
}{\sqrt{1-x^2}}.$$
Si $x$ est dans $]0,1[$, $f_3'(x)=-\frac{1}{2}\frac{1}{\sqrt{1-x^2}}=(-\frac{1}{2}\Arcsin)'(x)$.
Donc, il existe un réel $C$ tel que, pour tout $x$ de $[0,1]$ (par continuité) $f_3(x)=-\frac{1}{2}\Arcsin x+C$. $x=1$
fournit $C=\frac{\pi}{4}$. Donc,

\begin{center}
\shadowbox{
$\forall x\in[0,1],\;f_3(x)=\frac{\pi}{4}-\frac{1}{2}\Arcsin x=\frac{1}{2}\Arccos x.$
}
\end{center}
Si $x$ est dans $]-1,0[$, $f_3'(x)=\frac{3}{2}\frac{1}{\sqrt{1-x^2}}=(\frac{3}{2}\Arcsin)'(x)$. Donc il existe un
réel $C'$ tel que, pour tout $x$ de $]-1,0]$ (par continuité) $f_3(x)=\frac{3}{2}\Arcsin x+C'$.
$x=0$ fournit $\frac{\pi}{2}-\frac{\pi}{4}=C'$. Donc,

\begin{center}
\shadowbox{
$\forall x\in]-1,0],\;f_3(x)=\frac{3}{2}\Arcsin x+\frac{\pi}{4}.$
}
\end{center}}
    \item \question{$f_4(x)=\Arctan\frac{1}{2x^2}-\Arctan\frac{x}{x+1}+\Arctan\frac{x-1}{x}$.}
\reponse{$f_4$ est dérivable sur $\mathcal{D}=\Rr\setminus\{-1,0\}$ et pour $x$ élément de $\mathcal{D}$, on a~:~

\begin{align*}
f_4'(x)&=-\frac{1}{x^3}\frac{1}{1+\frac{1}{4x^4}}-\frac{(x+1)-x}{(x+1)^2}\frac{1}{1+\frac{x^2}{(x+1)^2}}+\frac{x
-(x-1)}{x^2}\frac{1}{1+\frac{(x-1)^2}{x^2}}\\
 &=-\frac{4x}{4x^4+1}-\frac{1}{2x^2+1+2x}+\frac{1}{2x^2+1-2x}=-\frac{4x
}{4x^4+1}+\frac{4x}{(2x^2+1)^2-4x^2}=0.
\end{align*}
$f_4$ est donc constante sur chacun des trois intervalles
$]-\infty,-1[$, $]-1,0[$ et $]0,+\infty[$. Pour $x>0$, $f(x)=f(1)=0$. Pour $-1<x<0$,
$\displaystyle f(x)=\lim_{
\substack{
t\rightarrow-1\\
t>-1}
}f(t)=\Arctan\frac{1}{2}-(-\frac{\pi}{2})+\Arctan2=\frac{\pi}{2}+\frac{\pi}{2}=\pi$.
Pour $x<-1$, $f(x)=\lim_{t\rightarrow -\infty}f(t)=0$ et donc

\begin{center}
\shadowbox{
$
\forall x\in\Rr\setminus\{-1;0\},\;f_4(x)=\left\{
\begin{array}{l}
0\;\mbox{si}\;x\in]-\infty,-1[\cup]0,+\infty[\\
\pi\;\mbox{si}\;x\in]-1,0[
\end{array}\right.
.$
}
\end{center}}
\end{enumerate}
}
