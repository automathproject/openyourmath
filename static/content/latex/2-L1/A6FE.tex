\uuid{A6FE}
\exo7id{4464}
\auteur{quercia}
\datecreate{2010-03-14}
\isIndication{false}
\isCorrection{true}
\chapitre{Série numérique}
\sousChapitre{Autre}

\contenu{
\texte{
Soit $(u_n)$ une suite définie par la donnée de $u_0 \in \R^*$ et la
relation : $\forall\ n\in\N,\ \frac{u_{n+1}}{u_n} = \frac{n+a}{n+b}$
où $a$, $b$ sont deux constantes réelles ($-a,-b\notin\N$).
}
\begin{enumerate}
    \item \question{Montrer que $u_n$ est de signe constant à partir d'un certain rang.}
    \item \question{On pose $v_n = (n+b-1)u_n$. \'Etudier la convergence de la suite $(v_n)$
    (on introduira la série de terme général $\ln(v_{n+1})-\ln(v_n)$).}
    \item \question{En déduire que la série $\sum u_n$ converge si et seulement si $a-b+1 < 0$
    et calculer sa somme en fonction de $a,b,u_0$.}
\reponse{
$\ln(v_{n+1})-\ln(v_n) = \ln\left(1+\frac{a-b+1}{n+b-1}\right)  \Rightarrow 
             \begin{cases} \text{si } a-b+1 > 0, v_n\to+\infty \cr
                     \text{si } a-b+1 = 0, v_n = \text{cste}\cr
                     \text{si } a-b+1 < 0, v_n\to0.\cr \end{cases}$
$(n+b)u_{n+1} - (n+a)u_n = 0  \Rightarrow 
              (n+b)u_{n+1} + (b-a-1)\sum_{k=1}^n u_k - au_0 = 0  \Rightarrow 
              \sum_{k=0}^\infty u_k = \frac{(b-1)u_0}{b-a-1}$.
}
\end{enumerate}
}
