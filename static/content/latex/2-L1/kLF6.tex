\uuid{kLF6}
\exo7id{612}
\auteur{bodin}
\organisation{exo7}
\datecreate{1998-09-01}
\isIndication{true}
\isCorrection{true}
\chapitre{Continuité, limite et étude de fonctions réelles}
\sousChapitre{Limite de fonctions}

\contenu{
\texte{

}
\begin{enumerate}
    \item \question{Montrer que toute fonction p\'eriodique
et non constante n'admet pas de limite en $+\infty$.}
\reponse{Soit $p>0$ la p\'eriode: pour tout $x\in \Rr$,
$f(x+p) = f(x)$. Par une r\'ecurrence facile on montre :
$$ \forall n \in \Nn \qquad \forall x\in \Rr \qquad f(x+np)=f(x).$$
Comme $f$ n'est pas constante il existe $a,b \in \Rr$ tels que $f(a)\not= f(b)$. Notons $x_n = a +np$ et $y_n = b+np$.
Supposons, par l'absurde, que $f$ a une limite $\ell$ en $+\infty$.
Comme $x_n \rightarrow +\infty$ alors $f(x_n) \rightarrow \ell$.
Mais $f(x_n) = f(a +np) = f(a)$, donc $\ell = f(a)$. 
De m\^eme avec la suite $(y_n)$: $y_n \rightarrow +\infty$ donc $f(y_n) \rightarrow \ell$ et $f(y_n) = f(b +np) = f(b)$, donc $\ell = f(b)$. 
Comme $f(a) \not= f(b)$ nous obtenons une contradiction.}
    \item \question{Montrer que toute fonction croissante
et major\'ee admet une limite finie en $+\infty$.}
\reponse{Soit $f: \Rr \longrightarrow \Rr$ une fonction croissante et major\'ee par $M\in \Rr$. 
Notons 
$$ F = f(\Rr) = \{ f(x) \ | \ x \in \Rr \}.$$
$F$ est un ensemble (non vide) de $\Rr$, notons $\ell = \sup F$.
Comme $M\in \Rr$ est un majorant de $F$, alors
$\ell < +\infty$.
Soit $\epsilon > 0$, par les propri\'et\'es du $\sup$
il existe $y_0 \in F$ tel que $\ell - \epsilon \leq y_0 \leq \ell$.
Comme $y_0\in F$, il existe $x_0 \in \Rr$ tel que $f(x_0) = y_0$.
Comme $f$ est croissante alors:
$$ \forall x\geq x_0 \qquad f(x) \geq f(x_0) = y_0 \geq \ell - \epsilon.$$
De plus par la d\'efinition de $\ell$:
$$\forall x\in \Rr \ \ f(x) \leq \ell.$$
Les deux propri\'et\'es pr\'ec\'edentes s'\'ecrivent:
$$ \forall x\geq x_0 \qquad \ell - \epsilon \leq f(x) \leq \ell.$$
Ce qui exprime bien que la limite de $f$ en $+\infty$ est $\ell$.}
\indication{\begin{enumerate}
    \item Raisonner par l'absurde.
    \item Montrer que la limite est la borne sup\'erieure de l'ensemble des valeurs atteintes $f(\Rr)$.
\end{enumerate}}
\end{enumerate}
}
