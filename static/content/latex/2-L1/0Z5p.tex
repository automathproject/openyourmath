\uuid{0Z5p}
\exo7id{5693}
\auteur{rouget}
\organisation{exo7}
\datecreate{2010-10-16}
\isIndication{false}
\isCorrection{true}
\chapitre{Série numérique}
\sousChapitre{Autre}

\contenu{
\texte{
Soit $\sigma$ une injection de $\Nn^*$ dans lui-même. Montrer que la série de terme général $\frac{\sigma(n)}{n^2}$ diverge.
}
\reponse{
Soit $\sigma$ une permutation de $\llbracket1,n\rrbracket$. Montrons que la suite $S_n=\sum_{k=1}^{n}\frac{\sigma(k)}{k^2}$, $n\geqslant1$, ne vérifie pas le critère de \textsc{Cauchy}. Soit $n\in\Nn^*$.

\begin{align*}\ensuremath
S_{2n}-S_n&=\sum_{k=n+1}^{2n}\frac{\sigma(k)}{k^2}\geqslant\frac{1}{(2n)^2}\sum_{k=n+1}^{2n}\sigma(k)\\
 &\geqslant\frac{1}{4n^2}(1+2+...+n)\;(\text{car les}\;n\;\text{entiers}\;\sigma(k),\;1\leqslant k\leqslant n,\;\text{sont strictement positifs et deux à deux distincts})\\
 &=\frac{n(n+1)}{8n^2}\geqslant\frac{n^2}{8n^2}=\frac{1}{8}.
\end{align*}

Si la suite $(S_n)$ converge, on doit avoir $\lim_{n \rightarrow +\infty}(S_{2n}-S_n)=0$ ce qui contredit l'inégalité précédente. Donc la série de terme général $\frac{\sigma(n)}{n^2}$, $n\geqslant1$, diverge.
}
}
