\uuid{573h}
\exo7id{3925}
\auteur{quercia}
\datecreate{2010-03-11}
\isIndication{false}
\isCorrection{true}
\chapitre{Fonctions circulaires et hyperboliques inverses}
\sousChapitre{Fonctions circulaires inverses}

\contenu{
\texte{
Résoudre :
}
\begin{enumerate}
    \item \question{$\arctan 2x + \arctan 3x = \frac \pi4$.}
\reponse{$x = \frac 16$.}
    \item \question{$\arctan\left(\frac{x-1}{x-2}\right) + \arctan\left(\frac{x+1}{x+2}\right) = \frac\pi4$.}
\reponse{$x = \pm1{\sqrt 2}$.}
    \item \question{$\arctan\left(\frac1x\right) + \arctan\left(\frac{x-1}{x+1}\right) = \frac\pi4$.}
\reponse{$x \in {]-\infty,-1[} \cup ]0,+\infty[$.}
    \item \question{$\arctan(x-3) + \arctan(x) + \arctan(x+3) = \frac {5\pi}4$.}
\reponse{$x^3 - 3x^2 - 12x + 10 = 0  \Rightarrow  x = 5,\,-1\pm\sqrt3$.
Seule la solution $x=5$ convient.}
\end{enumerate}
}
