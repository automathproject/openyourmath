\uuid{GQUY}
\exo7id{806}
\auteur{cousquer}
\datecreate{2003-10-01}
\isIndication{false}
\isCorrection{true}
\chapitre{Calcul d'intégrales}
\sousChapitre{Longueur, aire, volume}

\contenu{
\texte{
Représenter la courbe définie par son équation polaire
$\rho=a\sin^3\frac{\theta}{3}$. Calculer sa longueur $L$ et les aires
$A_1$ et $A_2$ limitées par les deux boucles qu'elle forme.
}
\reponse{
$\displaystyle L=\frac{3\pi a}{2},\quad
A_1=\frac{5\pi-9\sqrt3}{32}a^2,\quad
A_2=\frac{5\pi+18\sqrt3}{32}a^2$.
}
}
