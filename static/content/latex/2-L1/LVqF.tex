\uuid{LVqF}
\exo7id{5398}
\auteur{rouget}
\datecreate{2010-07-06}
\isIndication{false}
\isCorrection{true}
\chapitre{Continuité, limite et étude de fonctions réelles}
\sousChapitre{Continuité : théorie}

\contenu{
\texte{
Soit $f$ continue sur $\Rr^+$ à valeurs dans $\Rr$ admettant une limite réelle quand $x$ tend vers $+\infty$. Montrer que $f$ est uniformément continue sur $\Rr^+$.
}
\reponse{
Posons $\ell=\lim_{x\rightarrow +\infty}f(x)$.

Soit $\varepsilon>0$. $\exists A>0/\;\forall x\in\Rr^+,\;(x\geq A\Rightarrow|f(x)-\ell|<\frac{\varepsilon}{3}$.
 
Soit $(x,y)\in[A,+\infty[^2$. Alors, $|f(x)-f(y)|\leq|f(x)-\ell|+|\ell-f(y)|<\frac{2\varepsilon}{3}(<\varepsilon)$. D'autre part, $f$ est continue sur le segment $[0,A]$ et donc est uniformément continue sur ce segment d'après le théorème de \textsc{Heine}. Donc, $\exists\alpha>0/\;\forall(x,y)\in[0,A]^2,\;|x-y|<\alpha\Rightarrow|f(x)-f(y)|<\frac{\varepsilon}{3}(<\varepsilon)$.

Résumons. $\alpha>0$ étant ainsi fourni, soient $x$ et $y$ deux réels de $[0,+\infty[$ vérifiant $|x-y|<\alpha$.

Si $(x,y)\in[0,A]^2$, on a $|f(x)-f(y)|<\frac{\varepsilon}{3}<\varepsilon$.

Si $(x,y)\in[A,+\infty[^2$, on a $|f(x)-f(y)|<\frac{2\varepsilon}{3}<\varepsilon$.

Si enfin on a $x\leq A\leq y$, alors, puisque $|A-x|\leq|x-y|<\alpha$, on a $|f(x)-f(A)|<\frac{\varepsilon}{3}$ et puisque $A$ et $y$ sont dans $[A,+\infty[$, on a $|f(y)-f(A)|<\frac{2\varepsilon}{3}$. Mais alors,

$$|f(x)-f(y)|\leq|f(x)-f(A)|+|f(y)-f(A)|<\frac{\varepsilon}{3}+\frac{2\varepsilon}{3}=\varepsilon.$$

On a montré que $\forall\varepsilon>0,\;\exists\alpha>0/\;\forall(x,y)\in[0,+\infty[^2,\;(|x-y|<\alpha\Rightarrow|f(x)-f(y)|<\varepsilon)$. $f$ est donc uniformément continue sur $[0,+\infty[$.
}
}
