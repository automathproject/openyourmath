\uuid{5XPk}
\exo7id{572}
\auteur{bodin}
\organisation{exo7}
\datecreate{1998-09-01}
\isIndication{true}
\isCorrection{true}
\chapitre{Suite}
\sousChapitre{Suites équivalentes, suites négligeables}

\contenu{
\texte{

}
\begin{enumerate}
    \item \question{Soient $a,b > 0$. Montrer que $\sqrt{ab} \leqslant \frac{a+b}{2}$.}
\reponse{Soient $a,b >0$. On veut d\'emontrer
    que $\sqrt{ab}\leqslant \frac{a+b}{2}$. Comme les deux membres de cette
in\'egalit\'e sont positifs, cette in\'egalit\'e est \'equivalente
\`a $ ab\leqslant (\frac{a+b}{2})^2$. De plus,
$$ ab\leqslant \left( \frac{a+b}{2}\right)^2  \Leftrightarrow 4ab\leqslant a^2+2ab+b$$
$$ \Leftrightarrow 0\leqslant a^2-2ab+b^2$$ ce qui
est toujours vrai car $a^2-2ab+b^2=(a-b)^2$ est un carr\'e parfait. On a
donc bien l'in\'egalit\'e voulue.}
    \item \question{Montrer les in\'egalit\'es suivantes ($b \geqslant a > 0$) :
$$ a \leqslant \frac{a+b}{2} \leqslant b \qquad \text{et} \qquad a \leqslant \sqrt{ab} \leqslant b.$$}
\reponse{Quitte \`a \'echanger
$a$ et $b$ (ce qui ne change pas les moyennes arithm\'etique et
g\'eom\'etrique, et qui pr\'eserve le fait d'\^etre compris entre
$a$ et $b$), on peut supposer que $a\leqslant b$. Alors en ajoutant les
deux in\'egalit\'es $$a/2 \leqslant a/2 \leqslant b/2$$ $$a/2 \leqslant b/2 \leqslant
b/2,$$ on obtient $$a\leqslant \frac{a+b}{2}\leqslant b.$$

De m\^eme, comme tout est positif, en multipliant les deux
in\'egalit\'es
$$\sqrt{a} \leqslant \sqrt{a} \leqslant \sqrt{b}$$ $$\sqrt{a} \leqslant \sqrt{b} \leqslant
\sqrt{b}$$ on obtient $$a\leqslant \sqrt{ab} \leqslant b.$$}
    \item \question{Soient $u_0$ et $v_0$ des r\'eels strictement positifs avec
$u_0 < v_0$. On d\'efinit deux
suites $(u_n)$ et $(v_n)$ de la fa\c{c}on suivante :
$$ u_{n+1} = \sqrt{u_nv_n} \quad \text{et}\quad v_{n+1}=\frac{u_n+v_n}{2}.$$
    \begin{enumerate}}
\reponse{Il faut avant tout remarquer que pour tout $n$,
    $u_n$ et $v_n$ sont strictement positifs, ce qui
permet de dire que les deux suites sont bien d\'efinies. On le
d\'emontre par r\'ecurrence: c'est clair pour $u_0$ et $v_0$, et
si $u_n$ et $v_n$ sont strictement positifs alors leurs moyennes
g\'eom\'etrique (qui est $u_{n+1}$) et arithm\'etique (qui est $v_{n+1}$) sont
strictement positives.
    \begin{enumerate}}
    \item \question{Montrer que $u_n \leqslant v_n$ quel que soit $n\in\Nn$.}
\reponse{On veut montrer que pour chaque $n$, $u_n\leqslant v_n$. L'in\'egalit\'e est claire pour $n=0$
     gr\^ace aux hypoth\`eses faites sur $u_0$ et $v_0$.
     Si maintenant $n$ est plus grand que 1, $u_{n}$ est la
     moyenne g\'eom\'etrique de $u_{n-1}$ et $v_{n-1}$ et $v_{n}$
     est la moyenne arithm\'etique de $u_{n-1}$ et $v_{n-1}$,
     donc, par 1., $u_n\leqslant v_n$.}
    \item \question{Montrer que $(v_n)$ est une suite d\'ecroissante.}
\reponse{On sait d'apr\`es 2. que $u_n\leqslant u_{n+1}\leqslant v_n$.
 En particulier, $u_n\leqslant u_{n+1}$ i.e. $(u_n)$ est croissante.
 De m\^eme, d'apr\`es 2., $u_n\leqslant v_{n+1}\leqslant v_n$. En particulier,
 $v_{n+1}\leqslant v_n$ i.e. $(v_n)$ est d\'ecroissante.}
    \item \question{Montrer que $(u_n)$ est croissante En d\'eduire que
les suites $(u_n)$ et $(v_n)$ sont convergentes et quelles ont m\^eme limite.}
\reponse{Pour tout $n$, on a $u_0\leqslant u_n\leqslant v_n\leqslant v_0$.
        $(u_n)$ est donc croissante et major\'ee, donc converge
        vers une limite $\ell$. Et $(v_n)$ est d\'ecroissante et
        minor\'ee et donc converge vers une limite $\ell'$. 
Nous savons maintenant que 
$u_{n} \rightarrow \ell$, donc aussi $u_{n+1} \rightarrow \ell$, et $v_{n} \rightarrow \ell'$ ;
la relation $u_{n+1}=\sqrt{u_n v_n}$ s'écrit à la limite :
$$\ell=\sqrt{\ell\ell'}.$$
De même la relation $v_{n+1}=\frac{u_n+v_n}{2}$ donnerait à la limite :
$$\ell'=\frac{\ell+\ell'}{2}.$$
Un petit calcul avec l'une ou l'autre de ces égalités implique $\ell=\ell'$.}
\indication{\begin{enumerate}
  \item Regarder ce que donne l'inégalité en élevant au carré de chaque coté.
  \item Petites manipulations des inégalités.
  \item 
  \begin{enumerate}
     \item Utiliser 1.
     \item Utiliser 2.
     \item Une suite croissante et majorée converge ; une suite décroissante et minorée aussi.
  \end{enumerate}
\end{enumerate}}
\end{enumerate}
}
