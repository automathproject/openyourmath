\uuid{yv8I}
\exo7id{686}
\auteur{vignal}
\organisation{exo7}
\datecreate{2001-09-01}
\isIndication{false}
\isCorrection{true}
\chapitre{Continuité, limite et étude de fonctions réelles}
\sousChapitre{Etude de fonctions}

\contenu{
\texte{
D\'eterminer les domaines de d\'efinition des fonctions suivantes

$$f(x)=\sqrt{\frac{2+3\,x}{5-2\,x}}\ ;\quad g(x)=\sqrt{x^2-2\,x-5}\ ;\quad
h(x)=\ln\left(4\,x+3\right).$$
}
\reponse{
Il faut que le d\'enominateur ne s'annule pas donc $x \not= \frac 52$. En plus il faut
que le terme sous la racine soit positif ou nul, c'est-\`a-dire $(2+3x)\times(5-2x) \geq 0$, soit
$x \in [-\frac23, \frac52]$. L'ensemble de d\'efinition est donc  $[-\frac23, \frac52[$.
Il faut $x^2-2\,x-5\geq 0$, soit $x \in ]-\infty, 1-\sqrt6] \cup [1+\sqrt6,+\infty[$.
Il faut $4x+3>0$ soit $x > -\frac34$, l'ensemble de d\'efinition \'etant $]-\frac34,+\infty[$.
}
}
