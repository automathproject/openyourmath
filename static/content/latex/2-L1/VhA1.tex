\uuid{VhA1}
\exo7id{5459}
\auteur{rouget}
\datecreate{2010-07-10}
\isIndication{false}
\isCorrection{true}
\chapitre{Suite}
\sousChapitre{Convergence}

\contenu{
\texte{
Déterminer les limites quand $n$ tend vers $+\infty$ de
 
$$1)\;u_n=\frac{1}{n!}\int_{0}^{1}\Arcsin^nx\;dx\;2)\;\int_{0}^{1}\frac{x^n}{1+x}\;dx\;3)\;\int_{0}^{\pi}\frac{n\sin x}{x+n}\;dx.$$
}
\reponse{
Soit $n\in\Nn$. Pour $x\in[0,\frac{\pi}{2}]$, $0\leq\Arcsin^x\leq(\frac{\pi}{2})^n$ et donc, par croissance de l'intégrale,

$$0\leq u_n\leq\frac{1}{n!}\int_{0}^{1}(\frac{\pi}{2})^ndx=\frac{1}{n!}(\frac{\pi}{2})^n.$$

D'après un théorème de croissances comparées, $\frac{1}{n!}(\frac{\pi}{2})^n$ tend vers $0$ quand $n$ tend vers $+\infty$. D'après le théorème des gendarmes, $u_n$ tend vers $0$ quand $n$ tend vers $+\infty$.
$0\leq\int_{0}^{1}\frac{x^n}{1+x}\;dx\leq\int_{0}^{1}\frac{x^n}{1+0}\;dx=\frac{1}{n+1}$. Comme $\frac{1}{n+1}$ tend vers $0$ quand $n$ tend vers $+\infty$, $\int_{0}^{1}\frac{x^n}{1+x}\;dx$ tend vers $0$ quand $n$ tend vers $+\infty$.
Soit $n\in\Nn^*$.

\begin{align*}\ensuremath
\left|\int_{0}^{\pi}\frac{n\sin x}{x+n}\;dx-\int_{0}^{\pi}\sin x\;dx\right|=\left|
\int_{0}^{\pi}\frac{-x\sin x}{x+n}\;dx\right|\leq\int_{0}^{\pi}\left|\frac{-x\sin x}{x+n}\;dx\right|
\leq\int_{0}^{\pi}\frac{\pi}{0+n}\;dx=\frac{\pi^2}{n}.
\end{align*}

Or, $\frac{\pi^2}{n}$ tend vers $0$ quand $n$ tend vers $+\infty$, et donc $\int_{0}^{\pi}\frac{n\sin x}{x+n}\;dx$ tend vers $\int_{0}^{\pi}\sin x\;dx=2$ quand $n$ tend vers $+\infty$.
}
}
