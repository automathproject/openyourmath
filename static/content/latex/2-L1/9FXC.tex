\uuid{9FXC}
\exo7id{4452}
\auteur{quercia}
\organisation{exo7}
\datecreate{2010-03-14}
\isIndication{false}
\isCorrection{true}
\chapitre{Série numérique}
\sousChapitre{Autre}

\contenu{
\texte{
Soit $(u_n)$ une suite réelle positive, $U_n=\sum_{i=0}^n u_i$ et
$\alpha>0$ un réel donné. On suppose $\frac{U_n}{nu_n}\to \alpha$ lorsque $n\to\infty$. 
Étudier la suite de terme général $\frac1{n^2u_n}\sum_{k=0}^n ku_k$.
}
\reponse{
On remarque déjà que $\sum u_i$ diverge car $u_n\sim\frac{U_n}{n\alpha}\ge \frac{U_1}{n\alpha}$.
On calcule $\sum_{k=0}^n ku_k$ par parties~:
$$\sum_{k=0}^n ku_k = \sum_{k=1}^n k(U_k - U_{k-1}) = nU_n -\sum_{k=0}^n U_k$$
Comme $U_n\sim\alpha nu_n$, terme général strictement positif d'une série divergente,
on a $\sum_{k=0}^n U_k \sim \alpha\sum_{k=0}^n ku_k$ d'où~:
$(1+\alpha)\sum_{k=0}^n ku_k\sim nU_n$ et lorsque $n\to\infty$ :
$$\frac1{n^2u_n}\sum_{k=0}^n ku_k\sim \frac{nU_n}{(1+\alpha)n^2u_n} \to \frac\alpha{1+\alpha}.$$
}
}
