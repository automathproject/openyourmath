\uuid{KPgD}
\exo7id{4454}
\auteur{quercia}
\datecreate{2010-03-14}
\isIndication{false}
\isCorrection{true}
\chapitre{Série numérique}
\sousChapitre{Autre}

\contenu{
\texte{
Soit $(u_n)_{n\ge 1}$ une suite réelle positive décroissante telle que
$\sum u_n$ converge.
}
\begin{enumerate}
    \item \question{Montrer que $nu_n \to 0$ lorsque $n\to\infty$.
    $\Bigl($considérer $\sum_{k=n+1}^{2n} u_k \Bigr)$}
    \item \question{Montrer que $\sum_{n=1}^\infty n(u_n - u_{n+1})$ converge et a même somme que $\sum_{n=1}^\infty u_n$.}
    \item \question{Application : calculer pour $0 \le r < 1$ :
    $\sum_{k=1}^\infty kr^k$ et $\sum_{k=1}^\infty k^2r^k$.}
\reponse{
$kr^k = k(u_k - u_{k+1})$ avec $u_k = \frac{r^k}{1-r}$
    donc $\sum_{k=1}^\infty kr^k = \sum_{k=1}^\infty\frac{r^k}{1-r} = \frac r{(1-r)^2}$.
    \par
    De même, $S_n = \sum_{k=n}^\infty kr^k = \frac{(n-1)r^n}{1-r} +
     \sum_{k=n}^\infty\frac{r^k}{1-r} = \frac{nr^n}{1-r} + \frac{r^{n+1}}{(1-r)^2}$.
    
    $k^2r^k = k(S_k - S_{k+1})$ et $(S_k)$ décroît d'où
    $\sum_{k=1}^\infty k^2r^k = \sum_{k=1}^\infty S(k)
    = \sum_{k=1}^\infty \Bigl(\frac{kr^k}{1-r}+\frac{r^{k+1}}{(1-r)^2}\Bigr)
    = \frac{r+r^2}{(1-r)^3}$.
}
\end{enumerate}
}
