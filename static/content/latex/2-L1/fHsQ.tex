\uuid{fHsQ}
\exo7id{524}
\auteur{monthub}
\datecreate{2001-11-01}
\isIndication{true}
\isCorrection{true}
\chapitre{Suite}
\sousChapitre{Convergence}

\contenu{
\texte{
Soit $q$ un entier au moins {\'e}gal {\`a} $2$. Pour tout $n\in
\N$, on pose $u_n=\cos{\dfrac{2n\pi}{q}}$.
}
\begin{enumerate}
    \item \question{Montrer que $u_{n+q}=u_n$ pour tout $n \in \N$.}
\reponse{$u_{n+q} = \cos \left( \frac{2(n+q)\pi}{q} \right) = \cos \left(\frac{2n\pi}{q}+2\pi\right) = \cos \left(\frac{2n\pi}{q}\right) = u_n$.}
    \item \question{Calculer $u_{nq}$ et $u_{nq+1}$. En d{\'e}duire que la suite $(u_n)$
  n'a pas de limite.}
\reponse{$u_{nq} = \cos \left(\frac{2nq\pi}{q}\right) = \cos \left({2n\pi}\right)= 1 = u_0$ et $u_{nq+1} = \cos \left(\frac{2(nq+1)\pi}{q}\right) = \cos \left(\frac{2\pi}{q}\right) = u_1$.
Supposons, par l'absurde que $(u_n)$ converge vers $\ell$.
Alors la sous-suite $(u_{nq})_n$ converge vers $\ell$ comme 
$u_{nq}= u_0 = 1$ pour tout $n$ alors $\ell = 1$. D'autre part
la sous-suite $(u_{nq+1})_n$ converge aussi vers $\ell$, mais 
$u_{nq+1}= u_1 = \cos \frac{2\pi}{q}$, donc $\ell = \cos \frac{2\pi}{q}$. Nous obtenons une contradiction car pour $q\geq 2$, nous avons $\cos \frac{2\pi}{q} \not= 1$. Donc la suite $(u_n)$ ne converge pas.}
\indication{Pour la deuxi\`eme question, raisonner par l'absurde et trouver deux sous-suites ayant des limites distinctes.}
\end{enumerate}
}
