\uuid{OTsx}
\exo7id{7173}
\auteur{megy}
\datecreate{2017-07-26}
\isIndication{true}
\isCorrection{true}
\chapitre{Propriétés de R}
\sousChapitre{Autre}

\contenu{
\texte{
%[Application directe]
Soient $a$, $b$ et $c$ des réels positifs. Montrer que 
\[ 
\frac{a^2}{bc}+\frac{b^2}{ca}+\frac{c^2}{ab} \geq 3.
\]
}
\indication{Utiliser l'inégalité arithmético-géométrique.}
\reponse{
On applique l'inégalité arithmético-géométrique à $\frac{a^2}{bc}$, $\frac{b^2}{ca}$ et $\frac{c^2}{ab}$ ce qui donne
\[ \frac{a^2}{bc}+\frac{b^2}{ca}+\frac{c^2}{ab}
\geq3\sqrt[3]{\frac{a^2}{bc}\frac{b^2}{ca}\frac{c^2}{ab}}
=3
\]
}
}
