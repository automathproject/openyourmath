\uuid{PnGR}
\exo7id{1930}
\auteur{roussel}
\organisation{exo7}
\datecreate{2001-09-01}
\isIndication{false}
\isCorrection{false}
\chapitre{Série numérique}
\sousChapitre{Série à  termes positifs}

\contenu{
\texte{
Soient, pour $n>0$,
$u_n=\displaystyle{\frac{n!e^n}{n^{n+\frac{1}{2}}}}$ et $v_n=\ln u_n$.
}
\begin{enumerate}
    \item \question{Etudier la serie de terme g\'en\'eral ~$w_n$ o\`u, pour $n \geq 2,~w_n = v_n-v_{n-1}$
et $w_1=v_1$.}
    \item \question{En d\'eduire, en utilisant la convergence de la suite des sommes
partielles de $w_n$, que la suite $u_n$ converge vers $\lambda >0$.}
    \item \question{D\'eterminer $\lambda$ en utilisant la formule de Wallis :
$\lim _{n \rightarrow + \infty} \displaystyle{\frac{2^{2n}(n!)^2}{\sqrt{n}(2n)!}}=
\sqrt{\pi}$. En d\'eduire
un \'equivalent de $n!$.

\emph{Indication} : Exprimer $n!$ (respectivement $(2n)!$) en fonction de
$u_n$ (resp. de $u_{2n}$) et remplacer-les dans la formule de Wallis.}
\end{enumerate}
}
