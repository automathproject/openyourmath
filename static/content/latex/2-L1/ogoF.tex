\uuid{ogoF}
\exo7id{709}
\auteur{ridde}
\datecreate{1999-11-01}
\isIndication{false}
\isCorrection{true}
\chapitre{Dérivabilité des fonctions réelles}
\sousChapitre{Calculs}

\contenu{
\texte{
Soit $f (x) = \exp (-\frac 1{x^2})$ si $x \neq 0$ et $f (0) = 0$.
Montrer que $f$ est $C^{\infty}$ et que $\forall n \in \Nn \, \, f^{ (n)} (0) = 0$.
}
\reponse{
La limite de $f$ en $0$ est $0$, donc $f$ est continue en $0$.
De même le taux d'accroissement de $f$ en $0$ est
$f(x)/x$ qui tend vers $0$. Donc $f$ est dérivable en $0$ et 
$f'(0)=0$. En dehors de $0$, on a $f'(x) = 2{e^{-{x}^{-2}}}{x}^{-3}$
donc $f'$ est continue en $0$.

On continue de la même fa\c{c}on en remarquant que si
$f^{(n)}(x) = P(1/x) \exp(-1/x^2)$  
où $P$ est un polynôme et $f^{(n)}(0) = 0$.
Donc $f^{(n)}(x)$ tend vers $0$ si $x$ tend vers $0$. Donc $f^{(n)}$
est continue. De plus $f^{(n)}$ est dérivable en $0$ car son taux
d'accroissement vaut $1/x P(1/x) \exp(-1/x^2)$ qui tend vers $0$, donc 
$f^{(n+1)}(0) = 0$.
En dehors de $0$ on $f^{(n+1)}(x) = Q(1/x) \exp(-1/x^2)$ où $Q$ est un polynôme. 
Et on recommence...
}
}
