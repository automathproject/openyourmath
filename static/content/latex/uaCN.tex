\uuid{uaCN}
\titre{ Temps de transport }
\theme{probabilités}
\auteur{ }
\organisation{ AMSCC }

\contenu{
    \texte{ Une commune a lancé une étude concernant les problèmes liés au transport. Sur une ligne de bus, une enquête a permis de révéler que le retard (ou l'avance) sur l'horaire officiel du bus à une station donnée est donné par une variable aléatoire $X$, en minutes, suivant une loi normale $\mathcal{N}(5,\sigma)$ où $\sigma >0$ est pour l'instant indéterminé. Cependant, on sait que la probabilité que le retard soit inférieur à $7$ minutes est de $p=84{,}13\%$. }

    \begin{enumerate}
        \item \question{  Justifier que $\sigma = 2$.  }
        \reponse{ 
On sait que $\prob\left(X \leq 7 \right) = 0{,}8413$ d'où $\prob\left(\frac{X-5}{\sigma} \leq \frac{2}{\sigma}\right) = 0{,}8413$. Par lecture de table, $\frac{2}{\sigma} = 1$ d'où $\sigma = 2$.
        }
        \item \question{ 
            Sachant que le retard est supérieur à $3$ minutes, quelle est la probablité que ce retard soit inférieur à $7$ minutes ?
        }
        \reponse{
            On cherche $\prob(X \leq 7 \mid X \geq 3) = \frac{\prob(3 \leq X \leq 7)}{\prob(X \geq 3)}$. Or $\prob(3 \leq X \leq 7) = \prob\left(-1 \leq \frac{X-5}{2} \leq 1\right) = 2 \Phi(1) - 1 \approx 2 \times 0{,}8413 - 1 = 0{,}6826$ et $\prob(X \geq 3) = \prob\left( \frac{X-5}{2} \geq -1\right) \approx 0{,}8413$ d'où $\prob(X \leq 7 \mid X \geq 3) \approx 0{,}8114$. 
        }
        \item \texte{ Une dame fréquente cette ligne de bus une fois par jour pendant 10 jours. On suppose que les retards journaliers sont indépendants. On note $Y$ le nombre de jours où la dame a attendu moins de $7$ minutes. }
        \begin{enumerate}
            \item \question{ Déterminer la loi de $Y$, son espérance et sa variance. }
            \reponse{
                On a $Y \sim \mathcal{B}(10,p)$ d'où $\E(Y) = 10p = 8{,}413$ et $\var(Y) = 10p(1-p) = 1{,}3351431$. 
            }
            \item \question { 
            Soit $T$ la variable aléatoire définie de la manière suivante : si la dame attend moins de $7$ minutes chaque jour pendant les 10 jours, $T$ prend la valeur $0$ ; sinon $T$ pend la valeur du $k$-ème jour ($k \in \{1,...,10\}$) où pour la première fois, la dame attend plus de $7$ minutes. Déterminer, en fonction de $p$, la loi de probabilité de $T$. 
        }
        \reponse{
            Pour $k \in \{1,...,10\}$, $\prob(X=k) = p^{k-1}(1-p)$ (on reconnait la loi géométrique de paramètre $1-p$ pour ces $10$ cas). Pour $k=0$, $\prob(T=0) = \sum_{k=11}^{+\infty} p^{k-1}(1-p) = (1-p) \sum_{k=11}^{+\infty} p^{k-1} = (1-p) \times \frac{p^{10}}{1-p} = p^{10}$.
            } 
        \end{enumerate}
    \end{enumerate}
}