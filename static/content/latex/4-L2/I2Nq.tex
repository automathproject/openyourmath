\uuid{I2Nq}
\exo7id{3610}
\auteur{quercia}
\organisation{exo7}
\datecreate{2010-03-10}
\isIndication{false}
\isCorrection{false}
\chapitre{Réduction d'endomorphisme, polynôme annulateur}
\sousChapitre{Sous-espace stable}

\contenu{
\texte{
Soit $M \in \mathcal{M}_n(\R)$ et $\lambda = a+ib$ une valeur propre non réelle de $M$
($a \in \R$, $b \in \R^*$).
On note X un vecteur propre complexe de $M$.
}
\begin{enumerate}
    \item \question{Montrer que $\overline X$ est aussi vecteur propre de $M$.}
    \item \question{Montrer que $(X,\overline X)$ est libre dans $\C^n$.}
    \item \question{Soient $U = \frac12(X+\overline X)$, $V = \frac1{2i}(X-\overline X)$.
    Montrer que $(U,V)$ est libre dans $\R^n$.}
\end{enumerate}
}
