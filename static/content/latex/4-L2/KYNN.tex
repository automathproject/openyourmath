\uuid{KYNN}
\exo7id{3666}
\auteur{quercia}
\datecreate{2010-03-11}
\isIndication{false}
\isCorrection{true}
\chapitre{Espace euclidien, espace normé}
\sousChapitre{Produit scalaire, norme}

\contenu{
\texte{
Soit $E$ un espace euclidien de dimension 4, ${\cal B} = (\vec e_1, \dots, \vec e_4)$
une base orthonormée de $E$, et $F$ le sous-espace vectoriel d'équations dans $\cal B$ :
$$\begin{cases} x+y+z+t = 0\cr x+2y+3z+4t =0 \cr\end{cases}$$
}
\begin{enumerate}
    \item \question{Trouver une base orthonormée de $F$.}
\reponse{$\biggl(\frac 1{\sqrt6}(1,-2,1,0), \frac 1{\sqrt{30}}(2,-1,-4,3) \biggr)$}
    \item \question{Donner la matrice dans $\cal B$ de la projection orthogonale sur $F$.}
\reponse{$\frac 1{10}\begin{pmatrix}3 &-4 &-1 &2  \cr
                          -4 &7  &-2 &-1 \cr
                          -1 &-2 &7 &-4  \cr
                           2 &-1 &-4 &3  \cr\end{pmatrix}$.}
    \item \question{Calculer $d(\vec e_1, F)$.}
\reponse{$\sqrt{\frac 7{10}}$.}
\end{enumerate}
}
