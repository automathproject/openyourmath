\uuid{qlMk}
\exo7id{1648}
\auteur{bodin}
\organisation{exo7}
\datecreate{1999-11-01}
\isIndication{false}
\isCorrection{false}
\chapitre{Réduction d'endomorphisme, polynôme annulateur}
\sousChapitre{Diagonalisation}

\contenu{
\texte{
Soit $f$ un endomorphisme de $\C^3$ dont la
matrice par rapport \`a la base canonique est :
$$
M=
\left(
\begin{array}{ccc}
2 & -1 & 1 \\
-1 & k & 1 \\
1 &1 & 2
\end{array}
\right),
\; \mbox{o\`u} \; k\in\C.
$$
\emph{(a)} D\'eterminer, suivant les valeurs de
$k$, la dimension du noyau de $f$.  \\
\emph{(b)} Montrer que $M$ admet une valeur propre
r\'eelle enti\`ere ind\'ependante de $k$, et
calculer toutes les valeurs propres de $M$.  \\
\emph{(c)} Indiquer toutes les valeurs de $k$ pour
lesquelles on obtient des valeurs propres
multiples.  Pour quelles valeurs de ces $k$ la
matrice $M$ est-elle semblable \`a une matrice
diagonale?
}
}
