\uuid{ZS4J}
\exo7id{1665}
\auteur{ortiz}
\organisation{exo7}
\datecreate{1999-04-01}
\isIndication{false}
\isCorrection{false}
\chapitre{Réduction d'endomorphisme, polynôme annulateur}
\sousChapitre{Diagonalisation}

\contenu{
\texte{
Soient $E$ un $K$-espace vectoriel de dimension
$n$ et $f$ un endomorphisme de $E$ de rang $1$.
}
\begin{enumerate}
    \item \question{Montrer que si $f$ est diagonalisable alors $\text{tr}(f)\not= 0$.}
    \item \question{Montrer qu'il existe $\lambda\in K$ tel que le
polyn\^ome cararact\'eristique de $f$ s'\'ecrive
$${\chi_f=(-1)^nX^{n-1}(X-\lambda)}.$$}
    \item \question{\begin{enumerate}}
    \item \question{Montrer que $f$ est diagonalisable si et seulement si $\text{tr}(f)\not= 0$.}
    \item \question{R\'eduire sans calcul la matrice $A=\left( \begin{smallmatrix}
1&1&-1\\
-2&-2&2\\
1&1&-1
\end{smallmatrix}\right) \in \mathcal{M}_3(\Rr)$ et donner sans calcul les sous-espaces
vectoriels propres.}
\end{enumerate}
}
