\uuid{FiIz}
\exo7id{5364}
\auteur{rouget}
\datecreate{2010-07-06}
\isIndication{false}
\isCorrection{true}
\chapitre{Déterminant, système linéaire}
\sousChapitre{Calcul de déterminants}

\contenu{
\texte{
Calculer :
}
\begin{enumerate}
    \item \question{$\mbox{det}(|i-j|)_{1\leq i,j\leq n}$}
\reponse{Pour $n\geq2$, posons $\Delta_n=\left|
\begin{array}{ccccc}
0&1&2&\ldots&n-1\\
1&0&1& &n-2\\
2&1&0&\ddots&\vdots\\
\vdots& &\ddots&\ddots&1\\
n-1&n-2&\ldots&1&0
\end{array}
\right|$. Tout d'abord, on fait apparaître beaucoup de $1$.
Pour cela, on effectue les transformations $C_1\leftarrow C_1-C_2$ puis $C_2\leftarrow C_2-C_3$ puis \ldots puis $C_{n-1}=C_{n-1}-C_n$. On obtient

$$\Delta_n=\mbox{det}(C_1-C_2,C_2-C_3,...,C_{n-1}-C_n,C_n)=\left|
\begin{array}{ccccc}
-1&-1&\ldots&-1&n-1\\
1&-1& &\vdots&n-2\\
1&1&-1&\vdots&\vdots\\
\vdots& &\ddots&-1&1\\
1&1&\ldots&1&0
\end{array}
\right|.$$
On fait alors apparaître un déterminant triangulaire en constatant que $\mbox{det}(L_1,L_2,...,L_n)=\mbox{det}(L_1,L_2+L_1,...,L_{n-1}+L_1,L_n+L_1)$. On obtient

$$\Delta_n=\left|
\begin{array}{ccccc}
-1&\times&\ldots&\ldots&\times\\
0&-2&\ddots& &\vdots\\
0&0&\ddots&\ddots&\vdots\\
\vdots& &\ddots&-2&\times\\
0&\ldots&\ldots&0&n-1
\end{array}
\right|=(1-n)(-2)^{n-2}.$$.

\begin{center}
\shadowbox{
$\forall n\geq2,\;\Delta_n=(1-n)(-2)^{n-2}$.
}
\end{center}}
    \item \question{$\mbox{det}(\sin(a_i+a_j))_{1\leq i,j\leq n}$ ($a_1$,...,$a_n$ étant $n$ réels donnés)}
\reponse{$\forall(i,j)\in\llbracket1,n\rrbracket^2,\;\sin(a_i+a_j)=\sin a_i\cos a_j+\cos a_i\sin a_j$ et donc si on pose $C=\left(
\begin{array}{c}
\cos a_1\\
\cos a_2\\
\vdots\\
\cos a_n
\end{array}
\right)$ et $S=\left(
\begin{array}{c}
\sin a_1\\
\sin a_2\\
\vdots\\
\sin a_n
\end{array}
\right)$,
on a $\forall j\in\llbracket1,n\rrbracket,\;C_j=\cos a_jS+\sin a_jC$. En particulier, $\mbox{Vect}(C_1,...,C_n)\subset\mbox{Vect}(C,S)$ et le rang de la matrice proposée est inférieur ou égal à $2$. Donc,

\begin{center}
\shadowbox{
$\forall n\geq3,\;\mbox{det}(\sin(a_i+a_j))_{1\leq i,j\leq n}=0.$
}
\end{center}
Si $n=2$, $\mbox{det}(\sin(a_i+a_j))_{1\leq i,j\leq 2}=\sin(2a_1)\sin(2a_2)-\sin^2(a_1+a_2)$.}
    \item \question{$\left|
\begin{array}{cccccc}
a&0&\ldots& &\ldots&b\\
0&a&\ddots& &b&0\\
\vdots&0&\ddots& &0&\vdots\\
\vdots&0& &\ddots&0&\vdots\\
0&b& & &a&0\\
b&0&\ldots& &\ldots&a
\end{array}
\right|$}
\reponse{L'exercice n'a de sens que si le format $n$ est pair. Posons $n=2p$ où $p$ est un entier naturel non nul.

\begin{align*}\ensuremath
\Delta_n&=\left|\begin{array}{cccccc}
a&0&\ldots&\ldots&0&b\\
0&\ddots&0&0& &0\\
\vdots&0&a&b&0&\vdots\\
\vdots&0&b&a&0&\vdots\\
0& &0&0&\ddots&0\\
b&0&\ldots&\ldots&0&a
\end{array}
\right|
=\left|\begin{array}{cccccc}
a+b&0&\ldots&\ldots&0&b\\
0&\ddots&0&0& &0\\
\vdots&0&a+b&b&0&\vdots\\
\vdots&0&b+a&a&0&\vdots\\
0& &0&0&\ddots&0\\
b+a&0&\ldots&\ldots&0&a
\end{array}
\right|\;(\mbox{pour}\;1\leq j\leq p,\;C_j\leftarrow C_j+C_{2p+1-j})\\
 &=(a+b)^p\left|\begin{array}{cccccc}
1&0&\ldots&\ldots&0&b\\
0&\ddots&0&0& &0\\
\vdots&0&1&b&0&\vdots\\
\vdots&0&1&a&0&\vdots\\
0& &0&0&\ddots&0\\
1&0&\ldots&\ldots&0&a
\end{array}
\right|\;(\mbox{par linéarité par rapport aux colonnes}\;C_1,\;C_2,...,\;C_p)
\end{align*}

$$
=(a+b)^p\left|\begin{array}{cccccc}
1&0&\ldots&\ldots&0&b\\
0&\ddots&0&0& &0\\
\vdots&0&1&b&0&\vdots\\
 & &\ddots&a-b&0&\vdots\\
\vdots& & &\ddots&\ddots&0\\
0&0&\ldots&\ldots&0&a-b
\end{array}
\right| (\mbox{pour}\;p+1\leq i\leq2p,\;L_i\leftarrow L_i-L_{2p+1-i}).
$$
et $\Delta_n=(a+b)^p(a-b)^p=(a^2-b^2)^p$.

\begin{center}
\shadowbox{
$\forall p\in\Nn^*,\;\Delta_{2p}=(a^2-b^2)^p$.
}
\end{center}}
    \item \question{$\left|\begin{array}{ccccc}
1&1&\ldots& &1\\
1&1&0&\ldots&0\\
\vdots&0&\ddots&\ddots&\vdots\\
\vdots&\vdots&\ddots&1&0\\
1&0&\ldots&0&1
\end{array}
\right|$}
\reponse{On retranche à la première colonne la somme de toutes les autres et on obtient

$$D_n=\left|\begin{array}{ccccc}
1&1&\ldots& &1\\
1&1&0&\ldots&0\\
\vdots&0&\ddots&\ddots&\vdots\\
\vdots&\vdots&\ddots&1&0\\
1&0&\ldots&0&1
\end{array}
\right|=\left|\begin{array}{ccccc}
-(n-2)&1&\ldots& &1\\
0&1&0&\ldots&0\\
\vdots&\ddots&\ddots&\ddots&\vdots\\
\vdots& &\ddots&1&0\\
0&\ldots&\ldots&0&1
\end{array}
\right|=-(n-2).$$}
    \item \question{$\mbox{det}(C_{n+i-1}^{j-1})_{1\leq i,j\leq p+1}$}
\reponse{Pour $1\leq i\leq p$,

$$L_{i+1}-L_i=(C_{n+i}^0-C_{n+i-1}^0,C_{n+i}^1-C_{n+i-1}^1,...,C_{n+i}^p-C_{n+i-1}^p)=(0,C_{n+i-1}^0,C_{n+i-1}^1,...,C_{n+i-1}^{p-1}).$$
On remplace alors dans cet ordre $L_p$ par $L_p-L_{p-1}$ puis $L_{p-1}$ par $L_{p-1}-L_{p-2}$ puis ... puis $L_2$ par $L_2-L_1$ pour obtenir, avec des notations évidentes

$$\mbox{det}(A_p)=\left|
\begin{array}{cc}
1& \\
0&A_{p-1}
\end{array}\right|=\mbox{det}(A_{p-1}).$$
Par suite, $\mbox{det}(A_p)=\mbox{det}(A_{p-1})=...=\mbox{det}(A_1)=1$.}
    \item \question{$\left|\begin{array}{cccccc}
-X&1&0&\ldots&\ldots&0\\
0&-X&1&\ddots& &\vdots\\
\vdots&\ddots&\ddots&\ddots&\ddots&\vdots\\
\vdots& &\ddots&\ddots&\ddots&0\\
0&\ldots&\ldots&0&-X&1\\
a_0&\ldots& &\ldots&a_{n-2}&a_{n-1}-X
\end{array}
\right|$}
\reponse{En développant suivant la dernière ligne, on obtient~:

$$D_n=(a_{n-1}-X)(-X)^{n-1}+\sum_{k=0}^{n-2}(-1)^{n+k+1}a_k\Delta_k,$$
où $\Delta_k=\left|
\begin{array}{ccc|ccc}
-X&1&0& & & \\
0&\ddots&1& & & \\
0&0&-X& & & \\
\hline
0& &0&1&0&0\\
 & & &-X&\ddots&0\\
0& &0&0&-X&1\\
\end{array}
\right|=(-1)^{k}X^k$ et donc 

\begin{center}
\shadowbox{
$\forall n\geq2,\;D_n=(-1)^n\left(X^n-\sum_{k=0}^{n-1}a_kX^k\right).$
}
\end{center}}
\end{enumerate}
}
