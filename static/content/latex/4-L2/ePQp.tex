\uuid{ePQp}
\exo7id{3524}
\auteur{quercia}
\datecreate{2010-03-10}
\isIndication{false}
\isCorrection{false}
\chapitre{Réduction d'endomorphisme, polynôme annulateur}
\sousChapitre{Polynôme caractéristique, théorème de Cayley-Hamilton}

\contenu{
\texte{
Soit $A = (a_{ij}) \in \mathcal{M}_{3}(\R)$.
}
\begin{enumerate}
    \item \question{Vérifier que
    $\chi_A(\lambda) = -\lambda^3 + (\mathrm{tr} A)\lambda^2
       - \left( \begin{vmatrix} a_{11} &a_{12} \cr a_{21} &a_{22} \end{vmatrix}
              + \begin{vmatrix} a_{11} &a_{13} \cr a_{31} &a_{33} \end{vmatrix}
              + \begin{vmatrix} a_{22} &a_{23} \cr a_{32} &a_{33} \end{vmatrix} \right)\lambda
       +\det(A)$.}
    \item \question{Soit $\lambda$ une valeur propre de $A$ et $L_1,L_2$ deux lignes non
    proportionnelles de $A-\lambda I$ (s'il en existe).
    On calcule $L = L_1\wedge L_2$ (produit vectoriel) et $X = {}^tL$.
    Montrer que $X$ est vecteur propre de $A$ pour la valeur propre $\lambda$.}
\end{enumerate}
}
