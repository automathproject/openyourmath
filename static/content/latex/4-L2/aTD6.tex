\uuid{aTD6}
\exo7id{2572}
\auteur{delaunay}
\organisation{exo7}
\datecreate{2009-05-19}
\isIndication{false}
\isCorrection{false}
\chapitre{Réduction d'endomorphisme, polynôme annulateur}
\sousChapitre{Diagonalisation}

\contenu{
\texte{
({\it 9 points})  Soit $A$ la matrice de $M_3(\R)$ suivante :
$$A=\begin{pmatrix}1&0&1 \\  -1&2&1 \\  1&-1&1\end{pmatrix}$$
\begin {enumerate}
  \item  D\'emontrer que les valeurs propres de $A$ sont $1$ et $2$.
  \item D\'eterminer les sous-espaces propres de $A$. La matrice $A$ est-elle diagonalisable ?

  \item D\'eterminer les sous-espaces caract\'eristiques de $A$.
  \item D\'eterminer une base de $\R^3$ dans laquelle la matrice de l'endomorphisme associ\'e \`a $A$ est 
$$B=\begin{pmatrix}2&0&0 \\  0&1&1 \\  0&0&1\end{pmatrix}$$
 En d\'eduire la d\'ecomposition de 
Dunford de $B$.
  \item  R\'esoudre le syst\`eme diff\'erentiel

$$\left\{\begin{align*}x'&=x+z \\  y'&=-x+2y+z \\  z'&=x-y+z\end{align*}\right.$$
\end {enumerate}
}
}
