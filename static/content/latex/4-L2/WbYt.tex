\uuid{WbYt}
\exo7id{2460}
\auteur{matexo1}
\organisation{exo7}
\datecreate{2002-02-01}
\isIndication{false}
\isCorrection{false}
\chapitre{Espace euclidien, espace normé}
\sousChapitre{Produit scalaire, norme}

\contenu{
\texte{
{\sc Identit\'e du parall\'elogramme}

\bigskip

Soit $E$ un $\R$-espace vectoriel norm\'e.
}
\begin{enumerate}
    \item \question{On suppose que la norme de $E$ v\'erifie la relation
\begin{equation} \label{eq}
\forall x,y\in E, \qquad
 2\bigl(\|x\|^2 +\|y\|^2\bigr)
=\|x+y\|^2 + \|x-y\|^2.
\end{equation}
On d\'efinit $p : E\times E \to \R$ par
$$p(x,y) =
\frac 1 2 \bigl(\|x+y\|^2 - \|x\|^2 -\|y\|^2\bigr).$$
Montrer que $p$ est un produit scalaire sur $E$.}
    \item \question{R\'eciproquement, si $E$ est un espace euclidien dont le
produit scalaire est not\'e $\langle x,y \rangle$, montrer que la norme euclidienne
(d\'efinie par $\|x\| =\sqrt{\langle x,x \rangle}$) v\'erifie (\ref{eq}),
et que $\langle x,y \rangle=p(x,y)$.}
    \item \question{Dans le cas o\`u $E=\R^n$, pour quelles valeurs de $q\geq 1$
les normes $\|\cdot\|_q$ v\'erifient-elles (\ref{eq})\,?}
\end{enumerate}
}
