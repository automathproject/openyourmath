\uuid{QHhI}
\exo7id{5655}
\auteur{rouget}
\organisation{exo7}
\datecreate{2010-10-16}
\isIndication{false}
\isCorrection{true}
\chapitre{Réduction d'endomorphisme, polynôme annulateur}
\sousChapitre{Valeur propre, vecteur propre}

\contenu{
\texte{
Soit $E=\Rr_{3}[X]$. Pour $P$ élément de $E$, soit $f(P)$ le reste de la division euclidienne de $AP$ par $B$ où $A= X^4-1$ et $B = X^4 - X$.

Vérifier que $f$ est un endomorphisme de $E$ puis déterminer $\text{Ker}f$, $\text{Im}f$ et les valeurs et vecteurs propres de $f$.
}
\reponse{
Soit $P=aX^3+bX^2+cX+d\in\Rr_3[X]$

\begin{center} 
$AP - (X^4-X)P= (X-1)P = aX^4+(b-a)X^3+(c-b)X^2+(d-c)X-d = a(X^4-X)+(b-a)X^3+(c-b)X^2+(a+d-c)X-d$.
\end{center}

et donc $AP=(X^4-X)(P+a)+(b-a)X^3+(c-b)X^2+(a+d-c)X-d$ et donc $f(P) = (b-a)X^3+(c-b)X^2+(a+d-c)X-d$. Par suite, $f$ est un endomorphisme de $E$ et la matrice de $f$ dans la base canonique $(1,X,X^2,X^3)$ de $E$ est

\begin{center}
$A=\left(
\begin{array}{cccc}
-1&0&0&0\\
1&-1&0&1\\
0&1&-1&0\\
0&0&1&-1
\end{array}
\right)$.
\end{center}

puis

\begin{align*}\ensuremath
\chi_A&=\left|
\begin{array}{cccc}
-1-X&0&0&0\\
1&-1-X&0&1\\
0&1&-1-X&0\\
0&0&1&-1-X
\end{array}
\right|= (-1-X)\left|
\begin{array}{ccc}
-1-X&0&1\\
1&-1-X&0\\
0&1&-1-X
\end{array}
\right|\\
 &= -(X+1)(-(X+1)^3+1) =X(X+1)(X^2+3X+3).
\end{align*}

 
$A$ admet quatre valeurs propres simples dans $\Cc$, deux réelles 0 et -1 et deux non réelles $-1+j$ et $-1+j^2$. $\chi_f$ n'est pas scindé sur $\Rr$ et donc $f$ n'est pas diagonalisable.

\textbullet~Soit $P\in E$. $P\in\text{Ker}f\Leftrightarrow b-a = c-b = a+d-c = -d = 0\Leftrightarrow a=b=c\;\text{et}\;d=0$. $\text{Ker}f=\text{Vect}(X^3+X^2+X)$.

\textbullet~Soit $P\in E$. $P\in\text{Ker}(f+Id)\Leftrightarrow b=c=a+d = 0\Leftrightarrow b=c=0\;\text{et}\;d=-a$. $\text{Ker}(f+Id)= \text{Vect}(X^3-1)$.

\textbullet~$\text{rg}(f) = 3$ et immédiatement $\text{Im}f =\text{Vect}(X-1,X^2-X,X^3-X^2)$.

Si $\Kk=\Cc$, on peut continuer :

$P\in\text{Ker}(f+(1-j)Id)\Leftrightarrow b-ja = c-jb = a + d - jc = - jd = 0\Leftrightarrow b = ja,\;c = j^2a\;\text{et}\;d=0$.

Donc  
$\text{Ker}(f+(1-j)Id)=\text{Vect}(X^3+jX^2+j^2X)$ et en conjuguant $\text{Ker}(f+(1-j^2)Id) =\text{Vect}(X^3+j^2X^2+jX)$.

\textbf{Remarque.} $B =X(X-1)(X-j)(X-j^2)$ et on a trouvé pour base de vecteurs propres les quatre polynômes de \textsc{Lagrange} $X^3-1 = (X-1)(X-j)(X-j^2)$ puis $X^3+X^2+X = X(X-j)(X-j^2)$ puis $X^3+jX^2+j^2X =X(X-1)(X-j^2)$ et enfin $X^3+j^2X^2+jX = X(X-1)(X-j )$. C'est une généralité. On peut montrer que si $E =\Cc_n[X]$ et si $B$ a $n+1$ racines deux à deux distinctes dans $\Cc$ alors $f$ est diagonalisable et une base de vecteurs propres est fournie par les polynômes de \textsc{Lagrange} associés aux racines de $B$ et ceci pour un polynôme $A$ quelconque.
}
}
