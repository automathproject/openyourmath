\uuid{RCEp}
\exo7id{2607}
\auteur{delaunay}
\datecreate{2009-05-19}
\isIndication{false}
\isCorrection{true}
\chapitre{Réduction d'endomorphisme, polynôme annulateur}
\sousChapitre{Diagonalisation}

\contenu{
\texte{
Soit $A$ la matrice 
$$A=\begin{pmatrix}1&0&0 \\  -1&2&1 \\  0&0&2\end{pmatrix}$$ 
et $f$ l'endomorphisme de $\R^3$ associ\'e.
}
\begin{enumerate}
    \item \question{Déterminer les valeurs propres de $A$.}
\reponse{{\it Déterminons les valeurs propres de $A$.}

Calculons les racines du polynôme caractéristique $P_A(X)$ :
$$P_A(X)=\begin{vmatrix}1-X&0&0 \\  -1&2-X&1 \\  0&0&2-X\end{vmatrix}=(2-X)^2(1-X).$$
Les valeurs propres de $A$ sont $\lambda_1=1$, valeur propre simple et $\lambda_2=2$, valeur propre double.}
    \item \question{Déterminer, sans calculs, des vecteurs $\vec u$ et $\vec v$ tels que $f(\vec u)=2\vec u$ et $f(\vec v)=2\vec v+\vec u$.}
\reponse{{\it Déterminons, sans calculs, des vecteurs $\vec u$ et $\vec v$ tels que $f(\vec u)=2\vec u$ et $f(\vec v)=2\vec v+\vec u$.}

Si l'on note $(\vec e_1,\vec e_2, \vec e_3)$, la base dans laquelle est exprimée la matrice $A$ de l'endomorphisme $f$, on remarque que 
$$f(\vec e_2)=2\vec e_2\quad{\hbox{et}}\quad f(\vec e_3)=\vec e_2+2\vec e_3.$$
Ainsi, les vecteurs $\vec u=\vec e_2$ et $\vec v=\vec e_3$ répondent-ils à la question.}
    \item \question{Soit $\vec e$ tel que $f(\vec e)=\vec e$. Démontrer que $(\vec e,\vec u,\vec v)$ est une base de $\R^3$ et écrire la matrice de $f$ dans cette base.}
\reponse{{\it Soit $\vec e$ tel que $f(\vec e)=\vec e$. Démontrons que $(\vec e,\vec u,\vec v)$ est une base de $\R^3$ et écrivons la matrice de $f$ dans cette base.}

Notons $\vec e=(x,y,z)$ alors
$$f(\vec e)=\vec e\iff\left\{\begin{align*}x&=x \\  -x+2y+z&=y \\  2z&=z\end{align*}\right.
\iff\left\{\begin{align*}z&=0 \\   x&=y\end{align*}\right.$$
Le vecteur $\vec e=(1,1,0)$ convient. Les vecteurs $\vec e$, $\vec u$ et $\vec v$ sont linéairement indépendants, ils forment donc une base de $\R^3$. La matrice de $f$ dans cette base s'écrit
$$B=\begin{pmatrix}1&0&0 \\  0&2&1 \\  0&0&2\end{pmatrix}$$}
    \item \question{La matrice $A$ est-elle diagonalisable ? (Justifier.)}
\reponse{{\it La matrice $A$ est-elle diagonalisable ?} 

Le sous-espace propre associé à la valeur propre $2$ est l'ensemble des vecteurs $(x,y,z)$ tels que 
$$\left\{\begin{align*}x&=2x \\  -x+2y+z&=2y \\  2z&=2z\end{align*}\right.\iff\left\{\begin{align*}x&=0 \\   z&=0\end{align*}\right.$$
C'est une droite vectorielle, sa dimension n'est donc pas égale à la multiplicité de la valeur propre $2$ comme racine du polynôme caractéristique, la matrice $A$ n'est pas diagonalisable.}
\end{enumerate}
}
