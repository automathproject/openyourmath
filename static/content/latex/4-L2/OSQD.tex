\uuid{OSQD}
\exo7id{7371}
\auteur{mourougane}
\datecreate{2021-08-10}
\isIndication{false}
\isCorrection{true}
\chapitre{Groupe, anneau, corps}
\sousChapitre{Anneau}

\contenu{
\texte{

}
\begin{enumerate}
    \item \question{Ecrire une relation de Bezout entre $X^2+X+1$ et $X^3+X+1$ dans $\R[X]$.}
\reponse{$ X^3+X+1=(X^2+X+1)(X-1)+(X+2)$ et $X^2+X+1=(X+2)(X-1)+3$.
Une relation de Bezout entre $X^2+X+1$ et $X^3+X+1$ dans $\R[X]$ est 
$$3=(X^2+X+1)-(X+2)(X-1)=(X^2+X+1)-(X-1)[(X^3+X+1)-(X^2+X+1)(X-1)]$$
soit 
$$3=(-X+1)(X^3+X+1)+(X^2-2X+2)(X^2+X+1).$$}
    \item \question{La classe du polynôme $X^2+X+1$ est-elle inversible dans l'anneau quotient\\
$\R[X]/(X^3+X+1)$ ? Si oui, donner son inverse.}
\reponse{Puisque $X^3+X+1$ et $X^2+X+1$ sont premiers entre eux, la classe du polynôme $X^2+X+1$ inversible dans l'anneau quotient $\R[X]/(X^3+X+1)$.
Son inverse est la classe de $(X^2-2X+2)/3$, comme le donne la relation de Bezout.}
\end{enumerate}
}
