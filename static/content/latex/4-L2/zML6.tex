\uuid{zML6}
\exo7id{1564}
\auteur{legall}
\datecreate{1998-09-01}
\isIndication{false}
\isCorrection{false}
\chapitre{Endomorphisme particulier}
\sousChapitre{Endomorphisme orthogonal}

\contenu{
\texte{
Soit $  (E , \langle   ,   \rangle )  $ un espace euclidien 
de dimension $  4  $
et $  \epsilon =\{ e_1, \cdots , e_4\}   $ une base orthonorm\'ee de $  E  .$ Soit $  A  $ la matrice $  \displaystyle{ A=\frac{1}{ 4}  \begin{pmatrix}  0 &

-2 \sqrt{2} & 2 \sqrt2
& 0 \cr   2 \sqrt{2}  & 1 & 1 &  -\sqrt {6} \cr  
-2 \sqrt{2}  & 1 & 1 &  \sqrt {6}  \cr  0  & \sqrt {6} & \sqrt {6} &  2 \cr \end{pmatrix} } $
et $  u \in \mathcal{L} (E)  $ l'endomorphisme d\'etermin\'e par $  \hbox{Mat}(u, \epsilon )=A  .$
}
\begin{enumerate}
    \item \question{Montrer que $  u \in \mathcal{O} ^+(E)  .$}
    \item \question{Montrer que l'espace vectoriel $  F  $ engendr\'e par $  e_1  $ et $  u(e_1)  $ est stable par
$  u  .$ Montrer que la restriction de $  u  $ \`a $  F  $ est une rotation.}
    \item \question{Montrer que $  F^\perp   $ est stable par $  u  $ et est engendr\'e par $  e_4  $ et $  u(e_4)  .$ La restriction de $  u  $ \`a $  F^\perp   $ est-elle une rotation~?}
\end{enumerate}
}
