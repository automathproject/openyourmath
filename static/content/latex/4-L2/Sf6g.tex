\uuid{Sf6g}
\exo7id{3607}
\auteur{quercia}
\organisation{exo7}
\datecreate{2010-03-10}
\isIndication{false}
\isCorrection{true}
\chapitre{Réduction d'endomorphisme, polynôme annulateur}
\sousChapitre{Autre}

\contenu{
\texte{
Soit $E$ un $ K$-ev de dimension finie et $u\in \mathcal{L}(E)$.
Montrer que les ensembles ${\cal K} = \{\mathrm{Ker}(P(u)),\ P\in K[X]\}$ et
${\cal I} = \{\Im(P(u)),\ P\in K[X]\}$ sont finis et ont même cardinal.
}
\reponse{
Soit $\mu$ le polynôme minimal de~$u$ et ${\cal D}$ l'ensemble des
diviseurs unitaires de~$\mu$. Pour $P\in K[X]$ et $d=P\wedge\mu$ on
a facilement $\mathrm{Ker}(P(u)) = \mathrm{Ker}(d(u))$ et $\Im(P(u)) = \Im(d(u))$.
Ceci montre déjà que $\cal K$ et $\cal I$ sont finis.

De plus, si $d\in{\cal D}$ alors l'annulateur
minimal de $u_{|\Im(d(u))}$ est $\mu/d$ donc l'application $d \mapsto\Im(d(u))$
est injective sur $\cal D$ et $\mathrm{Card}\,({\cal I}) = \mathrm{Card}\,({\cal D})$.
De même, l'annulateur minimal de $u_{|\mathrm{Ker}(d(u))}$ est $d$ car
$\mathrm{Ker}(d(u))\supset \Im(\frac\mu d(u))$ et $d$ est l'annulateur minimal
de $u_{|\Im(\frac\mu d(u))}$ donc l'application $d \mapsto\mathrm{Ker}(d(u))$
est injective sur $\cal D$ et $\mathrm{Card}\,({\cal K}) = \mathrm{Card}\,({\cal D})$.
}
}
