\uuid{GRYL}
\exo7id{5644}
\auteur{rouget}
\organisation{exo7}
\datecreate{2010-10-16}
\isIndication{false}
\isCorrection{true}
\chapitre{Déterminant, système linéaire}
\sousChapitre{Calcul de déterminants}

\contenu{
\texte{
Soient $a_0$,...,$a_{n-1}$ $n$ nombres complexes. Calculer $\left|
\begin{array}{ccccc}
a_0&a_1&\ldots&a_{n-2}&a_{n-1}\\
a_{n-1}&a_0&\ddots& &a_{n-2}\\
\vdots&\ddots&\ddots&\ddots&\vdots\\
a_2& &\ddots&\ddots&a_1\\
a_1&a_2&\ldots&a_{n-1}&a_0
\end{array}
\right|=\text{det}A$.
Pour cela, on calculera d'abord $A\Omega$ où $\Omega=(\omega^{(j-1)(k-1)})_{1\leqslant j,k\leqslant n}$ avec $\omega=e^{2i\pi/n}$.
}
\reponse{
Le coefficient ligne $j$, colonne $k$, $(j,k)\in\llbracket1,n\rrbracket^2$, de la matrice $A$ vaut $a_{k-j}$ avec la convention : si $-(n-1)\leqslant u\leqslant -1$, $a_u = a_{n+u}$.

Le coefficient ligne $j$, colonne $k$, $(j,k)\in\llbracket1,n\rrbracket^2$, de la matrice  $A\Omega$ vaut 

\begin{align*}\ensuremath
\sum_{u=1}^{n}a_{u-j}\omega^{(u-1)(k-1)}&=\sum_{v=-(j-1)}^{n-j}a_v\omega^{(v+j-1)(k-1)}=\sum_{v=-(j-1)}^{-1}a_v\omega^{(v+j-1)(k-1)}+\sum_{v=0}^{n-j}a_v\omega^{(v+j-1)(k-1)}\\
 &=\sum_{v=-(j-1)}^{-1}a_{v+n}\omega^{(v+n+j-1)(k-1)}+\sum_{u=0}^{n-j}a_u\omega^{(u+j-1)(k-1)}\;(\text{car}\;a_{v+n}=a_v\;\text{et}\;\omega^n = 1)\\
 &=\sum_{u=n-j+1}^{n-1}a_{u}\omega^{(u+j-1)(k-1)}+\sum_{u=0}^{n-j}a_u\omega^{(u+j-1)(k-1)}=\sum_{u=0}^{n-1}a_u\omega^{(u+j-1)(k-1)}\\
 &=\omega^{(j-1)(k-1)}\sum_{u=0}^{n-1}a_u\omega^{u(k-1)}.
\end{align*}

Pour $k\in\llbracket1,n\rrbracket$, posons $S_k =\sum_{u=0}^{n-1}a_u\omega^{u(k-1)}$. Le coefficient ligne $j$, colonne $k$ de $A\Omega$ vaut donc $\omega^{(j-1)(k-1)}S_k$. Par passage au détereminant, on en déduit que :

\begin{center}
$\text{det}(A\Omega)=\text{det}\left(\omega^{(j-1)(k-1)}S_k\right)_{1\leqslant j,k\leqslant n}=\left(\prod_{k=1}^{n}S_k\right)\text{det}(\omega^{(j-1)(k-1)})_{1\leqslant j,k\leqslant n}$
\end{center}

($S_k$ est en facteur de la colonne $k$) ou encore $(\text{det}A)(\text{det}\Omega)=\left(\prod_{k=1}^{n}S_k\right)(\text{det}\Omega)$. Enfin, $\Omega$ est la matrice de \textsc{Vandermonde} des racines $n$-èmes de l'unité et est donc inversible puisque celles-ci sont deux à deux distinctes. Par suite $\text{det}\Omega\neq0$ et après simplification on obtient

\begin{center}
\shadowbox{
$\text{det}A=\prod_{k=1}^{n}S_k$ où $S_k=\sum_{u=0}^{n-1}a_u\omega^{u(k-1)}$.
}
\end{center}

Par exemple, $\left|
\begin{array}{ccc}
a&b&c\\
c&a&b\\
b&c&a
\end{array}
\right|=S_1S_2S_3 =(a+b+c)(a+jb+j^2c)(a+j^2b+jc)$ où $j =e^{2i\pi/3}$.

Un calcul bien plus simple sera fourni dans la planche \og Réduction \fg.
}
}
