\uuid{g64q}
\exo7id{3801}
\auteur{quercia}
\organisation{exo7}
\datecreate{2010-03-11}
\isIndication{false}
\isCorrection{true}
\chapitre{Espace euclidien, espace normé}
\sousChapitre{Endomorphismes auto-adjoints}

\contenu{
\texte{
On note $P$ l'ensemble des fonctions~$f$ polynomiales par morceaux,
continues sur $[0,1]$ et vérifiant ${f(0)=f(1)=0}$.
Si $f$ et~$g$ sont des fonctions de~$P$, on note
$(f\mid g) =  \int_{t=0}^1f'(t)g'(t)\,d t$.
}
\begin{enumerate}
    \item \question{Que dire de~$P$ muni de cette application~?}
\reponse{Que c'est un espace préhilbertien.}
    \item \question{Montrer que si $x\in{[0,1]}$, il existe $g_x\in P$ telle que
    $\forall\ f\in P,\ (g_x\mid f) = f(x)$.}
\reponse{$g_x(t) = \min(t(1-x),x(1-t))$}
    \item \question{On considère $n$ réels vérifiant~:
    $0<x_1<x_2<\dots<x_n<1$ et on donne $n$ réels $(\alpha_i)_{i\in{[[1,n]]}}$.
    On pose $\varphi(f) = \|f\|^2 + \sum_{i=1}^n (f(x_i)-\alpha_i)^2$
    et on demande de trouver le minimum de~$\varphi$ sur~$P$.}
\reponse{On note $g_i = g_{x_i}$~: $(g_1,\dots,g_n)$ est libre par
    considération des points anguleux, donc engendre un espace vectoriel $G$ de dimension~$n$.
    Soit $f\in P$~: $f = f_0+f_1$ avec $f_0\in G$ et $f_1\in G^\bot$.
    Alors $\varphi(f) = \varphi(f_0) + \|f_1\|^2$ donc $\varphi$ est minimale
    en~$f$ ssi $\varphi_{|G}$ est minimale en~$f_0$ et $f_1=0$.
    Désormais on suppose $f_1=0$ et~$f\in G$.
    
    L'application~: $$ u : G \to {\R^n}, f \mapsto {(f(x_1),\dots,f(x_n)) = ((f\mid g_1),\dots,(f\mid g_n))}$$
    est un isomorphisme linéaire. Soit $v$ l'endormophisme autoadjoint défini
    positif de $\R^n$ (pour le produit scalaire canonique) tel que~:
    $\forall\ t\in\R^n, (t\mid v(t)) = \|u^{-1}(t)\|^2$.

    On a donc en notant $\alpha = (\alpha_1,\dots,\alpha_n)$ et $\beta = (\mathrm{id}+v)^{-1}(\alpha)$~:
    \begin{align*}\forall\ t\in\R^n,\ \varphi(u^{-1}(t))
      &= (t\mid v(t)) + (t-\alpha\mid t-\alpha)\\
      &= (t\mid(\mathrm{id}+v)(t)) - 2(t\mid \alpha) + (\alpha\mid\alpha)\\
      &= (t-\beta\mid(\mathrm{id}+v)(t-\beta)) + (\alpha\mid\alpha-\beta).\\ \end{align*}
    $\mathrm{id}+v$ est autoadjoint défini positif donc le minimum de~$\varphi$
    est atteint pour $f=u^{-1}(\beta)$ (solution unique) et vaut
    $(\alpha\mid\alpha-\beta)$.}
\end{enumerate}
}
