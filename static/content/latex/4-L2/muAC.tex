\uuid{muAC}
\exo7id{5804}
\auteur{rouget}
\organisation{exo7}
\datecreate{2010-10-16}
\isIndication{false}
\isCorrection{true}
\chapitre{Espace euclidien, espace normé}
\sousChapitre{Problèmes matriciels}

\contenu{
\texte{
La matrice $\left(
\begin{array}{cccc}
n-1&-1&\ldots&-1\\
-1&\ddots&\ddots&\vdots\\
\vdots&\ddots&\ddots&-1\\
-1&\ldots&-1&n-1
\end{array}
\right)$ est-elle positive ? définie ?
}
\reponse{
\textbf{1ère solution.} (n'utilisant pas les valeurs propres)
Soient $A$ la matrice de l'énoncé puis $X =(x_i)_{1\leqslant i\leqslant n}$ un élément de $\mathcal{M}_{n,1}(\Rr)$.

\begin{align*}\ensuremath
{^t}XAX&= (n-1)\sum_{i=1}^{n}x_i^2 -\sum_{i\neq j}^{}x_ix_j=\sum_{i\neq j}^{}(x_i^2- x_ix_j)=\frac{1}{2}\left(\sum_{i\neq j}^{}x_i^2-2 x_ix_j+x_j^2\right)\\
 &=\sum_{i\neq j}(x_i-x_j)^2\geqslant 0
\end{align*}

et donc la matrice $A$ est positive. De plus, si $X=\left(1\right)_{1\leqslant i\leqslant n}\neq0$, ${^t}XAX= 0$ et donc la matrice $A$ n'est pas définie.

\textbf{2ème solution.} La matrice $A$ est symétrique réelle. Donc ses valeurs propres sont réelles et $A$ est diagonalisable. Par suite, la dimension de chacun de des sous-espaces propres de $A$ est égale à l'ordre de multiplicité de la valeur propre correspondante.

On note alors que $\text{rg}(A - nI_n) = 1$ et donc $n$ est valeur propre de $A$ d'ordre $n-1$. Soit $\lambda$ la valeur propre manquante.
\begin{center}
$(n-1)n+\lambda=\text{Tr}A = n(n-1)$.
\end{center}

Donc $\lambda= 0$. Ainsi, $\text{Sp}(A)\subset\Rr^+$ et donc la matrice $A$ est positive mais $0$ est valeur propre de $A$ et donc la matrice $A$ n'est pas définie.

\begin{center}
\shadowbox{
La matrice $A$ est positive et non définie.
}
\end{center}
}
}
