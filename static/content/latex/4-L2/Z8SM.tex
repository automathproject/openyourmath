\uuid{Z8SM}
\exo7id{3689}
\auteur{quercia}
\datecreate{2010-03-11}
\isIndication{false}
\isCorrection{true}
\chapitre{Espace euclidien, espace normé}
\sousChapitre{Produit scalaire, norme}

\contenu{
\texte{
Soit $P \in \R[X]$ de degré inférieur ou égal à 3 tel que
$ \int_{t=-1}^1 P^2(t)\,d t = 1$.

Montrer que $\sup\{ |P(x)|\text{ tq } -1\le x \le 1\} \le 2\sqrt2$.

Indications : Pour $a\in\R$ montrer qu'il existe $P_a\in\R_3[X]$ tel que :
$\forall\ P\in\R_3[X],\ P(a) =  \int_{t=-1}^1 P(t)P_a(t)\,d t$.
Calculer explicitement $P_a$, et appliquer l'inégalité de Cauchy-Schwarz.
}
\reponse{
$P_a(t) = \frac{3}{8}(3 -5t^2 -5a^2 +15a^2t^2)
                   +\frac{5at}8(15-21t^2 -21a^2 +35a^2t^2)$,\par
$8\|P_a\|^2 = 9 + 45a^2 - 165a^4 + 175a^6$ est maximal pour $a=\pm1  \Rightarrow  \|P_a\| = 2\sqrt2$.
}
}
