\uuid{p1C0}
\exo7id{1567}
\auteur{legall}
\organisation{exo7}
\datecreate{1998-09-01}
\isIndication{false}
\isCorrection{false}
\chapitre{Endomorphisme particulier}
\sousChapitre{Endomorphisme orthogonal}

\contenu{
\texte{
Soit $  (E , \langle   ,   \rangle )  $ un espace euclidien et
$  s \in \mathcal{L} (E)  $ telle que $  s^2=id  .$
}
\begin{enumerate}
    \item \question{Montrer que $  E=\hbox{Ker}(s-Id)\oplus \hbox{Ker}(s+Id)  .$}
    \item \question{Montrer que les propri\'et\'es suivantes sont \'equivalentes~:
\vskip1mm \hskip1mm {\em i)} $  s \in \mathcal{O} (E)  .$
\vskip1mm \hskip1mm {\em ii)} $  \hbox{Ker}(s-Id) \perp  \hbox{Ker}(s+Id)  .$
\vskip1mm \hskip1mm {\em iii)} $  s=s^*  .$}
    \item \question{On note d\'esormais $  s_F   $ l'unique sym\'etrie $  s \in \mathcal{O} (E)  $ telle que $ 
F=\hbox{Ker}(s+Id)  .$ Montrer que pour tout $  u\in \mathcal{O} (E)  $ on a~: $  us_Fu^{-1}=s_{u(F)}  .$}
    \item \question{Montrer que si $  f   $ est une application de $  E  $ dans lui-m\^eme laissant stables toutes
les droites vectorielles (c'est \`a dire que pour tout $  x\in E  $ il existe $  \lambda _x
\in \R  $ tel que $  f(x)=\lambda _x x  $) alors $  f   $ est lin\'eaire.}
    \item \question{En d\'eduire que $  Z(\mathcal{O} (E))=\{ id, -id\}   $ et que si $  n\geq 3  $ alors
$  Z(\mathcal{O} ^+ (E))=\{ id, -id\} \cap \mathcal{O} ^+(E)  .$ (on pourra appliquer 3.) dans le cas o\`u $  F  $
est une droite ou un plan.)}
    \item \question{Que se passe-t-il lorsque $  n=1  $ et $  n=2  ?$}
\end{enumerate}
}
