\uuid{6qjc}
\exo7id{3547}
\auteur{quercia}
\datecreate{2010-03-10}
\isIndication{false}
\isCorrection{true}
\chapitre{Réduction d'endomorphisme, polynôme annulateur}
\sousChapitre{Polynôme annulateur}

\contenu{
\texte{
Soit $E$ un ev de dimension finie et $u \in \mathcal{L}(E)$ tel que $\mathrm{rg}(u)=1$.
Montrer que : $$\Im u \subset \mathrm{Ker} u \Leftrightarrow u \text{ n'est pas diagonalisable.}$$
}
\reponse{
Si $\Im u\subset \mathrm{Ker} u$ alors $u^2 = 0$ donc $0$ est l'unique
         valeur propre de $u$ et $u\ne 0$ donc $u$ n'est pas diagonalisable.\par
         Si $\Im u \not \subset \mathrm{Ker} u$ alors $\Im u \cap \mathrm{Ker} u = \{\vec 0\}$
         et donc $\Im u + \mathrm{Ker} u = E$. Or $\Im u$ et $\mathrm{Ker} u$ sont des sous-espaces
         propres de $u$ donc $u$ est diagonalisable.
}
}
