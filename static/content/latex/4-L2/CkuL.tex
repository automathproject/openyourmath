\uuid{CkuL}
\exo7id{3471}
\auteur{quercia}
\organisation{exo7}
\datecreate{2010-03-10}
\isIndication{false}
\isCorrection{false}
\chapitre{Déterminant, système linéaire}
\sousChapitre{Système linéaire, rang}

\contenu{
\texte{
Soient $A \in \mathcal{M}_{n,p}(K)$ et $B \in \mathcal{M}_{n,q}(K)$. On considère
$C = (A\ B) \in \mathcal{M}_{n,p+q}(K)$.

Montrer que : $\mathrm{rg}(C) = \mathrm{rg}(A) \Leftrightarrow \exists\ P \in \mathcal{M}_{p,q}(K)$ tq $B = AP$.
}
}
