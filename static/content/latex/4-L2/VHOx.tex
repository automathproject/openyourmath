\uuid{VHOx}
\exo7id{3842}
\auteur{quercia}
\datecreate{2010-03-11}
\isIndication{false}
\isCorrection{true}
\chapitre{Espace euclidien, espace normé}
\sousChapitre{Espaces vectoriels hermitiens}

\contenu{
\texte{
Soit $E = \mathcal{C}([a,b],\R)$ et $(a_n)$ une suite d'éléments de $[a,b]$.

Pour $f,g \in E$,
on pose : $(f\mid g) = \sum_{n=0}^\infty \frac{f(a_n)g(a_n)}{2^n}$.
}
\begin{enumerate}
    \item \question{A quelle condition sur la suite $(a_n)$ définit-on un produit scalaire ?}
    \item \question{Soient $a=(a_n)$ et $b=(b_n)$ deux telles suites telles que les
    ensembles $\{a_n,n\in\N\}$ et $\{b_n,n\in\N\}$ sont distincts.
    Montrer que les normes correspondantes sont non équivalentes.}
    \item \question{Question ouverte~: à quelle condition les normes associées à
    deux suites $(a_n)$ et $(b_n)$ sont-elles équivalentes~?}
    \item \question{Montrer qu'il n'existe pas de suite $(a_n)$ pour laquelle $E$ soit
    complet.}
\reponse{
$(a_n)$ est partout dense.
Si les $a_n$ sont distincts, on choisit pour tout~$n$
une fonction $f_n$ comprise entre $0$ et~$1$ valant alternativement
$1$ et~$-1$ en $a_0,\dots,a_n$. Alors la suite $(f_n)$ est de Cauchy
mais ne converge pas car si $f_n\to f$ alors $f^2 \equiv 1$, absurde.
}
\end{enumerate}
}
