\uuid{PB8G}
\exo7id{5366}
\auteur{rouget}
\organisation{exo7}
\datecreate{2010-07-06}
\isIndication{false}
\isCorrection{true}
\chapitre{Déterminant, système linéaire}
\sousChapitre{Calcul de déterminants}

\contenu{
\texte{
Soit $A=(a_{i,j})_{1\leq i,j\leq n}$ où, pour tout $i$ et tout $j$, $a_{i,j}\in\{-1,1\}$. Montrer que $\mbox{det}\;A$ est un entier divisible par $2^{n-1}$.
}
\reponse{
On procède par récurrence sur $n\geq1$.
\textbullet~Pour $n=1$, c'est clair.
\textbullet~Soit $n\geq1$. Supposons que tout déterminant $\Delta_n$ de format $n$ et du type de l'énoncé soit divisible par $2^{n-1}$. Soit $\Delta_{n+1}$ un déterminant de format $n+1$, du type de l'énoncé.
Si tous les coefficients $a_{i,j}$ de $\Delta_{n+1}$ sont égaux à $1$, puisque $n+1\geq2$, $\Delta_{n+1}$ a deux colonnes égales et est donc nul. Dans ce cas, $\Delta_{n+1}$ est bien divisible par $2^n$.
Sinon, on va changer petit à petit tous les $-1$ en $1$.
Soit $(i,j)$ un couple d'indices tel que $a_{i,j}=-1$ et $\Delta_{n+1}'$ le déterminant dont tous les coefficients sont égaux à ceux de $\Delta_{n+1}$ sauf le coefficient ligne $i$ et colonne $j$ qui est égal à $1$.

$$\Delta_{n+1}-\Delta_{n+1}'=\mbox{det}(C_1,...,C_j,...,C_n)-\mbox{det}(C_1,...,C_j',...,C_n)=\mbox{det}(C_1,...,C_j-C_j',...,C_n),$$ 

où  $C_j-C_j'=\left(
\begin{array}{c}
0\\
\vdots\\
0\\
-2\\
0\\
\vdots\\
0
\end{array}
\right)$ ($-2$ en ligne $i$). En développant ce dernier déterminant suivant sa $j$-ème colonne, on obtient~:

$$\Delta_{n+1}-\Delta_{n+1}'=-2\Delta_n,$$ 
où $\Delta_n$ est un déterminant de format $n$ et du type de l'énoncé. Par hypothèse de récurrence, $\Delta_n$ est divisible par $2^{n-1}$ et donc $\Delta_{n+1}-\Delta_{n+1}'$ est divisible par $2^n$. Ainsi, en changeant les $-1$ en $1$ les uns après les autres, on obtient

\begin{center}
$\Delta_{n+1}\equiv\left|
\begin{array}{ccc}
1&\ldots&1\\
\vdots& &\vdots\\
1&\ldots&1
\end{array}\right|\;(\text{mod}\;2^n)$.
\end{center}
Ce dernier déterminant étant nul, le résultat est démontré par récurrence.
}
}
