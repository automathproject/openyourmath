\uuid{NdIl}
\exo7id{5296}
\auteur{rouget}
\organisation{exo7}
\datecreate{2010-07-04}
\isIndication{false}
\isCorrection{true}
\chapitre{Arithmétique}
\sousChapitre{Arithmétique de Z}

\contenu{
\texte{
Soient $A$ la somme des chiffres de $4444^{4444}$ et $B$ la somme des chiffres de $A$. Trouver la somme des chiffres de $B$. (Commencer par majorer la somme des chiffres de $n=a_0+10a_1...+10^pa_p$.)
}
\reponse{
Soit $n$ un entier naturel non nul. On note $\sigma(n)$ la somme de ses chiffres en base $10$ (voir l'exercice \ref{exo:suprou19}). Si $n=a_0+10a_1+...+10^ka_k$ où $k\in\Nn$, $0\leq a_i\leq9$ pour $0\leq i\leq k$ et $a_k\neq0$, alors

$$\sigma(n)=a_0+...+a_k\leq9(k+1)\leq 9(E(\log n)+1)\leq9(\log n+1).$$

Donc, 

$$A=\sigma(4444^{4444})\leq9(\log(4444^{4444})+1)\leq9(4444\log(10^5)+1)=9(4444.5+1)=9.22221=199989.$$

Puis, $B=\sigma(A)\leq1+5.9=46$, puis $\sigma(B)\leq\sigma(39)=12$. Donc, $1\leq\sigma(B)\leq12$.

D'autre part, on sait que modulo 9~:~$\sigma(B)\equiv B\equiv A=4444^{4444}$.
Enfin, $4444^{4444}=(9.443+7)^{4444}\equiv7^{4444}\;(9)$. De plus, $7\equiv-2\;(9)$ puis $7^2\equiv4\;(9)$ puis $7^3\equiv 28\equiv1\;(9)$ et donc $7^{4444}=(7^3)^{1481}.7\equiv(1^3)^{1481}.7\equiv7\;(9)$. Finalement, $1\leq\sigma(B)\leq12$ et $C\equiv7\;(9)$ ce qui impose $C=7$.
}
}
