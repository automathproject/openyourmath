\uuid{uO3B}
\exo7id{1171}
\auteur{liousse}
\datecreate{2003-10-01}
\isIndication{false}
\isCorrection{false}
\chapitre{Déterminant, système linéaire}
\sousChapitre{Système linéaire, rang}

\contenu{
\texte{
Mettre sous forme matricielle et r\'esoudre les syst\`emes suivants.\\
}
\begin{enumerate}
    \item \question{$ \left\{\begin{array}{rcl}2x+y+z&=&3\\3x-y-2z&=&0\\
x+y-z&=&-2\\x+2y+z&=&1\end{array}\right.$ \hfill}
    \item \question{$ \left\{\begin{array}{rcl}x+y+z+t&=&1\\x-y+2z-3t&=&2\\
2x+4z+4t&=&3\\2x+2y+3z+8t&=&2\\5x+3y+9z+19t&=&6\end{array}\right.$}
    \item \question{$ \left\{\begin{array}{rcl}2x+y+z+t&=&1\\x+2y+3z+4t&=&2\\
3x-y-3z+2t&=&5\\5y+9z-t&=&-6\end{array}\right.$ \hfill}
    \item \question{$ \left\{\begin{array}{rcl}x-y+z+t&=&5\\2x+3y+4z+5t&=&8\\
3x+y-z+t&=&7\end{array}\right.$}
    \item \question{$\left\{\begin{array}{rcl}x+2y+3z&=&0\\2x+3y-z&=&0\\
3x+y+2z&=&0\end{array}\right.$}
\end{enumerate}
}
