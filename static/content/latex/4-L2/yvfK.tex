\uuid{yvfK}
\exo7id{3832}
\auteur{quercia}
\datecreate{2010-03-11}
\isIndication{false}
\isCorrection{true}
\chapitre{Espace euclidien, espace normé}
\sousChapitre{Espaces vectoriels hermitiens}

\contenu{
\texte{
Soit $E$ un espace préhilbertien et $\vec u_1,\dots,\vec u_n \in E$.
On note $G = (g_{ij}) \in \mathcal{M}_n(K)$ la matrice de Gram de ces vecteurs
($g_{ij} = (\vec u_i\mid\vec u_j)$).
}
\begin{enumerate}
    \item \question{On suppose $E$ de dimension finie, rapporté à une base orthonormée
    ${\cal B} = (\vec e_1,\dots,\vec e_p)$.
    Exprimer $G$ en fonction de
    $M = \text{Mat}_{\cal B}(\vec u_1,\dots,\vec u_n)$.}
    \item \question{En déduire que $\det(G)$ est un réel positif ou nul, et nul si et
    seulement si les vecteurs $\vec u_i$ sont liés.}
    \item \question{Montrer le même résultat sans supposer que $E$ est de dimension finie.}
    \item \question{Examiner le cas particulier $n=2$.}
    \item \question{Application : Le tétraèdre $ABCD$ est tel que $AB=AC=AD=1$ et
    $(AB,AC) \equiv \frac \pi4$, $(AB,AD) \equiv \frac \pi3$,
    $(AC,AD) \equiv \frac\pi2$. Calculer son volume.}
\reponse{
$\frac1{12}$.
}
\end{enumerate}
}
