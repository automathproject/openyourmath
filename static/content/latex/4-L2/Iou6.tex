\uuid{Iou6}
\exo7id{1613}
\auteur{liousse}
\datecreate{2003-10-01}
\isIndication{false}
\isCorrection{false}
\chapitre{Réduction d'endomorphisme, polynôme annulateur}
\sousChapitre{Diagonalisation}

\contenu{
\texte{
Soit $E$ un espace vectoriel de 
dimension $3$ et de base $(e_1,e_2,e_3)$. On d\'esigne
par $I_E$ l'application identit\'e de $E$. Soit $f$ une 
application lin\'eaire de $E$ dans $E$ telle que
$f(e_1) = 2e_2+3e_3, f(e_2) = 2e_1 - 5e_2-8e_3, f(e_3) = -e_1+4e_2+6e_3$.
}
\begin{enumerate}
    \item \question{Donner la matrice de $f$ dans la base $(e_1, e_2,e_3)$.}
    \item \question{Donner la dimension et une base de $Ker (f-I_E)$.}
    \item \question{Donner la dimension et une base de $Ker (f^2+I_E)$.}
    \item \question{Montrer que la reunion des bases pr\'ec\'edentes constitue une base de E. 
Quelle est la matrice de $f$ dans cette nouvelle base ? Et celle de $f^2$ ?}
\end{enumerate}
}
