\uuid{Pwv0}
\exo7id{5652}
\auteur{rouget}
\datecreate{2010-10-16}
\isIndication{false}
\isCorrection{true}
\chapitre{Réduction d'endomorphisme, polynôme annulateur}
\sousChapitre{Applications}

\contenu{
\texte{
Résoudre dans $\mathcal{M}_3(\Rr)$ l'équation $X^2 = A$ où $A =\left(
\begin{array}{ccc}
3&0&0\\
8&4&0\\
5&0&1
\end{array}
\right)$.
}
\reponse{
Soit $X\in\mathcal{M}_3(\Rr)$. Si $X^2 = A$ alors $AX= X^3 = XA$ et donc $X$ et $A$ commutent.

$A$ admet trois valeurs propres réelles et simples à savoir $1$, $3$ et $4$. Donc $A$ est diagonalisable dans $\Rr$ et les sous espaces propres de $A$ sont des droites. $X$ commute avec $A$ et donc laisse stable les trois droites propres de $A$.

Ainsi une base de $\mathcal{M}_{3,1}(\Rr)$ formée de vecteurs propres de $A$ est également une base de vecteurs propres de $X$ ou encore, si $P$ est une matrice réelle inversible telle que $P^{-1}AP$ soit une matrice diagonale $D_0$ alors pour la même matrice $P$, 
$P^{-1}XP$ est une matrice diagonale $D$. De plus

\begin{center}
$X^2 =A\Leftrightarrow PD^2P^{-1}= PD_0P^{-1}\Leftrightarrow D^2 = D_0\Leftrightarrow D =\text{diag}(\pm\sqrt{3},\pm2, \pm1)$
\end{center}

ce qui fournit huit solutions deux à opposées. On peut prendre $P=\left(
\begin{array}{ccc}
2&0&0\\
-16&1&0\\
5&0&1
\end{array}
\right)$ puis $P^{-1}=\left(
\begin{array}{ccc}
1/2&0&0\\
8&1&0\\
-5/2&0&1
\end{array}
\right)$. 
D'où les solutions 

\begin{align*}\ensuremath
\left(
\begin{array}{ccc}
2&0&0\\
-16&1&0\\
5&0&1
\end{array}
\right)\left(
\begin{array}{ccc}
\sqrt{3}\varepsilon_1&0&0\\
0&2\varepsilon&0\\
0&0&\varepsilon_3
\end{array}
\right)\left(
\begin{array}{ccc}
1/2&0&0\\
8&1&0\\
-5/2&0&1
\end{array}
\right)&=\left(
\begin{array}{ccc}
2\sqrt{3}\varepsilon_1&0&0\\
-16\sqrt{3}\varepsilon_1&2\varepsilon_2&0\\
5\sqrt{3}\varepsilon_1&0&\varepsilon_3
\end{array}
\right)\left(
\begin{array}{ccc}
1/2&0&0\\
8&1&0\\
-5/2&0&1
\end{array}
\right)\\
 &=\left(
\begin{array}{ccc}
\sqrt{3}\varepsilon_1&0&0\\
-8\sqrt{3}\varepsilon_1+16\varepsilon_2&2\varepsilon_2&0\\
5(\sqrt{3}\varepsilon_1-\varepsilon_3)/2&0&\varepsilon_3
\end{array}
\right).
\end{align*}

où $(\varepsilon_1, \varepsilon_2,\varepsilon_3)\in\{-1,1\}^3$.
}
}
