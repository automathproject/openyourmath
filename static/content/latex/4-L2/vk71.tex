\uuid{vk71}
\exo7id{3798}
\auteur{quercia}
\organisation{exo7}
\datecreate{2010-03-11}
\isIndication{false}
\isCorrection{true}
\chapitre{Espace euclidien, espace normé}
\sousChapitre{Endomorphismes auto-adjoints}

\contenu{
\texte{
Soit $A\in M_n(\R)$ telle que $A^3={}^t\!A A$. $A$ est-elle diagonalisable dans $M_n(\R)$, dans
$M_n(\C)$ ?
}
\reponse{
${}^t\!AA$ est $\R$-diagonalisable donc annule un polynôme
$P$ scindé à racines simples. $A$ annule le polynôme $P(X^3)$, donc est $\C$-diagonalisable
si $0$ n'est pas racine de~$P$ ce que l'on peut imposer si $A$ est inversible.

Si $A$ n'est pas inversible, soit $P(X) = XQ(X)$ avec $Q(0)\ne 0$.

On a $\R^n = \mathrm{Ker}(A^3)\oplus \mathrm{Ker}(Q(A^3))$ et $\mathrm{Ker}(A^3) = \mathrm{Ker}({}^t\!AA) = \mathrm{Ker}(A)$
donc $AQ(A^3)=0$ et $A$ est encore $\C$-diagonalisable.

Contre-exemple pour la $\R$-diagonalisabilité~: prendre une
rotation d'angle $2\pi/3$ dans le plan.
}
}
