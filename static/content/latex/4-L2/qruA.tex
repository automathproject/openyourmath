\uuid{qruA}
\exo7id{1513}
\auteur{ortiz}
\datecreate{1999-04-01}
\isIndication{false}
\isCorrection{false}
\chapitre{Espace euclidien, espace normé}
\sousChapitre{Forme quadratique}

\contenu{
\texte{
Soit $q$ une forme quadratique sur un $\Rr$-espace
vectoriel $E$, que l'on suppose d\'efinie (\emph{i.e.} son
c\^one isotrope est $\left\{0\right\}$). Montrer que $q$
garde un signe constant sur $E$ (on pourra
raisonner par l'absurde et consid\'erer $q(a+tb)$
o\`u $a$ et $b$ sont des vecteurs bien choisis et
$t\in\Rr$).
}
}
