\uuid{aqKu}
\exo7id{3417}
\auteur{quercia}
\organisation{exo7}
\datecreate{2010-03-10}
\isIndication{false}
\isCorrection{true}
\chapitre{Déterminant, système linéaire}
\sousChapitre{Système linéaire, rang}

\contenu{
\texte{
\'Etudier l'existence de solutions du système :
 $$\left\{\begin{array}{ccccccc} ax &+&  by &+&  z &=& 1 \cr
             x &+& aby &+&  z &=& b \cr
             x &+&  by &+& az &=& 1.\cr\end{array}\right.$$
}
\reponse{
Pour éviter d'avoir à diviser par $a$ on réordonne nos lignes puis on applique la méthode du pivot :
{\small
$$\left\{\begin{array}{ccccccll}
x &+&  by &+& az &=& 1 & _{L_1} \cr
x &+& aby &+&  z &=& b & _{L_2} \cr
ax &+&  by &+&  z &=& 1 & _{L_3} \cr
\end{array}\right.
\iff
\left\{\begin{array}{ccccccll}
x &+&  by &+& az &=& 1 &_{L_1} \cr
  & & b(a-1)y &+&  (1-a)z &=& b-1 & _{L_2 \leftarrow L_2-L_1} \cr
  &&  b(1-a)y &+&  (1-a^2)z &=& 1-a & _{L_3 \leftarrow L_3-aL_1} \cr
\end{array}\right.
$$
}
On fait ensuite $L_3 \leftarrow L_3+L_2$ pour obtenir un système triangulaire équivalent au système initial :
$$\left\{\begin{array}{ccccccl}
x &+&  by &+& az &=& 1  \cr
  & & b(a-1)y &+&  (1-a)z &=& b-1  \cr
  &&   &&  (2-a-a^2)z &=& b-a  \cr
\end{array}\right.
$$
Nous allons maintenant discuter de l'existence des solutions.
Remarquons d'abord que $2-a-a^2= -(a-1)(a+2)$.
Donc si $a\neq 1$ et $a\neq -2$ alors $2-a-a^2\neq 0$
donc $z=\frac{a-b}{(a-1)(a+2)}$. On a donc trouvé la valeur de $z$.
La deuxième ligne du système triangulaire est 
$b(a-1)y +  (1-a)z = b-1$ on sait déjà $a-1\neq 0$.
Si $b\neq 0$ alors, en reportant la valeur de $z$ obtenue, on trouve la valeur 
$y= \frac{b-1 - (1-a)z}{b(a-1)}$.
Puis avec la première ligne on en déduit aussi $x=1-by-az$.

Donc si $a\neq 1$ et $a \neq -2$ et $b\neq 0$ alors il existe une unique solution $(x,y,z)$.
Il faut maintenant s'occuper des cas particuliers.
  \begin{enumerate}
Si $a=1$ alors notre système triangulaire devient :
$$\left\{\begin{array}{ccccccc}
x &+&  by &+&z&=& 1 \cr
  & & & &  0 &=& b-1 \cr
  &&   &&  0 &=& b-1 \cr
\end{array}\right.
$$ 
Si $b\neq 1$ il n'y a pas de solution.
Si $a=1$ et $b=1$
alors il ne reste plus que l'équation $x+y+z=1$.
On choisit par exemple $y,z$ comme paramètres, l'ensemble des solutions 
est $$\big\{(1-y-z,y,z) \mid y,z \in \Rr \big\}.$$
Si $a=-2$ alors le système triangulaire devient :
$$\left\{\begin{array}{ccccccc}
x &+&  by &-& 2z &=& 1 \cr
  & & -3by &+&  3z &=& b-1 \cr
  &&   &&  0 &=& b+2 \cr
\end{array}\right.
$$
Donc si $b\neq -2$ il n'y a pas de solution. 
Si $a=-2$ et $b=-2$ alors le système est
$$\left\{\begin{array}{ccccccc}
x &-&  2y &-& 2z &=& 1 \cr
  & & 2y &+&  z &=& -1 \cr
\end{array}\right.
$$
Si l'on choisit $y$ comme paramètre alors il y a une infinité de solutions
$$\big\{(-1-2y,y,-1-2y) \mid y \in \Rr \big\}.$$
Enfin si $b=0$
alors la deuxième et troisième ligne du système triangulaire sont :
$(1-a)z =-1 $ et $(2-a-a^2)z =-a$. Donc $z=\frac{-1}{1-a}=\frac{-a}{2-a-a^2}$
(le sous-cas $b=0$ et $a=1$ n'a pas de solution).
Dans tous les cas il n'y a pas de solution.
Conclusion :
\begin{itemize}
Si $a\neq 1$ et $a\neq -2$ et $b\neq 0$, c'est un système de Cramer : il admet une unique solution.
Si $a=1$ et $b\neq 1$ il n'y a pas de solution (le système n'est pas compatible).
Si $a=1$ et $b=1$ il y a une infinité de solutions (qui forment un plan dans $\Rr^3$).
Si $a=-2$ et $b\neq -2$ il n'y a pas de solution.
Si $a=-2$ et $b=-2$ il y a une infinité de solutions (qui forment une droite dans $\Rr^3$).
Si $b=0$ il n'y a pas de solution.
\end{itemize}
}
}
