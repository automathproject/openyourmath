\uuid{vDn3}
\exo7id{3525}
\auteur{quercia}
\organisation{exo7}
\datecreate{2010-03-10}
\isIndication{false}
\isCorrection{true}
\chapitre{Réduction d'endomorphisme, polynôme annulateur}
\sousChapitre{Polynôme caractéristique, théorème de Cayley-Hamilton}

\contenu{
\texte{
Soit $A \in \mathcal{M}_n(K)$ et $\lambda \in  K$ une valeur propre de $A$ telle que
$\mathrm{rg}(A-\lambda I) = n-1$.
}
\begin{enumerate}
    \item \question{Quelle est la dimension du sous espace propre $E_\lambda$ ?}
    \item \question{Montrer que les colonnes de ${}^t\text{com}(A - \lambda I)$ engendrent
    $E_\lambda$.}
    \item \question{Exemple : diagonaliser $A = \begin{pmatrix} 0 & \phantom-1 &  2 \cr
                                       1 & 1          &  1 \cr
                                       1 & 0          & -1 \cr \end{pmatrix}$.}
\reponse{
3. $P = \begin{pmatrix} -1 &\phantom-3 & 3 \cr 2 & 4 & 0 \cr -1 & 1 & -3 \cr\end{pmatrix}$,
    $D = \begin{pmatrix} 0 & 2 & -2 \cr \end{pmatrix}$.
}
\end{enumerate}
}
