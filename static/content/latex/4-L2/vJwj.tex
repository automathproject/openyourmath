\uuid{vJwj}
\exo7id{3751}
\auteur{quercia}
\datecreate{2010-03-11}
\isIndication{false}
\isCorrection{true}
\chapitre{Espace euclidien, espace normé}
\sousChapitre{Projection, symétrie}

\contenu{
\texte{
$E = \mathcal{M}_n(\R)$ muni du produit scalaire : $(A\mid B) = \mathrm{tr}({}^tAB)$.
}
\begin{enumerate}
    \item \question{Vérifier que c'est un produit scalaire.}
    \item \question{Soit $P \in {\cal O}(n)$. Montrer que les applications
    $\begin{cases} \phi_P : A \longmapsto AP \cr
             \psi_P : A \longmapsto P^{-1}AP \cr\end{cases}$
    sont orthogonales.}
    \item \question{Réciproquement, si $\phi_P$ ou $\psi_P \in {\cal O}(\mathcal{M}_n(\R))$,
    est-ce que $P \in {\cal O}(n)$ ?}
\reponse{
oui pour $\phi_P$.

Pour $\psi_P : \forall\ A,B,\ \mathrm{tr}({}^tAB) = \mathrm{tr}({}^tP^tA^t(P^{-1})P^{-1}BP )
      =\mathrm{tr}(P^tP^tA^t(P^{-1})P^{-1}B )$.

Donc $P^tP^tA^t(P^{-1})P^{-1} = A$, donc $P^tP$ est scalaire, donc $P$ est une
matrice de similitude.
}
\end{enumerate}
}
