\uuid{6ZMm}
\exo7id{1418}
\auteur{ortiz}
\datecreate{1999-04-01}
\isIndication{false}
\isCorrection{true}
\chapitre{Groupe, anneau, corps}
\sousChapitre{Groupe de permutation}

\contenu{
\texte{
Trouver la d\'ecomposition en produit de cycles \`a
supports disjoints, la signature, l'ordre et une d\'ecomposition
en produit de transpositions
des permutations suivantes de ${\cal S}_{10}:$%
$${
\sigma =\left(
\begin{array}{cccccccccc}
1 & 2 & 3 & 4 & 5 & 6 & 7 & 8 & 9 & 10 \\
3 & 7 & 1 & 4 & 2 & 6 & 9 & 8 & 5 & 10
\end{array}
\right) ,\
}$$
$${
\varphi =\left( 10,3,4,1\right) \left( 8,7\right) \left( 4,7\right) \left(
5,6\right) \left( 2,6\right) \left( 2,9\right)
.\bigskip }$$ Calculer $\sigma ^{1998}\;$et
$\varphi ^{1998}.$
}
\reponse{
$\sigma = (1,3)(2,7,9,5)= (2,7,9,5)(1,3)$ et
$\sigma^k = (1,3)^k(2,7,9,5)^k$. Les transpositions sont d'ordre
$2$ donc $(1,3)^k = \mathrm{id}$ si $k \equiv 0 (\mod 2)$ et $(1,3)^k =
(1,3)$ si $k \equiv 1 (\mod 2)$. Le cycle $(2,7,9,5)$ est d'ordre
$4$, et $(2,7,9,5)^k$ est respectivement égale à $\mathrm{id}, (2,7,9,5),
(2,9)(7,5), (5,9,7,2)$ si $k$ est respectivement congru à
$0,1,2,3$ modulo $4$. Le calcul de $\sigma^k$ donne donc $\mathrm{id},
(1,3)(2,7,9,5), (2,9)(7,5)$ ou $(1,3)(5,9,7,2)$ selon que $k$ est
congru à $0,1,2$ ou $3$ modulo $4$.
L'écriture de
$\varphi = (10,3,4,1)(8,7)(4,7)(5,6)(2,6)(2,9)$ est une
décomposition en produit de cycles mais ils ne sont pas à supports
disjoints. \'Ecrivons $\varphi$ sous la forme :
$$\phi = \begin{pmatrix}
1&2&3&4&5&6&7&8&9&10\\
10&9&4&8&6&2&1&7&5&3\\
\end{pmatrix}$$
Ce qui se décompose $\varphi = (1,10,3,4,8,7)(2,9,5,6) =
(2,9,5,6)(1,10,3,4,8,7)$. Le calcul de $\varphi^k =
(1,10,3,4,8,7)^k(2,9,5,6)^k$ est similaire au calcul précédent
(selon $k (\mod 12)$ )
}
}
