\uuid{PtnI}
\exo7id{5485}
\auteur{rouget}
\datecreate{2010-07-10}
\isIndication{false}
\isCorrection{true}
\chapitre{Espace euclidien, espace normé}
\sousChapitre{Produit scalaire, norme}

\contenu{
\texte{
Sur $\Rr[X]$, on pose $P|Q=\int_{0}^{1}P(t)Q(t)\;dt$. Existe-t-il $A$ élément de $\Rr[X]$ tel que $\forall P\in\Rr[X],\;P|A=P(0)$~?
}
\reponse{
Soit $A$ un éventuel polynôme solution c'est à dire tel que $\forall P\in\Rr[X],\;\int_{0}^{1}P(t)A(t)\;dt=P(0)$.

$P=1$ fournit $\int_{0}^{1}A(t)\;dt=1$ et donc nécessairement $A\neq0$. $P=XA$ fournit $\int_{0}^{1}tA^2(t)\;dt=P(0)=0$.
Mais alors, $\forall t\in[0,1],\;tA^2(t)=0$ (fonction continue positive d'intégrale nulle) puis $A=0$ (polynôme ayant une infinité de racines deux à deux distinctes). $A$ n'existe pas.
}
}
