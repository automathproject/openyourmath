\uuid{b4Ju}
\exo7id{3799}
\auteur{quercia}
\organisation{exo7}
\datecreate{2010-03-11}
\isIndication{false}
\isCorrection{true}
\chapitre{Espace euclidien, espace normé}
\sousChapitre{Endomorphismes auto-adjoints}

\contenu{
\texte{
Soit~$E$ un espace euclidien, $u$ et~$v$ deux endomorphismes auto-adjoints
de~$E$, $u$ étant défini positif.
}
\begin{enumerate}
    \item \question{Montrer qu'il existe un unique endomorphisme~$w$ tel que
    $u\circ w + w\circ u = v$. Que peut-on dire de~$w$~?}
\reponse{On se place dans une base propre pour~$u$, soient
    $U,V,W$ les matrices correspondantes avec $U = \mathrm{diag}(\lambda_i)$.
    On doit donc résoudre $(\lambda_i+\lambda_j)W_{ij} = V_{ij}$ d'où
    l'existence, l'unicité et la symétrie de~$w$.}
    \item \question{On suppose~$E$ de dimension~$3$, rapporté à une base orthonormale
    dans laquelle $u$ et~$v$ ont pour matrices respectives
    $A=\begin{pmatrix}4&1&1\cr 1&4&-1\cr1&-1&4\cr\end{pmatrix}$ et
    $B=\begin{pmatrix}0&\phantom-0&-1\cr 0&0&1\cr -1&1&3\cr\end{pmatrix}$. Déterminer~$w$.}
\reponse{\begin{verbatim}
> A := matrix([[4,1,1],[1,4,-1],[1,-1,4]]); B := matrix([[0,0,-1],[0,0,1],[-1,1,3]]);
> eigenvals(A); eigenvects(A);
> P := transpose(matrix([[1, 0, 1], [1, 1, 0],[-1, 1, 1]]));
> A1 := evalm(P^(-1)&*A&*P); B1 := evalm(P^(-1)&*B&*P);
> C1 := matrix(3,3);
> for i from 1 to 3 do for j from 1 to 3 do C1[i,j] := B1[i,j]/(A1[i,i]+A1[j,j]) od od;
> C := evalm(P&*C1&*P^(-1)); evalm(A&*C+C&*A-B);
\end{verbatim}
$ \Rightarrow  C=\frac1{140}\begin{pmatrix}11&-11&-33\cr -11&11&33\cr -33&33&69\cr\end{pmatrix}$.}
    \item \question{On revient au cas général. Si~$v$ est défini positif, que dire de~$w$~?
    Si~$w$ est défini positif, que dire de~$v$~?}
\reponse{Si $v$ est défini positif~: on a $(v(x)\mid x) = 2(u(x)\mid w(x))$ donc si $\lambda$ est
    une valeur propre de~$w$ et~$x$ est un vecteur propre associé, on a
    $\lambda = \frac{(v(x)\mid x)}{2(u(x)\mid x)} > 0$ d'où $w$ est défini positif.
    
    Cas $w$ défini positif et $v$ non positif~:
    $U=\begin{pmatrix}1&0\cr0&2\cr\end{pmatrix}$, $W=\begin{pmatrix}1&1\cr1&1+x\end{pmatrix}$,
    $V=\begin{pmatrix}2&3\cr3&4x+4\cr\end{pmatrix}$ avec $0<x<\frac18$.}
\end{enumerate}
}
