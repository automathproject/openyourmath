\uuid{7KE4}
\exo7id{1329}
\auteur{hilion}
\organisation{exo7}
\datecreate{2003-10-01}
\isIndication{false}
\isCorrection{false}
\chapitre{Groupe, anneau, corps}
\sousChapitre{Groupe, sous-groupe}

\contenu{
\texte{
Soit $E$ un ensemble muni d'une loi interne $\star$. On appelle translation à droite (resp. à gauche) par $a\in E$, l'application $d_{a}$ (resp. $g_{a})$ de $E$ dans $E$ définie par $d_{a}(x)=a\star x$ (resp. $g_{a}(x)=x\star a$).
}
\begin{enumerate}
    \item \question{Montrer que dans un groupe les translations à droite et à gauche sont des bijections.}
    \item \question{Réciproquement, si la loi $\star$ de $E$ est associative, et que les translations à droite et à gauche sont des bijections, on va montrer que $(E,\star)$ est un groupe.
\begin{enumerate}}
    \item \question{Montrer que pour tout $x\in E$, il existe un unique élément $e_{x}\in E$ (resp. $f_{x}\in E$) tel que $e_{x}\star x=x$ (resp. $x\star f_{x}=x$).}
    \item \question{Si $x,y \in E$, montrer que $e_{x}=e_{y}$ (noté $e$ dorénavant) et $f_{x}=f_{y}$ (noté $f$ dorénavant).}
    \item \question{Montrer que $e=f$ (noté $e$ dorénavant).}
    \item \question{Montrer que pour tout $x\in E$, il existe un unique élément $\bar{x}\in E$ (resp. $\bar{\bar{x}}\in E$) tel que $\bar{x}\star x=e$ (resp. $x\star\bar{\bar{x}}=e$).}
    \item \question{Montrer que $\bar{x}=\bar{\bar{x}}$.}
    \item \question{Conclure.}
\end{enumerate}
}
