\uuid{vY9e}
\exo7id{5681}
\auteur{rouget}
\datecreate{2010-10-16}
\isIndication{false}
\isCorrection{true}
\chapitre{Réduction d'endomorphisme, polynôme annulateur}
\sousChapitre{Autre}

\contenu{
\texte{
Trouver $A$ dans $\mathcal{M}_n(\Rr)$ telle que la comatrice de $A$ soit $\left(
\begin{array}{cccc}
1&0&\ldots&0\\
2&\vdots& &\vdots\\
\vdots&\vdots& &\vdots\\
n&0&\ldots&0
\end{array}
\right)$.
}
\reponse{
Soit $B$ la matrice de l'énoncé. $\text{rg}B = 1$ et si $A$ existe, nécessairement $\text{rg}A = n-1$ (exercice \ref{ex:rou18bis}).

Une matrice de rang $1$ admet l'écriture générale $U{^t}V$ où $U$ et $V$ sont des vecteurs colonnes non nuls. Ici $U=\left(
\begin{array}{c}
1\\
2\\
\vdots\\
n
\end{array}
\right)$  et $V =\left(
\begin{array}{c}
1\\
0\\
\vdots\\
0
\end{array}
\right)$.

Si $A$ existe, $A$ doit déjà vérifier $A{^t}tB={^t}BA = 0$ ou encore $AV{^t}U= 0$ (1) et $V{^t}UA = 0$ (2). En multipliant les deux membres de l'égalité (1) par $U$ à droite puis en simplifiant par le réel non nul ${^t}UU=\|U\|_2^2$, on obtient $AV = 0$. Ceci montre que la première colonne de $A$ est nulle (les $n-1$ dernières devant alors former une famille libre).

De même, en multipliant les deux membres de l'égalité (2) par ${^t}V$ à gauche, on obtient ${^t}UA = 0$ et donc les colonnes de la matrice $A$ sont orthogonales à $U$ (pour le produit scalaire usuel) ce qui invite franchement à considérer la matrice $A=\left(
\begin{array}{ccccc}
0&-2&\ldots&\ldots&-n\\
0&1&0&\ldots&0\\
\vdots&\ddots&\ddots&\ddots&\vdots\\
\vdots& &\ddots&\ddots&0\\
0&\ldots&\ldots&0&1
 \end{array}\right)$ qui convient.
}
}
