\uuid{PkJ8}
\exo7id{3530}
\auteur{quercia}
\datecreate{2010-03-10}
\isIndication{false}
\isCorrection{true}
\chapitre{Réduction d'endomorphisme, polynôme annulateur}
\sousChapitre{Polynôme caractéristique, théorème de Cayley-Hamilton}

\contenu{
\texte{
On considère la matrice de $M_n(\C)$, $A=\begin{pmatrix}0      & a      & \dots  & a\cr
                                                    b      & \ddots & \ddots & \vdots \cr
                                                    \vdots & \ddots & \ddots & a\cr
                                                    b      & \dots  & b      & 0\cr\end{pmatrix}$, $a\ne b$.
}
\begin{enumerate}
    \item \question{Montrer que le polynôme caractéristique de $A$ est $\frac{(-1)^n}{a-b}(a(X+b)^n-b(X+a)^n)$.}
\reponse{$\det(M+(t))$ est une fonction affine de~$t$.}
    \item \question{Montrer qu'en général les valeurs propres de $A$ sont sur un cercle.}
\reponse{$|\lambda+a| = k|\lambda+b|$ et $\lambda=x+iy  \Rightarrow 
    (1-k^2)(x^2+y^2)+\dots = 0$, équation d'un cercle si $|a|\ne |b|$.}
\end{enumerate}
}
