\uuid{sLMt}
\exo7id{2768}
\auteur{tumpach}
\organisation{exo7}
\datecreate{2009-10-25}
\isIndication{false}
\isCorrection{true}
\chapitre{Déterminant, système linéaire}
\sousChapitre{Système linéaire, rang}

\contenu{
\texte{

}
\begin{enumerate}
    \item \question{Résoudre de quatre manières différentes le système suivant 
(par substitution, par la méthode du pivot de Gauss, 
en inversant la matrice des coefficients, par la formule de Cramer)~:
$$
\left\{\begin{array}{ccccc}2x & + & y & = & 1\\ 3x & + & 7y & = & -2  \end{array}\right.
$$}
    \item \question{Choisir la méthode qui vous para\^it la plus rapide pour résoudre, 
selon les valeurs de $a$,  les systèmes suivants~:
$$
\left\{\begin{array}{ccccc} ax & + & y & = & 2\\ (a^2 + 1)x & + & 2ay & = & 1 \end{array} \right.
\qquad 
\left\{\begin{array}{ccccc} (a + 1)x & + & (a - 1)y & = & 1\\ (a - 1)x & + & (a + 1)y & = & 1 \end{array} \right.
$$}
\reponse{
\begin{enumerate}
\textbf{Par substitution.} La première équation s'écrit aussi $y=1-2x$.
On remplace maintenant $y$ dans la deuxième équation 
$$3x+7y=-2 \implies 3x+7(1-2x)=-2 \implies 11x=9 \implies  x=\frac{9}{11}.$$
On en déduit $y$ : $y=1-2x=1-2\frac{9}{11}=-\frac{7}{11}$.
La solution de ce système est donc le couple $(\frac{9}{11},-\frac{7}{11})$.

N'oubliez pas de vérifier que votre solution fonctionne !
\textbf{Par le pivot de Gauss.} On garde la ligne $L_1$ et on remplace la ligne 
$L_2$ par $2L_2-3L_1$ :
$$
\left\{\begin{array}{ccccc}2x & + & y & = & 1\\ 3x & + & 7y & = & -2  \end{array}\right.
\iff 
\left\{\begin{array}{ccccc}2x & + & y & = & 1\\  &  & 11y & = & -7  \end{array}\right.
$$
On obtient un système triangulaire : on en déduit $y=-\frac{7}{11}$ et alors la première ligne permet d'obtenir
$x=\frac{9}{11}$.
\textbf{Par les matrices.} En terme matriciel le système s'écrit

$$AX=Y \quad \text{  avec } A = \begin{pmatrix} 2 & 1 \\ 3 & 7 \\ \end{pmatrix}
\quad X = \begin{pmatrix} x \\ y \end{pmatrix} \quad Y = \begin{pmatrix} 1 \\ -2 \end{pmatrix}$$

On trouve la solution du système en inversant la matrice :
$$X = A^{-1} Y.$$

L'inverse d'une matrice $2\times 2$ se calcule ainsi
$$\text{ si } A = \begin{pmatrix} a & b \\ c & d \\ \end{pmatrix} \quad \text{ alors } \quad
A^{-1} = \frac{1}{ad-bc}\begin{pmatrix} d & -b \\ -c & a \\ \end{pmatrix}$$
Il faut bien sûr que le déterminant $\det A = \begin{vmatrix}  a & b \\ c & d \\ \end{vmatrix}=ad-bc$ 
soit différent de $0$.

Ici on trouve 
$$A^{-1} = \frac{1}{11}\begin{pmatrix} 7 & -1 \\ -3 & 2 \\ \end{pmatrix} \quad \text{ et } \quad
X = A^{-1} \begin{pmatrix} 1 \\ -2 \end{pmatrix} =  \frac{1}{11}\begin{pmatrix} 9 \\ -7 \end{pmatrix}$$
\textbf{Par les formules de Cramer.}
Les formules de Cramer pour un système de deux équations sont les suivantes si le déterminant vérifie $ad-bc\neq 0$ :
$$
\left\{\begin{array}{ccccc}ax & + & by & = & e\\ cx & + & dy & = & f  \end{array}\right.
\implies x = \frac{\begin{vmatrix}  e & b \\ f & d\end{vmatrix}}{\begin{vmatrix}  a & b \\ c & d \end{vmatrix}}
\quad \text{ et } \quad y = \frac{\begin{vmatrix}  a & e \\ c & f\end{vmatrix}}{\begin{vmatrix}  a & b \\ c & d \end{vmatrix}}$$

Ce qui donne ici :
$$x = \frac{\begin{vmatrix}  1 & 1 \\ -2 & 7\end{vmatrix}}{\begin{vmatrix}  2 & 1 \\ 3 & 7 \end{vmatrix}} = \frac{9}{11}
 \quad \text{ et } \quad y = \frac{\begin{vmatrix}  2 & 1 \\ 3 & -2 \end{vmatrix}}{\begin{vmatrix}  2 & 1 \\ 3 & 7 \end{vmatrix}} = -\frac{7}{11}$$
}
\end{enumerate}
}
