\uuid{At0Y}
\exo7id{3379}
\auteur{quercia}
\organisation{exo7}
\datecreate{2010-03-09}
\isIndication{false}
\isCorrection{true}
\chapitre{Réduction d'endomorphisme, polynôme annulateur}
\sousChapitre{Valeur propre, vecteur propre}

\contenu{
\texte{
Soient $A \in \mathcal{M}_{n,p}(K)$ et $B \in \mathcal{M}_{p,n}(K)$. On note $C = I_n - AB$ et
$D = I_p - BA$.\par
($I_n,I_p = $ matrices unité d'ordres $n$ et $p$)
}
\begin{enumerate}
    \item \question{Montrer que si $C$ est inversible, alors $D$ l'est aussi (résoudre $DX = 0$).}
    \item \question{Le cas échéant, exprimer $D^{-1}$ en fonction de $A,B,C^{-1}$.}
    \item \question{En déduire que $AB$ et $BA$ ont les mêmes valeurs propres non nulles.
    Examiner le cas de la valeur propre $0$ si $n = p$.}
\reponse{
2. $D^{-1} = BC^{-1}A + I_p$.
}
\end{enumerate}
}
