\uuid{goe5}
\exo7id{3635}
\auteur{quercia}
\organisation{exo7}
\datecreate{2010-03-10}
\isIndication{false}
\isCorrection{false}
\chapitre{Endomorphisme particulier}
\sousChapitre{Autre}

\contenu{
\texte{
Soit $E =  K_n[X]$, $Q \in E$ de degré $n$ et $Q_i = Q(X+i)$ ($0 \le i \le n$).
}
\begin{enumerate}
    \item \question{Montrer que $(Q,Q',Q'',\dots,Q^{(n)})$ est libre.}
    \item \question{Montrer que toute forme linéaire sur $E$ peut se mettre sous la forme :\par
    $f : P  \mapsto \alpha_0 P(0) + \alpha_1 P'(0) + \dots + \alpha_nP^{(n)}(0)$.}
    \item \question{Soit $f \in E^*$ telle que $f(Q_0) = \dots = f(Q_n) = 0$.
    Montrer que $f=0$.\par
    (considérer le polynôme $P = \alpha_0Q + \dots + \alpha_nQ^{(n)}$)}
    \item \question{Montrer que $(Q_0,\dots,Q_n)$ est une base de $E$.}
\end{enumerate}
}
