\uuid{G0ZW}
\exo7id{5678}
\auteur{rouget}
\organisation{exo7}
\datecreate{2010-10-16}
\isIndication{false}
\isCorrection{true}
\chapitre{Réduction d'endomorphisme, polynôme annulateur}
\sousChapitre{Polynôme caractéristique, théorème de Cayley-Hamilton}

\contenu{
\texte{
Soient $A$ et $B$ deux matrices carrées complexes de format $n$.
Montrer que $A$ et $B$ n'ont pas de valeurs propres communes si et seulement si la matrice $\chi_A(B)$ est inversible.
}
\reponse{
Soit $(\lambda_1,...,\lambda_n)$ la famille des valeurs propres de $A$. On a donc $\chi_A=(\lambda_1-X)\ldots(\lambda_n-X)$.

\begin{align*}\ensuremath
\chi_A(B)\;\text{inversible}&\Leftrightarrow(B-\lambda_1I)...(B-\lambda_nI)\;\text{inversible}\\ 
&\Leftrightarrow\forall k\in\llbracket1,n\rrbracket,\;B-\lambda_kI\;\text{inversible}\;(\text{car}\;\text{det}((B-\lambda_1I)...(B-\lambda_nI))=\text{det}(B-\lambda_1I)\times...\times\text{det}(B-\lambda_nI))\\
 &\Leftrightarrow\forall k\in\llbracket1,n\rrbracket,\;\lambda_k\;\text{n'est pas valeur propre de}\;B\\
 &\Leftrightarrow\text{Sp}A\cap\text{Sp}B =\varnothing.
\end{align*}
}
}
