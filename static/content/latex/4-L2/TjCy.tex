\uuid{TjCy}
\exo7id{1304}
\auteur{cousquer}
\datecreate{2003-10-01}
\isIndication{false}
\isCorrection{false}
\chapitre{Groupe, anneau, corps}
\sousChapitre{Groupe, sous-groupe}

\contenu{
\texte{
Étant donné un entier naturel $n$, on appelle classe d'un entier relatif $p$ modulo
$n$ l'ensemble $\overline p = \{p+kn\mid k\in\mathbb{Z}\}$.
L'ensemble des classes modulo
$n$ est noté $\mathbb{Z}_n$.
}
\begin{enumerate}
    \item \question{Écrire la liste des éléments distincts de $\mathbb{Z}_2$, $\mathbb{Z}_3$, $\mathbb{Z}_4$
et~$\mathbb{Z}_5$.}
    \item \question{Montrer que si $x\in\overline p$ et $y\in\overline q$, alors $x+y\in
\overline{p+q}$ et $xy\in\overline{pq}$.}
    \item \question{En posant $\overline p+\overline q=\overline{p+q}$ et $\overline
p\cdot\overline q=\overline{pq}$, on définit deux lois de composition,
addition et multiplication sur $\mathbb{Z}_n$.\\
Écrire la table d'addition et de multiplication de $\mathbb{Z}_4$.}
\end{enumerate}
}
