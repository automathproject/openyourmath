\uuid{ZiSn}
\exo7id{1414}
\auteur{bodin}
\datecreate{1999-11-01}
\isIndication{false}
\isCorrection{false}
\chapitre{Groupe, anneau, corps}
\sousChapitre{Groupe de permutation}

\contenu{
\texte{
\hfil I \\

Soit $(G,\cdot)$ un groupe. On d\'efinit le centre
$\mathcal{Z}(G)$ de $G$ par :
$$\mathcal{Z}(G) = \big\{ x \in G\  /\  \forall a \in G \ \ ax=xa \big\}.$$

Montrer que $\mathcal{Z}(G)$ est un sous-groupe de $G$.

Que peut-on dire de $\mathcal{Z}(G)$ si $G$ est ab\'elien ?


\bigskip

\hfil II \\

On d\'esigne par $\mathcal{A}_n$ le groupe altern\'e d'ordre $n$
(rappel : c'est le sous-groupe de
$(\mathcal{S}_n,\circ)$ form\'e des permutations
de $E_n = \{1,2,\ldots,n\}$
de signature $+1$.)

On se propose de d\'eterminer le centre de
$\mathcal{A}_n$ pour $n\geq 3$.
}
\begin{enumerate}
    \item \question{Donner la liste des \'el\'ements de $\mathcal{A}_3$ et de  $\mathcal{Z}(\mathcal{A}_3)$.}
    \item \question{On suppose d\'esormais $n\geq 4$.
        Dans cette question on fixe $i,j,k$ trois \'el\'ements
distincts de $E_n$.

  \begin{enumerate}}
    \item \question{V\'erifier que le $3$-cycle $(i,j,k)$ est dans $\mathcal{A}_n$.}
    \item \question{Soit $s\in \mathcal{S}_n$, montrer que
$s\circ(i,j,k) = (s(i),s(j),s(k)) \circ s$.}
    \item \question{En d\'eduire que si $s \in \mathcal{Z}(\mathcal{A}_n)$
alors l'image de  $\{i,j,k\}$ par $s$ est $\{i,j,k\}$.}
\end{enumerate}
}
