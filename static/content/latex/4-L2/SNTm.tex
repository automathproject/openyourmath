\uuid{SNTm}
\exo7id{1669}
\auteur{roussel}
\organisation{exo7}
\datecreate{2001-09-01}
\isIndication{false}
\isCorrection{false}
\chapitre{Réduction d'endomorphisme, polynôme annulateur}
\sousChapitre{Diagonalisation}

\contenu{
\texte{
On d\'esigne par $E$ l'espace vectoriel des polyn\^ome s \`a coefficients r\'eels,
et par $E_n$, le sous-espace des polyn\^ome s de degr\'e au plus $n$.
}
\begin{enumerate}
    \item \question{Montrer que pour tout $x$ dans $\mathbb{R}, ~~
\Delta P(x)=(x+1)P'(x)+2P(x)$ d\'efinit une application lin\'eaire de $E$ dans
$E$. Quel est le degr\'e de $\Delta P$ lorsque $P$ appartient \`a $E_n$?}
    \item \question{On consid\`ere $\Delta_2$, la restriction de $\Delta $ au sous-espace
$E_2$. D\'eterminer les valeurs propres ~de $\Delta_2$. L'endomorphisme ~$\Delta_2$
est-il diagonalisable ? Est-ce que $\Delta_2$ est un isomorphisme ?}
    \item \question{En utilisant la d\'efinition des valeurs propres, calculer les valeurs propres ~et les
polyn\^ome s propres de $\Delta $.}
\end{enumerate}
}
