\uuid{I4wJ}
\exo7id{3650}
\auteur{quercia}
\datecreate{2010-03-10}
\isIndication{false}
\isCorrection{true}
\chapitre{Endomorphisme particulier}
\sousChapitre{Autre}

\contenu{
\texte{
Soit $E$ un $ K$-ev de dimension finie $n$,
${\cal B} = ({\vec e}_1,\dots,{\vec e}_n)$ une base de $E$, et
${\cal B}' = ({\vec e}_1{}',\dots,{\vec e}_n{}')$ déduite de $\cal B$ par une
opération élémentaire (échange de deux vecteurs, multiplication d'un vecteur
par un scalaire non nul, addition à un vecteur d'un multiple d'un autre).

\'Etudier comment on passe de la base duale ${\cal B}^*$ à ${\cal B}'^*$
en fonction de l'opération effectuée.
}
\reponse{
\leavevmode\hbox{\vtop{\halign{&$#$\hfil\quad\cr
{\vec e}_i \leftrightarrow {\vec e}_j                   : &e_i^* \leftrightarrow e_j^*. \cr
{\vec e}_i \leftarrow \alpha{\vec e}_i              : &e_i^* \leftarrow e_i^*/\alpha. \cr
{\vec e}_i \leftarrow {\vec e_i} + \alpha{\vec e}_j : &e_j^* \leftarrow e_j^* - \alpha e_i^*. \cr
}}}
}
}
