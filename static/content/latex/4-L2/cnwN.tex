\uuid{cnwN}
\exo7id{3800}
\auteur{quercia}
\datecreate{2010-03-11}
\isIndication{false}
\isCorrection{true}
\chapitre{Espace euclidien, espace normé}
\sousChapitre{Endomorphismes auto-adjoints}

\contenu{
\texte{
Soit $E$ un espace euclidien et $s$ une symétrie de~$E$.
}
\begin{enumerate}
    \item \question{Que dire de $s^*\circ s$~?}
\reponse{c'est un endomorphisme autoadjoint positif de déterminant~$1$.}
    \item \question{Un polynôme $P$ est dit réciproque si $P(X) = X^nP\Bigl(\frac1X\Bigr)$,
    pour $P$ de degré~$n$.\par
    Montrer que~:
    $P(X) = \det(X\mathrm{id}+s^*\circ s)$ est un polynôme réciproque.}
\reponse{$X^n\det\Bigl(\frac{\mathrm{id}}X+s^*\circ s\Bigr) = \det(\mathrm{id}+Xs^*\circ s)
             =\det(s^*\circ(\mathrm{id}+Xs^*\circ s)\circ s) = \det(s^*\circ s + X\mathrm{id})$.}
    \item \question{Montrer que $P(1)\ge 2^n$. A quelle condition y a-til égalité~? Y a-t-il des
    conditions sur~$s$~?}
\reponse{$s^*\circ s$ est diagonalisable avec des valeurs propres $(\lambda_i)$
    réelles positives deux à deux inverses pour la même multiplicité.
    $P^2(1) = \prod_{1\le i\le n}(1+\lambda_i)\Bigl(1+\frac1{\lambda_i}\Bigr)$
    et $(1+x)\Bigl(1+\frac1x\Bigr)\ge 4$ pour tout $x>0$ avec égalité ssi
    $x=1$.
    
    Si $P(1) = 2^n$ alors toutes les valeurs propres de~$s^*\circ s$ valent~$1$
    et $s^*\circ s$ est diagonalisable donc $s^*\circ s = \mathrm{id}$ et $s$ est une
    symétrie orthogonale. La réciproque est immédiate.}
    \item \question{Soit la matrice $A = \begin{pmatrix}A_1&A_2\cr A_3&A_4\cr\end{pmatrix}$, carrée, d'ordre~$n$,
    symétrique définie positive, où $A_1$ et $A_4$ sont carrées d'ordres respectifs
    $p$ et~$q$. Montrer que $\det(A)\le \det(A_1)\det(A_4)$.}
\reponse{Se ramener au cas $A_4=I$ puis calculer $\det A$ par pivotage.}
\end{enumerate}
}
