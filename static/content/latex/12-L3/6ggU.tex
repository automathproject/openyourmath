\uuid{6ggU}
\exo7id{6288}
\auteur{mayer}
\organisation{exo7}
\datecreate{2011-10-16}
\isIndication{false}
\isCorrection{false}
\chapitre{Différentielle d'ordre supérieur, formule de Taylor}
\sousChapitre{Différentielle d'ordre supérieur, formule de Taylor}

\contenu{
\texte{
Soient $E$ et $F$ des espaces de Banach et $f: E\to F$ une
application de classe $C^2$.
}
\begin{enumerate}
    \item \question{Soit $h\in E$ et $\varphi _h : E\to F$ l'application définie
par $\varphi_h (x) = Df(x) (h)$. Justifier que
$$D^2 f(a) (k,h) = D\varphi _h(a) (k) \quad \text{pour tout} \;\; k\in E\;
.$$}
    \item \question{Supposons que, pour tous $t\in \Rr$ et $x\in E$, $f(tx) =
t^2 f(x)$. Montrer que $D^2f(0) (x,x) =2f(x)$ pour tout $x\in E$.}
    \item \question{Soit $a,h,k\in E$ et soit $\Psi : \Rr ^2 \to F$ définie
par $\Psi (t,s) = f(a+th+sk)$. Calculer $\frac{\partial ^2
\Psi}{\partial t \partial s}(0,0)$.}
\end{enumerate}
}
