\uuid{LIQc}
\exo7id{6300}
\auteur{queffelec}
\datecreate{2011-10-16}
\isIndication{false}
\isCorrection{false}
\chapitre{Autre}
\sousChapitre{Autre}

\contenu{
\texte{
Une fonction $f$ de classe $C^4$ (par exemple à 2
variables) est dite \emph{biharmonique} si $$\Delta (\Delta f)
= {\partial^4 f\over \partial x^4} + 2 {\partial^4 f\over
\partial x^2 \partial y^2} + {\partial^4 f\over \partial
y^4} \equiv 0.$$
Ces fonctions interviennent en théorie de
l'Elasticité. Bien entendu toute fonction harmonique est
biharmonique. Montrez que, si $f$ et $g$ sont deux
fonctions harmoniques, alors la fonction $xf + (x^2 +
y^2)g$ est biharmonique.
}
}
