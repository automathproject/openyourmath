\uuid{MVRI}
\exo7id{2504}
\auteur{queffelec}
\datecreate{2009-04-01}
\isIndication{false}
\isCorrection{false}
\chapitre{Différentiabilité, calcul de différentielles}
\sousChapitre{Différentiabilité, calcul de différentielles}

\contenu{
\texte{

}
\begin{enumerate}
    \item \question{Soit $f$ une application de $E$ dans $F$ espaces vectoriels
norm\'es et supposons $f$ diff\'eren\-tiable en $a$; montrer que
pour tout vecteur $u\in E^*$, la d\'eriv\'ee de $f$ en $a$ dans la
direction $u$ existe , i.e. $\lim_{h\to 0} {\frac 1 h}\big(f(a+hu)-f(a)\big)$ et l'exprimer \`a l'aide de $f'(a)$.}
    \item \question{On consid\`ere $f:{\Rr^2}\to{\Rr}$ d\'efinie par $f(0,0)=0$
et, si $(x,y)\neq(0,0)$,\ $f(x,y)={\frac{x^3y}{x^4+y^2}}$. Montrer
que $f$ est d\'erivable en $(0,0)$ dans toutes les directions,
mais que $f$ n'est pas diff\'erentiable en $(0,0)$.}
\end{enumerate}
}
