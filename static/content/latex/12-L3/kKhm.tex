\uuid{kKhm}
\exo7id{2495}
\auteur{sarkis}
\datecreate{2009-04-01}
\isIndication{false}
\isCorrection{true}
\chapitre{Différentiabilité, calcul de différentielles}
\sousChapitre{Différentiabilité, calcul de différentielles}

\contenu{
\texte{
D\'ecrire la boule de centre
l'origine et de rayon $1$ dans les espaces suivants:
}
\begin{enumerate}
    \item \question{$\mathbb{R}$ muni de la distance $d(x,y)=|x-y|.$}
\reponse{On a par définition $B(0,1)=\{x \in \mathbb{R}; |x-0|=|x| <
1\}=[-1,1].$}
    \item \question{$\mathbb{R}^2$ muni de la distance
$d_1((x_1,x_2),(y_1,y_2))=\sqrt{(x_1-y_1)^2+(x_2-y_2)^2}.$}
\reponse{C'est la norme euclidienne sur $\mathbb{R}^2$,
$B_1(0,1)=\{(x,y)\in \mathbb{R}^2; \sqrt{x^2+y^2}=1\}$ c'est le
disque de centre l'origine et de rayon $1$.}
    \item \question{$\mathbb{R}^2$ muni de la distance $d_2((x_1,x_2),(y_1,y_2))=\sup
(|x_1-y_1|,|x_2-y_2|).$}
\reponse{$B_2(0,1)=\{(x,y); |x|<1 et |y|<1 \}.$ C'est un carr\'e.}
    \item \question{$\mathbb{R}^2$ muni de la distance
$d_3((x_1,x_2),(y_1,y_2))=|x_1-y_1|+|x_2-y_2|.$}
\reponse{$B_3(0,1)=\{(x,y); |x|+|y|<1\}.$ Dans le quart de plan
$P^{++}=\{(x,y); x\geq 0, y\geq $, on a $B_3(0,1) \cap
P^{++}=\{(x,y) \in p^{++}; x+y<1 \}$ c'est le triangle
d\'elimit\'e par les droites $x=0, y=0$ et $x+y=1$. En faisant de
même pour les 3 autres secteurs du plan, on trouve que $B_3(0,1)$
est un losange (ou carr\'e) dont les sommets sont les points
$(0,1), (1,0), (-1,0), (0,-1).$}
\end{enumerate}
}
