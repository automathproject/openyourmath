\uuid{tPHB}
\titre{ Calcul d'une somme de série entière}
\theme {séries entières}
\auteur{ }
\organisation{ AMSCC }

\contenu{
	\texte{ Soit la série entière $\displaystyle \sum_{n\geq 2} \frac{(-1)^n}{n(n-1)}x^n$. }
\begin{enumerate}
	\item \question{ Déterminer le rayon de convergence $R$ de cette série. Préciser son intervalle de convergence et le comportement de la série aux extrémités de cet intervalle. }
	\reponse{
		$R=1$ et $D=[-1;1]$
	}
	
	\item \question{ Déterminer le rayon de convergence de la série $\displaystyle \sum_{n\geq 2} \frac{(-1)^n}{(n-1)}x^{n-1}$ 
	et de la série $\displaystyle \sum_{n\geq 2}(-1)^n x^{n-2}$ ? \\
	Étudier le comportement de chacune de ces séries au bord de son intervalle de convergence. }
	\reponse{ 
		Pour la première, $R=1$ et $D=]-1;1]$. \\
		Pour la deuxième, $R=1$ et $D=]-1;1[$.
	}
	
	\item\question{  Calculer la somme de la série $\displaystyle \sum_{n\geq 2}(-1)^n x^{n-2}$.  }
	\reponse{ 
		\[ \forall x \in ]-1;1[, \qquad \sum_{n=2}^{+\infty}(-1)^n x^{n-2}
		=\frac{1}{1+x}.
		\]
	}
	
	\item \question{ En déduire la somme de la série $\displaystyle \sum_{n\geq 2} \frac{(-1)^n}{(n-1)}x^{n-1}$. } %et de $\displaystyle \sum_{n\geq 2} \frac{(-1)^n}{n(n-1)}x^n$.
	\reponse{ 
		\[ \forall x \in ]-1;1[, \qquad \sum_{n=2}^{+\infty} \frac{(-1)^n}{(n-1)}x^{n-1}
		=\ln(1+x).
		\]
	}
\end{enumerate}
}