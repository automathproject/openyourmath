\uuid{CLjr}
\titre{Multiplicité d'une racine et factorisation d'un polynôme}
\theme{polynomes}
\auteur{}
\organisation{AMSCC}
\contenu{


\texte{ Déterminer l'ordre de multiplicité de la racine $x_0$ du polynôme $P$, et en déduire la factorisation du polynôme, dans les cas suivants : }

\begin{enumerate}
	\item  \question{ $P(X)=X^4-X^3-3 X^2+5 X-2 \quad x_0=1$.}
\reponse{
$$
P(X)=X^4-X^3-3 X^2+5 X-2 .
$$ 

 On a :
$$
\begin{aligned}
P(1) & =1-1-3+5-2=0 \\
P^{\prime}(X) & =4 X^3-3 X^2-6 X+5 \\
P^{\prime}(1) & =4-3-6+5=0 \\
P^{\prime \prime}(X) & =12 X^2-6 X-6 \\
P^{\prime \prime}(1) & =12-6-6=0 \\
P^{(3)}(X) & =24 X-6 \\
P^{(3)}(1) & \neq 0
\end{aligned}
$$
Ainsi $x_0=1$ est racine d'ordre 3 de $P$ et :
$$
\begin{gathered}
P(X)=(X-1)^3 \cdot(X-a)=\left(X^3-3 X^2+3 X-1\right) \cdot(X-a)=X^4 \ldots+a \\
=X^4-X^3-3 X^2+5 X-2 \\
\quad \Rightarrow a=-2 \\
P(X)=(X-1)^3 \cdot(X+2)
\end{gathered}
$$ }

\item \question{ $P(X)=X^3-iX^2+X-i \quad x_0=i$. }
\reponse{ $$
P(X)=X^3-i . X^2+X-i .
$$

On a :
$$
\begin{aligned}
P(i) & =i^3-i . i^2+i-i=0 \\
P^{\prime}(X) & =3 X^2-2 i X+1 \\
P^{\prime}(i) & =3 \cdot i^2-2 . i . i+1=0 \\
P^{\prime \prime}(X) & =6 X-2 i \\
P^{\prime \prime}(i) & =6 i-2 i \neq 0
\end{aligned}
$$
Ainsi $x_0=1$ est racine double de $P$ et :
$$
\begin{aligned}
P(X)= & (X-i)^2 \cdot(X-a)=\left(X^2-2 i X-1\right) \cdot(X-a)=X^3 \ldots+a \\
= & X^3-i X^2+X-i \\
& \Rightarrow a=-i
\end{aligned}
$$
d'où 
$$P(X) = (X-i)^2(X+i)$$ }

\end{enumerate}}
