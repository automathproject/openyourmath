\uuid{7ICd}
\exo7id{7865}
\auteur{mourougane}
\organisation{exo7}
\datecreate{2021-08-11}
\isIndication{false}
\isCorrection{false}
\chapitre{Théorème de Sylow}
\sousChapitre{Théorème de Sylow}

\contenu{
\texte{
Soit $p$ un nombre premier impair, on se propose de décrire les groupes d'ordre $p^2$ à isomorphisme près.
}
\begin{enumerate}
    \item \question{Soit $G$ un groupe de cardinal $p^2$, montrer que, ou bien $G$ est cyclique ou bien
tous les éléments différents de l’élément neutre sont d’ordre $p$.}
    \item \question{Soit $G$ un groupe non cyclique d’ordre $p^2$, soit $K$ un sous-groupe d’ordre $p$, montrer que $K$ est distingué dans $G$ et qu’il existe $H$ sous-groupe d’ordre $p$ tel que $K\cap H = \{e\}$. En déduire que $G$ est ismorphe à un produit semi-direct de $\Z/p\Z$
par $\Z/p\Z$.}
    \item \question{Montrer que tout groupe de cardinal $p^2$ est isomorphe à $\Z/p^2\Z$ ou $\Z/p\Z \times \Z/p\Z$.}
\end{enumerate}
}
