\uuid{xZGp}
\exo7id{7761}
\auteur{mourougane}
\organisation{exo7}
\datecreate{2021-08-11}
\isIndication{false}
\isCorrection{false}
\chapitre{Action de groupe}
\sousChapitre{Action de groupe}

\contenu{
\texte{

}
\begin{enumerate}
    \item \question{Soit $p$ un nombre premier. Montrer que le centre d'un $p$-groupe 
$G$,( i.e. un groupe fini d'ordre une
puissance non nulle de $p$), n'est pas réduit à l'élément neutre.}
    \item \question{Montrer que si le quotient d'un groupe par son centre est
 cyclique alors le groupe est abélien, donc égal à son centre.}
    \item \question{Montrer qu'un groupe d'ordre $p^2$ est abélien.}
    \item \question{Montrer que le centre d'un groupe non abélien d'ordre $p^3$ est d'ordre $p$. En déduire que le nombre de classes de conjugaison est $p^2+p-1$. (On pourra étudier l'action de $G$ sur lui-même par
conjugaison~: ses points fixes, l'orbite des éléments, le
stabilisateur des éléments...)}
\end{enumerate}
}
