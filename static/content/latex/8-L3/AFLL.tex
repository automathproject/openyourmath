\uuid{AFLL}
\exo7id{7740}
\auteur{mourougane}
\datecreate{2021-08-11}
\isIndication{false}
\isCorrection{true}
\chapitre{Théorème de Sylow}
\sousChapitre{Théorème de Sylow}

\contenu{
\texte{
Soit $\mathcal{P}_n$ un polygone régulier à $n$ côtés dans le plan euclidien orienté.
On appelle groupe diédral $D_{n}$ le groupe des isométries de $\mathcal{P}_n$.
}
\begin{enumerate}
    \item \question{Parmi les translations, les rotations, les symétries orthogonales, et les symétries glissées, décrire des isométries du plan qui conservent le polygone régulier $\mathcal{P}_n$.}
    \item \question{Déterminer, à l'aide de l'action naturelle de $D_{n}$ sur l'ensemble des sommets de $\mathcal{P}_n$, le cardinal de $D_{n}$. En déduire la liste complète des éléments de $D_{n}$.}
    \item \question{On suppose $n$ impair. Déterminer les $2$-Sylow de $D_{n}$ et vérifier (sans référence au cours) qu'ils sont conjugués.}
    \item \question{On suppose $n=6$. Déterminer un $2$-Sylow de $D_{6}$.
Ce $2$-Sylow est-il distingué ?
Déterminer deux sous-groupes d'ordre $2$ de $D_6$ non conjugués dans $D_6$.
Donner un $3$-Sylow de $D_{6}$.}
\reponse{
On note $O$ le centre de gravité de $\mathcal{P}_n$.
  Dans $D_{n}$, toutes les isométries, qui sont affines, conservent le centre de gravité. Il n'y a donc aucune translation ni symétrie glissée. Il y a les $n$ rotations de centre $O$ et d'angle $2k\pi/n$ avec $k = 0, \cdots, n-1$. Si $n$ est pair, il y a les $n/2$ symétries orthogonales d'axe $(OS)$ pour chaque sommet $S$ et les $n/2$ symétries d'axe médiateur d'un des $n$ côtés. Si $n$ est impair, il y a les $n$ symétries d'axe médiateur d'un des $n$ côtés.
On considère l'action de $D_{n}$ sur l'ensemble fini des $n$ sommets de $\mathcal{P}_n$.
 \`A l'aide des rotations de centre $O$ et d'angle $2k\pi/n$, on montre que l'action est transitive.
 Le stabilisateur d'un sommet $S$ est le sous-groupe d'ordre $2$ engendré par la symétrie orthogonale d'axe $(OS)$ où $O$ est le centre du polygone. Par la seconde formule des classes, $D_{n}$ est d'ordre $2n$.
 La liste précédente est donc complète.
Comme $n$ est impair, les $2$-Sylow sont d'ordre $2$. Ce sont les sous-groupes engendrés par les symétries orthogonales.
 Si $s$ est une symétrie d'axe $d$ et $r$ la rotation de centre $O$ et d'angle $2k\pi/n$, 
 $rsr^{-1}$ est la symétrie d'axe $r(d)$.
 Comme $n$ est impair, le groupe engendré par $r$ agit transitivement sur les axes de symétrie. Toutes les symétries sont donc conjuguées.
Les $2$-Sylow sont d'ordre $4$ dans $D_{6}$ d'ordre $2^2\times 3$.
 Le groupe engendré par deux symétries d'axe orthogonal (qui commutent) est d'ordre $4$. En conjuguant par la rotation de centre $O$ et d'angle $2\pi/6$, on obtient un autre $2$-Sylow.
 Les $2$-Sylow ne sont donc pas distingués. Le nombre de $2$-Sylow est congru à $1$ modulo $2$ et divise $12$ et n'est pas $1$.
 Il y a donc trois $2$-Sylow.
La conjuguée, par une rotation d'angle $2k\pi/6$ ou par une symétrie, d'une symétrie d'axe médiateur d'un coté est une symétrie d'axe médiateur d'un coté. 
 Les symétries d'axe médiateur d'un coté et les symétries d'axe passant par un sommet (qui diffèrent car $6$ est pair) ne sont donc pas conjuguées dans $D_{6}$.
Le sous-groupe engendré par une rotation d'angle $4\pi/6$ est d'ordre $3$ et c'est un $3$-Sylow.
}
\end{enumerate}
}
