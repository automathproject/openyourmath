\uuid{IaVr}
\exo7id{2168}
\auteur{debes}
\organisation{exo7}
\datecreate{2008-02-12}
\isIndication{true}
\isCorrection{false}
\chapitre{Action de groupe}
\sousChapitre{Action de groupe}

\contenu{
\texte{
\label{ex:deb68}
On appelle cycle une permutation $\sigma $ v\'erifiant la propri\'et\'e
suivante:  il existe une partition de $\{ 1, \dots , n \} $ en deux sous-ensembles
$I$ et $J$ tels que la restriction de $\sigma $ \`a $I$ est l'identit\'e
de $I$
et il existe $a\in J$ tel que $J= \{ a, \sigma (a) , \dots , \sigma ^{r-1}
(a)\} $ o\`u $r$ est le cardinal de $J$. Le sous-ensemble $J$ est appel\'e le
support du cycle $\sigma $.

Un tel cycle sera not\'e $(a, \sigma (a), \dots , \sigma ^{r-1} (a) )$
\smallskip

(a) Soit
$\sigma
\in S_n$ une permutation. On consid\`ere le sous-groupe
$C$ engendr\'e par
$\sigma$ dans
$S_n$. Montrer que la restriction de $\sigma$ \`a chacune des orbites de
$\{1, \dots ,n\} $ sous
l'action de $C $ est un cycle, que ces diff\'erents cycles commutent entre
eux, et que $\sigma $
est le produit de ces cycles.
\smallskip

(b) D\'ecomposer en cycles les permutations suivantes de $\{ 1, \dots , 7\} $ :

$\hskip 20mm 1\hskip 2mm 2\hskip 2mm 3\hskip 2mm 4\hskip 2mm 5\hskip 2mm 6\hskip 2mm 7 \hskip
20mm  1\hskip 2mm 2\hskip 2mm 3\hskip 2mm 4\hskip 2mm 5\hskip 2mm 6\hskip 2mm 7\hskip 20mm   
1\hskip 2mm 2\hskip 2mm 3\hskip 2mm 4\hskip 2mm 5\hskip 2mm 6\hskip 2mm 7$

$\hskip 20mm 3\hskip 2mm 6\hskip 2mm 7\hskip 2mm 2\hskip 2mm 1\hskip 2mm 4\hskip 2mm 5 \hskip
20mm  7\hskip 2mm 4\hskip 2mm 2\hskip 2mm 3\hskip 2mm 5\hskip 2mm 6\hskip 2mm 1\hskip 20mm    
1\hskip 2mm 3\hskip 2mm 7\hskip 2mm 2\hskip 2mm 4\hskip 2mm 5\hskip 2mm 6$

\smallskip
(c) Montrer que si $\sigma $ est un cycle, $\sigma =(a, \sigma (a), \dots ,
\sigma^{r-1}(a)) $, la
conjugu\'ee $\tau \sigma \tau ^{-1} $ est un cycle et que
$\tau \sigma \tau ^{-1} =(\tau (a), \tau (\sigma  (a)), \dots , \tau
(\sigma ^{r-1} (a)))$.
\smallskip

(d) D\'eterminer toutes les classes de conjugaison des permutations dans $S_5$
(on consid\'erera leur d\'ecomposition en cycles). D\'eterminer tous les
sous-groupes distingu\'es de $S_5$.
}
\indication{(a) est une simple v\'erification.
\smallskip

(b) Les trois permutations s'\'ecrivent respectivement $(1\hskip 2pt 3\hskip 2pt 7\hskip 2pt
5)\hskip 2pt (2\hskip 2pt 6\hskip 2pt 4)$, $(1\hskip 2pt 7)\hskip 2pt (2\hskip 2pt 4\hskip
2pt 3)$ et $(2\hskip 2pt 3\hskip 2pt 7\hskip 2pt 6\hskip 2pt 5\hskip 2pt 4)$.
\smallskip

(c) est une simple v\'erification.
\smallskip

(d) {\bf Rappel:} De fa\c con g\'en\'erale, on dit qu'une permutation $\omega \in S_n$ est de
type $1^{r_1}$-$2^{r_2}$-$\cdots$-$d^{r_d}$ o\`u $d, r_1,\ldots,r_d$ sont des entiers $\geq 0$
tels que $r_1+\cdots+r_d=n$, si dans la d\'ecomposition de $\omega$ en cycles \`a support
disjoints, figurent $r_1$ $1$-cycles (ou points fixes), $r_2$ $2$-cycles, ... et $r_d$
$d$-cycles. En utilisant la question (c), il n'est pas difficile de montrer que deux
permutations sont conjugu\'ees dans $S_n$ si et seulement si elles sont de m\^eme type.    
Les classes de conjugaison de $S_n$ correspondent donc exactement \`a tous les types
possibles.

\smallskip
On obtient ainsi facilement les classes de conjugaison de $S_5$. Soit maintenant $H$ un
sous-groupe distingu\'e non trivial de $S_5$. D\`es que $H$ contient un \'el\'ement de $S_5$,
il contient sa classe de conjugaison; $H$ est donc une r\'eunion de classes de conjugaison.
En consid\'erant toutes les classes possibles que peut contenir $H$, on montre que $H=A_5$ ou
$H=S_5$. Par exemple, si $H$ contient la classe 1-2-2, alors $H$ contient $(1\hskip 2pt
2)\hskip 2pt (3\hskip 2pt 4) \times (1\hskip 2pt 3)\hskip 2pt (2\hskip 2pt 5) = (1\hskip 2pt
4\hskip 2pt 3\hskip 2pt 2\hskip 2pt 5)$ et donc la classe des $5$-cycles. D'apr\`es
l'exercice \ref{ex:deb67}, $H$ contient alors $A_5$. Le groupe $H$ est donc $A_5$ ou $S_5$. Les autres cas
sont similaires.}
}
