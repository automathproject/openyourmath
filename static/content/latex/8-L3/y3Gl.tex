\uuid{y3Gl}
\exo7id{2135}
\auteur{debes}
\datecreate{2008-02-12}
\isIndication{true}
\isCorrection{true}
\chapitre{Ordre d'un élément}
\sousChapitre{Ordre d'un élément}

\contenu{
\texte{
Montrer que tout entier $n>0$ divise toujours $\varphi(2^n-1)$ (o\`u $\varphi$ est la fonction indicatrice d'Euler).
}
\indication{Trouver l'ordre de $2$ modulo $2^n-1$.}
\reponse{
Comme $2^n \equiv 1$ modulo $2^n-1$, l'ordre de $2$ modulo $2^n-1$, disons $m$, divise $n$. Si $m<n$, on aurait $2^m \equiv 1$ modulo $2^n-1$, c'est-\`a-dire $2^n-1$ divise $2^m-1$, ce qui n'est pas possible. L'ordre de $2$ modulo $2^n-1$ est donc $n$, et celui-ci doit diviser l'ordre de $(\Zz/(2^n-1)\Zz)^\times$, qui vaut $\varphi(2^n-1)$.
}
}
