\uuid{G2vt}
\exo7id{2109}
\auteur{debes}
\datecreate{2008-02-12}
\isIndication{true}
\isCorrection{true}
\chapitre{Ordre d'un élément}
\sousChapitre{Ordre d'un élément}

\contenu{
\texte{
Soit $E$ un ensemble muni d'une loi $\star $ associative 

(i) admettant un \'el\'ement neutre \`a gauche $e$ (i.e. $\forall x \in E \quad e\star x=x$)
et

(ii) tel que tout \'el\'ement poss\`ede un inverse \`a gauche (i.e. $\forall x \in E \quad \exists y\in E \quad y\star x =e$).

Montrer que $E$ est un groupe pour la loi $\star$.
}
\indication{On pourra montrer les points suivants:

(a) $x\star y=e \Rightarrow y\star x =e$

(b) L'\'el\'ement neutre \`a gauche est unique.

(c) L'\'el\'ement neutre \`a gauche est un \'el\'ement neutre \`a droite aussi.

(d) Tout \'el\'ement est inversible.}
\reponse{
(a) Pour $x,y\in E$ quelconques, notons $x^\prime$ et $y^\prime$ leurs inverses \`a gauche
respectifs. Si $xy=e$, on a aussi $yx=(x^\prime x)yx=x^\prime (xy) x =x^\prime x = e$.
\smallskip

(b) Soit $f$ un \'el\'ement neutre \`a gauche. On a donc $fe=e$. D'apr\`es (a), on a aussi
$ef=e$, c'est-\`a-dire $f=e$.
\smallskip

(c) Pour tout $x\in E$, on a $xe=x(x^\prime x)=(x x^\prime) x = x$ puisque d'apr\`es (a), $x
x^\prime = e$.
\smallskip

(d) r\'esulte alors de (a), (b) et (c).
\smallskip
}
}
