\uuid{afRR}
\exo7id{1449}
\auteur{hilion}
\organisation{exo7}
\datecreate{2003-10-01}
\isIndication{false}
\isCorrection{false}
\chapitre{Groupe quotient, théorème de Lagrange}
\sousChapitre{Groupe quotient, théorème de Lagrange}

\contenu{
\texte{
Rappel: si $A$ est un anneau (en particulier, si $A$ est un corps), on note $GL_n(A)$ l'ensemble des matrices carrées de dimension $n$ à coefficient dans $A$, qui sont inversibles.
$GL_n(A)$ forme un groupe pour la loi $\times$ de multiplication des matrices, appelé groupe linéaire.
Une matrice carrée de dimension $n$ est dans $GL_n(A)$ ssi son déterminant est un inversible de l'anneau $A$ (ce qui revient à dire, lorsque $A$ est un corps, que son déterminant est non nul).

Pour simplifier, on suppose dans l'exercice que $A$ est un corps, noté $\mathbb{K}$.
}
\begin{enumerate}
    \item \question{Montrer que $\det:GL_n(\mathbb{K})\rightarrow\mathbb{K}^*$ est un morphisme de groupes.}
    \item \question{On note $SL_n(\mathbb{K})=\ker(\det)$.
Dire pourquoi $SL_n(\mathbb{K})$ est un sous-groupe distingué de $GL_n(\mathbb{K})$ et montrer que $GL_n(\mathbb{K})/SL_n(\mathbb{K})\cong \mathbb{K}^*$.}
    \item \question{Reconnaître $GL_1(\mathbb{K})$ et $SL_1(\mathbb{K})$.}
    \item \question{Montrer que les matrices diagonales (resp. triangulaires supérieures) de $GL_n(\mathbb{K})$ forment un sous-groupe. Sont-ils distingués?}
    \item \question{Montrer que $Z(GL_n(\mathbb{K}))$ est le sous-groupe formé par les homothéties.}
\end{enumerate}
}
