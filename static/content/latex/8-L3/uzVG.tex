\uuid{uzVG}
\exo7id{2187}
\auteur{debes}
\datecreate{2008-02-12}
\isIndication{true}
\isCorrection{false}
\chapitre{Action de groupe}
\sousChapitre{Action de groupe}

\contenu{
\texte{
D\'ecrire le groupe $D_n$ des isom\'etries du plan affine
euclidien qui laissent invariant un
polygone r\'egulier \`a $n$ c\^ot\'es. Montrer que $D_n$ est engendr\'e
par deux \'el\'ements $\sigma $ et $\tau
$ qui v\'erifient les relations: $\sigma ^ n =1$, $\tau ^2=1$ et $\tau
\sigma \tau ^{-1} =\sigma
^{-1}$.  Quel est l'ordre de $D_n$? D\'eterminer le centre de $D_n$.
Montrer que $D_3 \simeq S_3$.
}
\indication{On se ram\`ene \`a la situation o\`u le polygone est inscrit dans le plan
complexe et a pour sommets les racines de l'unit\'e $e^{2ik\pi/n}, k=0,1,\ldots,n-1$. Une
isom\'etrie laissant invariant le polygone fixe n\'ecessairement l'origine. Elle est donc de
la forme $z\rightarrow az$ ou $z\rightarrow a \overline z$ avec $|a|=1$. On voit ensuite que
$a$ est n\'ecessairement une racine $n$-i\`eme de $1$. Notons $\sigma$ l'isom\'etrie
$z\rightarrow e^{2i\pi/n} z$ et $\tau$ la conjugaison complexe. On a $D_n=\{\sigma^k
\tau^\varepsilon \hskip 2pt | \hskip 2pt k=0,\ldots,n-1, \varepsilon = \pm 1\}$. On v\'erifie
que $\sigma$ et $\tau$ engendrent le groupe $D_n$ et satisfont les relations $\sigma^n=1$,
$\tau^2=1$ et $\tau \sigma \tau^{-1}= \sigma^{-1}$. Autrement dit, $D_n$ est isomorphe au
groupe di\'edral d'ordre $2n$. Si $n$ est impair, son centre est trivial et si $n=2m$ est
pair, son centre est $\{1,\sigma^m\}$. Le groupe $D_n$ se plonge naturellement dans $S_n$;
comme $|D_3|=|S_3|=6$, ce plongement est un isomorphisme pour $n=3$.}
}
