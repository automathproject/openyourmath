\uuid{FKpl}
\titre{Comparaison d'estimateurs}
\theme{estimateurs}
\auteur{Maxime Nguyen}
\organisation{AMSCC}

\contenu{
\texte{ 	On considère trois variables aléatoires indépendantes $X_1,X_2,X_3$ suivant chacune une loi de probabilité de même moyenne $\mu$ et de variance $\sigma^2$. On pose $$M_1=\frac{X_1+X_2+X_3}{3} \text{ et } M_2=\frac{X_1+2X_2+3X_3}{6}$$ }
\begin{enumerate}
	\item \question{ Démontrer que ce sont deux estimateurs sans biais de la moyenne $\mu$. }
	\reponse{
	\begin{align*}
		\E(M_1) &= \E\left(\frac{X_1+X_2+X_3}{3}\right) \\
		&= \frac{1}{3}\E(X_1+X_2+X_3) \\
		&= \frac{1}{3}(\E(X_1)+\E(X_2)+\E(X_3)) \\
		&= \frac{1}{3}(3\mu) \\
		&= \mu
	\end{align*}
	Donc $B(M_1) = \E(M_1 - \mu) = \E(M_1) - \mu = \mu - \mu = 0$. \\
	Donc $M_1$ est un estimateur sans biais de la moyenne $\mu$. \\

	De même, on a :
	\begin{align*}
		\E(M_2) &= \E\left(\frac{X_1+2X_2+3X_3}{6}\right) \\
		&= \frac{1}{6}\E(X_1+2X_2+3X_3) \\
		&= \frac{1}{6}(\mu+2\mu+3\mu) \\
		&= \frac{6\mu}{6} \\
		&= \mu
	\end{align*}
	Donc $B(M_2) = \E(M_2 - \mu) = \E(M_2) - \mu = \mu - \mu = 0$. \\
	Donc $M_2$ est un estimateur sans biais de la moyenne $\mu$. 
	}
	\item \question{ Lequel de ces deux estimateurs est le plus efficace ? }
	\reponse{
	\begin{align*}
		\V(M_1) &= \V\left(\frac{X_1+X_2+X_3}{3}\right) \\
		&= \frac{1}{9}\V(X_1+X_2+X_3) \\
		&= \frac{1}{9}(\V(X_1)+\V(X_2)+\V(X_3)) \\
		&= \frac{1}{9}(3\sigma^2) \\
		&= \frac{\sigma^2}{3}
	\end{align*}
	\begin{align*}
		\V(M_2) &= \V\left(\frac{X_1+2X_2+3X_3}{6}\right) \\
		&= \frac{1}{36}\V(X_1+2X_2+3X_3) \\
		&= \frac{1}{36}(\V(X_1)+4\V(X_2)+9\V(X_3)) \\
		&= \frac{1}{36}(3\sigma^2+4\sigma^2+9\sigma^2) \\
		&= \frac{16\sigma^2}{36} \\
		&= \frac{4\sigma^2}{9}
	\end{align*}
	Donc $\V(M_1) = \frac{\sigma^2}{3}$ et $\V(M_2) = \frac{4\sigma^2}{9}$. \\
	Donc $\V(M_1) < \V(M_2)$. \\
	Donc $M_1$ est plus efficace que $M_2$. 
	}
\end{enumerate}
}