\uuid{pfS1}
\titre{Estimation d'un paramètre}
\theme{statistiques}
\contenu{
	\texte{Dans une usine, une machine produit des pièces dont la longueur suit une loi normale de paramètres $\mu$ (en cm) et $\sigma^2 = 4$ (en cm²). On souhaite estimer $\mu$ à partir d'un échantillon de taille $n=5$. On note $(X_1,...,X_5)$ cet échantillon.

On considère les deux estimateurs suivants :
$$T_1 = \frac{1}{5}\sum_{i=1}^{5} X_i \qquad T_2 = \frac{1}{4}(X_1 + X_2) + \frac{1}{6}(X_3 + X_4 + X_5)$$

On pose également :
$$U = \sum_{i=1}^{5} (X_i - T_1)^2 \qquad V = \frac{1}{2}(X_1 - X_2)$$
}

\begin{enumerate}
\item \question{Montrer que $V$ suit une loi normale dont on précisera les paramètres.}
\reponse{
	Par linéarité de l'espérance : 
	$\E(V) = \frac{1}{2}(\E(X_1) - \E(X_2)) = \frac{1}{2}(\mu - \mu) = 0$
	
	Par indépendance des variables :
	$\V(V) = \frac{1}{4}(\V(X_1) + \V(X_2)) = \frac{1}{4}(4 + 4) = 2$
	
	Donc $V \sim \mathcal{N}(0,\sqrt{2})$
}
\item \question{ Calculer $\prob(-4{,}5 \leq V \leq 4{,}5)$ avec 8 chiffres significatifs.}
\reponse{

Avec le tableur et la formule \texttt{=LOI.NORMALE.N(4,5;0;RACINE(2);1)*2-1}, on trouve 		$\prob(-4{,}5 \leq V \leq 4{,}5) \approx 0{,}99853728$. 
}
\item \question{Déterminer la loi suivie par la variable $T_1$.}
\reponse{
	$T_1 = \frac{1}{5}\sum_{i=1}^{5} X_i$ est une somme de variables aléatoires normales indépendantes. Donc $T_1$ suit une loi normale de moyenne $\mu$ et de variance $\frac{4}{5}$.
}
\item \question{Étudier le biais des estimateurs $T_1$ et $T_2$.}
\reponse{
	Pour $T_1$ :
	\begin{align*}
		\E(T_1) &= \frac{1}{5}\sum_{i=1}^{5} \E(X_i) = \frac{5\mu}{5} = \mu
	\end{align*}
	Donc $T_1$ est sans biais.
	
	Pour $T_2$ :
	\begin{align*}
		\E(T_2) &= \frac{1}{4}(\E(X_1) + \E(X_2)) + \frac{1}{6}(\E(X_3) + \E(X_4) + \E(X_5)) \\
		&= \frac{2\mu}{4} + \frac{3\mu}{6} = \mu
	\end{align*}
	Donc $T_2$ est aussi sans biais.
}

\item \question{Lequel de ces deux estimateurs est le plus efficace ?}
\reponse{
	Pour $T_1$, par indépendance :
	\begin{align*}
		\V(T_1) &= \frac{1}{25}\sum_{i=1}^{5} \V(X_i) = \frac{5 \times 4}{25} = \frac{4}{5}
	\end{align*}
	
	Pour $T_2$ :
	\begin{align*}
		\V(T_2) &= \frac{1}{16}\V(X_1 + X_2) + \frac{1}{36}\V(X_3 + X_4 + X_5) \\
		&= \frac{8}{16} + \frac{12}{36} \\
		&= \frac{1}{2} + \frac{1}{3} > \frac{4}{5}
	\end{align*}
	
	Comme $\frac{4}{5} < \frac{30}{36}$, $T_1$ est plus efficace que $T_2$.
}

\item \question{Déterminer un coefficient $a \in \R$ tel que $aU$ suive une loi du $\chi^2$ dont on précisera le nombre de degrés de liberté.}
\reponse{
	$U = \sum_{i=1}^{5} (X_i - T_1)^2$ où $T_1$ est la moyenne empirique.
	D'après le cours (théorème de Fisher), $\frac{U}{\sigma^2}$ suit une loi du $\chi^2$ à $n-1 = 4$ degrés de liberté.
	Donc $\frac{1}{4}U$ suit une loi du $\chi^2$ à 4 degrés de liberté.
}

\item \question{En déduire la loi suivie par la variable $W = \frac{T_1 - \mu}{\sqrt{\frac{U}{20}}}$.}
\reponse{
	On sait que $\frac{T_1 - \mu}{\sigma/\sqrt{5}} \sim \mathcal{N}(0,1)$ indépendant de $\frac{U}{4} \sim \chi^2(4)$
	
	Donc $W = \frac{T_1 - \mu}{\sqrt{\frac{U}{20}}} = \frac{\frac{T_1 - \mu}{\sqrt{\frac{4}{5}}}}{\sqrt{\frac{\frac{U}{4}}{4}}} \sim St(4)$.	
	Ainsi, $W \sim St(4)$
}
\end{enumerate}
}