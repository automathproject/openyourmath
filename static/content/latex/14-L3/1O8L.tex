\uuid{1O8L}
\exo7id{5937}
\auteur{tumpach}
\datecreate{2010-11-11}
\isIndication{false}
\isCorrection{true}
\chapitre{Tribu, fonction mesurable}
\sousChapitre{Tribu, fonction mesurable}

\contenu{
\texte{
Soit $(\Omega, \Sigma, \mu)$ un espace mesur\'e et
$f\in\mathcal{M}^+(\Omega, \Sigma)$ (i.e $f$ est une fonction
r\'eelle mesurable positive). Pour tout $E\in\Sigma$, on pose~:
$$
\lambda(E) = \int_{E} f\,d\mu = \int_{\Omega} \mathbf{1}_{A}\cdot f
\,d\mu.
$$
Monter que $\lambda$ d\'efinit une mesure sur $(\Omega, \Sigma)$.
}
\reponse{
Soit $(\Omega, \Sigma, \mu)$ un espace mesur\'e et
$f\in\mathcal{M}^+(\Omega, \Sigma)$. Pour tout $E\in\Sigma$, on
pose~:
$$
\lambda(E) = \int_{E} f\,d\mu = \int_{\Omega} \mathbf{1}_{A}\cdot f
\,d\mu.
$$
Montrons que $\lambda$ d\'efinit une mesure sur $(\Omega,
\Sigma)$.\\

\emph{$1^{\text{\`ere}}$ m\'ethode~:} On montre d'abord que
l'affirmation est vraie pour les fonctions simples.  D'apr\`es
l'exercice~\ref{ex:barb20}, toute fonction $f\in\mathcal{M}^{+}(\Omega, \Sigma)$
s'\'ecrit $f = \sup_{n\in\mathbb{N}} \varphi_{n}$, o\`u les
$\varphi_{n}$ sont des fonctions simples. Puisque le supremum
d'une famille quelconque de mesure est une mesure, on conclut que $\lambda$ est une mesure. \\

\emph{$2^{\text{de}}$ m\'ethode~:} On a clairement
$\lambda(\emptyset) = 0$. Il suffit donc de v\'erifier la
$\sigma$-additivit\'e de $\lambda$. Soit
$\{E_{i}\}_{i\in\mathbb{N}}\subset\Sigma$ une suite d'\'el\'ements
deux \`a deux disjoints. On a
\begin{eqnarray*}
\lambda\left(\bigcup_{i=1}^{\infty} E_{i}\right) & = &
\int_{\bigcup_{i=1}^{\infty} E_{i}} f\,d\mu = \int_{\Omega}
\mathbf{1}_{\bigcup_{i=1}^{\infty} E_{i}} f\,d\mu\\
& = &\int_{\Omega} \left(\sum_{i=1}^{\infty}\mathbf{1}_{E_{i}} \right)
f\,d\mu = \int_{\Omega} \sum_{i=1}^{\infty} \left( \mathbf{1}_{E_{i}}
f\right)
\,d\mu\\
& = & \sum_{i=1}^{\infty} \int_{\Omega} \left( \mathbf{1}_{E_{i}} f\right)
d\mu = \sum_{i=1}^{\infty} \int_{E_{i}} f\,d\mu \\ & = &
\sum_{i=1}^{\infty} \lambda(E_{i}).
\end{eqnarray*}
}
}
