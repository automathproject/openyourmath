\uuid{l1b6}
\exo7id{7832}
\auteur{mourougane}
\datecreate{2021-08-11}
\isIndication{false}
\isCorrection{false}
\chapitre{Forme bilinéaire}
\sousChapitre{Forme bilinéaire}

\contenu{
\texte{
On considère dans l'espace euclidien $E$ de dimension $3$ une
quadrique $\mathcal{Q}$ d'équation $q(x,y,z)=0$ o{ù} $q$ est un polyn{ô}me
de degré 2. On note $h$ sa partie homogène de degré $2$. C'est une
forme quadratique sur $\vec{E}$. On l'appelle forme quadratique à
l'infini.  On note $Q$ l'homogénéisée de $q$. C'est une forme
quadratique sur $\widehat{E}$ qui définit la complétion projective de
$\mathcal{Q}$.
}
\begin{enumerate}
    \item \question{Montrer que chaque argument de la signature de $Q$ est plus
 grand que l'argument correspondant de la signature de $h$.}
    \item \question{On suppose que la signature de $h$ est $(3,0)$ ou $(0,3)$.
Déterminer, suivant la signature de $Q$ une forme réduite (dans un bon
repère) pour $q$ et  représenter dans chaque cas la quadrique
$\mathcal{Q}$.}
    \item \question{Indiquer le résultat pour les autres signatures.}
    \item \question{Affecter aux différents cas les noms suivants :

plan double réel, couple de plans imaginaires conjugués parallèles
distincts, couple de plans réels parallèles distincts,

cylindre à base parabolique, cylindre à base hyperbolique, cylindre à
base elliptique, cylindre imaginaire,

c{ô}ne imaginaire de sommet réel, c{ô}ne de base une conique propre
réelle,

ellipsoïde imaginaire, ellipsoïde réel,

paraboloïde hyperbolique, paraboloïde elliptique,

hyperboloïde à une nappe, hyperboloïde à deux nappes.}
\end{enumerate}
}
