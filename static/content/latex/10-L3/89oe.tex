\uuid{89oe}
\exo7id{6506}
\auteur{drutu}
\datecreate{2011-10-16}
\isIndication{false}
\isCorrection{false}
\chapitre{Anneau, corps}
\sousChapitre{Anneau, corps}

\contenu{
\texte{
Soit $M_n(F)$ l'anneau des matrices sur un corps $F$.
}
\begin{enumerate}
    \item \question{Montrer que si $X$ est une matrice non-dégénérée, alors $XY\neq 0$ et $YX\neq 0,\; \forall Y\neq 0$.}
    \item \question{Soit $X\in M_n(F)$ et soit l'application linéaire $f_X:F^n\to F^n,\; f_X(v):=Xv$. Trouver une écriture de la relation $XY=0,\; X,\; Y\in M_n(F)$ en termes de noyaux et images des applications $f_X,\; f_Y$. Même question pour la relation $YX=0$.
En déduire que toute matrice dégénérée non-nulle est un diviseur de 0.}
    \item \question{Montrer que tout idéal bilatère de $M_n(F)$ contient, avec une matrice de rang $r$, toutes les matrices diagonales de rang $r$.}
    \item \question{Montrer qu'un idéal qui contient une matrice non-dégénérée co\"{\i}ncide avec $M_n(F)$.}
    \item \question{Montrer que les seuls idéaux de $M_n(F)$ sont $0$ et $M_n(F)$.}
\end{enumerate}
}
