\uuid{p2C6}
\exo7id{2264}
\auteur{barraud}
\datecreate{2008-04-24}
\isIndication{false}
\isCorrection{true}
\chapitre{Polynôme}
\sousChapitre{Polynôme}

\contenu{
\texte{
Pour $n,m\ge 2$, d\'eterminer le reste de la division
euclidienne du polynôme \mbox{$(x-2)^m+(x-1)^n-1$} par $(x-1)(x-2)$ 
dans $\Zz[x]$.
}
\reponse{
Notons $(Q,R)$ le quotient et le reste de cette division euclidienne:
  $(x-2)^{m}+(x-1)^{n}-1=Q(x-2)(x-1)+R$ avec $\deg(R)\leq 1$. Notons
  $R=ax+b$. En évaluant en $1$, on obtient $(-1)^{m}-1=a+b$, et en
  évaluant en $2$, $2a+b=0$. On en déduit $b=-2a$ et $a=1-(-1)^{m}$, soit
  $R=(1-(-1)^{m})(x-2)$.
}
}
