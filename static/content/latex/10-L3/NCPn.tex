\uuid{NCPn}
\exo7id{6552}
\auteur{drutu}
\datecreate{2011-10-16}
\isIndication{false}
\isCorrection{false}
\chapitre{Extension de corps}
\sousChapitre{Extension de corps}

\contenu{
\texte{
Soit $K$ un corps et $k,\; A,\; B$ des sous-corps tels que $k\subset A$ et $k\subset B$, $[A:k]=m,\; [B:k]=n$. Soit $L$ le plus petit sous-corps de $K$ qui contient $A\cup B$.
}
\begin{enumerate}
    \item \question{Montrer que $[L:A]\leq n,\; [L:B]\leq m,\; [L:k]\leq mn$. Caractériser le cas $[L:k]= mn$ à l'aide d'une propriété de $A$ par rapport à $B$.}
    \item \question{Si $[K:k]=4,\, m=n=2$ montrer l'équivalence des propriétés suivantes
  \begin{enumerate}}
    \item \question{[(b$_1$)] $A\neq B$ ;}
    \item \question{[(b$_2$)] $L=K$ ;}
    \item \question{[(b$_3$)] il existe $a\in A$ et $b\in B$ tels que $\{ 1,\, a,\, b,\, ab \}$ soit une base de $L$ sur $k$.}
\end{enumerate}
}
