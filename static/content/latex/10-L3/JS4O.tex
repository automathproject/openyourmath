\uuid{JS4O}
\exo7id{2292}
\auteur{barraud}
\datecreate{2008-04-24}
\isIndication{false}
\isCorrection{true}
\chapitre{Polynôme}
\sousChapitre{Polynôme}

\contenu{
\texte{
Soit $(x^3-x+2)$ l'id\'eal principal engendr\'e
par $x^3-x+2$ dans l'anneau  $\Qq[x]$.
}
\begin{enumerate}
    \item \question{Montrer que l'anneau quotient $\Qq[x]/(x^3-x+2)$ est un corps.}
\reponse{Soit $P=x^{3}-x+2$. Sa réduction  $\bar{P}=x^{3}-x-1$ modulo $3$ est
    de degré $3$ et n'a pas de racine, donc $\bar{P}$ est irréductible
    dans $\Zz_{3}[x]$. Comme $P$ est primitif, on en déduit que $P$ est
    irréductible dans $\Zz[x]$, puis dans $\Qq[x]$. Comme $\Qq[x]$ est
    principal, on en déduit que $(P)$ est maximal, et donc que
    $\Qq[x]/(P)$ est un corps.}
    \item \question{Soit $y$ l'image de $x$ dans $\Qq[x]/(x^3-x+2)$ par la surjection
canonique. Calculer son inverse.}
\reponse{Dans $\Qq[x]/(P)$, on a $y^{3}-y+2=0$, donc $y(y^{2}-1)=-2$ et
    finalement $y(\frac{1}{2}(1-y^{2}))=1$. Ainsi $y^{-1}=\frac{1}{2}(1-y^{2})$.}
    \item \question{Montrer que $1+y+y^2$ est non nul et calculer son inverse.}
\reponse{$1+y+y^{2}=\pi(1+x+x^{2})$. On a $\pgcd(P,1+x+x^{2})=1$, et plus
    précisément, en utilisant l'algorithme d' Euclide~:
    $13=(x+4)P-(x^{2}+3x-5)(x^{2}+x+1)$ donc
    $(y^{2}+y+1)^{-1}=\frac{-1}{13}(y^{2}+3y-5)$.}
\end{enumerate}
}
