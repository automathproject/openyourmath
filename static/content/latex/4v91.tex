\uuid{4v91}
\titre{Convergence d'une suite de variables aléatoires}
\theme{variables aléatoires, convergence en loi, convergence en probabilité}
\auteur{}
\datecreate{2023-01-11}
\organisation{AMSCC}
\contenu{

\texte{ Soit la suite de variables aléatoires $(X_n)_{n \geq 1}$ définie par
$$
\left\{\begin{array}{l}
	P\left(X_n=0\right)=1-\frac{1}{n} \\
	P\left(X_n=n\right)=\frac{1}{n}
\end{array}\right.
$$ }

\begin{enumerate}
	\item  \question{ Montrer que la suite $(X_n)_{n \geq 1}$ converge en loi vers $X=0$.  }
	\reponse{ On peut utiliser la fonction caractéristique : 
		$$
		\varphi_{X_n}(t)=\mathrm{E}\left[\mathrm{e}^{\mathrm{it} X_{\mathrm{n}}}\right]=1-\frac{1}{n}+\frac{1}{n} e^{i t n}
		$$
		donc
		$$
		\lim _{n \rightarrow+\infty} \varphi_{X_n}(t)=1=\mathrm{E}\left[\mathrm{e}^{\mathrm{it} 0}\right]
		$$
		et par théorème du cours, on a $X_n \xrightarrow[n \to +\infty]{\text{en loi}} 0$ }
	\item \question{ En revenant à la définition, montrer que la suite $\left(X_n\right)_{n \in \mathbb{N}}$ converge en probabilité vers $X=0$.  }
	\reponse{ Remarque :
		$$
		\forall m \neq n \text { et } m \neq 0, \quad P\left(X_n=m\right)=0 
		$$
		Soit $\varepsilon>0$
		$$
		P\left(\left|X_n\right|>\varepsilon\right)=P\left(X_n=n\right)=\frac{1}{n}
		$$
		donc
		$$
		\lim _{n \rightarrow+\infty} P\left(\left|X_n\right|>\varepsilon\right)=0
		$$ }
\end{enumerate}
}
