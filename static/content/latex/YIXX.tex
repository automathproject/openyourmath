\uuid{YIXX}
\titre{Changement de variables d'un couple}
\theme{variables aléatoires à densité, loi conjointe}
\auteur{}
\datecreate{2022-11-15}
\organisation{AMSCC}
\contenu{


\texte{ 	Soit $(X,Y)$ un couple de variables aléatoires admettant pour densité la fonction $f$ définie par $$ f(x,y)=   \frac{3}{8}(x^2+y^2) \textbf{1}_{[-1;1]^2}(x,y)$$ }
	
\question{ 	On cherche à déterminer la loi de $(X+Y,X-Y)$.  }
	


\reponse{Pour cela, on pose $U=X+Y$, $V=X-Y$ et on étudie la loi du couple $(U,V)$. D'après le théorème de transfert, pour toute fonction $h$  continue bornée, 
	
	$$\mathbb{E}(h(U,V))= \frac{3}{8} \int_{[-1;1]^2}^{} h(x+y,x-y) (x^2+y^2)dxdy$$
	
	Pour faire apparaître la densité de $(U,V)$, on effectue le changement de variable $$(u,v)=(x+y,x-y)$$ (c'est une bijection de classe $\mathcal{C}^1$). La réciproque s'écrit $(x,y)=(\frac{u+v}{2},\frac{u-v}{2})$. La matrice jacobienne est $\begin{pmatrix}
	\frac{1}{2} & \frac{1}{2} \\
	\frac{1}{2} & -\frac{1}{2} \\
	\end{pmatrix} $ et la valeur absolue de son déterminant est $\frac{1}{2}$. On peut donc écrire $dxdy=\frac{1}{2}dudv$ et on a finalement :
	
	$$\mathbb{E}(h(U,V))= \frac{3}{8} \int_{C}^{} h(u,v) (\frac{u^2+v^2}{2})\frac{1}{2}dudv$$  où $C= \{(u,v) \in \R^2 \slash -2 \leq u+v \leq 2 \text{ et } -2 \leq u-v \leq 2 \}$ est le carré image de $[-1;1]^2$ par le changement de variables.
	
	On en déduit que $(U,V)$ a pour densité la fonction $g$ définie par $$g(u,v)=\frac{3}{16}(u^2+v^2) \textbf{1}_C(u,v)$$
	Pour avoir indépendamment la loi de $(X+Y)$ et $(X-Y)$, il resterait à calculer les lois marginales.
}
}