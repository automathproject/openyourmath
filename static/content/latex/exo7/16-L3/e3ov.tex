\uuid{e3ov}
\exo7id{6676}
\auteur{queffelec}
\organisation{exo7}
\datecreate{2011-10-16}
\isIndication{false}
\isCorrection{false}
\chapitre{Formule de Cauchy}
\sousChapitre{Formule de Cauchy}

\contenu{
\texte{
Calculer les intégrales suivantes.
}
\begin{enumerate}
    \item \question{$\displaystyle\int_\gamma {\cos z\over z}\ dz$ où $\gamma (t)=e^{it}$ ($t\in [0,2\pi]$)}
    \item \question{$\displaystyle\int_\gamma {\sin z\over z}\ dz$ où $\gamma (t)=e^{it}$ ($t\in [0,2\pi]$)}
    \item \question{$\displaystyle\int_\gamma {\cos {z^2}\over z}\ dz$ où $\gamma (t)=e^{it}$ ($t\in [0,2\pi]$)}
    \item \question{$\displaystyle\int_\gamma {e^{\pi z}\over z^3+z}\ dz$ où
$\gamma (t) =2e^{it}$ ($t\in [0,2\pi]$)}
    \item \question{$\displaystyle\int_\gamma {dz\over(z-a)^n}$ ($n\in{\Zz}$), où $\gamma $
est un chemin fermé ne passant pas par $a$}
    \item \question{$\displaystyle\int_{\gamma _r}{3z^2-12z+11\over z^3-6z^2+11z-6}\ dz$,
où $\gamma _r(t)=re^{it}$ ($t\in [0,2\pi]$)}
    \item \question{$\displaystyle\int_{\gamma }\left({z\over z-1}\right)^n\ dz$, 
$n\in{\Nn}^*$,
où $\gamma (t)=1+e^{it}$ ($t\in [0,2\pi]$)}
\end{enumerate}
}
