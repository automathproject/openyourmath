\uuid{GL6U}
\exo7id{7631}
\auteur{mourougane}
\organisation{exo7}
\datecreate{2021-08-10}
\isIndication{false}
\isCorrection{true}
\chapitre{Autre}
\sousChapitre{Autre}

\contenu{
\texte{
Soit $f$ une fonction holomorphe sur le disque unité $\Delta$. 
Le but de l'exercice est de montrer que $|f(z)|$ ne tend pas vers $+\infty$ quand $|z|$ tend vers $1$.
}
\begin{enumerate}
    \item \question{Vérifier l'énoncé pour l'application $\Delta\to\Cc$, $z\mapsto \frac{1}{z^6-1}$.}
\reponse{Cette application est continue en $i$ : elle ne tend donc pas vers $+\infty$ en module quand $z$ tend vers $i$.
  Elle ne tend donc pas vers $+\infty$ en module quand $|z|$ tend vers $1$.}
    \item \question{Supposons d'abord que $f$ n'a pas de zéro dans $\Delta$. 
Soit $r_n$ une suite de réels de $[0,1[$, tels que pour tout entier naturel non nul $n$ et tout $z$ de $\Delta$ de module supérieur à $r_n$, $f(z)$ est de module supérieur à $n$. 
Montrer alors que pour tout $n$, $|f(0)|\geq n$ et conclure.}
\reponse{Par le principe du maximum appliqué à la fonction holomorphe $\Delta\to\Cc$, $z\mapsto 1/f(z)$,
 $$|1/f(0)|\leq \sup_{|z|=r_n}|1/f(z)|=\frac{1}{\inf_{|z|=r_n} |f(z)|}\leq 1/n.$$
 Donc, pour tout $n$, $|f(0)|\geq n$, ce qui est absurde.
 Il n'existe donc pas de telle suite. Donc, dans le cas où $f$ n'a pas de zéro, $|f(z)|$ ne tend pas vers $+\infty$ quand $|z|$ tend vers $1$.}
    \item \question{Dans le cas général, supposons que $|f(z)|$ tende vers $+\infty$ quand $|z|$ tend vers $1$.
Montrer qu'on peut écrire $f$ comme produit sur $\Delta$ d'un polynôme et d'une application holomorphe $g$ qui n'a pas de zéro dans $\Delta$. Conclure.}
\reponse{Comme $|f(z)|$ tend vers $+\infty$ quand $|z|$ tend vers $1$, il existe $r\in ]0,1[$ tel que pour tout $z\in\Delta$ avec $|z|>r$, $|f(z)|>1$. L'application $f$ ne s'annule donc pas sur $\Delta-\overline{\Delta_r}$.
 Par le théorème des zéros isolés, puisque $f$ holomorphe n'est pas constante, elle n'admet qu'un nombre fini de zéro dans le compact $\overline{\Delta_r}$ et par suite dans tout le disque $\Delta$.
 Soit $P(z)$ le polynôme unitaire qui a pour zéros les zéros de $f$ avec les mêmes multiplicités.
 L'application $h=\frac{f}{P}$ a des singularités isolées en les zéros de $f$. Au voisinage de ces zéros, le développement en séries entières de $f$ montre que les singularités 
 sont apparentes et que $h$ tend vers une valeur finie non nulle en ces points. L'application $h$ se prolonge donc par continuité en une application holomorphe $g$ sur $\Delta$. 
 L'égalité $f=hP$ sur $\Delta-f^{-1}(0)$ se prolonge par continuité en $f=gP$ sur $\Delta$.
 Par construction, par le choix des multiplicités, l'application $g$ n'a pas de zéro aux zéros de $f$, et n'a donc pas de zéro sur $\Delta$.
 Comme $|f(z)|$ tend vers $+\infty$ quand $|z|$ tend vers $1$ et que $|P|$ tend vers une valeur finie en tout point de $\partial\Delta$, $|g|$ tend vers $+\infty$ quand $|z|$ tend vers $1$. Le cas précédent montre que l'existence d'une telle application $g$ est absurde.
 Donc, $|f(z)|$ ne tend pas vers $+\infty$ quand $|z|$ tend vers $1$.}
\end{enumerate}
}
