\uuid{3v1k}
\exo7id{7583}
\auteur{mourougane}
\organisation{exo7}
\datecreate{2021-08-10}
\isIndication{false}
\isCorrection{false}
\chapitre{Théorème des résidus}
\sousChapitre{Théorème des résidus}

\contenu{
\texte{

}
\begin{enumerate}
    \item \question{Montrer que le polynôme $f(z)=3+7z+2z^4$ a, comme le polynôme $3+7z$, exactement un zéro dans $\Delta$.}
    \item \question{Déterminer le nombre de zéros (comptés avec multiplicité) de $z^5+\frac{z^3}{3}+\frac{z^2}{4}+\frac{1}{3}$ dans $\Delta$.}
    \item \question{Déterminer le nombre de zéros (comptés avec multiplicité) de $z^5+\frac{z^3}{3}+\frac{z^2}{4}+\frac{1}{3}$ dans $\Delta_{\frac{1}{2}}$.}
    \item \question{Déterminer le nombre de zéros du polynôme $ z^{4} -5z-1 $ dans la couronne $ 1<\arrowvert z \arrowvert < 2 $.}
\end{enumerate}
}
