\uuid{WYv0}
\exo7id{5842}
\auteur{rouget}
\organisation{exo7}
\datecreate{2010-10-16}
\isIndication{false}
\isCorrection{true}
\chapitre{Topologie}
\sousChapitre{Autre}

\contenu{
\texte{

}
\begin{enumerate}
    \item \question{Montrer que $GL_n(\Rr)$ est un ouvert de $\mathcal{M}_n(\Rr)$, dense dans $\mathcal{M}_n(\Rr)$.}
\reponse{Soit $\begin{array}[t]{cccc}
d~:&\mathcal{M}_n(\Rr)&\rightarrow&\Rr\\
 &M&\mapsto&\text{det}(M)
\end{array}
$. On sait que l'application $d$ est continue sur $\mathcal{M}_n(\Rr)$ (muni de n'importe quelle norme) et que $\Rr^*$ est un ouvert de $\Rr$ en tant que réunion de deux intervalles ouverts.

Par suite, $GL_n(\Rr)=d^{-1}(\Rr^*)$ est un ouvert de $\mathcal{M}_n(\Rr)$ en tant qu'image réciproque d'un ouvert par une application continue.

Soit $A\in \mathcal{M}_n(\Rr)$. Le polynôme$\text{det}(A-xI)$ n'a qu'un nombre fini de racines (éventuellement nul) donc pour $p$ entier naturel supérieur ou égal à un certain $p_0$, $\text{det}\left(A- \frac{1}{p}I\right)\neq 0$. La suite $\left(A- \frac{1}{p}I\right)_{p\geqslant p_0}$ est une suite d'éléments de $GL_n(\Rr)$ convergente de limite $A$. Ceci montre que l'adhérence de $GL_n(\Rr)$ est $\mathcal{M}_n(\Rr)$ ou encore $GL_n(\Rr)$ est dense dans $\mathcal{M}_n(\Rr)$.

\begin{center}
\shadowbox{
$GL_n(\Rr)$ est un ouvert de $\mathcal{M}_n(\Rr)$, dense dans $\mathcal{M}_n(\Rr)$.
}
\end{center}}
    \item \question{Montrer que $\mathcal{M}_n(\Rr)\setminus GL_n(\Rr)$ est fermé mais non compact (pour $n\geqslant2$).}
\reponse{$\mathcal{M}_n(\Rr)\setminus GL_n(\Rr)$ est fermé en tant que complémentaire d'un ouvert.

Soit $n\geqslant2$. Les matrices $A_p = pE_{1,1}$, $p\in\Nn$, sont non inversibles et la suite $(A_p)_{p\in\Nn}$ est non bornée. Par suite $Mn(R)\setminus GLn(R)$  est non borné et donc non compact.

\begin{center}
\shadowbox{
$\forall n\geqslant2$, $M_n(\Rr)\setminus GL_n(\Rr)$  est fermé mais non compact.
}
\end{center}}
    \item \question{Montrer que $O_n(\Rr)$ est compact. $O_n(\Rr)$ est-il convexe ?}
\reponse{\textbullet~Montrons que $O_n(\Rr)$ est fermé. Posons $\begin{array}[t]{cccc}
g~:&\mathcal{M}_n(\Rr)&\rightarrow&(\mathcal{M}_n(\Rr))^2\\
 &M&\mapsto&(M,{^t}M)
\end{array}$, $\begin{array}[t]{cccc}h :&(\mathcal{M}_n(\Rr))^2&\rightarrow&\mathcal{M}_n(\Rr)\\
 &(M,N)&\mapsto MN\end{array}$puis $\begin{array}[t]{cccc}f~:&\mathcal{M}_n(\Rr)&\rightarrow&\mathcal{M}_n(R)\\
  &M&\mapsto&M{^t}M\end{array}$.
		
		
$g$ est continue sur $\mathcal{M}_n(\Rr)$ car linéaire sur un espace de dimension finie. $h$ est continue sur $(M_n(\Rr))^2$ car bilinéaire sur un espace de dimension finie. On en déduit que
$f = h\circ g$ est  continue sur $\mathcal{M}_n(\Rr)$.

Enfin $O_n(\Rr) =f^{-1}({I_n})$ est fermé en tant qu'image réciproque d'un fermé par une application continue.

\textbullet~Montrons que $O_n(\Rr)$ est borné. $\forall A\in O_n(\Rr)$, $\forall(i,j)\in\llbracket1,n\rrbracket^2$, $|a_{i,j}|\leqslant 1$ et donc $\forall A\in O_n(\Rr)$, $\|A\|_{\infty}\leqslant1$.

D'après le théorème de \textsc{Borel}-\text{Lebesgue}, puisque $O_n(\Rr)$ est un fermé borné de l'espace de dimension finie $\mathcal{M}_n(\Rr)$, $O_n(\Rr)$ est un compact de $\mathcal{M}_n(\Rr)$.

$O_n(\Rr)$ n'est pas convexe. En effet, les deux matrices $I_n$ et $-I_n$ sont orthogonales mais le milieu du segment joignant ces deux matrices est $0$ qui n'est pas une matrice orthogonale.

\begin{center}
\shadowbox{
$O_n(\Rr)$  est compact mais non convexe.
}
\end{center}}
    \item \question{Montrer que $S_n(\Rr)$ est fermé.}
\reponse{$S_n(\Rr)$ est un sous espace vectoriel de l'espace de dimension finie $\mathcal{M}_n(\Rr)$ et est donc un fermé de $\mathcal{M}_n(\Rr)$.

\begin{center}
\shadowbox{
$S_n(\Rr)$  est fermé.
}
\end{center}}
    \item \question{Soit $p\in\llbracket0,n\rrbracket$. Montrer que l'ensemble des matrices de rang inférieur ou égal à $p$ est un fermé de $\mathcal{M}_n(\Rr)$.}
\reponse{Soit $A\in\mathcal{M}_n(\Rr)$ et $p$ un élément fixé de $\llbracket1,n-1\rrbracket$ (le résultat est clair si $p=0$ ou $p=n$).

$A$ est de rang inférieur ou égal à $p$ si et seulement si tous ses mineurs de format $p+1$ sont nuls (hors programme).

Soient $I$ et $J$ deux sous-ensembles donnés de $\llbracket1,n\rrbracket$ de cardinal $p+1$ et $A_{I,J}$ la matrice extraite de $A$ de format $p+1$ dont les numéros de lignes sont dans $I$ et les numéros de colonnes sont dans $J$.

Pour $I$ et $J$ donnés, l'application $A\mapsto A_{I,J}$ est continue car linéaire de $\mathcal{M}_n(\Rr)$ dans $\mathcal{M}_{p+1}(\Rr)$. Par suite, l'application 

$f_{I,J}~:~A\mapsto\text{det}(A_{I,J})$ est continue sur $\mathcal{M}_n(\Rr)$. L'ensemble des matrices $A$ telles que $\text{det}(A_{I,J}) = 0$ est donc un fermé de $\mathcal{M}_n(\Rr)$ (image réciproque du fermé $\{0\}$ de $\Rr$ par l'application continue $f_{I,J}$) et l'ensemble des matrices de rang inférieur ou égal à $p$ est un fermé de $\mathcal{M}_n(\Rr)$ en tant qu'intersection de fermés.}
    \item \question{Montrer que l'ensemble des matrices diagonalisables dans $\mathcal{M}_n(\Cc)$ est dense dans $\mathcal{M}_n(\Cc)$. Peut-on remplacer $\mathcal{M}_n(\Cc)$ par $\mathcal{M}_n(\Rr)$ ?}
\reponse{Soit $A\in\mathcal{M}_n(\Cc)$. Posons $\text{Sp}(A)=(\lambda_i)_{1\leqslant i\leqslant n}$. On sait que toute matrice est triangulable dans $\Cc$ et donc il existe $P\in GL_n(\Cc)$ et $T\in\mathcal{T}_n(\Cc)$ avec $\forall i\in\llbracket1,n\rrbracket$, $t_{i,i}=\lambda_i$ telle que $A=PTP^{-1}$.

On munit dorénavant $\mathcal{M}_n(\Cc)$ d'une norme multiplicative notée $\|\;\|$. 
Puisque toutes les normes sont équivalentes en dimension finie, il existe un réel strictement positif $K$ telle que pour toute matrice $M$, $\|M\|\leqslant K\|M\|_\infty$.

Soit $\varepsilon> 0$. Il existe un $n$-uplet de réels $(\lambda_1,...,\lambda_n)$ tels que $\forall k\in\llbracket1,n\rrbracket$, $0\leqslant\varepsilon_k < \frac{\varepsilon}{K\|P\|\|P^{-1}\|}$ et les $\lambda_k+\varepsilon_k$ sont deux à deux distincts. (On prend $\varepsilon_1 = 0$ puis $\varepsilon_2$ dans $\left[0, \frac{\varepsilon}{K\|P\|\|P^{-1}\|}\right[$ tel que $\lambda_2+\varepsilon_2 \neq \lambda_1+\varepsilon_1$ ce qui est possible puisque $\left[0, \frac{\varepsilon}{K\|P\|\|P^{-1}\|}\right[$ est infini puis $\varepsilon_3$ dans $\left[0, \frac{\varepsilon}{K\|P\|\|P^{-1}\|}\right[$ tel que $\lambda_3+\varepsilon_3$ soit différent de $\lambda_1+\varepsilon_1$ et $\lambda_2+\varepsilon_2$ ce qui est possible puisque $\left[0, \frac{\varepsilon}{K\|P\|\|P^{-1}\|}\right[$ est infini ...)

On pose $D=\text{diag}(\lambda_i)_{1\leqslant i\leqslant n}$ puis $T' = T+D$ et enfin $A'=PT'P^{-1}$. Tout d'abord les valeurs propres de $A'$ sont deux à deux distinctes (ce sont les $\lambda_i+\varepsilon_i$, $1\leqslant i\leqslant n$) et donc $A'$ est diagonalisable. Ensuite 

\begin{center}
$\|A'-A\| =\|PDP^{-1}\|\leqslant\|P\|\|D\|\|P^{-1}\|\leqslant K\|P\|\|P^{-1}\|\|D\|_\infty <\varepsilon.$
\end{center}

En résumé, $\forall A\in\mathcal{M}_n(\Cc)$, $\forall\varepsilon>0$, $\exists A'\in\mathcal{M}_n(\Cc)/\;\|A'-A\|<\varepsilon$ et $A'$ diagonalisable. On a montré que

\begin{center}
\shadowbox{
L'ensemble des matrices complexes diagonalisables dans $\Cc$ est dense dans $\mathcal{M}_n(\Cc)$.
}
\end{center}

On ne peut remplacer $\mathcal{M}_n(\Cc)$ par $\mathcal{M}_n(\Rr)$.

Soient $A=\left(
\begin{array}{cc}
0&-1\\
1&0
\end{array}
\right)$ et $E=\left(
\begin{array}{cc}
a&c\\
b&d
\end{array}
\right)\in\mathcal{M}_2(\Rr)$.

\begin{center}
$\chi_{A+E}=\left|
\begin{array}{cc}
a-X&c-1\\
b+1&d-X
\end{array}
\right|=X^2 - (a+d)X+(ad-bc) +(b-c)+ 1$.
\end{center}

Le discriminant  de $\chi_{A+E}$ est $\Delta= (a+d)^2 - 4(ad-bc) - 4(b-c) -4$. Supposons de plus que $\|E\|_\infty\leqslant \frac{1}{4}$. Alors 

\begin{center}
$\Delta= (a+d)^2 - 4(ad-bc) - 4(b-c) -4\leqslant \frac{1}{4}+ 4\left( \frac{1}{16}+ \frac{1}{16}\right)+4\left( \frac{1}{4}+ \frac{1}{4}\right)-4=- \frac{5}{4}<0$.
\end{center}

Par suite, aucune des matrices $A+E$ avec $\|E\|_\infty\leqslant \frac{1}{4}$ n'a de valeurs propres réelles et donc aucun donc diagonalisable dans $\Rr$. On a montré que l'ensemble des matrices réelles diagonalisables dans $\Rr$ n'est pas dense dans $\mathcal{M}_n(\Rr)$.}
    \item \question{Propriétés topologiques de l'ensemble des triplets de réels $(a,b,c)$ tels que la forme quadratique $(x,y)\mapsto ax^2+2bxy+cy^2$ soit définie positive ?}
\reponse{La matrice de la forme quadratique $Q~:~(x,y)\mapsto ax^2+2bxy+cy^2$ dans la base canonique est $\left(
\begin{array}{cc}
a&b\\
b&c
\end{array}
\right)$.

Les valeurs propres de cette matrice sont strictement positives si et seulement si $a+c>0$ et $ac -b^2>0$. L'application $(a,b,c)\mapsto a+c$  est continue sur $\Rr^3$ car linéaire sur $\Rr^3$ qui est de dimension finie et l'application $(a,b,c)\mapsto ac -b^2$ est continue sur $\Rr^3$ en tant que polynôme.

L'ensemble des triplets considéré est l'intersection des images réciproques par ces applications de l'ouvert $]0,+\infty[$ de $\Rr$ et est donc un ouvert de $\Rr^3$.}
    \item \question{Montrer que l'ensemble des matrices stochastiques (matrices $(a_{i,j})_{1\leqslant i,j\leqslant n}\in\mathcal{M}_n(\Rr)$ telles que $\forall (i,j)\in\llbracket1,n\rrbracket^2$, $a_{i,j}\geqslant0$ et $\forall i\in\llbracket1,n\rrbracket$, $\sum_{j=1}^{n}a_{i,j}= 1$) est un compact convexe de $\mathcal{M}_n(\Rr)$.}
\reponse{Notons $\mathcal{S}$ l'ensemble des matrices stochastiques.

\textbullet~Vérifions que $\mathcal{S}$ est borné. Soit $A=(a_{i,j})_{1\leqslant i,j\leqslant n}\in\mathcal{S}$. $\forall(i,j)\in\llbracket1,n\rrbracket^2$, $0\leqslant a_{i,j}\leqslant1$ et donc $\|A\|_\infty\leqslant1$. Ainsi, $\forall A/in\mathcal{S}$, $\|A\|_\infty\leqslant 1$ et donc $\mathcal{S}$ est borné.

\textbullet~Vérifions que $\mathcal{S}$ est fermé. 

Soit $(i,j)\in\llbracket1,n\rrbracket^2$. L' application $f_{i,j}~:~A\mapsto a_{i,j}$ est continue sur $\mathcal{M}_n(\Rr)$ à valeurs dans $\Rr$ car linéaire sur $\mathcal{M}_n(\Rr)$ qui est de dimension finie. $[0,+\infty[$ est un fermé de $\Rr$ car son complémentaire $]-\infty,0[$ est un ouvert de $\Rr$. Par suite, $\left\{A=(a_{k,l})_{1\leqslant k,l\leqslant n}/\;a_{i,j}\geqslant0\right\}=f_{i,j}^{-1}([0,+\infty[)$ est un fermé de $\mathcal{M}_n(\Rr)$ en tant qu'image réciproque d'un fermé par une application continue.

Soit $i\in\llbracket1,n\rrbracket$. L' application $g_i~:~A\mapsto \sum_{j=1}^{n}a_{i,j}$ est continue sur $\mathcal{M}_n(\Rr)$ à valeurs dans $\Rr$ car linéaire sur $\mathcal{M}_n(\Rr)$ qui est de dimension finie. Le singleton $\{1\}$ est un fermé de $\Rr$. Par suite, $\left\{A=(a_{k,l})_{1\leqslant k,l\leqslant n}/\;\sum_{j=1}^{n}a_{i,j}=1\right\}=g_i^{-1}(\{1\})$ est un fermé de $\mathcal{M}_n(\Rr)$ en tant qu'image réciproque d'un fermé par une application continue.

$\mathcal{S}$ est donc un fermé de $\mathcal{M}_n(\Rr)$ en tant qu'intersection de fermé de $\mathcal{M}_n(\Rr)$.

En résumé, $\mathcal{S}$ est un fermé borné de l'espace $\mathcal{M}_n(\Rr)$ qui est de dimension finie et donc $\mathcal{S}$ est un compact de $\mathcal{M}_n(\Rr)$ d'après le théorème de \textsc{Borel}-\textsc{Lebesgue}.

\textbullet~Vérifions que $\mathcal{S}$ est convexe. Soient $(A,B)\in(\mathcal{S})^2$ et $\lambda\in[0,1]$. D'une part, $\forall(i,j)\in\llbracket1,n\rrbracket^2$, $(1-\lambda)a_{i,j}+\lambda b_{i,j}\geqslant0$ et d'autre part, pour $i\in\llbracket1,n\rrbracket$

\begin{center}
$\sum_{j=1}^{n}((1-\lambda)a_{i,j}+\lambda b_{i,j})=(1-\lambda)\sum_{j=1}^{n}a_{i,j}+\lambda\sum_{j=1}^{n}b_{i,j}=(1-\lambda)+\lambda=1$,
\end{center}

ce qui montre que $(1-\lambda)A+\lambda B\in\mathcal{S}$. On a montré que $\forall(A,B)\in\mathcal{S}^2$, $\forall \lambda\in[0,1]$, $(1-\lambda)A+\lambda B\in\mathcal{S}$ et donc $\mathcal{S}$ est convexe.

\begin{center}
\shadowbox{
l'ensemble des matrices stochastiques est un compact convexe de $\mathcal{M}_n(\Rr)$.
}
\end{center}}
    \item \question{Montrer que l'ensemble des matrices diagonalisables de $\mathcal{M}_n(\Rr)$ est connexe par arcs.}
\reponse{Soient $A$ et $B$ deux matrices réelles diagonalisables. Soient $\begin{array}[t]{cccc}
\gamma_1~:&[0,1]&\rightarrow&\mathcal{M}_n(\Rr)\\
 &t&\mapsto&(1-t).A+t.0=(1-t)A
\end{array}
$ et 

$\begin{array}[t]{cccc}
\gamma_2~:&[0,1]&\rightarrow&\mathcal{M}_n(\Rr)\\
 &t&\mapsto &tB
 \end{array}$. Soit enfin $\begin{array}[t]{cccc}
\gamma~:&[0,1]&\rightarrow&\mathcal{M}_n(\Rr)\\
 &t&\mapsto &\left\{
 \begin{array}{l}
 \rule[-4mm]{0mm}{10mm}\gamma_1(2t)\;\text{si}\;t\in\left[0, \frac{1}{2}\right]\\
 \gamma_2(2t-1)\;\text{si}\;t\in\left[ \frac{1}{2},1\right]
 \end{array}
 \right.
 \end{array}$.
 
 
$\gamma_1$ est un chemin continu joignant la matrice $A$ à la matrice nulle et $\gamma_2$ est un chemin continu joignant la matrice nulle à la matrice $B$. Donc $\gamma$ est un chemin continu joignant la matrice $A$ à la matrice $B$. De plus, pour tout réel $t\in[0,1]$, la matrice $\gamma_1(t)=(1-t)A$ est diagonalisable (par exemple, si $A=P\text{diag}(\lambda_i)_{1\leqslant i\leqslant n}P^{-1}$ alors $(1-t)A=P\text{diag}((1-t)\lambda_i)_{1\leqslant i\leqslant n}P^{-1}$) et de même, pour tout réel $t\in[0,1]$, la matrice $\gamma_2(t)=tB$ est diagonalisable. Finalement $\gamma$ est un chemin continu joignant les deux matrices $A$ et $B$ diagonalisables dans $\Rr$, contenu dans l'ensemble des matrices diagonalisables dans $\Rr$. On a montré que

\begin{center}
\shadowbox{
l'ensemble des matrices diagonalisables dans $\Rr$ est connexe par arcs.
}
\end{center}}
\end{enumerate}
}
