\uuid{dESw}
\exo7id{2624}
\auteur{debievre}
\organisation{exo7}
\datecreate{2009-05-19}
\isIndication{true}
\isCorrection{true}
\chapitre{Fonction de plusieurs variables}
\sousChapitre{Dérivée partielle}

\contenu{
\texte{
Soit $f:\R^2\rightarrow \R$ la fonction
d\'efinie par $f(x,y)=(x^2+y^2)^x$ pour $(x,y)\not=(0,0)$ et
$f(0,0)= 1$.
}
\begin{enumerate}
    \item \question{La fonction $f$ est-elle continue en $(0,0)$?}
\reponse{$f(x,y)=(x^2+y^2)^x= \mathrm e^{x \log (x^2+y^2)}=
\mathrm e^{2r\cos \varphi\log r}$. Puisque $\cos \varphi$
est born\'e,
 $\lim_{\begin{smallmatrix} r \to 0\\ r >0
\end{smallmatrix}} 2r\cos \varphi\log r =0
$
d'o\`u
\[
\mathrm{lim}_{\begin{smallmatrix} (x,y) \to (0,0)\\ (x,y) \ne (0,0)
\end{smallmatrix}} f(x,y)=
\mathrm e^{\mathrm{lim}_{\begin{smallmatrix} r \to 0\\ r >0
\end{smallmatrix}} 2r\cos \varphi\log r}=\mathrm e^0=1,
\]
car la fonction exponentielle est continue.}
    \item \question{D\'eterminer les d\'eriv\'ees partielles de $f$ en un point
quelconque distinct de l'origine.}
\reponse{Dans $\R^2 \setminus \{(0,0)\}$ les d\'eriv\'ees partielles
par rapport aux variables $x$ et $y$ se calculent ainsi:
\begin{align*}\frac{\partial f}{\partial x}&
= \left(\ln (x^2+y^2)+\frac {2x^2}{x^2+y^2}\right)(x^2+y^2)^x
\\
\frac{\partial f}{\partial y}&
= \left(\frac {2xy}{x^2+y^2}\right)(x^2+y^2)^x
\end{align*}}
    \item \question{La fonction $f$ admet-elle des d\'eriv\'ees partielles par
rapport \`a $x$, \`a $y$ en $(0,0)$?}
\reponse{Pour que la d\'eriv\'ee partielle
$\frac{\partial f}{\partial x}(0,0)$ existe, il faut et il suffit que
\[
\mathrm{lim}_{\begin{smallmatrix} x \to 0\\ x \ne 0
\end{smallmatrix}}\frac {f(x,0) -1}x =
\mathrm{lim}_{\begin{smallmatrix} x \to 0\\ x \ne 0
\end{smallmatrix}}\frac {(x^2)^x-1}x = 
\mathrm{lim}_{\begin{smallmatrix} x \to 0\\ x > 0
\end{smallmatrix}}\frac {\mathrm e^{2x \log x} -1}x
\]
existe. Si $x>0$,
\[
\frac {\mathrm e^{2x \log x} -1}x = 2 \log x + \varepsilon(x)
\]
o\`u $\mathrm{lim}_{\begin{smallmatrix} x \to 0\\ x > 0
\end{smallmatrix}}\varepsilon(x)=0$. Par cons\'equent,
 la d\'eriv\'ee partielle
$\frac{\partial f}{\partial x}(0,0)$ n'existe pas.
D'autre part,
\[
\frac{\partial f}{\partial y}(0,0)=
\mathrm{lim}_{\begin{smallmatrix} y \to 0\\ y \ne 0
\end{smallmatrix}}\frac {f(0,y) -1}y =
\mathrm{lim}_{\begin{smallmatrix} y \to 0\\ y \ne 0
\end{smallmatrix}}\frac {(y^2)^0-1}y = 0
\]
existe.}
\indication{\begin{enumerate} 
\item Utiliser les coordonn\'ees polaires $(r,\varphi)$ dans le plan
et le fait que $\lim_{\begin{smallmatrix} r \to 0\\ r >0
\end{smallmatrix}} r \log r =0$.
\end{enumerate}}
\end{enumerate}
}
