\uuid{DBhb}
\exo7id{5895}
\auteur{rouget}
\organisation{exo7}
\datecreate{2010-10-16}
\isIndication{false}
\isCorrection{true}
\chapitre{Fonction de plusieurs variables}
\sousChapitre{Différentiabilité}

\contenu{
\texte{
Soit $\begin{array}[t]{cccc}
f~:&GL_n(\Rr)&\rightarrow&M_n(\Rr)\\
 &A&\mapsto&A^{-1}
\end{array}$. Montrer que $f$ est différentiable en tout point de $M_n(\Rr)\setminus\{0\}$ et déterminer sa différentielle.
}
\reponse{
On munit $\mathcal{M}_n(\Rr)$ d'une norme sous-multiplicative $\|\;\|$. Soit $A\in GL_n(\Rr)$. On sait que $GL_n(\Rr)$ est un ouvert de $\mathcal{M}_n(\Rr)$ et donc pour $H\in\mathcal{M}_n(\Rr)$ de norme suffisamment petite, $A+H\in GL_n(\Rr)$. Pour un tel $H$

\begin{center}
$(A+H)^{-1}-A^{-1}=(A+H)^{-1}(I_n-(A+H)A^{-1})=-(A+H)^{-1}HA^{-1}$
\end{center}

puis

\begin{align*}\ensuremath
(A+H)^{-1}-A^{-1}+A^{-1}HA^{-1}&=-(A+H)^{-1}HA^{-1}+A^{-1}HA^{-1}=(A+H)^{-1}(-HA^{-1}+(A+H)A^{-1}HA^{-1})\\
 &=(A+H)^{-1}HA^{-1}HA^{-1}.
\end{align*}

Par suite, $\left\|f(A+H)-f(A)+A^{-1}HA^{-1}\right\|=\left\|(A+H)^{-1}-A^{-1}+A^{-1}HA^{-1}\right\|\leqslant\left\|(A+H)^{-1}\right\|\left\|A^{-1}\right\|^2\left\|H\right\|^2$. 

Maintenant, la formule $M^{-1}= \frac{1}{\text{det}(M)}{^t}(\text{com}(M))$, valable pour tout $M\in GL_n(\Rr)$, et la continuité du déterminant montre que l'application $M\mapsto M^{-1}$ est continue sur l'ouvert $GL_n(\Rr)$. On en déduit que $\left\|(A+H)^{-1}\right\|$ tend vers $\left\|A^{-1}\right\|$ quand $H$ tend vers $0$. Par suite,

\begin{center}
$\lim_{H \rightarrow 0}\left\|(A+H)^{-1}\right\|\left\|A^{-1}\right\|^2\left\|H\right\|=0$ et donc $\lim_{H \rightarrow 0} \frac{1}{\|H\|}\left\|(A+H)^{-1}-A^{-1}+A^{-1}HA^{-1}\right\|=0$.
\end{center}

Comme l'application $H\mapsto -A^{-1}HA^{-1}$ est linéaire, c'est la différentielle de $f$ en $A$.

\begin{center}
\shadowbox{
$\forall A\in GL_n(\Rr)$, $\forall H\in\mathcal{M}_n(\Rr)$, $df_A(H)=-A^{-1}HA^{-1}$.
}
\end{center}
}
}
