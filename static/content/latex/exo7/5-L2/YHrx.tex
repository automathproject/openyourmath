\uuid{YHrx}
\exo7id{4108}
\auteur{quercia}
\organisation{exo7}
\datecreate{2010-03-11}
\isIndication{false}
\isCorrection{true}
\chapitre{Equation différentielle}
\sousChapitre{Equations différentielles linéaires}

\contenu{
\texte{
Soit une application $A$ de classe $\mathcal{C}^\infty$ sur~$\R$ à valeurs dans~$\mathcal{M}_n(\C)$,
telle que les valeurs propres de $A(0)$ aient toutes une partie réelle
strictement positive. Soit~$F$ de classe $\mathcal{C}^\infty$ sur~$\R$, à valeurs
dans~$\C^n$. Montrer qu'il existe $X$ de classe $\mathcal{C}^\infty$ sur~$\R$, à valeurs
dans $\C^n$, solution de $tX'(t) + A(t)X(t) = F(t)$.

{\it Indication~: commencer par le cas $n=1$, $A$ constante.}
}
\reponse{
Pour $n=1$ et $A(t)=a>0$ on trouve après les incantations usuelles
(équation homogène, variation de la constante et mise en forme de l'intégrale) que
$X\ :\ t \mapsto \int_{u=0}^1F(tu)u^{a-1}\,d u$ est l'unique solution
prolongeable en $0$ et qu'elle est de classe $\mathcal{C}^\infty$ sur~$\R$.
Pour $n=1$ et $A$ non constante, on trouve de même~:
$$X(t) =  \int_{u=0}^1F(tu)u^{A(0)-1}\exp\Bigl( \int_{v=t}^{tu}\frac{A(v)-A(0)}v\,d v\Bigr)\,d u$$
et l'on voit que $X$ est $\mathcal{C}^\infty$ en écrivant
$\frac{A(v)-A(0)}v =  \int_{w=0}^1A'(vw)\,d w$.

Pour $n$ quelconque et $A$ constante~: alors la fonction $X\ :\ t \mapsto \int_{u=0}^1u^{A(0)-I}F(tu)\,d u$ est
l'unique solution prolongeable en $0$,
en convenant que $u^{A(0)-I} = \exp((A(0)-I)\ln(u))$ (l'intégrale converge en~$0$
car $u^{A(0)-I} = O(u^{\alpha-1}\ln(u)^n)$ pour tout $\alpha>0$
minorant les parties réelles des valeurs propres de~$A(0)$).

Pour $A$ non constante, on met l'équation sous forme intégrale~:
$$tX'(t) + A(t)X(t) = F(t)\Leftrightarrow X(t) =  \int_{u=0}^1u^{A(0)-I}\{F(tu)-(A(tu)-A(0))X(tu)\}\,d u .$$
Soit $a>0$ à choisir.
Posons $E = \mathcal{C}([-a,a],\C^n)$ et pour $X\in E$~: $$\Phi(X) = t \mapsto \int_{u=0}^1u^{A(0)-I}\{F(tu)-(A(tu)-A(0))X(tu)\}\,d u.$$
On a facilement~: $\Phi(X)\in E$ si $X\in E$ et $\Phi$ est contractante sur~$E$ pour $\|\ \|_\infty$
si $a$ est choisi suffisament petit. Donc l'équation $tX'(t) + A(t)X(t) = F(t)$
admet une solution (unique) définie au voisinage de $0$, et cette solution est prolongeable
en une solution sur~$\R$ car le théorème de Cauchy-Lipschitz s'applique
en dehors de~$0$. Par ailleurs, on a~:
$$\forall\ t\in\R,\ X(t) =  \int_{u=0}^1u^{A(0)-I}\{F(tu)-(A(tu)-A(0))X(tu)\}\,d u,$$
ce qui montre par récurrence sur~$k\in\N$ que $X$ est de classe $\mathcal{C}^k$ sur~$\R$.
}
}
