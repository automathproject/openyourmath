\uuid{OXXP}
\exo7id{4739}
\auteur{quercia}
\organisation{exo7}
\datecreate{2010-03-16}
\isIndication{false}
\isCorrection{false}
\chapitre{Topologie}
\sousChapitre{Topologie des espaces vectoriels normés}

\contenu{
\texte{
Soit $E$ un evn, et $\vec a \in E$, $r > 0$.
On note $\overline B = \overline B(\vec a,r)$ et $\mathring B = \mathring B(\vec a,r)$.
}
\begin{enumerate}
    \item \question{Montrer que $\overline B$ et $\mathring B$ sont convexes.}
    \item \question{Si la norme est euclidienne, montrer que si $\vec u,\vec v \in \overline B$ avec
    $\vec u \ne \vec v$, alors $]\vec u,\vec v\,[ \subset \mathring B$.

    $(]\vec u,\vec v\,[ = \{ (1-t)\vec u + t\vec v \text{ tq } t \in {]0,1[}\,\})$}
    \item \question{En d{\'e}duire que si la norme est euclidienne, toute partie $A$ telle que
    $\mathring B\subset A \subset \overline B$ est convexe.}
    \item \question{Donner un contre-exemple avec une norme non euclidienne.}
\end{enumerate}
}
