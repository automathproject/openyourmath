\uuid{Mou2}
\exo7id{865}
\auteur{bodin}
\organisation{exo7}
\datecreate{1998-09-01}
\isIndication{false}
\isCorrection{true}
\chapitre{Equation différentielle}
\sousChapitre{Résolution d'équation différentielle du deuxième ordre}

\contenu{
\texte{
R\'esoudre l'\'equation suivante :
$$4y^{\prime\prime}+4y^\prime + 5y = \sin x e^{-x/2} .$$
}
\reponse{
% $4y^{\prime\prime}+4y^\prime + 5y = \sin x e^{-x/2} $. L'\'equation
%caract\'eristique a 2 racines complexes $r_1 = -1/2+i$ et
%$r_2=\overline{r_1}$ et les solutions de l'\'equation homog\`ene
%sont :
%$$ y(x) = e^{-x/2}(c_1\cos x +c_2\sin x) \hbox{ avec } c_1,c_2\in \R $$
%On a $\sin x e^{-x/2} =\mathop{\mathrm{Im}}\nolimits(e^{(-1/2+i)x})$, on commence donc par
%chercher une solution $z_p$ de l'\'equation  avec le nouveau
%second membre $e^{(-1/2+i)x}$.Comme $-1/2+i$ est racine de
%l'\'equation caract\'eristique, on cherchera $
%z_p(x)=P(x)e^{(-1/2+i)x}$ avec $P$ de degr\'e 1. Par cons\'equent
%la condition $(*)$ sur $P$ :
%$$ 4P^{\prime\prime}+f^\prime(-1/2+i)P^\prime+f(-1/2+i)P = 1$$
%s'\'ecrit ici : $8iP^\prime =1$ ( $P^{\prime\prime} = 0$,
%$f(-1/2+i)=0$ et $f^\prime(-1/2+i)=8i$), on peut donc prendre
%$P(x)=-i/8x$ et $z_p(x)=-i/8xe^{(-1/2+i)x}$, par cons\'equent sa
%partie imaginaire $y_p(x)=\mathop{\mathrm{Im}}\nolimits(-i/8xe^{(-1/2+i)x})= 1/8 x\sin
%xe^{-x/2}$ est une solution de notre \'equation. Les solutions
%sont donc toutes les fonctions de la forme :
%$$y(x) = e^{-x/2}(c_1\cos x +(c_2+1/8x)\sin x) \hbox{ avec } c_1,c_2\in \R.$$

La solution générale est de la forme
$$y(x) = K_1 \cos(x) e^{-x/2} + K_2 \sin(x) e^{-x/2} - \frac18 x e^{-x/2} \cos(x)$$
($K_1$ et $K_2$ constantes réelles) et les conditions initiales donnent $K_1=0$, $K_2=1/8$.
}
}
