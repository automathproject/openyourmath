\uuid{IqoM}
\exo7id{5851}
\auteur{rouget}
\organisation{exo7}
\datecreate{2010-10-16}
\isIndication{false}
\isCorrection{true}
\chapitre{Topologie}
\sousChapitre{Topologie de la droite réelle}

\contenu{
\texte{
Soit $f~:~\Rr\rightarrow\Rr$ une application uniformément continue sur $\Rr$. Montrer qu'il existe deux réels $a$ et $b$ tels que $\forall x\in\Rr$, $|f(x)|\leqslant a|x| +b$.
}
\reponse{
Soit $f$ une application uniformément continue sur $\Rr$. 
$\exists\alpha> 0/\;\forall(x,y)\in\Rr^2,\;(|x-y|\leqslant\alpha\Rightarrow|f(x)-f(y)|\leqslant1)$.

Soit $x\in\Rr^+$ (le travail est analogue si $x\in\Rr^-$). 

Pour $n\in\Nn$

\begin{center}
$|x-n\alpha|\leqslant\alpha\Leftrightarrow-\alpha\leqslant x-n\alpha\leqslant\alpha\Leftrightarrow \frac{x}{\alpha}-1\leqslant n\leqslant \frac{x}{\alpha}+1\Leftarrow n = E\left( \frac{x}{\alpha}\right).$
\end{center}

On pose $n_0=E\left( \frac{x}{\alpha}\right)$.

\begin{align*}\ensuremath
|f(x)| &\leqslant|f(x)-f(x-\alpha)|+|f(x-\alpha)-f(x-2\alpha)|+\ldots+|f(x-(n_0-1)\alpha)-f(x-n_0\alpha)|+|f(x-n_0\alpha)-f(0)|+|f(0)|\\
 &\leqslant n_0+1+|f(0)|\;(\text{car}\;|x-n_0\alpha-0|\leqslant\alpha)\\
 &\leqslant \frac{x}{\alpha}+2+|f(0)| .
\end{align*}

Ainsi, $\forall x\in\Rr^+$, $|f(x)|\leqslant \frac{x}{\alpha}+2+|f(0)|$. Par symétrie des calculs, $\forall x\in\Rr^-$, $|f(x)|\leqslant \frac{-x}{\alpha}+2+|f(0)|$ et donc $\forall x\in\Rr$, $|f(x)|\leqslant \frac{|x|}{\alpha}+2+|f(0)|$.

\begin{center}
\shadowbox{
$f$ uniformément continue sur $\Rr\Rightarrow\exists(a,b)\in\Rr^2/\;\forall x\in\Rr,\;|f(x)\leqslant a|x|+b$.
}
\end{center}
}
}
