\uuid{gyMl}
\exo7id{4662}
\auteur{quercia}
\organisation{exo7}
\datecreate{2010-03-14}
\isIndication{false}
\isCorrection{true}
\chapitre{Série de Fourier}
\sousChapitre{Autre}

\contenu{
\texte{

}
\begin{enumerate}
    \item \question{Soit $f:\R \to \R$ de classe $\mathcal{C}^1$ 1-périodique,
    $\alpha \in \R\setminus \Q$ et $x \in \R$.

    Montrer que $\frac{f(x) + f(x+\alpha) + \dots + f(x+n\alpha)}{n+1}
    \to  \int_{t=0}^1 f(t)\,d t$ lorsque $n\to\infty$.}
\reponse{Développer $f$ en série de Fourier.}
    \item \question{Montrer que le résultat est encore vrai en supposant seulement $f$ continue.}
\reponse{Densité des polynômes trigonométriques dans $\mathcal{C}^0$.}
    \item \question{En déduire la nature de la série $\sum_{n=1}^\infty \frac{\sin^2 n}n$.}
\reponse{$f(t) = \sin^2(\pi t)$, $\alpha = \frac1\pi$, $x = 0$ :
             $S_n = \sin^21 + \dots + \sin^2n \sim \frac n2$.\par
             Transformation d'Abel : $\sum_{n=1}^N \frac{\sin^2 n}n =
             -\sin^21 + \sum_{n=2}^{N-1} \frac{S_k}{k(k-1)} + \frac{S_N}N
             \to +\infty$ lorsque $N\to\infty$.
    Remarque : on a un raisonnement plus simple en écrivant $2\sin^2(n) = 1-\cos(2n)$.}
\end{enumerate}
}
