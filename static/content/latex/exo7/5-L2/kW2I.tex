\uuid{kW2I}
\exo7id{853}
\auteur{ridde}
\organisation{exo7}
\datecreate{1999-11-01}
\isIndication{false}
\isCorrection{false}
\chapitre{Equation différentielle}
\sousChapitre{Résolution d'équation différentielle du premier ordre}

\contenu{
\texte{
R\'esoudre le syst\`eme diff\'erentiel :
$\begin{cases} \dot{x} (t) = x (t) + y (t) \\ \dot{y} (t) = 3x (t)-y (t) \end{cases}$
et $ \begin{cases} x (0) = 2 \\ y (0) = -2 \end{cases}$.
}
\indication{Chercher deux réels $a,b$ tels que, si on pose $u=x+ay$ et $v=x+by$, alors
$\dot u(t)=A u(t)$ et $\dot v(t)=B v(t)$, où $A,B$ sont des constantes.}
\reponse{
% Correction : Sylvie Ruette
Si $z=x+\lambda y$, alors $\dot z=x(1+3\lambda)+y(1-\lambda)$.
Pour que $\dot z$ soit proportionnel à $z$, il suffit que $\lambda(1+3\lambda)
=1-\lambda$, autrement dit $3\lambda^2+2\lambda-1=0$. Il y a deux
solutions : $a=-1$ et $b=1/3$. On pose $u=x-y$ et $v=x+\frac 13 y$.
On a alors : $\dot u=-2u$ et $\dot v=2v$. On en déduit que les solutions
$u$ et $v$ sont de la forme 
$u(t)=C_1 e^{-2t}$ et $v(t)=C_2e^{2t}$, avec $C_1,C_2\in\mathbb{R}$.
De plus,
$$
\left\{\begin{array}{rcl}
u&=&x-y\\
v&=&x+\frac 13 y
\end{array}\right.
\Longleftrightarrow
\left\{\begin{array}{rcl}
x&=&\frac 14(u+3v)\\
y&=&\frac 34(v-u)
\end{array}\right.
$$
Donc les solutions du système différentiel sont les couples de fonctions 
$(x,y)$ de la forme 
$$x(t)=D_1 e^{-2t}+3D_2e^{2t},\ y(t)=-3D_1 e^{-2t}+3D_2e^{2t},\ 
D_1,D_2\in\mathbb{R}.$$

On veut $x(0)=2$ et $y(0)=-2$. Ceci est équivalent à
$$
\left\{\begin{array}{rcrcr}
D_1&+&3D_2&=&2\\
-3D_1&+&3D_2&=&-2
\end{array}\right.\Longleftrightarrow
\left\{\begin{array}{rcr}
D_1&=&1\\D_2&=&1/3
\end{array}\right.
$$
Conclusion : il y a une unique solution, donnée par
$$x(t)=e^{-2t}+e^{2t},\ y(t)=-3e^{-2t}+e^{2t}.$$
}
}
