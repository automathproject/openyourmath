\uuid{s0Fj}
\exo7id{2634}
\auteur{debievre}
\organisation{exo7}
\datecreate{2009-05-19}
\isIndication{true}
\isCorrection{true}
\chapitre{Fonction de plusieurs variables}
\sousChapitre{Différentiabilité}

\contenu{
\texte{
Utiliser une approximation affine bien choisie pour calculer une valeur approch\'ee des nombres suivants:
\[
\exp[\sin(3.16)\cos(0.02))], \quad \arctan[\sqrt{4.03}-2\exp(0.01)].
\]
}
\indication{Prendre
\begin{align*}
f(x,y)&=\exp[\sin(\pi+x)\cos y]=\exp[-\sin x \cos y ],
\\
h(x,y)&=\arctan[\sqrt{4+x}-2\exp(y)] .
\end{align*}}
\reponse{
\begin{align*}
\frac{\partial f}{\partial x}&= -\cos x \cos y\exp[-\sin x \cos y ]
\\
\frac{\partial f}{\partial y}&= \sin x \sin y\exp[-\sin x \cos y ]
\end{align*}
etc. d'o\`u, avec $\frac{\partial f}{\partial x}(0,0)=-1$ et
$\frac{\partial f}{\partial y}(0,0)=0$,
\[
f(x,y)
=
f(0,0)+\frac{\partial f}{\partial x}(0,0)x+\frac{\partial f}{\partial y}(0,0)y
+\ldots
=1-x +\ldots
\]
Avec $x=0,0184$ on trouve, pour
$\exp[\sin(3.16)\cos(0.02))]$,
la valeur approch\'ee $1-0,0184 =0,9816$. 

N.B. On peut faire mieux si n\'ecessaire : Avec
\begin{align*}
\frac{\partial^2 f}{\partial x^2}&= (\sin x \cos y
+\cos^2 x \cos^2 y)\exp[-\sin x \cos y ]
\\
\frac{\partial^2 f}{\partial x \partial y}&
= (\cos x \sin y+\cos x \cos y \sin x \sin y)\exp[-\sin x \cos y ]
\\
\frac{\partial^2 f}{\partial y^2}&=(\sin x \cos y
+\sin^2 x \sin^2 y)\exp[-\sin x \cos y ]
\end{align*}
on trouve
\[
f(x,y)
=
f(0,0)+\frac{\partial f}{\partial x}(0,0)x+\frac{\partial f}{\partial y}(0,0)y
+\ldots
=1-x++\tfrac 12 x^2 +\ldots
\]
etc.

De m\^eme,
\begin{align*}
\frac{\partial h}{\partial x}&= \frac 1 {2(1+(\sqrt{4+x}-2\exp(y))^2)\sqrt{4+x}}
\\
\frac{\partial h}{\partial y}&= -\frac {2\exp(y)} {1+(\sqrt{4+x}-2\exp(y))^2}
\end{align*}
etc. d'o\`u,  avec $\frac{\partial h}{\partial x}(0,0)=\tfrac 14$ et
$\frac{\partial h}{\partial y}(0,0)=-2$,
\[
h(x,y)
=
h(0,0)+\frac{\partial h}{\partial x}(0,0)x+\frac{\partial h}{\partial y}(0,0)y
+\ldots
=\tfrac 14 x-2y +\ldots .
\]
Avec $x=0,03$ et $y=0,01$ on trouve, pour
$ \arctan[\sqrt{4.03}-2\exp(0.01)]$,
la valeur approch\'ee $0,0075-0,02 =-0,00125$.
}
}
