\uuid{76hL}
\exo7id{5738}
\auteur{rouget}
\organisation{exo7}
\datecreate{2010-10-16}
\isIndication{false}
\isCorrection{true}
\chapitre{Suite et série de fonctions}
\sousChapitre{Suites et séries d'intégrales}

\contenu{
\texte{
Pour $n\in\Nn^*$, on pose $f_n(x)=\left\{\begin{array}{l}
\left(1-\frac{x^2}{n}\right)^n\;\text{si}\;x\in[0,\sqrt{n}]\\
\rule{0mm}{5mm}0\;\text{si}\;x>\sqrt{n}
\end{array}
\right.$.
}
\begin{enumerate}
    \item \question{Montrer que la suite $(f_n)_{n\in\Nn^*}$ converge simplement sur $\Rr^+$ vers la fonction $f~:~x\mapsto e^{-x^2}$.}
\reponse{Soit $x\in[0,+\infty[$. Pour $n> x^2$, $f_n(x)=\text{exp}\left(n\ln\left(1-\frac{x^2}{n}\right)\right)$ et donc $f_n(x)\underset{n\rightarrow+\infty}{=}\text{exp}(-x^2+o(1))$. Donc la suite $(f_n)_{n\in\Nn^*}$ converge simplement sur $\Rr^+$ vers la fonction $f~:~x\mapsto e^{-x^2}$.}
    \item \question{A l'aide de la suite $(f_n)_{n\in\Nn^*}$, calculer l'intégrale de \textsc{Gauss} $\int_{0}^{+\infty}e^{-x^2}\;dx$.}
\reponse{Chaque fonction $f_n$, $n\in\Nn^*$, est continue par morceaux sur $[0,+\infty[$ et nulle au voisinage de $+\infty$. Donc chaque fonction $f_n$, $n\in\Nn^*$, est intégrable sur $[0,+\infty[$.

La fonction $f$ est continue sur $[0,+\infty[$ et négligeable devant $\frac{1}{x^2}$ quand $x$ tend vers $+\infty$. Donc la fonction $f$ est intégrable sur $[0,+\infty[$.

Soit $n\in\Nn^*$. Par convexité de la fonction exponentielle, $\forall u\in\Rr$, $1+u\leqslant e^u$. Par suite, $\forall x\in\left[0,\sqrt{n}\right]$, $0\leqslant1-\frac{x^2}{n}\leqslant e^{-x^2/n}$ puis par croissance de la fonction $t\mapsto t^n$ sur $\Rr^+$, $0\leqslant f_n(x)=\left(1-\frac{x^2}{n}\right)^n\leqslant e^{-x^2}=f(x)$. D'autre part, pour $x>\sqrt{n}$, $f_n(x)=0\leqslant f(x)$. Finalement 

\begin{center}.
$\forall n\in\Nn^*$, $\forall x\in[0,+\infty[$, $|f_n(x)|\leqslant f(x)$.
\end{center}

En résumé,

\textbullet~chaque fonction $f_n$, $n\in\Nn^*$, est continue par morceaux et intégrable sur $[0,+\infty[$,

\textbullet~la suite de fonctions $(f_n)$ converge simplement vers la fonction $f$ sur $[0,+\infty[$ et la fonction $f$ est continue sur $[0,+\infty[$.

\textbullet~$\forall n\in\Nn^*$, $|f_n|\leqslant f$, la fonction $f$ étant intégrable sur $[0,+\infty[$.

D'après le théorème de convergence dominée, la suite $\left(\int_{0}^{+\infty}f_n(x)\;dx\right)_{n\in\Nn^*}$ converge vers $\int_{0}^{+\infty}f(x)\;dx$. Ainsi,

\begin{center}
$\int_{0}^{+\infty}e^{-x^2}\;dx=\lim_{n \rightarrow +\infty}\int_{0}^{\sqrt{n}}\left(1-\frac{x^2}{n}\right)^n\;dx$.
\end{center}

Soit $n\in\Nn^*$. En posant $t=\Arccos\left(\frac{x}{\sqrt{n}}\right)$ et donc $\frac{x^2}{n}=\cos^2t$ et $dx=-\sqrt{n}\sin t\;dt$, on obtient

\begin{center}
$\int_{0}^{\sqrt{n}}\left(1-\frac{x^2}{n}\right)^n\;dx=\int_{\pi/2}^{0}(1-\cos^2t)^n\times(-\sqrt{n}\sin t)\;dt=\sqrt{n}\int_{0}^{\pi/2}\sin^{2n+1}t\;dt=\sqrt{n}W_{2n+1}$,
\end{center}

où $W_n$ est la $n$-ème intégrale de \textsc{Wallis}. Classiquement, $W_n\underset{n\rightarrow+\infty}{\sim}\sqrt{\frac{\pi}{2n}}$ (voir Exercices Maths Sup) et donc

\begin{center}
$\frac{W_{2n+1}}{\sqrt{n}}\underset{n\rightarrow+\infty}{\sim}\sqrt{n}\sqrt{\frac{\pi}{2(2n+1)}}\underset{n\rightarrow+\infty}{\sim}\frac{\sqrt{\pi}}{2}$.
\end{center}

On a montré que

\begin{center}
\shadowbox{
$\int_{0}^{+\infty}e^{-x^2}\;dx=\frac{\sqrt{\pi}}{2}$.
}
\end{center}}
\end{enumerate}
}
