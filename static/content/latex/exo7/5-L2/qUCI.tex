\uuid{qUCI}
\exo7id{4805}
\auteur{quercia}
\organisation{exo7}
\datecreate{2010-03-16}
\isIndication{false}
\isCorrection{true}
\chapitre{Topologie}
\sousChapitre{Topologie des espaces vectoriels normés}

\contenu{
\texte{
$E$ est l'ensemble des fonctions de $\R$ dans $\R$ continues et born{\'e}es sur $\R$.

Pour $p \in \N$ et $f \in E$ on pose : 
$N_{p}(f) = \sup\{|t^{p} e^{-|t|} f(t)|,\ t\in\R\}$.
}
\begin{enumerate}
    \item \question{Montrer que $N_{p}$ est une norme sur $E$.}
    \item \question{Soit $c \in \R$ et ${P_{c}} : E \to \R, f \mapsto {f(c).}$
    {\'E}tudier la continuit{\'e} de $P_c$ sur $(E,N_p)$.}
    \item \question{Montrer que, pour $p$ et $q$ distincts dans $\N$, les normes $N_p$ et $N_q$ ne sont pas {\'e}quivalentes.}
\reponse{
Si $c\ne 0$ alors $P_c$ est continue pour toutes les normes $N_p$
    et $\|\!|P_c|\!\|_{N_p}=|c|^{-p}e^{|c|}$.
    Par contre $P_0$ n'est continue que pour~$N_0$ car si $p>0$ alors
    $N_p(x \mapsto e^{-n|x|})=\frac{p^pe^{-p}}{(n+1)^p}\xrightarrow[n\to\infty]{}0$
    donc la suite $(x \mapsto e^{-n|x|})$ converge vers la fonction nulle
    pour $N_p$, mais $P_0(x \mapsto e^{-n|x|})=1\rlap{\hskip9pt/}\to0$.
Si $p<q$ alors $N_p(x \mapsto e^{-n|x|})/N_q(x \mapsto e^{-n|x|}) \to \infty$ losrque $n\to \infty$.
}
\end{enumerate}
}
