\uuid{BlVz}
\exo7id{4788}
\auteur{quercia}
\organisation{exo7}
\datecreate{2010-03-16}
\isIndication{false}
\isCorrection{true}
\chapitre{Topologie}
\sousChapitre{Topologie des espaces vectoriels normés}

\contenu{
\texte{
Soit $N\in\mathcal{M}_n(\C)$. Montrer que $N$ est nilpotente si et seulement si
la matrice nulle est adh{\'e}rente {\`a} l'ensemble
$\{P^{-1}NP,\ P\in GL_n(\C)\}$.
}
\reponse{
Si $N$ est nilpotente, on peut se ramener au cas o{\`u} $N$
est triangulaire sup{\'e}rieure stricte.

Soit alors $P=\mathrm{diag}(1,\alpha,\dots,\alpha^{n-1})$ avec $\alpha\in\C^*$.
Le coefficient g{\'e}n{\'e}ral de $P^{-1}NP$ est $\alpha^{j-i}N_{ij}\xrightarrow[\alpha\to0]{}0$.

R{\'e}ciproquement, s'il existe une suite $(N_k)$ de matrices semblables
{\`a}~$N$ convergeant vers la matrice nulle, alors par continuit{\'e} du
polyn{\^o}me caract{\'e}ristique, on a $\chi_N=(-X)^n$ et $N$ est nilpotente.
}
}
