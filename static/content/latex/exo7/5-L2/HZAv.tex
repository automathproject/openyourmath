\uuid{HZAv}
\exo7id{4373}
\auteur{quercia}
\organisation{exo7}
\datecreate{2010-03-12}
\isIndication{false}
\isCorrection{true}
\chapitre{Intégration}
\sousChapitre{Intégrale de Riemann dépendant d'un paramètre}

\contenu{
\texte{
Soit~$(a_n)_{n\in\N}$ la suite définie par $a_0=1$ et $a_n =
\frac1{n!} \int_{t=0}^1t(t-1)\dots(t-n)\,d t$.
}
\begin{enumerate}
    \item \question{Quel est le rayon de convergence de la série entière $\sum_{n=0}^\infty
    a_nx^n$~?}
\reponse{Pour $0\le t\le 1$ on a $t(1-t)(n-1)! \le t(1-t)\dots(n-t)\le n!$
    d'où $\frac1{6n}\le|a_n|\le1$ et $R=1$.}
    \item \question{Donner un équivalent de~$a_n$.}
\reponse{$(-1)^na_n= \int_{t=0}^1t(1-t)(1-t/2)\dots(1-t/n)\,d t$.
    Pour $0\le x\le\frac12$ on a $x\le-\ln(1-x)\le x+x^2$ (étude
    de fonction) donc pour $k\ge 2$ et $0\le t\le 1$~:
    $e^{-t/k-t^2/k^2}\le 1-t/k\le e^{-t/k}$ d'où~:
 
    $$b_n =  \int_{t=0}^1t(1-t)e^{-t(H_n-1)-t^2K_n}\,d t\le (-1)^na_n
    \le  \int_{t=0}^1te^{-tH_n}\,d t = c_n$$
    avec $H_n = 1+1/2+\dots+1/n$ et $K_n = 1/2^2+\dots+1/n^2$.

    \'Equivalent du majorant~:
    $$c_n = \frac{1-(1+H_n)e^{-H_n}}{H_n^2}\sim\frac1{H_n^2}.$$

    \'Equivalent du minorant~:
    \begin{align*}b_n&\ge  \int_{t=0}^1t(1-t)(1-t^2K_n)e^{-t(H_n-1)}\,d t\cr
    &=  \int_{t=0}^1te^{-t(H_n-1)}\,d t -  \int_{t=0}^1t^2(1+t(1-t)K_n)e^{-t(H_n-1)}\,d t\cr
    &\ge  \int_{t=0}^1te^{-t(H_n-1)}\,d t - (1+{\textstyle\frac14}K_n) \int_{t=0}^1t^2e^{-t(H_n-1)}\,d t\cr
    &\ge \frac{1-H_ne^{1-H_n}}{(H_n-1)^2} - (1+{\textstyle\frac14}K_n)\frac{2-(H_n^2+1)e^{1-H_n}}{(H_n-1)^3}\cr
    &\sim\frac1{H_n^2}.\cr\end{align*}
    Finalement, $a_n\sim\frac{(-1)^n}{H_n^2}\sim\frac{(-1)^n}{\ln^2n}$.}
\end{enumerate}
}
