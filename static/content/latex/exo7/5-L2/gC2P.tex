\uuid{gC2P}
\exo7id{4081}
\auteur{quercia}
\organisation{exo7}
\datecreate{2010-03-11}
\isIndication{false}
\isCorrection{true}
\chapitre{Equation différentielle}
\sousChapitre{Equations différentielles linéaires}

\contenu{
\texte{
Soit $f$ continue et intégrable sur~$\R$. On considère l'équation
différentielle $(E)$ : $y' - y + f = 0$.
}
\begin{enumerate}
    \item \question{Montrer que $(E)$ admet une unique solution $F$ bornée sur~$\R$.}
\reponse{Formule de Duhamel : $y(t) = - \int_{x=0}^t e^{t-x}f(x)\,d x + \lambda e^t$.

    Par convergence dominée, l'intégrale tend vers $0$ quand $t$ tend vers $-\infty$
    donc toutes les solutions de~$(E)$ sont bornées au voisinage de $-\infty$.

    Pour $t\ge 0$ on a $y(t) = e^t\Bigl(\lambda- \int_{x=0}^t e^{-x}f(x)\,d x\Bigr)$
    donc il y a au plus une valeur de $\lambda$ telle que $y$ soit éventuellement
    bornée au voisinage de $+\infty$, c'est $\lambda =  \int_{x=0}^{+\infty} e^{-x}f(x)\,d x$.

    Pour ce choix on a~:
    $|y(t)| = \Bigl| \int_{x=t}^{+\infty}e^{t-x}f(x)\,d x\Bigr|
    \le \int_{x=t}^{+\infty}|f(x)|\,d x \to 0$ lorsque $t\to+\infty$.}
    \item \question{Montrer que $F$ est intégrable sur $\R$ et comparer $ \int_{-\infty}^{+\infty} F$ et  $ \int_{-\infty}^{+\infty} f$.}
\reponse{\begin{align*} \int_{t=a}^b|F(t)|\,d t
    &\le  \int_{t=a}^b \int_{x=t}^{+\infty}e^{t-x}|f(x)|\,d xd t\\
    &\le  \int_{x=a}^b \int_{t=a}^xe^{t-x}|f(x)|\,d td x
       +  \int_{x=b}^{+\infty} \int_{t=a}^be^{t-x}|f(x)|\,d td x\\
    &\le  \int_{x=a}^b(1-e^{a-x})|f(x)|\,d x
       +  \int_{x=b}^{+\infty}(e^{b-x}-e^{a-x})|f(x)|\,d td x\\
    &\le  \int_{x=-\infty}^{+\infty}|f(x)|\,d x\\ \end{align*}
    donc $F$ est intégrable. $F' = F-f$ est aussi intégrable et
    $ \int_{t=-\infty}^{+\infty}F'(t)\,d t = \Bigl[F(t)\Bigr]_{t=-\infty}^{+\infty} = 0$
    d'où $ \int_{-\infty}^{+\infty} F = \int_{-\infty}^{+\infty} f$.}
\end{enumerate}
}
