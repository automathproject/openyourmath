\uuid{ZOtv}
\exo7id{4132}
\auteur{quercia}
\organisation{exo7}
\datecreate{2010-03-11}
\isIndication{false}
\isCorrection{true}
\chapitre{Equation différentielle}
\sousChapitre{Equations différentielles non linéaires}

\contenu{
\texte{
On définit une suite de fonctions sur $[0,1]$ de la manière suivante~:
$f_0$ est la fonction constante $1$ et pour tout $x\in{[0,1]}$ et $n\in\N$,
$f_{n+1}(x) = 1 +  \int_{t=0}^x f_n(t-t^2)\,d t$.
}
\begin{enumerate}
    \item \question{En étudiant $f_{n+1}-f_n$ montrer que la suite $(f_n)$ converge
    uniformément sur~$[0,1]$. On note $f$ sa limite.}
\reponse{$f_{n+1}(x) - f_n(x) =  \int_{t=0}^x (f_n-f_{n-1})(t-t^2)\,d t$
    donc par récurrence $f_{n+1}-f_n \ge 0$.

    De plus $f_{n+1}(x) - f_n(x) \le x\|f_n-f_{n-1}\|_\infty$
    d'où $f_{n+2}(x) - f_{n+1}(x) \le \|f_n-f_{n-1}\|_\infty \int_{t=0}^x(t-t^2)\,d t
                                  \le \frac16\|f_n-f_{n-1}\|_\infty$
    et $\|f_{n+2} - f_{n+1}\|_\infty \le \frac16\|f_n-f_{n-1}\|_\infty$
    ce qui prouve que la série télescopique $\sum(f_{n+1}-f_n)$ est normalement
    convergente.}
    \item \question{Montrer que $f$ est de classe $\mathcal{C}^\infty$ sur $[0,1]$.
    Que valent $f'(0)$ et $f'(1)$~?}
\reponse{Par passage à la limite uniforme sous le signe intégral
    on a $f(x) = 1 +  \int_{t=0}^x f(t-t^2)\,d t$ d'où $f$ est $\mathcal{C}^1$
    et $f'(x) = f(x-x^2)$ ce qui entraine le caractère $\mathcal{C}^\infty$ de $f$ par
    récurrence.
    $f'(0) = f'(1) = f(0) = 1$.}
    \item \question{\'Etudier la concavité de~$f$.}
\reponse{$f'$ est positive d'après l'équation différentielle vérifiée par~$f$
    et $f''(x) = (1-2x)f'(x-x^2)$ est du signe de $1-2x$, c'est-à-dire que $f$
    est convexe sur $[0,\frac12]$ et concave sur $[\frac12,1]$.}
    \item \question{Montrer que pour tout $x\in{[0,1]}$ on a $1+x \le f(x)\le \exp(x)$.}
\reponse{$1+x = f_1(x) \le f(x) = f_1(x) + \sum_{k=1}^\infty(f_{k+1}(x) - f_k(x))$.
    De plus, $f'(x) = f(x-x^2) \le f(x)$ d'où $x \mapsto f(x)e^{-x}$ est décroissante
    et vaut $1$ en $0$ ce qui prouve que $f(x) \le e^x$.}
\end{enumerate}
}
