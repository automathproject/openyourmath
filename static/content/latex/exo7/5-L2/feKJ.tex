\uuid{feKJ}
\exo7id{2637}
\auteur{debievre}
\organisation{exo7}
\datecreate{2009-05-19}
\isIndication{true}
\isCorrection{true}
\chapitre{Fonction de plusieurs variables}
\sousChapitre{Différentielle seconde}

\contenu{
\texte{
Soit $g\colon\R_{>0}\times \R_{>0}\to \R$ une fonction
de classe $C^1$
telle que $g(1,1)=3$ et dont la diff\'erentielle vaille
\begin{equation}\label{eq}
\mathrm{d} g= (2xy+y^2)\mathrm{d} x + (x^2+2xy) \mathrm{d} y.
\end{equation} 
Soit 
\[
h\colon \R_{>0}\times\R_{>0} \longrightarrow \R_{>0}\times\R_{>0}
\]
l'application de classe $C^1$ d\'efinie par 
\[
h(x,y)= (u(x,y), v(x,y))=(x^2y, xy^2)\in \R_{>0}\times \R_{>0}. 
\]
}
\begin{enumerate}
    \item \question{Calculer $\mathrm{d} u+\mathrm{d} v$.}
\reponse{Un calcul imm\'ediat donne $\mathrm d u+\mathrm d v =\mathrm d g$.}
    \item \question{D\'eterminer $g$  \`a partir du calcul 
pr\'ec\'edent et (\ref{eq}), et sans autre calcul.}
\reponse{Par cons\'equent, $g=u+v+c$ o\`u la constante $c$ est d\'etermin\'ee par
la condition
\[
3=g(1,1)= u(1,1)+v(1,1)+c=1+1+c
\]
d'o\`u $c=1$.}
    \item \question{Montrer que $h$ est une bijection. (On pourra calculer explicitement $h^{-1}$.)}
\reponse{Un calcul direct montre que l'application r\'eciproque 
\[
k \colon \R_{>0}\times\R_{>0} \longrightarrow \R_{>0}\times\R_{>0}
\]
de $h$ 
est donn\'ee par la formule
\[
k(u,v)= (x(u,v),y(u,v))= 
\left(\left(\frac{u^2}{v}\right)^{1/3},\left(\frac{v^2}{u}\right)^{1/3}\right).
\]}
    \item \question{D\'eterminer explicitement $\mathrm{d} (g\circ h^{-1})$.}
\reponse{$\mathrm d (g \circ k)=\mathrm d (u \circ k)+ \mathrm d (v \circ k)
=\mathrm d u +\mathrm d v$ car $u(k(u,v))=u$ et $v(k(u,v))=v$.}
    \item \question{Calculer les matrixes jacobiennes
$J_h(x,y)$ et $J_{h^{-1}}(u,v)$ et v\'erifier par un calcul direct que
\[
J_h(x,y)J_{h^{-1}}(h(x,y))=I_2,
\]
 o\`u $I_2$ est la matrice identit\'e d'ordre 2.}
\reponse{Un calcul imm\'ediat donne 
$$J_h=\left [\begin{matrix} 2xy & x^2\\ y^2 &2xy \end{matrix}\right ], \quad 
J_k=\left [\begin{matrix} 
\frac 23 (uv)^{-1/3} & -\frac {u^{2/3}}{3v^{4/3}}
\\  -\frac {v^{2/3}}{3u^{4/3}} &\frac 23 (uv)^{-1/3} 
\end{matrix}
\right ]$$ d'o\`u $J_h(x,y)J_{k}(h(x,y))=I_2$.}
\indication{On va d\'eterminer une primitive d'une forme diff\'erentielle
de degr\'e 1 par un changement de variables tel que,
dans les nouvelles variables, la primitive soit presque \'evidente.}
\end{enumerate}
}
