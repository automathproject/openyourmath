\uuid{p40I}
\exo7id{4752}
\auteur{quercia}
\organisation{exo7}
\datecreate{2010-03-16}
\isIndication{false}
\isCorrection{false}
\chapitre{Topologie}
\sousChapitre{Topologie des espaces vectoriels normés}

\contenu{
\texte{
Soit $A \in \mathcal{M}_p(\R)$. On suppose que la suite de matrices :
$A_n = I + A + A^2 + \dots + A^n$ converge vers une matrice $B$.
Montrer que $I-A$ est inversible, et $B = (I-A)^{-1}$.

Remarque : La r{\'e}ciproque est fausse, c'est {\`a} dire que la suite $(A_n)$
peut diverger m{\^e}me si $I-A$ est inversible. Chercher un contre-exemple.
}
}
