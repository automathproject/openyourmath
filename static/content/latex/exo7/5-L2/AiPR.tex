\uuid{AiPR}
\exo7id{5749}
\auteur{rouget}
\organisation{exo7}
\datecreate{2010-10-16}
\isIndication{false}
\isCorrection{true}
\chapitre{Série entière}
\sousChapitre{Autre}

\contenu{
\texte{
Soient $P_n =\sum_{k=0}^{n}\frac{X^k}{k!}$ et $R > 0$ donné. Montrer que pour $n$ suffisamment grand, $P_n$ n'a pas de racine dans le disque fermé de centre $0$ et de rayon $R$.
}
\reponse{
Soit $R > 0$. Notons $D_R$ le disque fermé de centre $0$ et de rayon $R$.
Soient $z\in D_R$ et $n$ un entier naturel.

\begin{center}
$|P_n(z)| =\left|e^z - (e^z-P_n(z))\right|\geqslant|e^z|-|e^z-P_n(z)|\geqslant e^{-R}-|e^z-P_n(z)|$.
\end{center}

On sait que la suite de polynômes $(P_n)_{n\in\Nn}$ converge uniformément vers la fonction exponentielle sur $D_R$.Donc il existe un entier $n_0$ tel que pour tout $z\in D_R$ et tout entier $n\geqslant n_0$, $\left|e^z-P_n(z)\right|\leqslant\frac{1}{2}e^{-R}$. Pour $n\geqslant n_0$ et $z\in D_R$, $\left|P_n(z)\right|\geqslant\frac{1}{2}e^{-R}>0$ et $P_n$ ne s'annule pas dans $D_R$.
}
}
