\uuid{TQuA}
\exo7id{7523}
\auteur{mourougane}
\organisation{exo7}
\datecreate{2021-08-10}
\isIndication{false}
\isCorrection{false}
\chapitre{Topologie}
\sousChapitre{Connexité}

\contenu{
\texte{
Le but de cet exercice est de montrer qu'une partie $P$ ouverte de $\Cc$ est connexe
si et seulement si pour tout couple $(a,b)$ de points de $P$ il existe une ligne polygonale incluse dans $P$
qui joint $a$ et $b$.
}
\begin{enumerate}
    \item \question{On suppose d'abord que pour tout couple $(a,b)$ de points de $P$ il existe une ligne polygonale incluse dans $P$
qui joint $a$ et $b$. Supposons que $P$ n'est pas connexe. Montrer qu'il existe deux ouverts disjoints $A$ et $B$,
un point $a$ de $A$, un point $b$ de $B$ tel que le segment $[ab]$ soit inclus dans~$P$.}
    \item \question{Noter \begin{eqnarray*}
       \alpha&:=&\{t\in[0,1]\ \ /\ \ ta+(1-t)b\in A\}\\
       \beta&:=&\{t\in[0,1]\ \ /\ \ ta+(1-t)b\in B\}\\
       \end{eqnarray*}
Montrer que $\alpha$ et $\beta$ sont non vides ouverts et disjoints
et utiliser la connexité de $[0,1]$ pour obtenir une contradiction.}
    \item \question{On suppose réciproquement que $P$ est connexe et on fixe un point $o$ de $P$.
 On considère $$A:=\{b\in P \text{ il y a une ligne polygonale incluse dans } P \text{ qui joint } o \text{ et } b.\}.$$
 Montrer que $A$ est ouvert dans $P$, en montrant que pour tout point $b$ de $A$, il existe une boule $B(b,\epsilon)$ incluse dans $A$.}
    \item \question{Avec les notations précédentes, montrer que $P-A$ est ouvert dans $P$, en montrant que pour tout point $z$ de $P-A$
 et toute boule $B(z,\epsilon)$ incluse dans $P$, $B(z,\epsilon)$ est incluse dans $P-A$.}
    \item \question{Conclure.}
\end{enumerate}
}
