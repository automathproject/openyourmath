\uuid{C7D1}
\exo7id{7688}
\auteur{mourougane}
\organisation{exo7}
\datecreate{2021-08-11}
\isIndication{false}
\isCorrection{true}
\chapitre{Sous-variété}
\sousChapitre{Sous-variété}

\contenu{
\texte{
Calculer la torsion de la courbe de $\Rr^3$ paramétrée par $c~:~t\mapsto (4\cos t,5-5\sin t, -3\cos t)$.
La courbe  est-elle plane ?
}
\reponse{
Le vecteur vitesse est $\dot{c}(t)=(-4\sin t,-5\cos t,3\sin t)$.
Sa norme est $5$. Par conséquent, le paramétrage par 
$C~:~t\mapsto (4\cos t/5,5-5\sin t/5, -3\cos t/5)$ est un paramétrage par la longueur d'arc.
Le nouveau vecteur vitesse est $\dot{C}(t)=(-4/5\sin t/5,-\cos t/5,3/5\sin t/5)$.
Le vecteur accélération est $\ddot{C}(t)=(-4/25\cos t/5,1/5\sin t/5,3/25\cos t/5)$.
Sa norme $1/5$ est la courbure $\kappa$. Le vecteur normal unitaire est donc
$n(t)=(-4/5\cos t/5,\sin t/5,3/5\cos t/5)$.
Le vecteur binormal est $\dot{C}(t)\times n(t)=(-3/5,0,-4/5)$. Comme il est constant,
la torsion est nulle par les formules de Frenet.

Oui la courbe est plane, puisque la torsion est nulle. On peut aussi remarquer que la courbe est dans le plan d'équation
$3X+4Z=0$.
}
}
