\uuid{pcqZ}
\exo7id{6269}
\auteur{queffelec}
\organisation{exo7}
\datecreate{2011-10-16}
\isIndication{false}
\isCorrection{false}
\chapitre{Difféomorphisme, théorème d'inversion locale et des fonctions implicites}
\sousChapitre{Difféomorphisme, théorème d'inversion locale et des fonctions implicites}

\contenu{
\texte{

}
\begin{enumerate}
    \item \question{Soit $f$ la fonction de ${\Rr}^2$ dans ${\Rr}$ définie par
$f(x,y)=x^3+y^3-3xy$ et $C$ l'ensemble des $(x,y)\in\Rr^2$ tels que
$f(x,y)=0$.

En quels points $(a,b)$ peut-on appliquer le théorème des fonctions implicites
? Calculer la dérivée de la fonction implicite lorsqu'elle existe et
écrire l'équation de la tangente à $C$.}
    \item \question{Montrer que l'équation $e^x+e^y+x+y-2=0$ définit au voisinage de $0$ une
fonction implicite $\varphi$ de $x$ dont on calculera le développement limité à
l'ordre $3$ en $0$.}
    \item \question{Montrer que les équations $x+y-zt=0,\ xy-z+t=0$ définissent au voisinage de
$(0,1)$ deux fonctions implicites $x=\varphi_1(z,t),\ y=\varphi_2(z,t)$ avec $\varphi_1(0,1)=1$,
dont on calculera les différentielles en ce point.}
\end{enumerate}
}
