\uuid{a6Zg}
\exo7id{2555}
\auteur{tahani}
\organisation{exo7}
\datecreate{2009-04-01}
\isIndication{false}
\isCorrection{true}
\chapitre{Différentielle d'ordre supérieur, formule de Taylor}
\sousChapitre{Différentielle d'ordre supérieur, formule de Taylor}

\contenu{
\texte{
Trouver le volume maximum
d'une boite rectangulaire inscrite dans la sph\`ere d'\'equation
$x^2+y^2+z^2=R^2$.
}
\reponse{
Le volume d'une boite \'etant invariant par rotations, on peut
toujours supposer que toutes les boites sont centr\'ees \`a
l'origine et on des cot\'es parall\`eles aux axes de
coordonn\'ees. Par cons\'equent, la donn\'ee d'un point $(x,y,z)$
sur la sph\`ere d\'efinit de mani\`ere unique une boite
rectangulaire dont l'un des sommets est le point $(x,y,z)$. On
prendra $x,y$ et $z$ positifs car une telle boite a toujours un
sommet dans le secteur positif de l'espace. Par cons\'equent, on
doit maximiser la fonction volume $g(x,y,z)=8xyz$ sur la
sous-vari\'et\'e $S$ d\'efinie par l'\'equation $f=0$ avec
$f(x,y,z)=x^2+y^2+z^2-R^2$. Un point critique de $g$ sur $S$
v\'erifie $$Dg(x,y,z)=\lambda Df(x,y,z)$$ et $f(x,y,z)=0$. On a
obtient alors le syst\`eme d'\'equations:
$$\begin{array}{c}
8yz=2x \\
8xz=2y \\
8xy=2z \\
x^2+y^2+z^2=R^2
\end{array}$$

Par cons\'equent
$$\begin{array}{c}
8xyz=2x^2 \\
8xyz=2y^2 \\
8xyz=2z^2 \\
x^2+y^2+z^2=R^2
\end{array}$$
et donc comme $x, y$ et $z$ sont positifs, on a
$x=y=z=\frac{R}{\sqrt{3}}$. Or $g$ est continue et $S^+=S\cap \{x
\geq 0; y \geq 0; z \geq0\}$ est un compact. Comme $g$ est nulle
sur le bord de $S^+$, le maximum de $g$ est ateint en un point
critique de $g$ dans l'int\'erieur de $S^+$. Le seul point
critique de $g$ est donc bien ce maximum recherch\'e. Ici, il n'y
a pas eu besoin de calculer la hessienne de $g$ sur $S$ par la
formule: $$H=D^2g-\lambda D^2f$$ o\`u $\lambda$ est le coefficient
de lagrange trouv\'e pr\'ec\'edemment.
}
}
