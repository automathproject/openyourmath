\uuid{t2LG}
\exo7id{5970}
\auteur{tumpach}
\organisation{exo7}
\datecreate{2010-11-11}
\isIndication{false}
\isCorrection{true}
\chapitre{Autre}
\sousChapitre{Autre}

\contenu{
\texte{
\textbf{Th\'eor\`eme.}
L'espace $L^{\infty}(\mathbb{R}^{n})$ n'est pas s\'eparable.

\bigskip

Le but de cet exercice  est de d\'emontrer ce th\'eor\`eme.
}
\begin{enumerate}
    \item \question{Soit $E$ un espace de Banach. On suppose qu'il existe une
famille $(O_{i})_{i\in I}$ telle que
\begin{itemize}}
\reponse{Soit $E$ un espace de Banach. On suppose qu'il existe une
famille $(O_{i})_{i\in I}$ telle que
\begin{itemize}}
    \item \question{Pour tout $i\in I$, $O_i$ est un ouvert non vide de
$E$.}
\reponse{Pour tout $i\in I$, $O_i$ est un ouvert non vide de
$E$.}
    \item \question{$O_i \cap O_j = \emptyset $ si $i\neq j$.}
\reponse{$O_i \cap O_j = \emptyset $ si $i\neq j$.}
    \item \question{$I$ n'est pas d\'enombrable.
\end{itemize}
Montrer que $E$ n'est pas s\'eparable. (On pourra raisonner par
l'absurde).}
\reponse{$I$ n'est pas d\'enombrable.
\end{itemize}
Supposons que $E$ est s\'eparable. Soit $(u_{n})_{n\in\mathbb{N}}$
une suite dense dans $E$. Gr\^ace \`a (a), pour chaque $i\in I$,
$O_{i} \cap \{u_{n}, n\in\mathbb{N}\} \neq \emptyset$. On choisit
$n(i)$ tel que $u_{n(i)}\in O_i$. On a $n(i) = n(j) \Rightarrow
u_{n(i)} = u_{n(j)} \in O_{i}\cap O_j$ donc $i = j$ par (b). Ainsi
l'application $i \mapsto n(i)$ est injective. Par suite $I$ est
d\'enombrable ce qui contredit (c).}
    \item \question{Pour tout $a\in \mathbb{R}^{n}$,  on pose $f_{a} =
\mathbf{1}_{\mathcal{B}(a, 1)}$ o\`u $\mathcal{B}(a, 1)$ est la boule de
$\mathbb{R}^n$ de rayon $1$ centr\'ee en $a$. Montrer que la
famille
$$
O_a = \{ f\in L^{\infty}(\mathbb{R}^n), ~ \|f - f_a \|_{\infty} <
\frac{1}{2}\},
$$
o\`u $a$ parcourt les points de $\mathbb{R}^n$, satisfait (a), (b)
et (c). Conclure.}
\reponse{Pour tout $a\in \mathbb{R}^{n}$,  on pose $f_{a} =
\mathbf{1}_{\mathcal{B}(a, 1)}$ o\`u $\mathcal{B}(a, 1)$ est la boule de
$\mathbb{R}^n$ de rayon $1$ centr\'ee en $a$. Soit la famille
$$
O_a = \{ f\in L^{\infty}(\mathbb{R}^n), ~ \|f - f_a \|_{\infty} <
\frac{1}{2}\},
$$
o\`u $a$ parcourt les points de $\mathbb{R}^n$. L'ensemble des
points de $\mathbb{R}^n$ n'est pas d\'enombrable, donc (c) est
v\'erifi\'e.  L'ensemble $O_a$ est la boule ouverte de
$L^{\infty}(\mathbb{R}^n)$ de rayon $\frac{1}{2}$ centr\'ee en
$f_{a}$. En particulier (a) est v\'erifi\'e. Remarquons que
lorsque $a\neq b$, on a $\|f_{a} - f_{b}\|_{\infty} = 1$.
Supposons qu'il existe $f \in O_{a}\cap O_b$ avec $a\neq b$. Alors
$$
\|f_{a} - f_{b}\|_{\infty} \leq \|f_a - f\|_{\infty} + \|f -
f_{b}\|_{\infty} < \frac{1}{2} + \frac{1}{2} = 1.
$$
ce qui n'est pas possible. Donc (b) est v\'erifi\'e. On en conclut
que $L^{\infty}(\mathbb{R}^n)$ n'est pas s\'eparable.}
\end{enumerate}
}
