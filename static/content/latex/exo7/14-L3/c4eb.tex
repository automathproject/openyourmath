\uuid{c4eb}
\exo7id{5949}
\auteur{tumpach}
\organisation{exo7}
\datecreate{2010-11-11}
\isIndication{false}
\isCorrection{true}
\chapitre{Théorème de convergence dominée}
\sousChapitre{Théorème de convergence dominée}

\contenu{
\texte{
Donner un exemple de fonction
$f:\mathbb{R}\rightarrow \mathbb{R}$ qui est int\'{e}grable au
sens de Lebesgue mais pas au sens de Riemann.
}
\reponse{
La fonction de Dirichlet restreint
\`a l'intervalle $[a,b]$,
$f(x)=\mathbf{1}_\mathbb{Q}\left|_{[a,b]}(x)\right.,$ est int\'{e}grable
au sens de Lebesgue et son int\'egrale par rapport \`a la mesure
de Lebesgue vaut $0$. Mais elle n'est pas int\'{e}grable au sens
de Riemann: $\underline{S}(f,\tau)=0$ et
$\overline{S}(f,\tau)=b-a\;$ pour toute subdivision $\tau$ de
l'intervalle $[a,b].$
}
}
