\uuid{spfV}
\exo7id{6315}
\auteur{queffelec}
\organisation{exo7}
\datecreate{2011-10-16}
\isIndication{false}
\isCorrection{false}
\chapitre{Solution maximale}
\sousChapitre{Solution maximale}

\contenu{
\texte{
Soit $f,g$ deux fonctions réelles continues sur un intervalle $I$ de $\Rr$,
$x_0\in I$, et on suppose que $g^{-1}(0)$ est fini ou discret; montrer que la
solution de l'équation
$y'=f(x)g(y),\ y(x_0)=y_0$ s'obtient sous forme implicite, et préciser son
intervalle de définition.

Exemples : $f(x)=\displaystyle{1\over\sqrt{1-x^2}},\ g(y)=\sqrt{1-y^2}$;
$f(x)=1,\  g(y)=y-y^2$.
}
}
