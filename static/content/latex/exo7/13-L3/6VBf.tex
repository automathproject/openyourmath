\uuid{6VBf}
\exo7id{6344}
\auteur{queffelec}
\organisation{exo7}
\datecreate{2011-10-16}
\isIndication{false}
\isCorrection{false}
\chapitre{Système linéaire à  coefficients constants}
\sousChapitre{Système linéaire à  coefficients constants}

\contenu{
\texte{
On considère le système linéaire $x'(t) = A(t) x(t)$, où
$A\in {\cal C} ([0, \infty ))$. Soit $\varphi $ une solution
non-triviale de ce système et soit
$$ \gamma = \limsup_{t\to \infty } \frac{1}{t} \log \|\varphi (t) \|
\;\; , -\infty \leq \gamma \leq \infty \; .$$
}
\begin{enumerate}
    \item \question{Montrer que $\gamma$ ne dépend pas du choix de la norme
$\| . \|$ de $\Rr^n$.}
    \item \question{Montrer que $\gamma$ est une valeur finie si on suppose
 que les coefficients de la matrice $A(t)$ sont des fonctions bornées (on utilisera
 l'inégalité de Gronwall).}
    \item \question{Dans le cas où $A$ est une matrice constante diagonalisable, montrer que $\gamma$
est forcément la partie réelle d'une valeur propre de $A$.}
\end{enumerate}
}
