\uuid{PG0b}
\exo7id{5198}
\auteur{rouget}
\organisation{exo7}
\datecreate{2010-06-30}
\isIndication{false}
\isCorrection{true}
\chapitre{Géométrie affine euclidienne}
\sousChapitre{Géométrie affine euclidienne du plan}

\contenu{
\texte{
Soit $(ABDC)$ un parallélogramme. Déterminer les coordonnées de $D$ dans le repère $(A,\overrightarrow{AB},\overrightarrow{AC})$.
}
\reponse{
Puisque $(ABDC)$ un parallélogramme, $\overrightarrow{AD}=\overrightarrow{AB}+\overrightarrow{AC}$. Les coordonnées de $D$ dans le repère $(A,\overrightarrow{AB},\overrightarrow{AC})$ sont donc $(1,1)$.
}
}
