\uuid{BSVj}
\exo7id{7158}
\auteur{megy}
\organisation{exo7}
\datecreate{2017-05-13}
\isIndication{false}
\isCorrection{true}
\chapitre{Géométrie affine euclidienne}
\sousChapitre{Géométrie affine euclidienne du plan}

\contenu{
\texte{
Soit $\mathcal T=ABC$ un triangle et soit $g$ une isométrie de $\mathcal T$, c'est-à-dire une isométrie du plan telle que $g(\mathcal T)=\mathcal T$.
}
\begin{enumerate}
    \item \question{Montrer que $g$ envoie les sommets sur les sommets.}
    \item \question{Montrer que $g$ admet au moins un point fixe.}
    \item \question{Montrer que $g$ n'est ni une translation ni une réflexion glissée.}
\reponse{
Comme une isométrie est affine, elle conserve les barycentres. Soit $P$ un sommet du triangle. Comme ce n'est pas un barycentre d'autres points du triangle, son image par une isométrie fixant le triangle non plus, c'est-à-dire que son image est un sommet. On en déduit que les sommets sont envoyés sur les sommets.

Par préservation du barycentre une isométrie de $\mathcal T$ fixe son isobarycentre.

Comme elle a au moins un point fixe, ça ne peut être une translation ou une symétrie glissée.
}
\end{enumerate}
}
