\uuid{5IDH}
\exo7id{7099}
\auteur{megy}
\organisation{exo7}
\datecreate{2017-01-21}
\isIndication{true}
\isCorrection{true}
\chapitre{Géométrie affine euclidienne}
\sousChapitre{Géométrie affine euclidienne du plan}

\contenu{
\texte{
% tags : translation, homothétie
}
\begin{enumerate}
    \item \question{On donne deux droites parallèles et un point $A$ entre les deux droites. Dénombrer et tracer les cercles passant par $A$ et tangents aux deux droites.}
    \item \question{On donne deux droites sécantes et un point $A$ n'appartenant pas aux deux droites. Dénombrer et tracer les cercles passant par $A$ et tangents aux deux droites.}
\reponse{
Commencer par tracer la bissectrice, puis $(OA)$. Ensuite, tracer un cercle quelconque tangent aux deux droites (dans le même secteur angulaire), et utiliser une homothétie. 

Note : les exercices faisant intervenir des homothéties se résolvent plus facilement en \og partant de la fin\fg, c'est-à-dire en procédant par analyse-synthèse et en faisant une figure approximative de ce que sera la solution.
}
\indication{Sans la condition sur $A$, l'exercice est facile. Tracer n'importe quel cercle tangent aux droites. Ensuite, appliquer la méthodologie classique. % translation dans un cas, homothétie dans l'autre.}
\end{enumerate}
}
