\uuid{M2nz}
\exo7id{1956}
\auteur{liousse}
\organisation{exo7}
\datecreate{2003-10-01}
\video{zrMBUrw-HzU}
\isIndication{false}
\isCorrection{true}
\chapitre{Géométrie affine dans le plan et dans l'espace}
\sousChapitre{Géométrie affine dans le plan et dans l'espace}

\contenu{
\texte{
Soit $P$ un  plan muni d'un repère  $\mathcal{R}(O,\vec{i},\vec{j})$, les points et les vecteurs sont exprimés par leurs
coordonnées dans $\mathcal{R}$.
}
\begin{enumerate}
    \item \question{Donner un vecteur directeur, la pente 
 une équation paramétrique et une équation cartésienne des droites $(AB)$ suivantes :

 \begin{enumerate}}
\reponse{\begin{enumerate}}
    \item \question{$A(2,3)$ et $B(-1,4)$}
\reponse{Un vecteur directeur est $\overrightarrow{AB}$ dont les coordonnées sont $(x_B-x_A,y_B-y_A)=(-3,1)$.
Pour n'importe quel vecteur directeur $\vec v = (x_v,y_v)$ la pente est le réel $p=\frac{y_v}{x_v}$.
La pente est indépendante du choix du vecteur directeur. On trouve ici $p=-\frac13$.
Une équation paramétrique de la droite de vecteur directeur $\vec v$ passant par $A=(x_A,y_A)$ est donnée par
$\left\{ \begin{array}{l}
x = x_v t + x_A \\ y = y_v t+y_A \\ 
\end{array} \right..$
Donc ici pour le vecteur directeur $\overrightarrow{AB}$ on trouve l'équation paramétrique 
$\left\{ \begin{array}{l}
x = -3t + 2 \\ y = t+3 \\ 
\end{array} \right.$

Il y a plusieurs façons d'obtenir une équation cartésienne $ax+by+c=0$.


\textbf{Première méthode.} 
On sait que $A=(x_A,y_A)$ appartient à la droite donc ses coordonnées vérifient l'équation 
$ax_A+by_A+c=0$, idem avec $B$.
On en déduit le système 
$\left\{ \begin{array}{l}
2a+3b+c = 0 \\  -a+4b+c=0 \\ 
\end{array} \right..$
Les solutions s'obtiennent à une constante multiplicative près,
on peut fixer $a=1$ et on trouve alors $b=3$ et $c=-11$.
L'équation est donc $x+3y-11=0$.}
    \item \question{$A(-7,-2)$ et $B(-2,-5)$}
\reponse{On trouve $\vec v = \overrightarrow{AB}=(5,-3)$, $p=-\frac35$
et $\left\{ \begin{array}{l}
x = 5t -7 \\ y = -3t-2 \\ 
\end{array} \right.$

\textbf{Deuxième méthode.} 
Pour trouver l'équation cartésienne on part de l'équation
paramétrique réécrite ainsi
$\left\{ \begin{array}{l}
\frac{x+7}{5} = t \\ -\frac{y+2}{3} = t \\ 
\end{array} \right.$
On en déduit $\frac{x+7}{5}=-\frac{y+2}{3}$ ; d'où
l'équation $3x+5y+31=0$.}
    \item \question{$A(3,3)$ et $B(3,6)$}
\reponse{On trouve $\vec v = \overrightarrow{AB}=(0,3)$, la droite est donc verticale (sa pente est infinie)
une équation paramétrique est
$\left\{ \begin{array}{l}
x = 3 \\ y = 3t+6 \\ 
\end{array} \right.$.
Une équation cartésienne est simplement $(x=3)$.}
\end{enumerate}
}
