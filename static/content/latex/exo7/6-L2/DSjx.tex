\uuid{DSjx}
\exo7id{6881}
\auteur{gammella}
\organisation{exo7}
\datecreate{2012-05-29}
\isIndication{true}
\isCorrection{true}
\chapitre{Analyse vectorielle}
\sousChapitre{Forme différentielle, champ de vecteurs, circulation}

\contenu{
\texte{
En utilisant la formule de Green-Riemann, calculer
$I= \iint_{\mathcal{D}} xydxdy$
o\`u $$\mathcal{D}=\{(x,y)\in \Rr^2\, |\, x\geq0; y\geq 0;x+y\leq 1\}.$$
}
\indication{On rappelle la formule de Green-Riemann qui permet de faire le lien entre intégrale
double et intégrale curviligne :

\textbf{Théorème.} 
Soit $\mathcal{D}$ un domaine de $\Rr^2$ limité par une courbe
fermée $\mathcal{C}$ que l'on suppose coupée
par toute parallèle aux axes en deux points au plus. On considère une forme
différentielle
$\omega=Pdx+Qdy$  définie sur $\mathcal{D}$. Si
les fonctions $P$ et $Q$ sont de classe $C^{1}$, on a :
$$\int_{\mathcal{ C}^+} Pdx+Qdy=\iint_{\mathcal{D}} ( \frac{\partial Q}{\partial x}-\frac{\partial P}{\partial y} ) dx dy$$
o\`u l'on a noté $\mathcal{C}^{+}$ la courbe $\mathcal{C}$ que l'on a
orientée dans le sens direct.}
\reponse{
On rapporte le plan à un repère orthonormé direct d'origine $O$.
D'après la formule de Green-Riemann, en choisissant de prendre $P=0$ et
$Q=x^2y$ de sorte que  $ \frac{\partial Q}{\partial x}-\frac{\partial P}{\partial y}=xy$, on obtient :
$$I= \iint_{\mathcal{D}} xydxdy=\int_{T} x^2ydy$$
o\`u l'on a noté $T$ le triangle $OAB$ orienté dans le sens direct
avec $O(0,0)$, $A(1,0)$ et $B(1,1)$.
Ainsi 
$$I=\iint_{\mathcal{D}} xydxdy=\int_{\overset{\frown}{OA}} x^2ydy
+\int_{\overset{\frown}{AB}} x^2ydy+
\int_{\overset{\frown}{BO}} x^2ydy.$$
L'intégrale curviligne d'une forme différentielle sur un chemin est indépendant du paramétrage choisi 
pour ce chemin. Pour le calcul, nous choisissons de paramétrer
$\overset{\frown}{OA}$ par $x=t$ et $y=0$ avec $t$ variant de $0$ à $1$ et ainsi
$\int_{\overset{\frown}{OA}} x^2ydy=0$. 
De m\^eme, nous choisissons de paramétrer
$\overset{\frown}{BO}$ par $x=0$ et $y=t$ avec $t$ variant de $1$ à $0$ et ainsi
$\int_{\overset{\frown}{BO}} x^2ydy=0$. 
Enfin, nous choisissons de paramétrer
$\overset{\frown}{AB}$ par $x=t$ et $y=1-t$ avec $t$ allant de $1$ à $0$ et donc :
$$I= \iint_{\mathcal{D}} xydxdy=\int_{\overset{\frown}{AB}} x^2ydy
= \int_1^0  \frac{t^2(1-t) }{2}(-dt)
= \int_0^1  \frac{t^2(1-t)}{2} dt = 
   \frac{1}{24}.$$
 Remarquons qu'il n'aurait pas été plus difficile ici de calculer directement l'intégrale
 double sans utiliser la formule de Green-Riemann :
 $$\iint_{\mathcal{D}} xy dx dy=
  \int_{0^1} (\int_{0}^{1-x} xy dy) dx=  \int_0^1 x [\frac{y^2}{2}]_0^{1-x} dx
 = \int_0^1 x\frac{(1-x)^2}{2}dx =  \frac{1}{24}.$$
}
}
