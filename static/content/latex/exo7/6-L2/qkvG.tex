\uuid{qkvG}
\exo7id{2021}
\auteur{liousse}
\organisation{exo7}
\datecreate{2003-10-01}
\video{6HAwYKkwS8Y}
\isIndication{false}
\isCorrection{true}
\chapitre{Géométrie affine dans le plan et dans l'espace}
\sousChapitre{Géométrie affine dans le plan et dans l'espace}

\contenu{
\texte{

}
\begin{enumerate}
    \item \question{Déterminer la distance du point $A$ au plan $(P)$

\begin{enumerate}}
\reponse{La distance d'un point $A=(x_0,y_0,z_0)$ à un plan $P$ d'équation 
$ax+by+cz+d=0$ est donnée par la formule :
$$d(A,P) = \frac{|ax_0+by_0+cz_0+d|}{\sqrt{a^2+b^2+c^2}}.$$
On trouve donc
  \begin{enumerate}}
    \item \question{$A(1,0,2)$ et $(P): 2x+y+z+4=0$.}
\reponse{$d(A,P)=\frac{|2\cdot 1+1\cdot 0+1\cdot 2+4|}{\sqrt{2^2+1^2+1^2}}= \frac{8}{\sqrt6}=\frac{4\sqrt6}{3}$.}
    \item \question{$A(3,2,1)$ et $(P): -x+5y-4z=5$.}
\reponse{$d(A,P)=\frac{2}{\sqrt{42}}$.}
\end{enumerate}
}
