\uuid{VFPD}
\exo7id{5154}
\auteur{rouget}
\organisation{exo7}
\datecreate{2010-06-30}
\isIndication{false}
\isCorrection{true}
\chapitre{Suite}
\sousChapitre{Autre}

\contenu{
\texte{
Soient $x$ un réel. Déterminer $\lim_{n\rightarrow +\infty}\frac{E(x)+E(2x)+...+E(nx)}{n^2}$.
}
\reponse{
Soient $x\in\Rr$ et $n\in\Nn^*$. Pour $1\leq k\leq n$, on a

$$kx-1<E(kx)\leq kx.$$

En sommant ces inégalités, on obtient

$$\frac{E(x)+E(2x)+...+E(nx)}{n^2}\leq\frac{x+2x+...+nx}{n^2}=\frac{n(n+1)x}{2n^2}=\frac{(n+1)x}{2n},$$

et aussi,

$$\frac{E(x)+E(2x)+...+E(nx)}{n^2}>\frac{(x-1)+(2x-1)+...+(nx-1)}{n^2}=\frac{n(n+1)x/2-n}{n^2}
=\frac{(n+1)x}{2n}-\frac{1}{n}.$$

Finalement, pour tout naturel non nul,

$$\frac{(n+1)x}{2n}-\frac{1}{n}<\frac{E(x)+E(2x)+...+E(nx)}{n^2}\leq\frac{(n+1)x}{2n}.$$

Les deux membres extrêmes de cet encadrement tendent vers $\frac{x}{2}$ quand $n$ tend vers $+\infty$. D'après le
théorème des gendarmes, on peut affirmer que
\begin{center}
\shadowbox{
$\forall x\in\Rr,\;\lim_{n\rightarrow +\infty}\frac{E(x)+E(2x)+...+E(nx)}{n^2}=\frac{x}{2}.$
}
\end{center}
}
}
