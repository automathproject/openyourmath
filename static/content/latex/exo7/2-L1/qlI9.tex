\uuid{qlI9}
\exo7id{5385}
\auteur{rouget}
\organisation{exo7}
\datecreate{2010-07-06}
\isIndication{false}
\isCorrection{true}
\chapitre{Continuité, limite et étude de fonctions réelles}
\sousChapitre{Continuité : théorie}

\contenu{
\texte{
\label{exo:roudist}
Soit $A$ une partie non vide de $\Rr$. Pour $x\in\Rr$, on pose $f(x)=\mbox{Inf}\{|y-x|,\;y\in A\}$. Montrer que $f$ est continue en tout point de $\Rr$.
}
\reponse{
Soit $(x,y)\in\Rr^2$ et $z\in A$. $|x-z|\leq|x-y|+|y-z|$. Or, $forall z\in A,\;|x-z|\geq d(x,A)$ et donc $d(x,A)-|x-y|$ est un minorant de $\{|y-z|,\;z\in A\}$. Par suite, $d(x,A)-|x-y|\leq d(y,A)$. On a montré que 

$$\forall(x,y)\in\Rr^2,\;d(x,A)-d(y,A)\leq|y-x|.$$

En échangeant les roles de $x$ et $y$, on a aussi montré que $\forall(x,y)\in\Rr^2,\;d(y,A)-d(x,A)\leq|y-x|$.

Finalement, $\forall(x,y)\in\Rr^2,\|f(y)-f(x)|\leq|y-x|$. Ainsi, $f$ est donc $1$-Lipschitzienne et en particulier continue sur $\Rr$.
}
}
