\uuid{f9Ah}
\exo7id{2081}
\auteur{bodin}
\organisation{exo7}
\datecreate{2008-02-04}
\video{UXp2ntGBZNE}
\isIndication{false}
\isCorrection{true}
\chapitre{Calcul d'intégrales}
\sousChapitre{Théorie}

\contenu{
\texte{
Soit $f$ la fonction définie sur $[0,4]$ par 
\begin{equation*}
  f(x)=
  \begin{cases}
    -1 &\text{ si $x=0$}\\
    1 &\text{ si $0<x<1$}\\
    3 &\text{ si $x=1$}\\
    -2 &\text{ si $1<x\leq 2$}\\
    4 &\text{ si $2<x\leq 4$.}
  \end{cases}
\end{equation*}
}
\begin{enumerate}
    \item \question{Calculer $\int_0^4f(t) \, dt$.}
\reponse{On trouve $\int_0^4 f(t) \, dt = +7$. Il faut tout d'abord tracer le graphe de cette fonction. 
Ensuite la valeur d'une intégrale ne dépend pas de la valeur de la fonction en un point, 
c'est-à-dire ici les valeurs en $x=0$, $x=1$, $x=2$ n'ont aucune influence sur l'intégrale. 
Ensuite on revient à la définition de $\int_0^4f(t) \, dt$ : pour la subdivision de $[0,4]$
 définie par $\{x_0=0,x_1=1,x_2=2,x_3=3, x_4=4\}$, on trouve la valeur de l'intégrale (ici le sup et l'inf 
sont atteints et égaux pour cette subdivision et toute subdivision plus fine).
Une autre façon de faire est considérer que $f$ est une fonction en escalier (en <<oubliant>> les accidents en
$x=0$, $x=1$, $x=2$) dont on sait calculer l'intégrale.}
    \item \question{Soit $x\in [0,4]$, calculer $F(x)=\int_0^x  f(t) \, dt$.}
\reponse{C'est la même chose pour $\int_0^x  f(t) \, dt$, mais au lieu d'aller jusqu'à $4$ on s'arrête à $x$, on trouve
\begin{equation*}
  F(x)=
  \begin{cases}
   x &\text{ si $0\leqslant x \leqslant 1$}\\
   3-2x &\text{ si $1<x\leqslant 2$}\\
   4x-9 &\text{ si $2 < x \leqslant 4$.}\\
  \end{cases}
\end{equation*}}
    \item \question{Montrer que $F$ est une fonction continue sur $[0,4]$. La fonction
$F$ est-elle dérivable sur $[0,4]$ ?}
\reponse{Les seuls points à discuter pour la continuité sont les points $x=1$ et $x=2$,
mais les limites à droite et à gauche de $F$ sont égales en ces points donc $F$ est continue. 
Par contre $F$ n'est pas dérivable en $x=1$ (les dérivées à droite et à gauche 
sont distinctes), $F$ n'est pas non plus dérivable en $x=2$.}
\end{enumerate}
}
