\uuid{k3RY}
\exo7id{5401}
\auteur{rouget}
\organisation{exo7}
\datecreate{2010-07-06}
\isIndication{false}
\isCorrection{true}
\chapitre{Continuité, limite et étude de fonctions réelles}
\sousChapitre{Continuité : théorie}

\contenu{
\texte{
Soit $f$ périodique et continue sur $\Rr$. Montrer que $f$ est bornée et uniformément continue sur $\Rr$.
}
\reponse{
Soit $T$ une pèriode strictement positive de $f$. $f$ est continue sur le segment $[0,T]$ et donc est bornée sur ce segment. $f$ est par suite bornée sur $\Rr$ par $T$-périodicité.

Soit $\varepsilon>0$.

$f$ est continue sur le segment $[0,T]$ et donc, d'après le théorème de \textsc{Heine}, $f$ est uniformément continue sur ce segment. Donc,

$$\exists\alpha\in]0,T[/\;\forall(x,y)\in[0,T],\;(|x-y|<\alpha\Rightarrow|f(x)-f(y)|<\frac{\varepsilon}{2}.$$

Soient $x$ et $y$ deux réels tels que $|x-y|<\alpha$. Si il existe un entier naturel $k$ tel que $(x,y)\in[kT,(k+1)T]$, alors $x-kT\in[0,T]$, $y-kT\in[0,T]$, puis $|(x-kT)-(y-kT)|=|y-x|<\alpha$ et donc $|f(x)-f(y)|<\frac{\varepsilon}{2}<\varepsilon$.

Sinon, en supposant par exemple que $x\leq y$, puisque l'on a choisi $\alpha<T$, 

$$\exists k\in\Zz/\;(k-1)T\leq x\leq kT\leq y\leq(k+1)T.$$

Mais alors, $|x-kT|=|y-x|<\alpha$ et $|y-kT|\leq|y-x|<\alpha$. Par suite,

$$|f(x)-f(y)|\leq|f(x)-f(kT)|+|f(y)-f(kT)|<\frac{\varepsilon}{2}+\frac{\varepsilon}{2}=\varepsilon.$$

Dans tous les cas, si $|x-y|<\alpha$, alors $|f(x)-f(y)|<\varepsilon$. On a montré que 

$$\forall\varepsilon>0,\;\exists\alpha>0/\;\forall(x,y)\in\Rr^2,\;(|x-y|<\alpha\Rightarrow|f(x)-f(y)|<\varepsilon).$$

$f$ est donc uniformément continue sur $\Rr$.
}
}
