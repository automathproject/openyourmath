\uuid{aXTP}
\exo7id{4465}
\auteur{quercia}
\organisation{exo7}
\datecreate{2010-03-14}
\isIndication{false}
\isCorrection{true}
\chapitre{Série numérique}
\sousChapitre{Autre}

\contenu{
\texte{
On se donne $u_1$ et $a$ deux réels strictement positifs et l'on
définit par récurrence la suite $(u_n)$ par $u_{n+1}=u_n+\frac1{n^a u_n}\cdotp$
Étudiez la limite de la suite $(u_n)$, et, quand $a\le1$, en donner un
équivalent.
}
\reponse{
La suite $(u_n)$ est croissante donc tend vers $\ell\in{]0,+\infty]}$.
On a $\ell$ fini si et seulement si la série télescopique
$\sum(u_{n+1}-u_n) = \sum\frac1{n^au_n}$ est convergente, soit si et
seulement si $a>1$.

Pour $a< 1$ on a $u_{n+1}^2 = u_n^2 + \frac2{n^a} +  o\Bigl(\frac2{n^a}\Bigr)$
donc $u_{n+1}^2-u_n^2\sim \frac2{n^a}$ et $u_n\sim\sqrt{\frac{2n^{1-a}}{1-a}}$
(sommation des relations de comparaison).

Pour $a=1$ on a de même $u_n\sim \sqrt{2\ln n}$.
}
}
