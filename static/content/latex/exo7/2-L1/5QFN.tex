\uuid{5QFN}
\exo7id{4031}
\auteur{quercia}
\organisation{exo7}
\datecreate{2010-03-11}
\isIndication{false}
\isCorrection{false}
\chapitre{Développement limité}
\sousChapitre{Autre}

\contenu{
\texte{
Soit $f : \R \to \R$ une fonction de classe $\mathcal{C}^n$.
On pose
$g(x) =\begin{cases}\frac {f(x)-f(0)}x &\text{ si } x \ne 0 \cr f'(0) &\text{ si } x=0. \cr\end{cases}$
}
\begin{enumerate}
    \item \question{On suppose que $f(x) = o (x^n)$.
  \begin{enumerate}}
    \item \question{Démontrer que : $\forall\ p \le n,\ f^{(p)}(x) = o (x^{n-p})$,
        et : $\forall\ p < n,\ g^{(p)}(x) = o (x^{n-p-1})$.}
    \item \question{En déduire que $g$ est de classe $\mathcal{C}^{n-1}$ en $0$.}
\end{enumerate}
}
