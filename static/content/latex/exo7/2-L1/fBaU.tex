\uuid{fBaU}
\exo7id{5408}
\auteur{rouget}
\organisation{exo7}
\datecreate{2010-07-06}
\isIndication{false}
\isCorrection{true}
\chapitre{Dérivabilité des fonctions réelles}
\sousChapitre{Théorème de Rolle et accroissements finis}

\contenu{
\texte{
Soient $a$ et $b$ deux réels tels que $a<b$ et $n$ un entier naturel. Soit $f$ une fonction élément de $C^n([a,b],\Rr)\cap D^{n+1}(]a,b[,\Rr)$. Montrer qu'il existe $c\in]a,b[$ tel que 

$$f(b)=\sum_{k=0}^{n}\frac{f^{(k)}(a)}{k!}(b-a)^k+\frac{(b-a)^{n+1}f^{(n+1)}(c)}{(n+1)!}.$$

Indication. Appliquer le théorème de \textsc{Rolle} à la fonction $g(x)=f(b)-\sum_{k=0}^{n}\frac{f^{(k)}(x)}{k!}(b-x)^k-A\frac{(b-x)^{n+1}}{(n+1)!}$ où $A$ est intelligemment choisi.
}
\reponse{
On a déjà $g(b)=f(b)-f(b)=0$. Puisque $a\neq b$, on peut choisir $A$ tel que $g(a)=0$ (à savoir $A=\frac{(n+1)!}{(b-a)^{n+1}}(f(b)-\sum_{k=0}^{n}\frac{f^{(k)}(a)}{k!}(b-a)^k$).

Avec les hypothèses faites sur $f$, $g$ est d'autre part continue sur $[a,b]$ et dérivable sur $]a,b[$. Le théorème de \textsc{Rolle} permet alors d'affirmer qu'il existe $c\in]a,b[$ tel que $g'(c)=0$.

Pour $x\in]a,b[$, on a

\begin{align*}\ensuremath
g'(x)&=-\sum_{k=0}^{n}\frac{f^{(k+1)}(x)}{k!}(b-x)^k+\sum_{k=1}^{n}\frac{f^{(k)}(x)}{(k-1)!}(b-x)^{k-1}
+A\frac{(b-x)^n}{n!}\\
 &=-\sum_{k=0}^{n}\frac{f^{(k+1)}(x)}{k!}(b-x)^k+\sum_{k=0}^{n-1}\frac{f^{(k+1)}(x)}{k!}(b-x)^{k}
+A\frac{(b-x)^n}{n!}=-\sum_{k=0}^{n}\frac{f^{(n+1)}(x)}{n!}(b-x)^n+A\frac{(b-x)^n}{n!}\\
 &=\frac{(b-x)^n}{n!}(A-f^{(n+1)}(x)).
\end{align*}

Ainsi, il existe $c\in]a,b[$ tel que $\frac{(b-c)^n}{n!}(A-f^{(n+1)}(c))=0$, et donc, puisque $c\neq b$, tel que $A=f^{(n+1)}(c)$.

L'égalité $g(a)=0$ s'éxrit alors

$$f(b)=\sum_{k=0}^{n}\frac{f^{(k)}(a)}{k!}(b-a)^k+\frac{(b-a)^{n+1}f^{(n+1)}(c)}{(n+1)!},$$

pour un certain réel $c$ de $]a,b[$.
}
}
