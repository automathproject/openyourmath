\uuid{xMLU}
\exo7id{4703}
\auteur{quercia}
\organisation{exo7}
\datecreate{2010-03-16}
\isIndication{false}
\isCorrection{true}
\chapitre{Suite}
\sousChapitre{Suite définie par une relation de récurrence}

\contenu{
\texte{
Étudier la convergence de la suite $(u_n)$ définie par :
}
\begin{enumerate}
    \item \question{$u_0 = a > 1$, $u_{n+1} = \frac 12\left(u_n + \frac a{u_n}\right)$.}
    \item \question{$0 < u_0 < \frac {\sqrt5-1}2$, $u_{n+1} = 1 - u_n^2$.}
    \item \question{$u_{n+1} = u_n-u_n^2$.}
    \item \question{$u_0 = 0$, $u_{n+1} = u_n^2+\alpha$.}
    \item \question{$u_{n+1} = u_n + \frac{1+u_n}{1+2u_n}$.}
    \item \question{$u_0 \in\, [0,1]$, $u_{n+1} = \frac{\sqrt{u_n}}{\sqrt{u_n}+\sqrt{1-u_n}}$.}
    \item \question{$u_{n+1} = \sqrt{2-u_n}$.}
    \item \question{$u_{n+1} = \sqrt{4-3u_n}$.}
    \item \question{$u_{n+1} = \frac{u_n-\ln(1+u_n)}{u_n^2}$.}
    \item \question{$u_{n+1} = \frac3{2u_n^2+1}$.}
    \item \question{$u_0 > 0$, $u_{n+1} = u_n^\alpha$.}
    \item \question{$u_0 > 0$, $u_{n+1} = \alpha^{u_n}$.}
\reponse{
$u_n\searrow\sqrt a$, et $u_n-\sqrt a < \frac{a-\sqrt a}{(2\sqrt a)^{2^n-1}}$.
$u_{2n}\to0$, $u_{2n+1}\to1$.
Si $0\le u_0 \le 1$ : $u_n\searrow0$, sinon $u_n\searrow-\infty$.
\begin{itemize}
$\frac{1}{4} < \alpha$ : $u_n\to\infty$;
$- \frac{3}{4} < \alpha \le \frac{1}{4}$: $u_n\to\frac{1-\sqrt{1-4a}}2$,
$-1 < \alpha \le - \frac{3}{4}$: 
	  $1$ point fixe et deux points réciproques. $(u_n)$ ne converge pas.
	  \end{itemize}
Si $u_0 > {\scriptstyle-}\frac12$, $u_n\to\infty$;
             si $u_0 < {\scriptstyle-}\frac12$, $u_n\to-1$.
$u_n\to\frac12$.
Thm du point fixe sur $]-\infty, \frac74]  \Rightarrow  u_n\to 1$.
Si $u_0\ne 1$, $\exists\ n \text{ tq } 4-3u_n < 0  \Rightarrow $ suite finie.
$u_n\to\alpha \approx 0.39754$.
1 est point fixe, il y a deux points réciproques. $(u_n)$ ne converge pas.
\begin{itemize}
$1 < \alpha$: $u_n\to0$ si $u_0<1$, $u_n\to\infty$ si $u_0>1$
$-1 < \alpha <1$: $u_n\to 1$
$\alpha \le -1$: si $u_0 \ne 1$, $(u_n)$ diverge.
    	\end{itemize}
\begin{itemize}
       $e^{1/e}  < \alpha$: $u_n\to\infty$.
       $1 < \alpha < e^{1/e}$: 2 pts fixes, $\beta<\gamma$. $u_n\to\beta$ si $u_0<\gamma$, 
       et $u_n\to\infty$ si $u_0>\gamma$. 
       $e^{-e} \le \alpha <  1$: 1 pt fixe, $\beta$, et $u_n\to\beta$.
       $\alpha <  e^{-e}$: 1 point fixe et deux points réciproques. $(u_n)$ ne converge pas.
    	\end{itemize}
}
\end{enumerate}
}
