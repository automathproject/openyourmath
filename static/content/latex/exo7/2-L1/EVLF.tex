\uuid{EVLF}
\exo7id{4269}
\auteur{quercia}
\organisation{exo7}
\datecreate{2010-03-12}
\isIndication{false}
\isCorrection{false}
\chapitre{Calcul d'intégrales}
\sousChapitre{Intégrale impropre}

\contenu{
\texte{
%
  $\int_0^{+\infty} \frac{ d t}{(1+t^2)^2} = \frac{\pi}{4} $\par
  $\int_{-\infty}^{+\infty} \frac{ d t}{t^2+2t+2} = \pi $\par
  $\int_0^{+\infty} \frac{ d t}{(1+t^2)^4} = \frac{ 5\pi}{32} $\par
  $\int_{-\infty}^{+\infty} \frac{ d t}{(t^2+1)(t^2-2t\cos\alpha+1)} = \frac{\pi}{2|\sin\alpha|} $\par
  $\int_0^{+\infty} \frac{ 2t^2+1}{(t^2+1)^2} \,d t = \frac{ 3\pi}{4} $\par
  $\int_{-\infty}^{+\infty} \frac{ t^2\,d t}{(t^2+1)(t^2+a^2)} = \frac{ \pi}{1+|a|} $\par
  $\int_0^{+\infty} \frac{ d t}{1+t^4} = \frac{\pi}{2\sqrt2} $\par
  $\int_0^{+\infty} \frac{ t^2\,d t}{1+t^4} = \frac{\pi}{2\sqrt2} $\par
  $\int_1^{+\infty} \frac{ d t}{t^6(1+t^{10})} = \frac{ 4-\pi}{20} $\par
}
}
