\uuid{CnoG}
\exo7id{3872}
\auteur{quercia}
\organisation{exo7}
\datecreate{2010-03-11}
\isIndication{false}
\isCorrection{true}
\chapitre{Continuité, limite et étude de fonctions réelles}
\sousChapitre{Continuité : théorie}

\contenu{
\texte{
\
}
\begin{enumerate}
    \item \question{Existe-t-il toujours~$\varphi$ lipschitzienne telle que $\varphi\le f$
    où $f : {\R^n} \to \R$ est une application continue donnée~?}
\reponse{Non. Si $\varphi$ est lipschitzienne alors $\varphi(x) = O(\|x\|)$
    lorsque $\|x\|$ tend vers l'infini, donc toute fonction $f$ à décroissance
    suffisament rapide vers $-\infty$ n'est pas minorable par une fonction
    lipschitzienne. Contre-exemple explicite~: $f(x) = -\|x\|^2$.}
    \item \question{Soit~$k>0$. Trouver une CNS sur~$f$ pour qu'il existe $\varphi : {\R^n}\to \R$
    $k$-lipschitzienne minorant~$f$.}
\reponse{CNS~: $x \mapsto f(x)+k\|x\|$ est minorée.}
    \item \question{On suppose cette CNS vérifiée pour~$k_0>0$. Montrer que si $k\ge k_0$
    alors il existe $\varphi_k$, $k$-lipschitzienne minorant~$f$ et maximale
    pour l'ordre usuel des fonctions.}
\reponse{On pose $\varphi(x) = \sup\{g(x),\ g\ k$-lipschitzienne minorant $f\}$.
    Il suffit de vérifier que~$\varphi$ est $k$-lipschitzienne, ce sera alors
    la plus grande fonction $k$-lipschitzienne minorant~$f$. Soient~$x,y\in\R^n$,
    $\varepsilon>0$ et~$g_x,g_y$ des fonctions $k$-lipschitziennes minorant~$f$ telles
    que $g_y(x)\le\varphi(x)\le g_x(x)+\varepsilon$ et $g_x(y)\le\varphi(y)\le g_y(y)+\varepsilon$.
    On a~: 
\begin{align*}
    \varphi(y)&\le g_y(y)+\varepsilon\le g_y(x)+k\|x-y\|+\varepsilon\le\varphi(x)+k\|x-y\|+\varepsilon,\\
    \varphi(y)&\ge g_x(y)\ge g_x(x)-k\|x-y\|\ge\varphi(x)-k\|x-y\|-\varepsilon.\\
\end{align*}
    Donc $|\varphi(x)-\varphi(y)|\le k\|x-y\|+\varepsilon$ et on fait tendre $\varepsilon$
    vers~$0^+$.}
\end{enumerate}
}
