\uuid{4OCH}
\exo7id{6979}
\auteur{blanc-centi}
\organisation{exo7}
\datecreate{2014-05-06}
\video{bRKLAqnlbBM}
\isIndication{true}
\isCorrection{true}
\chapitre{Fonctions circulaires et hyperboliques inverses}
\sousChapitre{Fonctions hyperboliques et hyperboliques inverses}

\contenu{
\texte{
\'Etudier le domaine de définition de la fonction $f$ définie par 
$$f(x)=\Argch\left[\frac{1}{2}\left(x+\frac{1}{x}\right)\right]$$
et simplifier son expression lorsqu'elle a un sens.
}
\indication{On trouve $f(x)=|\ln x|$ pour tout $x>0$.}
\reponse{
La fonction $\Argch$ est définie sur $[1,+\infty[$. Or
\begin{eqnarray*}
\frac{1}{2}\left(x+\frac{1}{x}\right)\ge 1
 &\Longleftrightarrow&\frac{x^2+1}{x}\ge 2\\
 &\Longleftrightarrow&\frac{x^2+1-2x}{x}\ge 0\\
 &\Longleftrightarrow&\frac{(x-1)^2}{x}\ge 0\\
 &\Longleftrightarrow& x > 0 
\end{eqnarray*}
donc $f$ est définie sur $]0,+\infty[$. 

\medskip

Soit $x>0$, alors $y=\frac{1}{2}\left(x+\frac{1}{x}\right)\ge 1$
et on sait que $\Argch y=\ln(y+\sqrt{y^2-1})$. Ainsi
$\sqrt{y^2-1}=\sqrt{\frac{1}{4}\left(x+\frac{1}{x}\right)^2-1}
=\sqrt{\frac{(x^2+1)^2}{4x^2}-1}=\sqrt{\frac{(x^2-1)^2}{4x^2}}
=\left|\frac{x^2-1}{2x}\right|$,
on obtient
$$f(x)=\Argch y=\ln(y+\sqrt{y^2-1})=\ln\left(\frac{x^2+1}{2x}+\left|\frac{x^2-1}{2x}\right|\right)$$


On a supposé $x>0$, il suffit donc de distinguer les cas $x\ge 1$ et $0<x\le 1$.
\begin{itemize}
  \item Si $x\ge 1$, $\displaystyle f(x)=\ln\left(\frac{x^2+1}{2x}+\frac{x^2-1}{2x}\right)=\ln x$.
  \item Si $0<x\le 1$, $\displaystyle f(x)=\ln\left(\frac{x^2+1}{2x}+\frac{1-x^2}{2x}\right)=\ln \frac{1}{x}=-\ln x$.
\end{itemize}

Puisque $\ln x$ est positif si $x\ge 1$ et négatif si $x\le 1$, 
on obtient dans les deux cas $f(x)=|\ln x|$.

\begin{center}
\begin{tikzpicture}[scale=1.5]
      \draw[->,>=latex, gray] (-1,0)--(5,0) node[below,black] {$x$};
      \draw[->,>=latex, gray] (0,-1)--(0,2.7) node[right,black] {$y$};  
 
     \draw[ultra thick, color=red,domain=0.1:5,samples=100,smooth] plot (\x,{abs(ln(\x))}) node[above left] {$f(x)$}; 
     
     \fill (0,0) circle (1.5pt);     
     \fill (1,0) circle (1.5pt);
     \fill (0,1) circle (1.5pt);     
     \node at (0,1)[above left] {$1$};
     \node at (1,0)[below] {$1$};
     \node at (0,0)[below right] {$0$};
\end{tikzpicture}  
\end{center}
}
}
