\uuid{xKOf}
\exo7id{642}
\auteur{bodin}
\organisation{exo7}
\datecreate{1998-09-01}
\video{Uh9semrdJvk}
\isIndication{false}
\isCorrection{true}
\chapitre{Continuité, limite et étude de fonctions réelles}
\sousChapitre{Continuité : théorie}

\contenu{
\texte{
Soit $f:[a,b]\longrightarrow\R$ une fonction continue telle que
$f(a)=f(b)$. Montrer que la fonction $g(t)=f(t+\frac{b-a}{2})-f(t)$ s'annule en au moins un point
de $[a,\frac{a+b}{2}]$.

\emph{Application :} une personne parcourt 4 km en 1 heure.
Montrer qu'il existe un intervalle de 30 mn pendant lequel elle parcourt exactement 2 km.
}
\reponse{
$g(a) = f(\frac{a+b}{2})-f(a)$ et
$g(\frac{a+b}{2}) = f(b) - f(\frac{a+b}{2})$.
Comme $f(a) = f(b)$ alors nous obtenons que $g(a) = -g(\frac{a+b}{2})$.
Donc ou bien $g(a) \leq 0$ et $g(\frac{a+b}{2}) \geq 0$
ou bien $g(a) \geq 0$ et $g(\frac{a+b}{2}) \leq 0$.
D'après le théorème des valeurs intermédiaires, $g$
s'annule en $c$ pour un $c$ entre $a$ et $\frac{a+b}{2}$.
Notons $t$ le temps (en heure) et $d(t)$ la distance parcourue (en km) entre les instants $0$ et $t$.
Nous supposons que la fonction $t \mapsto d(t)$ est continue.
Soit $f(t) = d(t) - 4t$. Alors $f(0) = 0$ et par hypothèse $f(1) = 0$.
Appliquons la question précédente avec $a=0$, $b=1$. 
Il existe $c\in [0,\frac 12]$ tel que $g(c) = 0$, c'est-à-dire
$f(c+\frac12)= f(c)$. Donc $d(c+\frac 12)-d(c) = 4(c+\frac12)-4c = 2$.
Donc entre $c$ et $c+\frac 12$, (soit 1/2 heure), la personne parcourt
exactement $2$ km.
}
}
