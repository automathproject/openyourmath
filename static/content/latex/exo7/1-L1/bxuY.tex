\uuid{bxuY}
\exo7id{1052}
\auteur{legall}
\organisation{exo7}
\datecreate{1998-09-01}
\video{7AwRBOj2CYY}
\isIndication{true}
\isCorrection{true}
\chapitre{Matrice}
\sousChapitre{Propriétés élémentaires, généralités}

\contenu{
\texte{
Soit 
$A=\begin{pmatrix} 
1 & 0 & 2 \cr
0 & -1 & 1 \cr
1 & -2 & 0 \cr
\end{pmatrix}$. 
Calculer $A^3-A$. En déduire que $A$ est inversible puis déterminer $A^{-1}$.
}
\indication{Une fois que l'on a calculé $A^2$ et $A^3$ on peut en déduire $A^{-1}$ sans calculs.}
\reponse{
On trouve 
$$A^2 =
\begin{pmatrix} 
3 & -4 & 2 \cr
1 & -1 & -1 \cr
1 & 2 & 0 \cr
\end{pmatrix}
\qquad \text{ et } \qquad 
A^3=
\begin{pmatrix} 
5 & 0 & 2 \cr
0 & 3 & 1 \cr
1 & -2 & 4 \cr
\end{pmatrix}
.$$

Un calcul donne $A^3-A = 4 I$.
En factorisant par $A$ on obtient $A\times (A^2-I) = 4I$.
Donc $A \times \frac 1 4 (A^2-I) = I$, ainsi $A$ est inversible et
$$A^{-1} = \frac 1 4 (A^2-I) = \frac 1 4
\begin{pmatrix}
2&-4&2\\
1&-2&-1\\
1&2&-1\\
\end{pmatrix}.
$$
}
}
