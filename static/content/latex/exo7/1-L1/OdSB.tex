\uuid{OdSB}
\exo7id{5135}
\auteur{rouget}
\organisation{exo7}
\datecreate{2010-06-30}
\isIndication{false}
\isCorrection{true}
\chapitre{Nombres complexes}
\sousChapitre{Racine n-ieme}

\contenu{
\texte{
On considère l'équation $(E)~:~(z-1)^n-(z+1)^n=0$ où $n$ est un entier naturel supérieur ou
égal à $2$ donné.
}
\begin{enumerate}
    \item \question{Montrer que les solutions de $(E)$ sont imaginaires pures.}
\reponse{Soit $z\in\Cc$. Soient $M$, $A$ et $B$ les points d'affixes resectives $z$, $1$ et $-1$.

\begin{align*}
z\;\mbox{solution
de}\;(E)&\Rightarrow(z-1)^n=(z+1)^n\Rightarrow|(z-1)^n|=|(z+1)^n|\Rightarrow|z-1|^n=|z+1|^n
\Rightarrow|z-1|=|z+1|\\
 &\Rightarrow AM=BM\Rightarrow M\in\mbox{med}[AB]\Rightarrow M\in(Oy)\Rightarrow z\in i\Rr.
\end{align*}}
    \item \question{Montrer que les solutions de $(E)$ sont deux à deux opposées.}
\reponse{Soit $z\in\Cc$.

$$(-z-1)^n-(-z+1)^n=(-1)^n((z+1)^n-(z-1)^n)=-(-1)^n((z-1)^n-(z+1)^n).$$
Par suite,

$$z\;\mbox{solution de}\;(E)\Leftrightarrow(z-1)^n-(z+1)^n=0\Leftrightarrow(-z-1)^n-(-z+1)^n=0\Leftrightarrow-z\;\mbox{solution de}\;(E).$$}
    \item \question{Résoudre $(E)$.}
\reponse{Soit $z\in\Cc$.
\begin{align*}
z\;\mbox{solution de}\;(E)&\Leftrightarrow(z-1)^n=(z+1)^n\Leftrightarrow\exists k\in\llbracket0,n-1\rrbracket/\;z-1=e^{2ik\pi/n}(z+1)\\
&\Leftrightarrow\exists k\in\llbracket1,n-1\rrbracket/\;z=-\frac{e^{2ik\pi/n}+1}{e^{2ik\pi/n}-1}
\Leftrightarrow\exists k\in\llbracket1,n-1\rrbracket/\;z=-\frac{e^{ik\pi/n}+e^{-ik\pi/n}}{e^{ik\pi/n}-e^{-ik\pi/n}}\\
&\Leftrightarrow\exists k\in\llbracket1,n-1\rrbracket/\;z=-\frac{2\cos\frac{k\pi}{n}}{2i\sin\frac{k\pi}{n}}
\Leftrightarrow\exists k\in\llbracket1,n-1\rrbracket/\;z=i\cotan\frac{k\pi}{n}
\end{align*}}
\end{enumerate}
}
