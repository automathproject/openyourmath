\uuid{pO3p}
\exo7id{5321}
\auteur{rouget}
\organisation{exo7}
\datecreate{2010-07-04}
\isIndication{false}
\isCorrection{true}
\chapitre{Polynôme, fraction rationnelle}
\sousChapitre{Autre}

\contenu{
\texte{
Soit $P$ un polynôme à coefficients entiers relatifs de degré supérieur ou égal à $1$. Soit $n$ un entier relatif 
et $m=P(n)$.
}
\begin{enumerate}
    \item \question{Montrer que $\forall k\in\Zz,\;P(n+km)$ est un entier divisible par $m$.}
\reponse{Posons $P=\sum_{i=0}^{l}a_iX_i$ où $l\geq1$ et où les $a_i$ sont des entiers relatifs avec $a_l\neq0$.

$$P(n+km)=\sum_{i=0}^{l}a_i(n+km)^i=\sum_{i=0}^{l}a_i(n^i+K_im)=\sum_{i=0}^{l}a_in^i+Km=m+Km=m(K+1),$$

où $K$ est un entier relatif. $P(n+km)$ est donc un entier relatif multiple de $m=P(n)$.}
    \item \question{Montrer qu'il n'existe pas de polynômes non constants à coefficients entiers tels que $P(n)$ soit premier pour tout entier $n$.}
\reponse{Soit $P\in\Zz[X]$ tel que $\forall n\in\Nn,\;P(n)$ est premier.

Soit $n$ un entier naturel donné et $m=P(n)$ (donc, $m\geq2$ et en particulier $m\neq0$). Pour tout entier relatif $k$, $P(n+km)$ est divisible par $m$ mais $P(n+km)$ est un nombre premier ce qui impose $P(n+km)=m$. Par suite, le polynôme $Q=P-m$ admet une infinité de racines deux à deux distinctes (puisque $m\neq0$) et est donc le polynôme nul ou encore $P$ est constant.}
\end{enumerate}
}
