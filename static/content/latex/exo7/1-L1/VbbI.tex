\uuid{VbbI}
\exo7id{926}
\auteur{gourio}
\organisation{exo7}
\datecreate{2001-09-01}
\video{uLIAnT8Tc2w}
\isIndication{true}
\isCorrection{true}
\chapitre{Espace vectoriel}
\sousChapitre{Somme directe}

\contenu{
\texte{
Soit $$E=\big\{(u_{n})_{n\in
  \Nn}\in \Rr^{\Nn}\ |\ (u_{n})_{n} \text{ converge }\big\}.$$
Montrer que
l'ensemble des suites constantes et l'ensemble des suites convergeant
vers $0$ sont des sous-espaces suppl\'{e}mentaires dans $E.$
}
\indication{Pour une suite $(u_n)$ qui converge vers $\ell$ regarder la suite $(u_n-\ell)$.}
\reponse{
$F\cap G = \{0\}$. En effet une suite constante qui converge
  vers $0$ est la suite nulle.
$F+G=E$. Soit $(u_n)$ un \'el\'ement de $E$. Notons $\ell$ la
    limite de $(u_n)$.  Soit $(v_n)$ la suite d\'efinie par
    $v_n=u_n-\ell$, alors $(v_n)$ converge vers $0$. Donc $(v_n)\in
    G$.  Notons $(w_n)$ la suite constante \'egale \`a $\ell$. Alors nous
    avons $u_n=\ell+u_n-\ell$, ou encore $u_n=w_n+v_n$, ceci pour tout
    $n\in\Nn$. En terme de suite cela donne $(u_n)=(w_n)+(v_n)$. Ce
    qui donne la d\'ecomposition cherch\'ee.
}
}
