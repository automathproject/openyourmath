\uuid{gtDb}
\exo7id{6972}
\auteur{blanc-centi}
\organisation{exo7}
\datecreate{2014-04-08}
\isIndication{true}
\isCorrection{true}
\chapitre{Polynôme, fraction rationnelle}
\sousChapitre{Fraction rationnelle}

\contenu{
\texte{
Soit $T_n(x)=\cos\big(n \arccos(x)\big)$ pour $x\in [-1,1]$.
}
\begin{enumerate}
    \item \question{\begin{enumerate}}
\reponse{\begin{enumerate}}
    \item \question{Montrer que pour tout $\theta\in[0,\pi]$, $T_n(\cos\theta)=\cos(n\theta)$.}
\reponse{Si on pose $x=\cos \theta$ alors l'égalité $T_n(x)=\cos\big(n \arccos(x)\big)$
    devient $T_n(\cos\theta)=\cos(n\theta)$, car $\arccos (\cos \theta) = \theta$ pour $\theta \in [0,\pi]$.}
    \item \question{Calculer $T_0$ et $T_1$.}
\reponse{$T_0(x) = 1$, $T_1(x)=x$.}
    \item \question{Montrer la relation de récurrence $T_{n+2}(x) = 2xT_{n+1}(x)-T_n(x)$, pour tout $n \ge0$.}
\reponse{En écrivant $(n+2)\theta = (n+1)\theta + \theta$ et $n\theta = (n+1)\theta - \theta$ on obtient :
    $$\begin{array}{c}
    \cos\big( (n+2)\theta \big) = \cos\big((n+1)\theta\big)\cos \theta - \sin\big((n+1)\theta\big)\sin \theta     \\
    \cos\big( n\theta \big) = \cos\big((n+1)\theta\big)\cos \theta + \sin\big((n+1)\theta\big)\sin \theta     \\    
      \end{array}$$
    Lorsque l'on fait la somme de ces deux égalités on obtient :
    $$\cos\big( (n+2)\theta \big) + \cos\big( n\theta \big) = \cos\big((n+1)\theta\big)\cos \theta$$
    
    Avec $x = \cos \theta$ cela donne :
    $$T_{n+2}(x) + T_n(x) = 2x T_{n+1}(x)$$}
    \item \question{En déduire que $T_n$ une fonction polynomiale de degré $n$.}
\reponse{$T_0$ et $T_1$ étant des polynômes alors, par récurrence,  $T_n(x)$ est un polynôme.
    De plus, toujours par la formule de récurrence, il est facile de voir que le degré de $T_n$ est $n$.}
\indication{Pour 1. exprimer $\cos\big((n+2)\theta\big)$ et $\cos(n\theta)$ 
en fonction de $\cos\big((n+1)\theta\big)$.
Pour 3. chercher les racines de $T_n$ : 
$\omega_k=\cos\left(\frac{(2k+1)\pi}{2n}\right)$ pour $k=0,\ldots,n-1$.}
\end{enumerate}
}
