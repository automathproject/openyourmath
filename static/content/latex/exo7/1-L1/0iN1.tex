\uuid{0iN1}
\exo7id{3143}
\auteur{quercia}
\organisation{exo7}
\datecreate{2010-03-08}
\isIndication{false}
\isCorrection{true}
\chapitre{Arithmétique dans Z}
\sousChapitre{Nombres premiers, nombres premiers entre eux}

\contenu{
\texte{
Soit $n \in \N$, $n \ge 2$.
Montrer que $x_n = 1 + \frac 12 + \frac 13 + \dots + \frac 1n$ est de la forme :
$\frac {p_n}{2q_n}$ avec $p_n,q_n \in \N^*$ et $p_n$ impair.
}
\reponse{
$H_n  \Rightarrow  H_{2n}  \Rightarrow  H_{2n+1}$.
}
}
