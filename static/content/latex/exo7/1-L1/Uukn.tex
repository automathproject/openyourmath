\uuid{Uukn}
\exo7id{337}
\auteur{bodin}
\organisation{exo7}
\datecreate{1998-09-01}
\video{POYqKNGJr44}
\isIndication{true}
\isCorrection{true}
\chapitre{Arithmétique dans Z}
\sousChapitre{Nombres premiers, nombres premiers entre eux}

\contenu{
\texte{
D\'emontrer que, si $a$ et $b$ sont des entiers
premiers entre eux, il en est de m\^eme des entiers $a+b$ et $ab$.
}
\indication{Raisonner par l'absurde et utiliser le lemme de Gauss.}
\reponse{
Soit $a$ et $b$ des entiers premiers entre eux. Raisonnons par
l'absurde et supposons que $ab$ et $a+b$ ne sont pas premiers
entre eux. Il existe alors $p$ un nombre premier divisant
$ab$ et $a+b$. Par le lemme d'Euclide comme $p|ab$ alors
$p|a$ ou $p|b$. Par exemple supposons que $p|a$.
Comme $p|a+b$ alors $p$ divise aussi $(a+b)-a$, donc $p|b$.
$\delta$ ne divise pas $b$ cela implique que $\delta$ et $b$ sont
premiers entre eux.

D'apr\`es le lemme de Gauss, comme $\delta$ divise $ab$ et
 $\delta$ premier avec $b$ alors $\delta$ divise $a$.
Donc $p$ est un facteur premier de $a$ et de
$b$ ce qui est absurde.
}
}
