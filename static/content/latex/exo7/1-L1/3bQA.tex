\uuid{3bQA}
\exo7id{339}
\auteur{bodin}
\organisation{exo7}
\datecreate{1998-09-01}
\video{RR5gG5ZLCGs}
\isIndication{true}
\isCorrection{true}
\chapitre{Arithmétique dans Z}
\sousChapitre{Nombres premiers, nombres premiers entre eux}

\contenu{
\texte{
Soit $p$ un nombre premier.
}
\begin{enumerate}
    \item \question{Montrer que $\forall i\in \Nn, 0< i < p$ on a : $$C_p^i\text{  est divisible par }p .$$}
\reponse{\'Etant donn\'e $0< i < p$, nous avons 
$$C_p^i = \frac{p!}{i!(p-i)!} = \frac{p(p-1)(p-2)\ldots(p-(i+1))}{i!}$$
Comme $C_p^i$ est un entier alors $i!$ divise $ p(p-1)\ldots(p-(i+1))$.
Mais $i!$ et $p$ sont premiers entre eux (en utilisant l'hypoth\`ese $0 < i < p$).
Donc d'apr\`es le th\'eor\`eme de Gauss: $i!$ divise $(p-1)\ldots(p-(i+1))$, autrement dit
il existe $k\in\Zz$ tel que $k i! = (p-1)\ldots(p-(i+1))$. Maintenant nous avons
$C_p^i = p k$ donc $p$ divise $C_p^i$.}
    \item \question{Montrer par r\'ecurence que : $$\forall p \text{ premier}, \forall a\in \Nn^* ,\text{ on a } a^p-a \text{ est divisible par } p.$$}
\reponse{Il s'agit de montrer le petit th\'eor\`eme de Fermat: pour $p$ premier et $a\in\Nn^*$, alors
$a^p \equiv a \pmod{p}$. Fixons $p$. Soit l'assertion
$$(\mathcal{H}_a) \ \ \ a^p \equiv a \pmod{p}.$$
Pour $a=1$ cette assertion est vraie !
\'Etant donn\'e $a \geq 1$ supposons que $\mathcal{H}_a$ soit vraie.
Alors 
$$(a+1)^p = \sum_{i=0}^p {C_p^i}a^i.$$
Mais d'apr\`es la question pr\'ec\'edente pour $0 < i < p$, $p$ divise $C_p^i$.
En termes de modulo nous obtenons:
$$ (a+1)^p \equiv C_p^0 a^0 + C_p^pa^p \equiv 1+a^p \pmod{p}.$$
Par l'hypoth\`ese de r\'ecurrence nous savons que $a^p \equiv a \pmod{p}$, donc
$$(a+1)^p \equiv a+1 \pmod{p}.$$ Nous venons de prouver que $\mathcal{H}_{a+1}$ est vraie.
Par le principe de r\'ecurrence alors quelque soit $a\in \Nn^*$ nous avons:
$$a^p \equiv a \pmod{p}.$$}
\indication{\begin{enumerate}
  \item \'Ecrire $$C_p^i = \frac{p(p-1)(p-2)\ldots(p-(i+1))}{i!}$$
et utiliser le lemme de Gauss ou le lemme d'Euclide.
  \item Raisonner avec les modulos, c'est-\`a-dire prouver $a^p \equiv a \pmod{p}$.

\end{enumerate}}
\end{enumerate}
}
