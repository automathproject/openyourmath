\uuid{wn8l}
\exo7id{230}
\auteur{bodin}
\organisation{exo7}
\datecreate{1998-09-01}
\isIndication{false}
\isCorrection{true}
\chapitre{Dénombrement}
\sousChapitre{Binôme de Newton et combinaison}

\contenu{
\texte{
D\'emontrer les formules suivantes :
}
\begin{enumerate}
    \item \question{$C_n^m=C_m^{n-m}$ (on pourra utiliser le fait que $\mathcal{P}(E)
    \longrightarrow \mathcal{P}(E) A\mapsto A^c$ est une bijection.)}
    \item \question{$C_n^m=C_{n-1}^m+C_{n-1}^{m-1},$}
    \item \question{$C_{n}^{m}=C_{n-2}^{m} +2C_{n-2}^{m-1}+C_{n-2}^{m-2}.$}
\reponse{
L'application $\Phi$ est une bijection : son inverse est $\Phi$
elle-m\^eme.

Supposons que $E$ soit un ensemble fini. Notre bijection $\Phi$
envoie un ensemble $\mathcal{Q} \subset \mathcal{P}(E)$ sur un
ensemble de m\^eme cardinal.

Choisissons $E$ un ensemble \`a $n$ \'el\'ements, et soit $p\le
n$. Soit $\mathcal{Q} \subset \mathcal{P}(E)$ :
$$\mathcal{Q} = \left\lbrace F \subset E,\ \ \mathrm{Card} F = p \right\rbrace.$$

Nous savons que $\mathrm{Card} \mathcal{Q} =C_n^p$ (c'est la d\'efinition
de $C^p_n$).
 De plus
\begin{align*}
 \Phi(\mathcal{Q})
  &= \left\lbrace \Phi(F),\ \ F \subset E,\ \ \mathrm{Card} F = p \right\rbrace \\
  &= \left\lbrace \complement F,\ \ F \subset E,\ \ \mathrm{Card} F = p \right\rbrace \\
  &= \left\lbrace G\subset E,\ \ \mathrm{Card} G = n-p \right\rbrace.
\end{align*}

Donc $\mathrm{Card} \Phi(\mathcal{Q}) = C_n^{n-p}$. Et comme $\Phi$ est
une bijection, $\mathrm{Card} \Phi(\mathcal{Q}) = \mathrm{Card} (\mathcal{Q})$,
donc $C_n^{n-p}=C_n^p$.
}
\end{enumerate}
}
