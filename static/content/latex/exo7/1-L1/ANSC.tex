\uuid{ANSC}
\exo7id{3236}
\auteur{quercia}
\organisation{exo7}
\datecreate{2010-03-08}
\isIndication{false}
\isCorrection{false}
\chapitre{Polynôme, fraction rationnelle}
\sousChapitre{Racine, décomposition en facteurs irréductibles}

\contenu{
\texte{
Soit $\alpha \in \C$. On dit que $\alpha$ est {\it alg{\'e}brique\/}
s'il existe un polyn{\^o}me $P \in {\Q[X]}$ tel que $P(\alpha) = 0$. \\
Le polyn{\^o}me unitaire de plus bas degr{\'e} v{\'e}rifiant $P(\alpha) = 0$ est appel{\'e} :
{\it polyn{\^o}me minimal de $\alpha$}.
}
\begin{enumerate}
    \item \question{Soit $\alpha$ alg{\'e}brique de polyn{\^o}me minimal $P$.  D{\'e}montrer que
  $P$ est irr{\'e}\-duc\-tible dans ${\Q[X]}$ et que $\alpha$ est racine
  simple de $P$.}
    \item \question{Soit $\alpha$ alg{\'e}brique, et $P \in {\Q[X]}$ tel que $P(\alpha) = 0$.
     On suppose que la multiplicit{\'e} de $\alpha$ dans $P$
     est strictement sup{\'e}rieure {\`a} $\frac 12\deg P$. D{\'e}montrer que $\alpha \in \Q$.}
\end{enumerate}
}
