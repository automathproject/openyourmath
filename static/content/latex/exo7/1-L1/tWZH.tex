\uuid{tWZH}
\exo7id{5137}
\auteur{rouget}
\organisation{exo7}
\datecreate{2010-06-30}
\isIndication{false}
\isCorrection{true}
\chapitre{Dénombrement}
\sousChapitre{Binôme de Newton et combinaison}

\contenu{
\texte{
\emph{La difficulté va en augmentant graduellement de facile à assez difficile sans être
insurmontable.}
}
\begin{enumerate}
    \item \question{Calculer $\binom{n}{0}+\binom{n}{1}+...+\binom{n}{n}$.}
    \item \question{Montrer que $\binom{n}{0}+\binom{n}{2}+\binom{n}{4}+...=\binom{n}{1}+\binom{n}{3}+\binom{n}{5}+...$ et trouver la valeur commune des
deux sommes.}
    \item \question{Calculer les sommes $\binom{n}{0}+\binom{n}{3}+\binom{n}{6}+...$ et $\binom{n}{0}+\binom{n}{4}+\binom{n}{8}+...$.}
    \item \question{Montrer que $\forall n\in\Nn^*,\;\forall k\in\llbracket1,n\rrbracket,\;k\binom{n}{k}=n\binom{n-1}{k-1}$.}
    \item \question{Montrer que $\binom{n}{0}^2 +\binom{n}{1}^2 + ... +\binom{n}{n}^2 =\binom{2n}{n}$ (utiliser le polynôme $(1+x)^{2n}$).}
    \item \question{Calculer les sommes $0.\binom{n}{0}+1.\binom{n}{1}+ ...+n.\binom{n}{n}$ et $\frac{\binom{n}{0}}{1}+\frac{\binom{n}{1}}{2}+...
+\frac{\binom{n}{n}}{n+1}$ (considérer dans chaque cas un certain polynôme astucieusement choisi).}
    \item \question{Montrer que $\binom{p}{p}+\binom{p+1}{p}... +\binom{n}{p}=\binom{n+1}{p+1}$ où $0\leq p\leq n$. Interprétation dans le triangle de \textsc{Pascal}~?}
    \item \question{\begin{enumerate}}
    \item \question{Soit $I_n=\int_{0}^{1}(1-x^2)^n\;dx$. Trouver une relation de récurrence liant $I_n$ et $I_{n+1}$ et en déduire
$I_n$ en fonction de $n$ (faire une intégration par parties dans $I_n-I_{n+1}$).}
    \item \question{Démontrer l'identité valable pour
$n\geq1$~:~$1-\frac{\binom{n}{1}}{3}+\frac{\binom{n}{2}}{5}+...+(-1)^n\frac{\binom{n}{n}}{2n+1}
=\frac{2.4.....(2n)}{1.3...(2n+1)}$.}
\reponse{
D'après la formule du binôme de \textsc{Newton},
\begin{center}
\shadowbox{
$\forall n\in\Nn,\;\sum_{k=0}^{n}\binom{n}{k}=(1+1)^n=2^n.$
}
\end{center}
Soit $n$ un entier naturel non nul. Posons $S_1=\sum_{k=0}^{E(n/2)}\binom{n}{2k}$ et
$S_2=\sum_{k=0}^{E((n-1)/2)}\binom{n}{2k+1}$. Alors

$$S_1-S_2=\sum_{k=0}^{n}(-1)^k\binom{n}{k}=(1-1)^n=0\;(\mbox{car}\;n\geq1),$$
et donc $S_1=S_2$. Puis $S_1+S_2=\sum_{k=0}^{n}\binom{n}{k}=2^n$, et donc $S_1=S_2=2^{n-1}$.
\begin{center}
\shadowbox{
$\forall n\in\Nn^*,\;\binom{n}{0}+\binom{n}{2}+\binom{n}{4}+\ldots=\binom{n}{1}+\binom{n}{3}+\binom{n}{5}+\ldots=2^{n-1}.$
}
\end{center}
En posant $j=e^{2i\pi/3}$, on a~:

$$\sum_{k=0}^{n}\binom{n}{k}=(1+1)^n=2^n,\;\sum_{k=0}^{n}\binom{n}{k}j^k=(1+j)^n\;\mbox{et}\;\sum_{k=0}^{n}\binom{n}{k}j^{2k}
=(1+j^2)^n.$$

En additionnant ces trois égalités, on obtient

$$\sum_{k=0}^{n}\binom{n}{k}(1+j^k+j^{2k})=2^n+(1+j)^n+(1+j^2)^n.$$

Maintenant,
\begin{itemize}
[-] si $k\in3\Nn$, il existe $p\in\Nn$ tel que $k=3p$ et $1+j^k+j^{2k}=1+(j^3)^p+(j^3)^{2p}=3$ car $j^3=1$.
[-] si $k\in3\Nn+1$, il existe $p\in\Nn$ tel que $k=3p+1$ et $1+j^k+j^{2k}=1+j(j^3)^p+j^2(j^3)^{2p}=1+j+j^2=0$
[-] si $k\in3\Nn+2$, il existe $p\in\Nn$ tel que $k=3p+2$ et
$1+j^k+j^{2k}=1+j^2(j^3)^p+j^4(j^3)^{2p}=1+j^2+j=0$.
\end{itemize}

Finalement, $\sum_{k=0}^{n}\binom{n}{k}(1+j^k+j^{2k})=3\sum_{k=0}^{E(n/3)}\binom{n}{3k}$. Par suite,

\begin{align*}
\sum_{k=0}^{E(n/3)}\binom{n}{3k}&=\frac{1}{3}(2^n+(1+j)^n+(1+j^2)^n)=\frac{1}{3}(2^n+2\Re((1+j)^n))\\
 &=\frac{1}{3}(2^n+2\Re((-j^2)^n))=\frac{1}{3}(2^n+2\cos\frac{n\pi}{3})
\end{align*}
Pour $1\leq k\leq n$, on a

$$k\binom{n}{k}=k\frac{n!}{k!(n-k)!}=n\frac{(n-1)!}{(k-1)!((n-1)-(k-1))!}=k\binom{n-1}{k-1}.$$
$\binom{2n}{n}$ est le coefficient de $x^n$ dans le développement de $(1+x)^{2n}$. Mais d'autre part ,

$$(1+x)^{2n}=(1+x)^n(1+x)^n=(\sum_{k=0}^{n}\binom{n}{k}x^k)(\sum_{k=0}^{n}\binom{n}{k}x^k).$$

Dans le développement de cette dernière expression, le coefficient de $x^n$ vaut $\sum_{k=0}^{n}\binom{n}{k}\binom{n}{n-k}$ ou
encore $\sum_{k=0}^{n}\binom{n}{k}^2$. Deux polynômes sont égaux si et seulement si ils ont mêmes
coefficients et donc

$$\binom{2n}{n}=\sum_{k=0}^{n}\binom{n}{k}^2.$$
\begin{itemize}
[\textbf{1ère solution.}] Pour $x$ réel, posons $P(x)=\sum_{k=1}^{n}k\binom{n}{k}x^{k-1}$.

Pour $x$ réel, $$P(x)=(\sum_{k=0}^{n}\binom{n}{k}x^{k})'=((1+x)^n)'=n(1+x)^{n-1}.$$

En particulier, pour $x=1$, on obtient~:

$$\sum_{k=1}^{n}k\binom{n}{k}=n(1+1)^{n-1}=n2^{n-1}.$$
[\textbf{2ème solution.}] D'après 4),

$$\sum_{k=1}^{n}k\binom{n}{k}=\sum_{k=1}^{n}n\binom{n-1}{k-1}=n\sum_{k=0}^{n-1}\binom{n-1}{k}=n(1+1)^{n-1}=n2^{n-1}.$$
\end{itemize}

\begin{itemize}
[\textbf{1ère solution.}] Pour $x$ réel, posons $P(x)=\sum_{k=0}^{n}\binom{n}{k}\frac{x^{k+1}}{k+1}$. On a

$$P'(x)=\sum_{k=0}^{n}\binom{n}{k}x^k=(1+x)^n,$$

et donc, pour $x$ réel,

$$P(x)=P(0)+\int_{0}^{x}P'(t)\;dt=\int_{0}^{1}(1+t)^n\;dt=\frac{1}{n+1}((1+x)^{n+1}-1).$$

En particulier, pour $x=1$, on obtient

$$\sum_{k=0}^{n}\frac{\binom{n}{k}}{k+1}=\frac{2^{n+1}-1}{n+1}.$$
[\textbf{2ème solution.}] D'après 4), $(n+1)\binom{n}{k}=(k+1)\binom{n+1}{k+1}$ et donc

$$\sum_{k=0}^{n}\frac{\binom{n}{k}}{k+1}=\sum_{k=0}^{n}\frac{\binom{n+1}{k+1}}{n+1}=\frac{1}{n+1}\sum_{k=1}^{n+1}\binom{n+1}{k}
=\frac{1}{n+1}((1+1)^{n+1}-1)=\frac{2^{n+1}-1}{n+1}.$$
\end{itemize}
Pour $1\leq k\leq n-p$, $\binom{p+k}{p}=\binom{p+k+1}{p+1}-\binom{p+k}{p+1}$ (ce qui reste vrai pour $k=p$ en tenant
compte de $\binom{p}{p+1}=0$). Par suite,

\begin{align*}
\sum_{k=0}^{n-p}\binom{p+k}{p}&=1+\sum_{k=1}^{n-p}\binom{p+k+1}{p+1}-\binom{p+k}{p+1}=1+\sum_{k=2}^{n-p+1}\binom{p+k}{p+1}
-\sum_{k=1}^{n-p}\binom{p+k}{p+1}\\
 &=1+\binom{n+1}{p+1}-1=\binom{n+1}{p+1}.
\end{align*}

Interprétation dans le triangle de \textsc{Pascal}. Quand on descend dans le triangle de \textsc{Pascal}, le long de la
colonne $p$, du coefficient $\binom{p}{p}$ (ligne $p$) au coefficient $\binom{p}{n}$ (ligne $n$), et que l'on additionne ces
coefficients, on trouve$\binom{n+1}{p+1}$ qui se trouve une ligne plus bas et une colonne plus loin.
\begin{enumerate}
Pour $n$ naturel donné, posons $I_n=\int_{0}^{1}(1-x^2)^n\;dx$. Une intégration par parties fournit:

\begin{align*}
I_n-I_{n+1}&=\int_{0}^{1}((1-x^2)^n-(1-x^2)^{n+1})\;dx=\int_{0}^{1}x^2(1-x^2)^n\;dx=\int_{0}^{1}x.x(1-x^2)^{n+1}\;dx\\
 &=\left[-x\frac{(1-x^2)^{n+1}}{2(n+1)}\right]_{0}^{1}+\frac{1}{2(n+1)}\int_{0}^{1}(1-x^2)^{n+1}\;dx
=\frac{1}{2(n+1)}I_{n+1}
\end{align*}

et donc $2(n+1)(I_n-I_{n+1})=I_{n+1}$ ou encore~:

$$\forall n\in\Nn,\;(2n+3)I_{n+1}=2(n+1)I_n.$$

On a déjà $I_0=1$. Puis, pour $n\geq1$,

\begin{align*}
I_n=\frac{2n}{2n+1}I_{n-1}=\frac{2n}{2n+1}\frac{2n-2}{2n-1}...\frac{2}{3}I_0=\frac{(2n)(2n-2)...2}{(2n+1)(2n-1)...3
.1}.
\end{align*}
Pour $n$
naturel non nul donné~:

\begin{align*}
1-\frac{\binom{n}{1}}{3}+\frac{\binom{n}{2}}{5}+...+(-1)^n\frac{\binom{n}{n}}{2n+1}&=\int_{0}^{1}(1-\binom{n}{1}x^2+\binom{n}{2}x^4+...
+(-1)^n\binom{n}{n}x^{2n})\;dx\\
 &=\int_{0}^{1}(1-x^2)^n\;dx=I_n
=\frac{(2n)(2n-2)...2}{(2n+1)(2n-1)...3.1}.
\end{align*}
}
\end{enumerate}
}
