\uuid{PPCO}
\exo7id{107}
\auteur{bodin}
\organisation{exo7}
\datecreate{1998-09-01}
\video{ZCLCoGR3hcc}
\isIndication{false}
\isCorrection{true}
\chapitre{Logique, ensemble, raisonnement}
\sousChapitre{Logique}

\contenu{
\texte{
Soit  $f$  une application de  ${\Rr}$  dans
${\Rr}$. Nier, de la mani\`ere la plus pr\' ecise possible, les \'
enonc\' es qui suivent :
}
\begin{enumerate}
    \item \question{Pour tout  $x\in {\Rr}\ f(x)\leq 1$.}
\reponse{Cette assertion se d\'ecompose de la mani\`ere suivante : ( Pour tout $x\in
  {\Rr}$) ($f(x)\leq 1$). La n\'egation de ``( Pour tout $x\in
  {\Rr}$)" est ``Il existe $x\in {\Rr}$" et la n\'egation de ``($f(x)\leq 1$)" est
$ f(x)>1$. Donc la n\'egation de l'assertion compl\`ete est : ``Il
existe $x\in {\Rr},  f(x)>1$".}
    \item \question{L'application  $f$  est croissante.}
\reponse{Rappelons comment se traduit l'assertion ``L'application $f$ est
  croissante" : ``pour tout couple de r\'eels $(x_1,x_2)$, si $x_1\leq x_2$
  alors $f(x_1) \leq f(x_2)$". Cela se d\'ecompose en : ``(pour tout couple de
  r\'eels $x_1$ et $x_2$) ($x_1\leq x_2$ implique $f(x_1) \leq f(x_2)$)".
La n\'egation de la premi\`ere partie est : ``(il existe un couple de
r\'eels $(x_1,x_2)$)" et la n\'egation de la deuxi\`eme partie est :
``($x_1\leq x_2$ et $f(x_1) > f(x_2)$)". Donc la n\'egation de
l'assertion compl\`ete est : ``Il existe $x_1\in \Rr$ et $x_2\in
\Rr$ tels que $x_1 \leq x_2$ et $f(x_1) > f(x_2)$".}
    \item \question{L'application  $f$  est croissante et positive.}
\reponse{La n\'egation est : ``l'application $f$ n'est pas croissante ou n'est pas
  positive". On a d\'ej\`a traduit ``l'application $f$ n'est pas croissante",
  traduisons ``l'application $f$ n'est pas positive" : ``il existe $x \in \Rr,
  f(x)<0$". Donc la n\'egation de l'assertion compl\`ete est : ``
Il existe $x_1\in \Rr$ et $x_2\in \Rr$ tels que $x_1<x_2$ et
$f(x_1) \geq f(x_2)$, ou il existe $x \in \Rr,
  f(x)<0$".}
    \item \question{Il existe  $x\in {\Rr}^+$  tel que  $f(x)\leq 0$.}
\reponse{Cette assertion se d\'ecompose de la mani\`ere suivante : ``(Il existe $x\in
  {\Rr}^+$) ($f(x)\leq 0$)". La n\'egation de la premi\`ere partie est : ``(pour
  tout $x\in  {\Rr}^+$)", et celle de la seconde est :``($f(x)> 0$)".
  Donc la n\'egation de l'assertion compl\`ete est : ``Pour tout $x\in {\Rr}^+$,  $f(x)> 0$".}
    \item \question{Il existe $x\in \Rr$ tel que quel que soit $y\in \Rr$, si $x<y$ alors $f(x)>f(y)$.}
\reponse{Cette assertion se d\'ecompose de la mani\`ere suivante : 
``($\exists x \in {\Rr}$)($\forall y \in {\Rr}$)($ x<y \Rightarrow f(x)>f(y)$)".  
La n\'egation de la premi\`ere partie est ``($\forall x \in  {\Rr}$)", celle de la
seconde est ``($\exists y \in {\Rr}$)", et celle de la troisi\`eme est
``($ x<y \mathrm{\ et\ } f(x)\leq f(y)$)". Donc la n\'egation de
l'assertion compl\`ete est : `` $\forall x \in {\Rr}, \exists y \in
{\Rr} \text{ tels que } x<y \text{ et } f(x)\leq f(y)$".}
\end{enumerate}
}
