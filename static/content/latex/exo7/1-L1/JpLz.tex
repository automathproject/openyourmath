\uuid{JpLz}
\exo7id{5113}
\auteur{rouget}
\organisation{exo7}
\datecreate{2010-06-30}
\isIndication{false}
\isCorrection{true}
\chapitre{Logique, ensemble, raisonnement}
\sousChapitre{Ensemble}

\contenu{
\texte{
Soient $(A_i)_{i\in I}$ une famille de parties d'un ensemble $E$ indéxée par un ensemble $I$ et $(B_i)_{i\in I}$ une
famille de parties d'un ensemble $F$ indéxée par un ensemble $I$. Soit $f$ une
application de $E$ vers $F$. Comparer du point de vue de l'inclusion les parties suivantes~:
}
\begin{enumerate}
    \item \question{$f(\bigcup_{i\in I}A_i)$ et $\bigcup_{i\in I}f(A_i)$ (recommencer par $f(A\cup B)$ si on n'a pas les 
idées claires).}
\reponse{Soit $x\in E$.

\begin{align*}
x\in f\left(\bigcup_{i\in I}A_i\right)&\Leftrightarrow\exists y\in\bigcup_{i\in I}A_i/\;x=f(y)\Leftrightarrow\exists i\in I,\;\exists y\in
A_i/\;x=f(y)\\
 &\Leftrightarrow\exists i\in I/\;x\in f(A_i)\Leftrightarrow x\in\bigcup_{i\in I}f(A_i)
\end{align*}
Donc

\begin{center}
\shadowbox{
$f\left(\displaystyle\bigcup_{i\in I}A_i\right)=\displaystyle\bigcup_{i\in I}f(A_i)$.
}
\end{center}}
    \item \question{$f(\bigcap_{i\in I}A_i)$ et $\bigcap_{i\in I}f(A_i)$.}
\reponse{Soit $x\in E$.

\begin{align*}
x\in f\left(\bigcap_{i\in I}A_i\right)&\Leftrightarrow\exists y\in\bigcap_{i\in I}A_i/\;x=f(y)\Leftrightarrow\exists y\in E/\;\forall i\in I,\;
y\in A_i\;\mbox{et}\;x=f(y)\\
 &\Rightarrow\forall i\in I/\;\exists y\in A_i/\;x=f(y)\Leftrightarrow\forall i\in I/\;x\in f(A_i)\\
 &\Leftrightarrow x\in\bigcap_{i\in I}f(A_i)
\end{align*}
Donc

\begin{center}
\shadowbox{
$f\left(\displaystyle\bigcap_{i\in I}A_i\right)\subset\displaystyle\bigcap_{i\in I}f(A_i)$.
}
\end{center}
L'inclusion contraire n'est pas toujours vraie. Par exemple, pour $x$ réel on pose $f(x)=x^2$ puis $A=\{-1\}$ et
$B=\{1\}$. $A\cap B=\varnothing$ et donc $f(A\cap B)=\varnothing$ puis $f(A)=f(B)=\{1\}$ et donc $f(A)\cap f(B)=\{1\}$.}
    \item \question{$f(E\setminus A_i)$ et $F\setminus f(A_i)$.}
\reponse{Il n'y a aucune inclusion vraie entre $f(E\setminus A)$ et $F\setminus f(A)$. Par exemple, soit
$\begin{array}[t]{cccc}
f~:&\Rr&\rightarrow&\Rr\\
 &x&\mapsto&x^2
\end{array}$ et $A=[-1,2]$.
$f(A)=[0,4]$ et donc $C_{\Rr}(f(A))=]-\infty,0[\cup]4,+\infty[$ mais $f(C_{\Rr}A)=f(]-\infty,-1[\cup]2,+\infty[)
=]1,+\infty[$ et aucune inclusion entre les deux parties n'est vraie.}
    \item \question{$f^{-1}(\bigcap_{i\in I}B_i)$ et $\bigcap_{i\in I}f^{-1}(B_i)$.}
\reponse{Soit $x\in E$.

$$x\in f^{-1}\left(\bigcap_{i\in I}B_i\right)\Leftrightarrow f(x)\in\bigcap_{i\in I}B_i\Leftrightarrow\forall i\in I,\;f(x)\in B_i\Leftrightarrow\forall i\in
I,\;x\in f^{-1}(B_i)\Leftrightarrow x\in\bigcap_{i\in I}f^{-1}(B_i).$$
Donc,

\begin{center}
\shadowbox{
$f^{-1}(\displaystyle\bigcap_{i\in I}B_i)=\displaystyle\bigcap_{i\in I}f^{-1}(B_i)$.
}
\end{center}}
    \item \question{$f^{-1}(\bigcup_{i\in I}B_i)$ et $\bigcup_{i\in I}f^{-1}(B_i)$.}
\reponse{Soit $x\in E$.

$$x\in f^{-1}(\bigcup_{i\in I}B_i)\Leftrightarrow f(x)\in\bigcup_{i\in I}B_i\Leftrightarrow\exists i\in I,\;f(x)\in B_i\Leftrightarrow\exists i\in
I,\;x\in f^{-1}(B_i)\Leftrightarrow x\in\bigcup_{i\in I}f^{-1}(B_i).$$
Donc,

\begin{center}
\shadowbox{
$f^{-1}(\displaystyle\bigcup_{i\in I}B_i)=\displaystyle\bigcup_{i\in I}f^{-1}(B_i)$.
}
\end{center}}
    \item \question{$f^{-1}(F\setminus B_i)$ et $E\setminus f^{-1}(Bi)$.}
\reponse{Soit $x\in E$.

$$x\in f^{-1}(F\setminus B_i)\Leftrightarrow f(x)\in F\setminus B_i\Leftrightarrow f(x)\notin B_i\Leftrightarrow x\notin f^{-1}(B_i)\Leftrightarrow x\in
E\setminus f^{-1}(B_i).$$
Donc,

\begin{center}
\shadowbox{
$f^{-1}(F\setminus B_i)=E\setminus f^{-1}(B_i)$.
}
\end{center}}
\end{enumerate}
}
