\uuid{xO1E}
\exo7id{6961}
\auteur{blanc-centi}
\organisation{exo7}
\datecreate{2014-04-01}
\video{8EYEgxhsGFA}
\isIndication{true}
\isCorrection{true}
\chapitre{Polynôme, fraction rationnelle}
\sousChapitre{Racine, décomposition en facteurs irréductibles}

\contenu{
\texte{
Soit $n\in\Nn$. Montrer qu'il existe un unique $P\in\Cc[X]$ tel que 
$$\forall z\in\Cc^* \qquad  P\left(z+\frac{1}{z}\right) = z^n+\frac{1}{z^n}$$

Montrer alors que toutes les racines de $P$ sont réelles, simples, 
et appartiennent à l'intervalle $[-2,2]$.
}
\indication{Pour l'existence, preuve par récurrence sur $n$. Pour les racines,
montrer que $P(x)=2\cos(n\Arccos(x/2))$.}
\reponse{
Commençons par remarquer que si $P$ et $Q$ sont deux polynômes 
  qui conviennent, alors pour tout $z\in\Cc^*$, 
$P\left(z+\frac{1}{z}\right)-Q\left(z+\frac{1}{z}\right)=0$. 
En appliquant cette égalité à $z=e^{i\theta}$, on obtient 
$(P-Q)(2\cos\theta)=0$. Le polynôme $P-Q$ a une infinité 
de racines, donc il est nul, ce qui montre $P=Q$.
Montrons l'existence de $P$ par récurrence forte sur $n$:
  \begin{itemize}
Pour $n=0$, $P=2$ convient et pour $n=1$, $P=X$ convient.
Passage des rangs $k\le n$ au rang $n+1$. 
    Si on note $P_k$ le polynôme construit pour $k\le n$, on a 
$$z^{n+1}+\frac{1}{z^{n+1}}=(z+\frac{1}{z})(z^n+\frac{1}{z^n})-(z^{n-1}+\frac{1}{z^{n-1}})
=(z+\frac{1}{z})P_n(z+\frac{1}{z})-P_{n-1}(z+\frac{1}{z})$$
donc $P_{n+1}(X)=XP_n(X)-P_{n-1}(X)$ convient.
On a ainsi construit $P_n$ pour tout $n$ (avec $\deg P_n =n$). 
  \end{itemize}
Fixons $n$ et notons $P$ le polynôme obtenu.
  Pour tout $\theta\in\R$, $P(e^{i\theta}+e^{-i\theta})=e^{in\theta}+e^{-in\theta}$ 
  donc $P(2\cos(\theta))=2\cos(n\theta)$. 
  
  En posant $x=2\cos(\theta)$ et donc $\theta = \Arccos(\frac x2)$ on obtient la relation
  Ainsi,
$$P(x)=2\cos(n\Arccos(\frac x2)) \qquad \forall x\in[-2,2]$$
Le polynôme dérivée est $P'(x)=\frac{n}{\sqrt{1-(\frac{x}{2})^2}}\sin(n\Arccos(\frac x2))$, 
il s'annule en changeant de signe en chaque 
$\alpha_k = 2\cos(\frac{k\pi}{n})$, ainsi $P'(\alpha_k)=0$ pour $k = 0,\ldots,n$.

On calcule aussi que $P(\alpha_k) = \pm 2$.
Le tableau de signe montre que $P$ est alternativement croissante 
(de $-2$ à $+2$) puis décroissante (de $+2$ à $-2$) 
sur chaque intervalle
$[\alpha_{k+1}, \alpha_k]$, qui forment une partition de $[-2,2]$.
D'après le théorème des valeurs intermédiaires, $P$ possède $n$ racines simples 
(une dans chaque intervalle $[\alpha_{k+1}, \alpha_k]$) dans $[-2,2]$. 
Puisque $P$ est de degré $n$, on a ainsi obtenu toutes ses racines.
}
}
