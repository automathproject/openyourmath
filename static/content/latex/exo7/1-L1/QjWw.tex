\uuid{QjWw}
\exo7id{886}
\auteur{legall}
\organisation{exo7}
\datecreate{1998-09-01}
\video{1rxZQKF8yc0}
\isIndication{true}
\isCorrection{true}
\chapitre{Espace vectoriel}
\sousChapitre{Définition, sous-espace}

\contenu{
\texte{
D\' eterminer lesquels des
ensembles $E_1$, $E_2$, $E_3$ et $E_4$ sont des sous-espaces
vectoriels de ${\Rr}^3$. 

 $E_1 =\{ (x,y,z)\in {\Rr}^3\ \mid \ 3x-7y = z \} $ 

 $E_2 =\{(x,y,z)\in {\Rr}^3\ \mid \ x^2-z^2=0 \} $  

 $E_3=\{ (x,y,z)\in {\Rr}^3\ \mid \ x+y-z=x+y+z=0 \} $ 

 $E_4 =\{ (x,y,z)\in {\Rr}^3\ \mid \ z(x^2+y^2)=0 \} $
}
\indication{\begin{enumerate}
\item $E_1$ est un sous-espace vectoriel.
\item $E_2$ n'est pas un sous-espace vectoriel.
\item $E_3$ est un sous-espace vectoriel.
\item $E_4$ n'est pas un sous-espace vectoriel.
\end{enumerate}}
\reponse{
\begin{enumerate}
$(0,0,0) \in E_1$.
Soient $(x,y,z)$ et $(x',y',z')$ deux \' el\' ements
de $E_1$. On a donc $3x-7y=z$ et $3x'-7y'=z'$. Donc $3(x+x')-7(y+y')=(z+z')$, d'où $(x+x',y+y',z+z')$  
appartient \`a  $E_1$.
Soit $\lambda \in {\R}$ et $(x,y,z)\in E_1$.  Alors la relation $3x-7y=z$ implique
  que $3 (\lambda x) -7(\lambda y)=\lambda z$
  donc que $\lambda(x,y,z)=(\lambda x,\lambda  y,\lambda  z)$ appartient \`a  $E_1$.
}
}
