\uuid{JA0o}
\exo7id{5568}
\auteur{rouget}
\organisation{exo7}
\datecreate{2010-10-16}
\isIndication{false}
\isCorrection{true}
\chapitre{Espace vectoriel}
\sousChapitre{Système de vecteurs}

\contenu{
\texte{
Soit $f(x) =\ln(1+x)$ pour $x$ réel positif. Soient $f_1 = f$, $f_2=f\circ f$ et $f_3= f\circ f\circ f$. Etudier la liberté de $(f_1,f_2,f_3)$ dans $[0,+\infty[^{[0,+\infty[}$.
}
\reponse{
Les fonctions $f_1$, $f_2$ et $f_3$ sont bien définies sur $\Rr^+$.

Soient $a$, $b$ et $c$ trois réels tels que $af_1+bf_2 +cf_3=0$.

\textbf{Première solution.} Si $a$ est non nul, la fonction $af_1+bf_2+cf_3$ est équivalente au voisinage de $+\infty$ à $a\ln x$ et ne peut donc être égale à la fonction nulle. Donc $a=0$. Puis si $b$ est non nul, la fonction $af_1+bf_2+cf_3=bf_2+cf_3$ est équivalente à $b\ln(\ln x)$ et ne peut être égale à la fonction nulle. Donc $b =0$. Puis $c=0$.

\textbf{Deuxième solution.} On effectue un développement limité à un ordre suffisant de la fonction $af_1+bf_2+cf_3$ quand $x$ tend vers $0$ :

$f_1(x)=\ln(1+x)\underset{x\rightarrow0}{=}x-\frac{x^2}{2}+\frac{x^3}{3}+o(x^3)$ puis

\begin{align*}\ensuremath
f_2(x)&=\ln(1+f_1(x))\underset{x\rightarrow0}{=}\ln\left(1+x-\frac{x^2}{2}+\frac{x^3}{3}+o(x^3)\right) =\left(x-\frac{x^2}{2}+\frac{x^3}{3}\right)-\frac{1}{2}\left(x-\frac{x^2}{2}\right)^2+\frac{1}{3}x^3+o(x^3)\\
 &=x-x^2+\frac{7}{6}x^3+o(x^3)
\end{align*}

puis

\begin{align*}\ensuremath
f_3(x)&=\ln(1+f_2(x))\underset{x\rightarrow0}{=}\ln\left(1+x-x^2+\frac{7}{6}x^3+o(x^3)\right)=\left(x-x^2+\frac{7}{6}x^3\right)-\frac{1}{2}\left(x-x^2\right)^2+\frac{1}{3}x^3+o(x^3)\\
 &=x-\frac{3}{2}x^2+\frac{5}{2}x^3+o(x^3).
\end{align*}

Par suite, $af_1(x)+bf_2(x)+cf_3(x)\underset{x\rightarrow0}{=}(a+b+c)x+\left(-\frac{a}{2}-b-\frac{3c}{2}\right)x^2+\left(\frac{a}{3}+\frac{7b}{6}+\frac{5c}{2}\right)x^3+o(x^3)$. L'égalité $af_1+bf_2+cf_3=0$ fournit, par identification des parties régulières des développements limités à l'ordre trois en zéro :

\begin{center}
$\left\{
\begin{array}{l}
a+b+c=0\\
-\frac{a}{2}-b-\frac{3c}{2}=0\\
\rule{0mm}{6mm}\frac{a}{3}+\frac{7b}{6}+\frac{5c}{2}
\end{array}
\right.$ ou encore $\left\{
\begin{array}{l}
a+b+c=0\\
a+2b+3c=0\\
2a+7b+15c=0
\end{array}
\right.$.
 \end{center}
 
 
Comme $\left|
\begin{array}{ccc}
1&1&1\\
1&2&3\\
2&7&15
\end{array}
\right|=\left|
\begin{array}{ccc}
1&0&0\\
1&1&1\\
2&5&8
\end{array}
\right|=3\neq0$, on a donc $a=b=c=0$.
}
}
