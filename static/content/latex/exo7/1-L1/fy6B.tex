\uuid{fy6B}
\exo7id{2779}
\auteur{tumpach}
\organisation{exo7}
\datecreate{2009-10-25}
\isIndication{false}
\isCorrection{false}
\chapitre{Espace vectoriel}
\sousChapitre{Définition, sous-espace}

\contenu{
\texte{
Qu'est -ce qui emp\^eche de d\'efinir les m\^emes op\'erations que dans l'exercice pr\'ec\'edent sur les ensembles suivants~?
}
\begin{enumerate}
    \item \question{[(a)] $E = \textrm{l'ensemble des solutions}\,\,\, (x, y, z)\in \mathbb{R}^3\,\,\, \textrm{de l'\'equation} \quad  
\mathcal{S}_3~: 
x -  2y  +  3z  =  3
$ ;}
    \item \question{[(b)] $E = $ l'ensemble des fonctions $y(x)$ telles que $y''(x)\sin x  + x^3 y^2(x) + y(x) \log x = 0, \forall x >0$ ;}
    \item \question{[(c)]  $E = \mathbb{N}$ ;}
    \item \question{[(d)] $E = \mathbb{Z}$ ;}
    \item \question{[(e)] $E = \mathbb{R}^{+}$ ;}
    \item \question{[(f)] $ E = \mathbb{Q}^n$ ;}
    \item \question{[(g)] $E = \textrm{ l'ensemble des suites}\,\,\,  (x_n)_{n\in\mathbb{N}}$ de nombres positifs ;}
    \item \question{[(h)] $E = $ l'ensemble des fonctions r\'eelles qui prennent la valeur $1$ en $0$ ;}
    \item \question{[(i)] $E = $ l'ensemble des fonctions r\'eelles qui tendent vers $+\infty$ lorsque $x$ tend vers $+\infty$ ;}
\end{enumerate}
}
