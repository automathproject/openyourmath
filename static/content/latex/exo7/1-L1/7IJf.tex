\uuid{7IJf}
\exo7id{5572}
\auteur{rouget}
\organisation{exo7}
\datecreate{2010-10-16}
\isIndication{false}
\isCorrection{true}
\chapitre{Espace vectoriel}
\sousChapitre{Base}

\contenu{
\texte{
$E=\Rr_n[X]$. Pour $0\leqslant k\leqslant n$, on pose $P_k=X^k(1-X)^{n-k}$. Montrer que la famille $(P_k)_{0\leqslant k\leqslant n}$ est une base de $E$.
}
\reponse{
\textbf{Première solution.} Chaque $P_k$, $0\leqslant k\leqslant n$, est de degré $k+n-k=n$ et est donc dans $\Rr_n[X]$.

Les polynômes $P_k$, $0\leqslant k\leqslant n$ ont des valuations deux à deux distinctes et donc constituent une famille libre. Comme de plus $\text{card}(P_k)_{0\leqslant k\leqslant n}=n+1=\text{dim}(E)<+\infty$, la famille $(P_k)_{0\leqslant k\leqslant n}$ est une base de $E$.

\textbf{Deuxième solution.} La matrice carrée $M$ de la famille $(P_k)_{0\leqslant k\leqslant n}$ dans la base canonique de $\Rr_n[X]$ est triangulaire inférieure. Ses coefficients diagonaux sont tous non nuls car égaux à $1$. $M$ est donc inversible et $(P_k)_{0\leqslant k\leqslant n}$ est une base de $E$.
}
}
