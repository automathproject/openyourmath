\uuid{Sa7A}
\exo7id{3354}
\auteur{quercia}
\organisation{exo7}
\datecreate{2010-03-09}
\isIndication{false}
\isCorrection{true}
\chapitre{Application linéaire}
\sousChapitre{Morphismes particuliers}

\contenu{
\texte{
Soit $E$ un ev de dimension $n$ et
$\Phi : {\mathcal{L}(E)} \to {\mathcal{L}(E)}$ un automorphisme d'algèbre. On note
$(\vec e_1, \dots, \vec e_n)$ une base fixée de $E$, $(\varphi_{ij})$ la
base de $\mathcal{L}(E)$ associée
$\bigl(\varphi_{ij}(\vec e_k) = \delta_{jk}\vec e_i\bigr)$ et
$\psi_{ij} = \Phi(\varphi_{ij})$.
}
\begin{enumerate}
    \item \question{Simplifier $\psi_{ij} \circ \psi_{k\ell}$.}
\reponse{$\psi_{ij} \circ \psi_{k\ell} = \delta_{jk}\psi_{i\ell}$.}
    \item \question{En déduire qu'il existe $\vec u_1 \in E\setminus\{\vec 0\}$ tel que
     $\psi_{11}(\vec u_1) = \vec u_1$.}
\reponse{$\psi_{11}$ est un projecteur non trivial.}
    \item \question{On note $\vec u_i = \psi_{i1}(\vec u_1)$.
     Montrer que $\psi_{ij}(\vec u_k) = \delta_{jk}\vec u_i$ et en déduire que
     $(\vec u_i)$ est une base de $E$.}
\reponse{Si $\sum \lambda_k\vec u_k = \vec 0$, alors en appliquant
              $\psi_{1j}$ : $\lambda_j\vec u_1 = \vec 0  \Rightarrow  \lambda_j = 0$.}
    \item \question{Soit $f \in GL(E)$ définie par : $f(\vec e_i) = \vec u_i$.
     Montrer que : $\forall\ g \in \mathcal{L}(E),\ \Phi(g) = f\circ g \circ f^{-1}$.}
\reponse{Décomposer $g$ sur la base $(\varphi_{ij})$.}
\end{enumerate}
}
