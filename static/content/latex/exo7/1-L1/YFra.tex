\uuid{YFra}
\exo7id{2939}
\auteur{quercia}
\organisation{exo7}
\datecreate{2010-03-08}
\isIndication{false}
\isCorrection{true}
\chapitre{Nombres complexes}
\sousChapitre{Racine n-ieme}

\contenu{
\texte{
R{\'e}soudre :
}
\begin{enumerate}
    \item \question{$(z+1)^n = (z-1)^n$.}
    \item \question{$(z+1)^n = z^n = 1$.}
    \item \question{$z^4 - z^3 + z^2 - z + 1 = 0$.}
    \item \question{$1 + 2z + 2z^2 + \dots + 2z^{n-1} + z^n = 0$.}
    \item \question{$\left(\frac{1+ix}{1-ix}\right)^n = \frac{1+i\tan a}{1-i\tan a}$.}
    \item \question{$\overline x = x^{n-1}$.}
    \item \question{$\left(\frac{z+1}{z-1}\right)^3 + \left(\frac{z-1}{z+1}\right)^3 = 0$.}
\reponse{
$z=-i\mathrm{cotan}\frac {k\pi}n$.
$6\mid n \Rightarrow  z = j$ ou $j^2$. Sinon, pas de solution.
$z = \exp\frac{(2k+1)i\pi}5$, $k = 0,1,3,4$.
$z = -1$ ou $z = \exp\frac{2ik\pi}n$, $1\le k< n$.
$x = \tan\left(\frac {a+2k\pi}n\right)$.
$z = \pm i,\ \pm i(2\pm\sqrt3)$.
}
\end{enumerate}
}
