\uuid{oPyj}
\exo7id{2353}
\auteur{queffelec}
\organisation{exo7}
\datecreate{2003-10-01}
\isIndication{true}
\isCorrection{true}
\chapitre{Continuité, uniforme continuité}
\sousChapitre{Continuité, uniforme continuité}

\contenu{
\texte{
Soit $X$ un espace topologique et $f:X\to \Rr$.
}
\begin{enumerate}
    \item \question{Montrer que $f$ est continue si et seulement si pour tout $\lambda\in \Rr$, 
les ensembles $\{x\ ;\ f(x)<\lambda\}$ et $\{x\ ;\ f(x)>\lambda\}$ sont des
ouverts de $X$.}
\reponse{Sens direct. Si $f$ est continue alors $\{x \mid f(x) < \lambda\} = 
f^{-1}(]-\infty,\lambda[)$ est un ouvert comme image réciproque par une application continue de l'intervalle ouvert $]-\infty,\lambda[$. De même avec
$]\lambda,+\infty[$.

Réciproque. Tout d'abord, tout intervalle ouvert $]a,b[$, ($a<b$) peut s'écrire
$$]a,b[= ]-\infty,b[ \cap ]a,+\infty[.$$
Donc 
$$f^{-1}(]a,b[) = f^{-1}(]-\infty,b[) \cap f^{-1}(]a,+\infty[)$$
est une intersection de deux ouverts donc un ouvert de $X$.
Soit $O$ un ouvert de $\Rr$, alors $O$ peut s'écrire comme l'union dénombrables d'intervalles ouverts :
$$O = \bigcup_{i\in I} {]a_i,b_i[}.$$
Donc 
$$f^{-1}(O)= \bigcup_{i\in I} f^{-1}(]a_i,b_i[)$$
est une union d'ouvert donc un ouvert de $X$ .}
    \item \question{Montrer que si $f$ est continue, pour tout $\omega$ ouvert de $\Rr$,
$f^{-1}(\omega)$ est un $F_\sigma$ ouvert de $X$ ($F_\sigma$= r\'eunion
d\'enombrable de ferm\'es).}
\reponse{Nous le faisons d'abord pour un intervalle ouvert $]a,b[$.
$$]a,b[ = \bigcup_{j\in\Nn^*} [a+\frac 1n,b-\frac 1n].$$
Donc 
$$f^{-1}(]a,b[) = \bigcup_{j\in\Nn^*} f^{-1}([a+\frac 1j,b-\frac 1j]),$$
est une union dénombrable de fermés.
Maintenant comme pour la première question, tout ouvert $O$ de $\Rr$
s'écrit $O = \bigcup_{i\in I} ]a_i,b_i[$, avec $I$ dénombrable.
Donc on peut écrire 
$$f^{-1}(O)= \bigcup_{i\in I}\bigcup_{j\in\Nn^*} f^{-1}([a_i+\frac 1j,b_i-\frac 1j]),$$
qui est une union dénombrable de fermés (mais c'est un ouvert !).}
\indication{\begin{enumerate}
  \item Utiliser le fait que tout ouvert de $\Rr$ est l'union dénombrable d'intervalles ouverts.
  \item \'Ecrire un intervalle fermé comme union dénombrable d'intervalles
ouverts, puis utiliser la même remarque que ci-dessus.
   \end{enumerate}}
\end{enumerate}
}
