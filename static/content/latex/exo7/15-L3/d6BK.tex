\uuid{d6BK}
\exo7id{6232}
\auteur{queffelec}
\organisation{exo7}
\datecreate{2011-10-16}
\isIndication{false}
\isCorrection{false}
\chapitre{Théorème du point fixe}
\sousChapitre{Théorème du point fixe}

\contenu{
\texte{

}
\begin{enumerate}
    \item \question{Soit $X$ un espace métrique et $(f_n)$ une suite d'applications
conti\-nues à valeurs dans un espace métrique $Y$, convergeant vers $f$
uniformément sur $X$. Montrer que si
$(x_n)$ est une suite de points de $X$ convergeant vers $x\in X$, alors
$f_n(x_n)$ tend vers
$f(x)$.}
    \item \question{Application : Soit $X$ un espace métrique compact, et soit $(f_n)$ une suite
d'applications continues de $X$ dans $X$, ayant chacune un point
fixe; on suppose que la suite $(f_n)$ converge vers une fonction $f$
uniformément sur $X$. Montrer que $f$ a aussi un point fixe.}
    \item \question{Soit $K$ un convexe compact de $\Rr^n$ et $f$ une application continue de
$K$ dans $K$ vérifiant
$$\Vert f(x)-f(y)\Vert \leq \Vert x-y\Vert;$$
En considérant les fonctions $f_n$ définies sur $K$ par $f_n(x)={1\over
n}f(x_0)+(1-{1\over n})f(x)$, où $x_0\in K$, montrer que
$f$ a un point fixe. Est-il unique ? Que se passe-t-il si $K$ n'est plus
convexe ?}
\end{enumerate}
}
