\uuid{HMcb}
\exo7id{2400}
\auteur{mayer}
\organisation{exo7}
\datecreate{2003-10-01}
\isIndication{true}
\isCorrection{true}
\chapitre{Espace métrique complet, espace de Banach}
\sousChapitre{Espace métrique complet, espace de Banach}

\contenu{
\texte{
Soit $E$ un espace vectoriel norm\'e. On dit qu'une s\'erie $\sum u_k$ est normalement convergente
si la s\'erie $\sum \| u_k \|$ est convergente. On veut d\'emontrer que $E$ est complet si et seulement
si toute s\'erie normalement convergente est convergente.
}
\begin{enumerate}
    \item \question{Soit $(x_n)$ une suite de Cauchy de $E$; montrer qu'on peut en extraire une sous-suite
$(x_{n_k})$ telle que la s\'erie de terme g\'en\'eral $u_k = x_{n_{k+1}} - x_{n_k}$ 
soit normalement convergente.
En d\'eduire que si toute s\'erie normalement convergente est convergente, alors $E$ est complet.}
\reponse{Soit $(x_n)$ une suite de Cauchy. 
Pour $\epsilon = 1$ il existe $n_0\in \Nn$ tel que
$$\forall q \ge n_0 \qquad \| x_{n_0}-x_q \| < 1.$$
Puis pour $\epsilon = \frac 12$ il existe $n_1 > n_0$ tel que
$$\forall q \ge n_1 \qquad \| x_{n_1}-x_q \| < \frac 12.$$
Puis par récurrence pour $\epsilon = \frac 1{2^k}$, on pose
$n_k > n_{k-1}$ tel que 
$$\forall q \ge n_k \qquad \| x_{n_k}-x_q \| < \frac 1{2^k}.$$

Donc en particulier à chaque étape on a
$$\| x_{n_{k+1}}-x_{n_k} \| < \frac 1 {2^k}.$$
Posons $u_k = x_{n_{k+1}} - x_{n_k}$ 
Alors $\|u_k\| \le \frac 1 {2^k}$ donc
$$\sum_{k\ge 0} \| u_k \| \le \sum_{k\ge 0} \frac 1 {2^k} = 2.$$
Donc la série $\sum_{k\ge 0} u_k$ est normalement convergente.
Si cette série converge notons $T=\sum_{k= 0}^{+\infty} u_k$ sa somme,
C'est-à-dire la limite de $T_N = \sum_{k= 0}^{N} u_k$. 
Mais alors $T_N = x_{n_{N+1}}-x_{n_0}$ converge vers $T$.
Donc la suite extraite $(x_{n_k})_k$ converge (vers $T+x_{n_0}$).
Conséquence : si toute s\'erie normalement convergente est convergente, alors on peut extraire de toute suite de Cauchy une sous-suite convergente donc $E$ est complet.}
    \item \question{Soit $\sum u_k$ une s\'erie normalement convergente. On note $S_n = \sum _{k=0}^nu_k$. 
Montrer que $S_n$ est une suite de Cauchy.
En d\'eduire que si $E$ est complet, alors toute série normalement convergente est convergente.}
\reponse{Soit $p \le q$.
$$\| S_q-S_p\| = \| \sum_{k=p+1}^q u_k \| \le  \sum_{k=p+1}^{q} \| u_k \|
\le  \sum_{k=p+1}^{+\infty} \| u_k \|$$
Or la série $\sum_{k\ge 0} u_k$ est normalement convergente donc le reste
$\sum_{k=p+1}^{+\infty} \| u_k \|$ tend vers $0$ quand $p\rightarrow +\infty$.
Fixons $\epsilon >0$, il existe donc $N \in \Nn$ tel que pour $p\ge N$ on a
$\sum_{k=p+1}^{+\infty} \| u_k \| \le \epsilon$, donc pour tout $p,q \ge N$ on a aussi $\| S_q-S_p\| \le \epsilon$. Donc la suite $(S_n)$ est de Cauchy.
Si $E$ est complet alors $(S_n)$ converge, notons $S$ sa limite. Donc $\| S_n -S\|$ tend vers $0$. Dons la série $\sum_{k\ge 0} u_k$ est convergente (de somme $S$).}
\indication{\begin{enumerate}
  \item \'Ecrire ce que donne la définition de ``$(x_n)$ est une suite de Cauchy''
pour $\epsilon=1$, puis $\epsilon = \frac 12$, ..., puis $\epsilon = \frac 1{2^k}$.
Faire la somme. Remarquer que si $T_N = \sum _{k=0}^N u_k$ alors $T_N =  x_{n_{N+1}}-x_{n_0}$.

  \item ...
\end{enumerate}}
\end{enumerate}
}
