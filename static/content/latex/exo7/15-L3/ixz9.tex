\uuid{ixz9}
\exo7id{2372}
\auteur{mayer}
\organisation{exo7}
\datecreate{2003-10-01}
\isIndication{true}
\isCorrection{true}
\chapitre{Compacité}
\sousChapitre{Compacité}

\contenu{
\texte{
Soient $K,F\subset \Rr^n$ des parties non vides, $K$ compact et $F$ ferm\'e.
Montrer qu'il existe $a\in K$ et $b \in F$ tel que $\|a-b\| = \mathrm{dist}(K,F)$.
}
\indication{Extraire des sous-suites...}
\reponse{
Notons $\ell = \mathrm{dist}(K,F)$. Alors il existe $(x_n)$ suite d'éléments de $K$ et $(y_n)$ suite d'éléments de $F$ telles que 
$\|x_n - y_n\| \rightarrow \ell$. Comme $K$ est compact alors on peut extraire de $(x_n)$ une sous-suite $(x_{\phi(n)})$ qui converge dans $K$. Notons $a\in K$ cette limite
Alors la suite extraite $(y_{\phi(n)})$ est bornée car
$$\| y_{\phi(n)} \| \le \| y_{\phi(n)} - x_{\phi(n)}\|  + \|x_{\phi(n)}\|.$$
La suite $(x_{\phi(n)})$ qui converge est donc bornée, et
la suite $(\| y_{\phi(n)} - x_{\phi(n)}\|)$ qui converge dans $\Rr$ (vers $\ell$) est bornée également. Donc la suite $(y_{\phi(n)})$ est bornée on peut donc en extraire une sous-suite convergente $(y_{\phi\circ \psi (n)})$.
De plus comme $F$ est fermé alors cette suite converge vers $b\in F$.
La suite $(x_{\phi\circ \psi (n)})$ extraite de $(x_{\phi(n)})$ converge vers $a\in K$. Et comme nous avons extrait deux suites $(x_n)$ et $(y_n)$ on a toujours $\|x_{\phi\circ \psi (n)} - y_{\phi\circ \psi (n)}\| \rightarrow \ell$.
A la limite nous obtenons $\| a-b\| = \ell$  avec $a\in K$ et $b\in F$.
Remarque : si $K$ était supposé fermé mais pas compact alors le résultat précédent pourrait être faux. Par exemple pour $K = \{ (x,y) \in \Rr^2 \mid xy \ge 1 \text { et } y \ge 0\}$ et $F = \{ (x,y) \in \Rr^2 \mid y \le0\}$ nous avons $d(K,F) = 0$ mais $K\cap F = \varnothing$.
}
}
