\uuid{a8JU}
\exo7id{2152}
\auteur{debes}
\organisation{exo7}
\datecreate{2008-02-12}
\isIndication{true}
\isCorrection{false}
\chapitre{Sous-groupe distingué}
\sousChapitre{Sous-groupe distingué}

\contenu{
\texte{
\label{ex:le17}
Etant donn\'es deux entiers $m, n>0$, d\' eterminer tous les morphismes de groupe de $\Zz/m\Zz$ dans $\Zz/n\Zz$, puis tous les automorphismes de $\Zz/n\Zz$.
}
\indication{L'ensemble $\textrm{Hom}(\Zz/m\Zz,\Zz/n\Zz)$ des morphismes de groupe de $\Zz/m\Zz$ dans $\Zz/n\Zz$ est un groupe ab\'elien pour l'addition naturelle des morphismes. On note $\delta$ le pgcd de $m$ et $n$ et $m^\prime$ et $n^\prime$ les entiers $m/\delta$ et $n/\delta$. Si $p:\Zz\rightarrow \Zz/m\Zz$ d\'esigne la surjection canonique, la correspondance associant \`a tout $f\in \textrm{Hom}(\Zz/m\Zz,\Zz/n\Zz)$ l'\'el\'ement $f\circ p(1)$ induit un isomorphisme de groupe entre $\textrm{Hom}(\Zz/m\Zz,\Zz/n\Zz)$ et le sous-groupe $n^\prime \Zz/n\Zz$ du groupe additif $ \Zz/n\Zz$, lequel est isomorphe \`a $\Zz/\delta \Zz$. 

L'ensemble $\textrm{Aut}(\Zz/n\Zz)$ des automorphismes de $\Zz/n\Zz$ est un groupe pour la composition. La correspondance pr\'ec\'edente induit un isomorphisme entre $\textrm{Aut}(\Zz/n\Zz)$ et le groupe $(\Zz/n\Zz)^\times$ des inversibles de $\Zz/n\Zz$.}
}
