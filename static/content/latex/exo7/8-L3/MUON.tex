\uuid{MUON}
\exo7id{2163}
\auteur{debes}
\organisation{exo7}
\datecreate{2008-02-12}
\isIndication{true}
\isCorrection{false}
\chapitre{Sous-groupe distingué}
\sousChapitre{Sous-groupe distingué}

\contenu{
\texte{
(a) Soit $G$ un groupe et $H$ un sous-groupe distingu\'e de $G$. On note $\varphi $ la
surjection canonique $\varphi : G \rightarrow G/H$. Montrer que l'ordre d'un \'el\'ement $x$ de $G$
est un multiple de l'ordre de $\varphi (x)$.

\smallskip
(b) Pour tout $x\in G$ on pose $\tau _x$ l'application de $G$ dans $G$ d\'efinie par $\tau
_x(y)=xyx^{-1}$. Montrer que $\tau _x$ est un automorphisme de $G$ et que l'application 
$$x\rightarrow \tau _x$$  est un morphisme de groupes de $G$ dans $\hbox{\rm Aut}(G)$. Quel
est le noyau de ce morphisme?

\smallskip

(c) On suppose que $G$ est fini et que $H$ est un sous-groupe distingu\'e dont l'ordre est le
plus petit nombre premier $p$ divisant l'ordre de $G$. Montrer que pour tout $x\in G$ l'ordre
de la restriction \`a $H$ de $\tau _x$ est un diviseur de $p-1$ et de l'ordre de $G$. En
d\'eduire que $\tau _x$ restreint \`a $H$ est l'identit\'e pour tout $x$ et donc que $H$ est
contenu dans le centre de $G$.
}
\indication{Les questions (a) et (b) ne pr\'esentent aucune difficult\'e.
\smallskip

Pour la question (c), noter que, pour tout $x\in G$, on a $(\tau_x)^{|G|}=1$, et que la
restriction de $\tau_x$ \`a $H$ appartient \`a $\hbox{\rm Aut}(H)\simeq \hbox{\rm
Aut}(\Z/p\Z)$ (et utiliser l'exercice \ref{ex:le25}).}
}
