\uuid{5BVJ}
\exo7id{2201}
\auteur{debes}
\organisation{exo7}
\datecreate{2008-02-12}
\isIndication{true}
\isCorrection{false}
\chapitre{Théorème de Sylow}
\sousChapitre{Théorème de Sylow}

\contenu{
\texte{
(a) Donner l'ensemble ${\cal D}$ des
ordres possibles des \'el\'ements du groupe altern\'e $A_5$ et pour
chaque $d\in {\cal D}$, indiquer le nombre d'\'el\'ements de $A_5$
d'ordre $d$.

(b) Montrer que, pour $d=2$ et $d=3$, les \'el\'ements
d'ordre $d$ sont conjugu\'es, et que les sous-groupes d'ordre $5$ sont
conjugu\'es.

(c) D\'eduire une preuve de la simplicit\'e de
$A_5$.
}
\indication{Pour le (c), pour $H\not=\{1\}$ sous-groupe
distingu\'e de $A_5$, raisonner sur les \'el\'ements d'ordre $2$,   
$3$ et $5$ contenus dans $H$.}
}
