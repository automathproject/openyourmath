\uuid{dT7x}
\exo7id{1434}
\auteur{ortiz}
\organisation{exo7}
\datecreate{1999-04-01}
\isIndication{false}
\isCorrection{false}
\chapitre{Groupe quotient, théorème de Lagrange}
\sousChapitre{Groupe quotient, théorème de Lagrange}

\contenu{
\texte{
Soit $G$ un groupe non r\'eduit \`a un
\'el\'ement. Un sous-groupe $M$ de $G$ est dit
\textit{maximal} si le seul sous-groupe de $G$,
distinct de $G$ et contenant $M,$ est $M$
lui-m\^eme. Les questions sont ind\'ependantes.
}
\begin{enumerate}
    \item \question{\begin{enumerate}}
    \item \question{Montrer que $6\Zz$ n'est pas un sous-groupe maximal de $\Zz$.}
    \item \question{Montrer que $5\Zz$ est un sous-groupe maximal de $\Zz$.}
\end{enumerate}
}
