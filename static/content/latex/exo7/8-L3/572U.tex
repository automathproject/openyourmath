\uuid{572U}
\exo7id{2169}
\auteur{debes}
\organisation{exo7}
\datecreate{2008-02-12}
\isIndication{false}
\isCorrection{true}
\chapitre{Action de groupe}
\sousChapitre{Action de groupe}

\contenu{
\texte{
\label{ex:deb69}
Montrer que  les permutations circulaires engendrent $S_n$ si
$n$ est pair, et $A_n$ si $n$ est impair.
}
\reponse{
Les cas $n=1$ et $n=2$ sont imm\'ediats. On peut supposer $n\geq 3$. On v\'erifie
ais\'ement la formule $(a_1 \hskip 2pt a_2\hskip 2pt\ldots\hskip 2pt a_{n-1}\hskip 2pt a_n)\ 
(a_{n-1} \hskip 2pt a_n\hskip 2pt a_{n-2} \ldots\hskip 2pt a_2\hskip 2pt a_1) = (a_1 \hskip
2pt a_n \hskip 2pt  a_{n-1})$ o\`u $a_1,\ldots,a_n$ sont les \'el\'ements d'un ensemble de
cardinal $n$. On en d\'eduit que le groupe $PC_n$ engendr\'e par les permutations circulaires
contient les $3$-cycles et donc le groupe altern\'e $A_n$ (voir exercice \ref{ex:deb67}). Les
permutations circulaires sont de signature $(-1)^{n-1}$. Si $n$ est impair, elles sont donc
paires d'o\`u $PC_n\subset A_n$ et donc finalement $PC_n = A_n$ dans ce cas. Si $n$ pair, les
permutations circulaires sont impaires, donc $PC_n\not=A_n$. L'indice de $PC_n$ dans $S_n$
devant diviser $2$ (puisque $PC_n\supset A_n$), il vaut $1$, c'est-\`a-dire $PC_n=S_n$.
}
}
