\uuid{89da}
\exo7id{2121}
\auteur{debes}
\organisation{exo7}
\datecreate{2008-02-12}
\isIndication{false}
\isCorrection{true}
\chapitre{Ordre d'un élément}
\sousChapitre{Ordre d'un élément}

\contenu{
\texte{
Montrer que dans un groupe $G$, toute partie non vide finie stable par la
loi de composition est un sous-groupe. Donner un contre-exemple \`a la propri\'et\'e
pr\'ec\'edente dans le cas d'une partie infinie.
}
\reponse{
Soit $H$ une partie finie non vide de $G$ stable par la loi de composition. Pour montrer que
$H$ est un sous-groupe, il reste \`a voir que pour tout $x\in H$, $x^{-1}\in H$. Les
puissances $x^k$ o\`u $k\in \N$ restant dans $H$, il existe $m,n\in \N$ tels que $m>n$ et
$x^m=x^n$. On a alors $x^{m-n-1} \cdot x=1$, soit $x^{-1}= x^{m-n-1}$, ce qui montre que
$x^{-1} \in H$.\smallskip

Si $H$ est infini, la propri\'et\'e pr\'ec\'edente n'est pas vraie en g\'en\'eral. Par
exemple $\N$ est une partie stable de $\Z$ pour l'addition mais n'en est pas un sous-groupe.
}
}
