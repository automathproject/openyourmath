\uuid{FFcp}
\exo7id{6541}
\auteur{drutu}
\organisation{exo7}
\datecreate{2011-10-16}
\isIndication{false}
\isCorrection{false}
\chapitre{Polynôme}
\sousChapitre{Polynôme}

\contenu{
\texte{
Soit $D$ un anneau intègre fini contenant $n$ éléments distincts $c_1,c_2,\dots ,c_n$. Soit le polyn\^ome $P_0:=\Pi_{i=1}^n(X-c_i)$.
}
\begin{enumerate}
    \item \question{Démontrer que deux polyn\^omes $Q, R$ dans $D[X]$ ont la même fonction polynomiale associée si et seulement si $P_0 | Q-R$.}
    \item \question{Calculer le polyn\^ome $P_0$ pour $D=\mathbb{F}_3,\; \mathbb{F}_5 $.}
\end{enumerate}
}
