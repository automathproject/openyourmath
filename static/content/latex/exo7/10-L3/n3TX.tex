\uuid{n3TX}
\exo7id{2284}
\auteur{barraud}
\organisation{exo7}
\datecreate{2008-04-24}
\isIndication{false}
\isCorrection{true}
\chapitre{Anneau, corps}
\sousChapitre{Anneau, corps}

\contenu{
\texte{
Montrer que un id\'eal propre  $I$ de l'anneau
$A$ est premier ssi quand le produit de deux id\'eaux est 
contenue dans $I$, alors l'un de deux est contenu dans $I$.
En d\'eduire que si $M$ est un id\'eal maximal de $A$, alors
le seul id\'eal premier de $A$ qui contient $M^n$ est $M$.
}
\reponse{
On rappelle que le produit de deux idéaux $I$ et $J$ est l'idéal
  engendré par les produits de la forme $ab$ avec $a\in I$, $b\in J$~:
  $$
   I\cdot J=\{\sum_{i=0}^{N}a_{i}b_{i}, N\in\Nn, a_{i}\in I, b_{i}\in J\}
  $$
  \begin{itemize}
  \item Si $I$ est un idéal premier~: Soient $J$ et  $K$ deux idéaux tels
    que $J\cdot K\subset I$. Alors si $J\not\subset I$, $\exists a\in x
    \setminus I$. Soit $y\in K$. On a $xy\in J\cdot K$ donc $xy\in I$.
    Comme $I$ est premier, $x\in I$ ou $y\in I$. Mais $x\notin I$ donc
    $y\in I$. Ainsi $\forall y\in K, y\in I$~: on a montré que~:
    $J\not\subset I\Rightarrow K\subset I$. On a donc bien $J\subset I$
    ou $K\subset I$.

  \item
    Si $\forall J,K \text{ idéaux},( J\cdot K\subset I\Rightarrow
    J\subset I \text{ ou }K\subset I)$~: Soit $a,b\in A$ avec $ab\in I$.
    Alors $(a)\cdot(b)=(ab)$ donc $(a)\subset I$ ou $(b)\subset I$ et
    donc  $a\in I$ ou $b\in I$. $I$ est donc premier.
  \end{itemize}

  On a $M^{n}=M\cdot M^{n-1}$. Donc si $I$ est premier et contient
  $M^{n}$ alors $I$ contient $M$ ou $M^{n-1}$, et par une récurrence
  finie, on obtient que $I$ contient $M$. Ainsi~: $M\subset I\subsetneq
  A$. Comme $M$ est maximal on en déduit que $M=I$.
}
}
