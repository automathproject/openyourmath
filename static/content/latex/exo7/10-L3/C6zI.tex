\uuid{C6zI}
\exo7id{2309}
\auteur{barraud}
\organisation{exo7}
\datecreate{2008-04-24}
\isIndication{false}
\isCorrection{true}
\chapitre{Anneau, corps}
\sousChapitre{Anneau, corps}

\contenu{
\texte{
Soit $\sqrt{d}$ non rationel. Dans l'anneau 
$$
{\mathbb{Z}}[\sqrt{d}]=\{n+m\sqrt{d}\,|\,n,m\in \mathbb{Z}\}
$$
on definit la ``conjugaison" $\bar{z}$ :
\medskip

\centerline{si $z=n+m\sqrt{d}$, alors $\bar{z}=n-m\sqrt{d}$.}
\medskip

On peut aussi d\'efinir la norme  
$N_d:{\mathbb Z}[\sqrt{d}] \rightarrow{\mathbb Z}$ 
par $N_d(z) = z\bar{z}=(n+m\sqrt{d})(n-m\sqrt{d})$.
\medskip

0.  Montrer que les aplications $\bar{z}$ et $N(z)$ sont multiplicatives :
$$
\overline{z_1\cdot z_2}=\bar{z_1}\cdot \bar{z_2},
\qquad N_d(z_1\cdot z_2)=N_d(z_1)\cdot N_d(z_2).
$$
}
\reponse{
Soit $z=n+m\sqrt{d},z'=n'+m'\sqrt{d}\in \Z[\sqrt{d}]$. Alors
\begin{align*}
\overline{zz'}
&=\overline{(n+m\sqrt{d})(n'+m'\sqrt{d})} \\
&=\overline{(nn'+mm'd)+(nm'+n'm)\sqrt{d}} \\
&=          (nn'+mm'd)-(nm'+n'm)\sqrt{d}  \\
&=(n-m\sqrt{d})(n'-m'\sqrt{d}) \\
&=\bar{z}\, \bar{z}'
\end{align*}

Donc $\forall z,z'\in\Z[\sqrt{d}], \overline{zz'}=\bar{z}\,\bar{z}'$.

On a alors $\forall z,z'\in\Z[\sqrt{d}],\
 N(zz')=zz'\,\overline{zz'}=z\bar{z}\,z'\bar{z}'=N(z)\,N(z')$.
}
}
