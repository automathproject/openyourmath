\uuid{kdoJ}
\exo7id{1710}
\auteur{barraud}
\organisation{exo7}
\datecreate{2003-09-01}
\isIndication{false}
\isCorrection{true}
\chapitre{Réduction d'endomorphisme, polynôme annulateur}
\sousChapitre{Applications}

\contenu{
\texte{
Donner toutes les suites $(x_{n})$, $(y_{n})$ et $(z_{n})$ telles que~:
{\it(on notera $\omega=e^{\frac{i\pi}{3}}$)}
$$
\forall n\in\N,\ 
\left\{
\begin{array}{r}
  x_{n+1}= x_{n}+y_{n}\\
  y_{n+1}= y_{n}+z_{n}\\
  z_{n+1}= z_{n}+x_{n}
\end{array}
\right.
$$

Parmi les solutions de ce système, donner celle qui satisfait $x_{0}=2$
et $y_{0}=z_{0}=1$.
}
\reponse{
Soit $A=\Big(\begin{smallmatrix}%
  1&1&0\\0&1&1\\1&0&1%
\end{smallmatrix}\Big)$.
$\chi_{A}=(2-X)(\omega-X)(\bar\omega-X)$ donc $A$ est diagonalisable sur
$\C$.

$\ker(A-2I)=\C\Big(\begin{smallmatrix}%
  1\\1\\1%
\end{smallmatrix}\Big)$  

$\Big(\begin{smallmatrix}x\\y\\z\end{smallmatrix}\Big)\in\ker(A-\omega I)%
\Leftrightarrow\Big\{\begin{smallmatrix}%
  (1-\omega)x+y=0\\(1-\omega)y+z=0\\(1-\omega)z+x=0%
\end{smallmatrix}
\Leftrightarrow\big\{\begin{smallmatrix}%
  y=(\omega-1)x\\z=(\omega-1)^{2}x%
\end{smallmatrix} 
\Leftrightarrow\big\{\begin{smallmatrix}%
  y=\omega^{2}x\\z=\omega^{4}x%
\end{smallmatrix}$ 
donc $\ker(A-\omega I)=\C\Big(\begin{smallmatrix}1\\\omega^{2}\\\omega^{4}\end{smallmatrix}\Big)$

On en déduit que $\ker(A-\bar\omega I)=\C\Big(\begin{smallmatrix}1\\\bar\omega^{2}\\\bar\omega^{4}\end{smallmatrix}\Big)$.

Ainsi en posant $P=\Big(\begin{smallmatrix}
  1&1&1\\1&\omega^{2}&\bar\omega^{2}\\1&\omega^{4}&\bar\omega^{4}\end{smallmatrix}\Big)$ on obtient
$P^{-1}AP=\Big(\begin{smallmatrix}
  2&0&0\\
  0&\omega&0\\
  0&0&\bar\omega\end{smallmatrix}\Big)$

On en déduit que les solutions sont les suites de la forme%
$\Big(\begin{smallmatrix}
  x_{n}\\
  y_{n}\\
  z_{n}  
\end{smallmatrix}\Big)=
P\Big(\begin{smallmatrix}
  2^n&0&0\\
  0&\omega^{n}&0\\
  0&0&\bar\omega^{n}\end{smallmatrix}\Big)
\Big(\begin{smallmatrix}
  a\\b\\c\end{smallmatrix}\Big)
=
P\Big(\begin{smallmatrix}2^na\\\omega^n b\\\bar\omega^nc\end{smallmatrix}\Big)
$ soit~:
$\Big\{\begin{smallmatrix}
  x_{n}=&2^na&+\omega^{n}b  &+\bar\omega^{n}c\\
  y_{n}=&2^na&+\omega^{n+2}b&+\bar\omega^{n+2}c\\
  z_{n}=&2^na&+\omega^{n+4}b&+\bar\omega^{n+4}c\\
\end{smallmatrix}$ où $a,b,c$ sont trois complexes.

En résolvant le système%
$\Big\{\begin{smallmatrix}
  a+b+c&=2\\
  a+b\omega^{2}+c\bar\omega^{2}&=1\\
  a+b\omega^{4}+c\bar\omega^{4}&=1\\
\end{smallmatrix}$
on obtient la solution particulière cherchée,  c'est la solution
associée à $a=4/3, b=c=1/3$ :
$$\left\{\begin{array}{llll}
  x_{n}=&2^{n+2}/3 &+&2/3 \cos(n\pi/3)\\
  y_{n}=&2^{n+2}/3 &+&2/3 \cos((n+2)\pi/3)\\
  z_{n}=&2^{n+2}/3 &+&2/3 \cos((n+4)\pi/3)\\
\end{array}\right..$$
}
}
