\uuid{M5ey}
\exo7id{2999}
\auteur{quercia}
\organisation{exo7}
\datecreate{2010-03-08}
\isIndication{false}
\isCorrection{false}
\chapitre{Groupe, anneau, corps}
\sousChapitre{Autre}

\contenu{
\texte{
Soit $p$ un entier naturel premier. On appelle $G$ l'ensemble des $z\in \C$
pour lesquels existe $n\in\N$ tel que $z^{p^n}=1$.
}
\begin{enumerate}
    \item \question{Montrer que $G$ est un groupe multiplicatif infini o{\`u} tout {\'e}l{\'e}ment est d'ordre fini.}
    \item \question{Montrer que tout sous-groupe $H$ de $G$, distinct de $G$, est cyclique
    (on pourra consid{\'e}rer un {\'e}l{\'e}ment $z_0$ de $G\setminus H$ et montrer que l'ordre des
    {\'e}l{\'e}ments de $H$ n'exc{\`e}de pas celui de $z_0$).}
\end{enumerate}
}
