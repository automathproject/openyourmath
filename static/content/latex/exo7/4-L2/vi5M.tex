\uuid{vi5M}
\exo7id{5653}
\auteur{rouget}
\organisation{exo7}
\datecreate{2010-10-16}
\isIndication{false}
\isCorrection{true}
\chapitre{Réduction d'endomorphisme, polynôme annulateur}
\sousChapitre{Applications}

\contenu{
\texte{
Soit $A = \left(
\begin{array}{ccc}
3&1&0\\
-4&-1&0\\
4&8&-2
\end{array}
\right)$.
}
\begin{enumerate}
    \item \question{Vérifier que $A$ n'est pas diagonalisable.}
\reponse{$\chi_A=-(2+X)((3-X)(-1-X)+4)=-(X+2)(X^2-2X+1)=-(X+2)(X-1)^2$. 

$A$ diagonalisable $\Rightarrow\text{dim}(\text{Ker}(A-I))=2\Rightarrow\text{rg}(A-I) = 1$ ce qui n'est pas . Donc $A$ n'est pas diagonalisable. De plus, $E_{-2}=\text{Vect}(e_1)$ où $e_1=\left(
\begin{array}{c}
0\\
0\\
1
\end{array}
\right)$ et $E_1=\text{Vect}(e_2)$ où $e_2=\left(
\begin{array}{c}
1\\
-2\\
-4
\end{array}
\right)$.}
    \item \question{Déterminer $\text{Ker}(A-I)^2$.}
\reponse{$(A-I)^2=\left(
\begin{array}{ccc}
2&1&0\\
-4&-2&0\\
4&8&-3
\end{array}
\right)\left(
\begin{array}{ccc}
2&1&0\\
-4&-2&0\\
4&8&-3
\end{array}
\right)=\left(
\begin{array}{ccc}
0&0&0\\
0&0&0\\
-36&-36&9
\end{array}
\right)$ et donc $\text{Ker}(A-I)^2$ est le plan d'équation $4x+4y-z=0$.}
    \item \question{Montrer que $A$ est semblable à une matrice de la forme $\left(
\begin{array}{ccc}
a&0&0\\
0&b&c\\
0&0&b
\end{array}
\right)$}
\reponse{On note $f$ l'endomorphisme de $\Rr^3$ dont la matrice dans la base canonique de $\Rr^3$ est $A$. Le théorème de \textsc{Cayley}-\textsc{Hamilton} et le théorème de décomposition des noyaux permettent d'affirmer

\begin{center}
$\mathcal{M}_3(\Rr)=\text{Ker}(A+2I)\oplus\text{Ker}(A-I)^2$.
\end{center}

De plus,  chacun des sous-espaces $\text{Ker}(A+2I)$ et $\text{Ker}(A-I)^2$ étant stables par $f$, la matrice de $f$ dans toute base adaptée à cette décomposition est diagonale par blocs. Enfin, $\text{Ker}(A-I)$ est une droite vectorielle contenue dans le plan $\text{Ker}(A-I)^2$ et en choisissant une base de $\text{Ker}(A-I)^2$ dont l'un des deux vecteurs est dans $\text{Ker}(A-I)$, la matrice de $f$ aura la forme voulue.

On a déjà choisi $e_1=\left(
\begin{array}{c}
0\\
0\\
1
\end{array}
\right)$ et $e_2=\left(
\begin{array}{c}
1\\
-2\\
-4
\end{array}
\right)$ puis on prend $e_3=\left(
\begin{array}{c}
1\\
-1\\
0
\end{array}
\right)$. On note $P=\left(
\begin{array}{ccc}
0&1&1\\
0&-2&-1\\
1&-4&0
\end{array}
\right)$. $P$ est inversible d'inverse $P^{-1}=\left(
\begin{array}{ccc}
-4&-4&1\\
-1&-1&0\\
2&1&0
\end{array}
\right)$. On peut déjà affirmer que $P^{-1}AP$ est de la forme $\left(
\begin{array}{ccc}
-2&0&0\\
0&1&\times\\
0&0&1
\end{array}
\right)$.
Plus précisément 

\begin{center}
$Ae_3-e_3=\left(
\begin{array}{ccc}
3&1&0\\
-4&-1&0\\
4&8&-2
\end{array}
\right)\left(
\begin{array}{c}
1\\
-1\\
0
\end{array}
\right)-\left(
\begin{array}{c}
1\\
-1\\
0
\end{array}
\right)=\left(
\begin{array}{c}
1\\
-2\\
-4
\end{array}
\right)=e_2$
\end{center}

 et donc $Ae_3=e_2+e_3$ puis
 
 \begin{center}
 \shadowbox{
 $A=PTP^{-1}$ où $P=\left(
\begin{array}{ccc}
0&1&1\\
0&-2&-1\\
1&-4&0
\end{array}
\right)$, $P^{-1}=\left(
\begin{array}{ccc}
-4&-4&1\\
-1&-1&0\\
2&1&0
\end{array}
\right)$ et $T=\left(
\begin{array}{ccc}
-2&0&0\\
0&1&1\\
0&0&1
\end{array}
\right)$.
}
\end{center}}
    \item \question{Calculer $A^n$ pour $n$ entier naturel donné.}
\reponse{Soit $n\in\Nn$. Posons $T = D+N$ où $D=\text{diag}(-2,1,1)$ et $N =E_{2,3}$. On a $ND = DN$ et $N^2 = 0$. Puisque les matrices $D$ et $N$ commutent, la formule du binôme de \textsc{Newton} permet d'écrire

\begin{align*}\ensuremath 
T^n&= D^n+nD^{n-1}N=\text{diag}((-2)^n,1,1) + n\text{diag}((-2)^{n-1},1,1)E_{2,3}=\text{diag}((-2)^n,1,1) +nE_{2,3}\\
 &=\left(
 \begin{array}{ccc}
 (-2)^n&0&0\\
 0&1&n\\
 0&0&1
 \end{array}
 \right).
\end{align*}

Puis

\begin{align*}\ensuremath
A^n&=PT^nP^{-1}=\left(
\begin{array}{ccc}
0&1&1\\
0&-2&-1\\
1&-4&0
\end{array}
\right)\left(
 \begin{array}{ccc}
 (-2)^n&0&0\\
 0&1&n\\
 0&0&1
 \end{array}
 \right)\left(
\begin{array}{ccc}
-4&-4&1\\
-1&-1&0\\
2&1&0
\end{array}
\right)\\
 &= \left(
\begin{array}{ccc}
0&1&n+1\\
0&-2&-2n-1\\
(-2)^n&-4&-4n
\end{array}
\right)\left(
\begin{array}{ccc}
-4&-4&1\\
-1&-1&0\\
2&1&0
\end{array}
\right)=\left(
\begin{array}{ccc}
2n+1&n&0\\
-4n&-2n+1&0\\
-4(-2)^n-8n+4&-4(-2)^n-4n+4&(-2)^n
\end{array}
\right).
\end{align*}

\begin{center}
\shadowbox{
$\forall n\in\Nn$, $A^n=\left(
\begin{array}{ccc}
2n+1&n&0\\
-4n&-2n+1&0\\
-4(-2)^n-8n+4&-4(-2)^n-4n+4&(-2)^n
\end{array}
\right)$.
}
\end{center}}
\end{enumerate}
}
