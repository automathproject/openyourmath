\uuid{zIAt}
\exo7id{1342}
\auteur{legall}
\organisation{exo7}
\datecreate{1998-09-01}
\isIndication{false}
\isCorrection{true}
\chapitre{Groupe, anneau, corps}
\sousChapitre{Ordre d'un élément}

\contenu{
\texte{
Le groupe $  (\Q ,+)  $ est-il monog\`ene ?
}
\reponse{
Par l'absurde supposons que $(\Qq,+)$ est engendré par un seul
élément $\frac pq$ ($p$ et $q$ premiers entre eux) alors tout
élément de $\Qq$ s'écrit $n\frac pq$ avec $n\in \Zz$. Il s'ensuit
que $\frac p{2q}$ (qui appartient à $\Qq$) doit s'écrire $n\frac
pq$, mais alors $2n=1$ avec $n \in \Zz$ ce qui est impossible.
Conclusion $(\Qq,+)$ n'est pas monogène.
}
}
