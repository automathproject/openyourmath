\uuid{f0YJ}
\exo7id{1673}
\auteur{gineste}
\organisation{exo7}
\datecreate{2001-11-01}
\isIndication{false}
\isCorrection{true}
\chapitre{Réduction d'endomorphisme, polynôme annulateur}
\sousChapitre{Diagonalisation}

\contenu{
\texte{
Soit
$K$ un corps commutatif quelconque, et soit $F=\mathcal{M}_n(K)$
l'espace vectoriel sur $K$ des matrices carrées d'ordre $n$ à
coefficients dans $K.$ Si $i$ et $j$ sont des entiers compris
entre 1 et $n$, on note par $F_{ij}$ l'élément de $F$ dont le
coefficient $(i,j)$ est 1 et dont les autres coefficients sont
nuls. Montrer que les $F_{ij}$ forment une base de $F$. Dimension
de $F$? Soit $D$ dans $F$ et diagonale. Soient $\alpha$ et $\beta$
dans $K$ et soit l'endomorphisme $\Phi$ de $F$ qui à la matrice
$X$ fait correspondre la matrice $\Phi(X)=\alpha XD+ \beta DX.$
Calculer $\Phi(F_{ij}).$ $\Phi$ est il un endomorphisme
diagonalisable? Donner son polynôme caractéristique en fonction
des coefficients de $D$ et de $\alpha$ et $\beta$.
}
\reponse{
Si
$X=(x_{ij})_{1\leq i,j\leq n}\in F,$ il est clair que
$X=\sum_{1\leq i,j\leq n}x_{ij}F_{ij}.$ C'est donc une famille
g\'en\'eratrice. Elle est ind\'ependante, car si $\sum_{1\leq
i,j\leq n}x_{ij}F_{ij}$ est la matrice nulle, cela implique que
$x_{ij} =0$ pour tous $i$ et $j.$ C'est donc une base de $F$. Elle
est de taille $n^2$, donc $F$ est de dimension $n^2.$ Ensuite, si
$D=\mathrm{diag}(d_1,\ldots,d_n)$ et si $X=(x_{ij})_{1\leq i,j\leq
n}$ alors le coefficient $(i,j)$ de la matrice $\Phi(X)=\alpha
XD+\beta DX$ est $(\alpha d_j+\beta d_i)x_{ij}.$ Donc $\Phi
(F_{ij})=(\alpha d_j+\beta d_i)F_{ij},$ ce qui est dire que
$F_{ij}$ est un vecteur propre de $\Phi$ pour la valeur propre
$\alpha d_j+\beta d_i.$ L'espace $F$ admet donc une base de
vecteurs propres de $\Phi.$ D'apr\`es le cours, cela entra\^ine
que $\Phi$ est diagonalisable. Si on le repr\'esente dans la base
de vecteurs propres, le d\'eterminant de $\Phi$ est donc le
produit des \'el\'ements diagonaux, c'est \`a dire $\det
\Phi=\prod_{i=1}^n\prod_{j=1}^n(\alpha d_j+\beta d_i).$ Plus
g\'en\'eralement
 $\det (\Phi-\lambda \mathrm{id}_F)=
\prod_{i=1}^n\prod_{j=1}^n(\alpha d_j+\beta d_i-\lambda).$
}
}
