\uuid{dfpT}
\exo7id{1728}
\auteur{legall}
\organisation{exo7}
\datecreate{1998-09-01}
\isIndication{false}
\isCorrection{false}
\chapitre{Réduction d'endomorphisme, polynôme annulateur}
\sousChapitre{Applications}

\contenu{
\texte{
{{\em Les parties I, II, III et IV peuvent \^etre trait\'ees
ind\'ependamment les unes des autres.}}


 Soient
$  M_a=\begin{pmatrix} a+1 & 1-a & a-1    \cr
                                   -1 & 3 & 2a-3   \cr
                                    a-2 & 2-a & 3a-2  \cr  \end{pmatrix}\in M_3(\R )  $ une matrice d\'ependant
d'un param\`etre r\'eel $  a  $ et  $  f_a   $ l'endomorphisme lin\'eaire de $  \R
^3  $ ayant pour matrice $  M_a   $ dans la base canonique de $  \R ^3  .$
\vskip1mm
On nomme {\em racine carr\'ee} d'une matrice $  M\in M_n(\R )  $ toute matrice
$  N \in M_n(\R )  $ telle que $  N^2=M  .$
\vskip1mm On d\'esigne par $  I  $ la matrice identit\'e et, pour toute base $  \epsilon   $ de
$  \R ^3  ,$ on
note $  \hbox{Mat}(f_a,\epsilon)  $ la matrice repr\'esentant
l'endomorphisme $  f_a  $ dans la base $  \epsilon  .$
 \vskip2mm
 \centerline{{\bf I}}
}
\begin{enumerate}
    \item \question{Calculer les valeurs propres de $  M_a  $ en fonction de $  a  .$ Pour quelle raison la matrice $  M_a  $ est-elle triangularisable~?}
    \item \question{Pour quelles valeurs du
param\`etre $  a  $ la matrice $  M_a  $ est-elle diagonalisable~?

\vskip2mm
 \centerline{{\bf II}}
\centerline{{\em On pose maintenant (questions 3 et 4) $  a=2  .$}}}
    \item \question{Diagonaliser $  M_2  .$ D\'eterminer une racine carr\'ee $  A $ de
$  M_2  .$}
    \item \question{\begin{enumerate}}
    \item \question{Soit $  g \in \mathcal{L} (\R ^3)  $ telle que $  g^2=f_2  .$ Montrer que
$  g   $ est diagonalisable (on pourra d\'eterminer le polyn\^ome minimal de $  f_2  $).
 Montrer que les sous-espaces propres de $  f_2  $ sont laiss\'es stables par
$  g  .$}
    \item \question{D\'emontrer que la matrice $ \begin{pmatrix} 4 & 0     \cr
                                   0 & 4    \cr \end{pmatrix}  $ a une infinit\'e de racines carr\'ees.
En d\'eduire l'existence d'une infinit\'e de racines carr\'ees de $  M_2  .$}
\end{enumerate}
}
