\uuid{Hpdo}
\exo7id{1609}
\auteur{barraud}
\organisation{exo7}
\datecreate{2003-09-01}
\isIndication{false}
\isCorrection{false}
\chapitre{Réduction d'endomorphisme, polynôme annulateur}
\sousChapitre{Valeur propre, vecteur propre}

\contenu{
\texte{
Soit $f$ un endomorphisme de $E=\C^{n}$. Soit $\pi_{1},...,\pi_{N}$ des
  endomorphismes tous non nuls de $E$ et $\lambda_{1},...,\lambda_{N}$
  $N$ nombres complexes distincts. On suppose que~:
  $$
  \forall m\in\N \quad f^{m}=\sum_{k=1}^{N}\lambda_{k}^{m}\pi_{k}.
  $$
}
\begin{enumerate}
    \item \question{Montrer que $\forall P\in\C[X],\quad
    P(f)=\sum_{k=1}^{N}P(\lambda_{k})\pi_{k}$

\medskip

  On considère le polynôme $Q=\prod_{1\leq k\leq N}(X-\lambda_{k})$ et
  pour chaque $p\in\{1,...,N\}$ les polynômes suivants~:  
  $$
  Q_{p}=\prod_{\substack{ 1\leq k\leq N \\k\neq p}}
             (X-\lambda_{k})\qquad\text{et}\qquad
  \tilde Q_{p}=\frac{1}{Q_{p}(\lambda_{p})}\ Q_{p}
  $$}
    \item \question{Calculer $Q(f)$. Qu'en déduit-on pour $f$~?}
    \item \question{Montrer que $Sp(f)\subset\{\lambda_{1},...,\lambda_{N}\}$}
    \item \question{Montrer que $\tilde Q_{p}(f)=\pi_{p}$. Vérifier alors que
    $\pi_{p}\circ\pi_{q}= \left\{
    \begin{array}{l}
      0       \text{ si } p\neq q\\
      \pi_{p} \text{ si } p=q
    \end{array}
    \right. $}
    \item \question{Calculer $f\circ\pi_{p}$. En déduire que
    $Sp(f)=\{\lambda_{1},...,\lambda_{N}\}$.


\medskip
  On note $E_{p}$ l'espace propre associé à la valeur propre
  $\lambda_{p}$.}
    \item \question{Montrer que $\mathrm{Im}\pi_{p}\subset E_{p}$. Réciproquement, pour $x\in
    E_{p}$, montrer que $x\in\mathrm{Ker}\pi_{q}$ pour $q\neq p$ (on calculera
    par exemple $\pi_{q}\circ f(x)$ de deux façons différentes) puis que
    $x=\pi_{p}(x)$. En déduire que $E_{p}\subset\mathrm{Im}\pi_{p}$.}
    \item \question{En déduire que $\mathrm{Im}\pi_{p}=E_{p}$ et que
    $\mathrm{Ker}\pi_{p}=\bigoplus_{q\neq p}E_{q}$. Décrire géométriquement
    $\pi_{p}$.}
\end{enumerate}
}
