\uuid{Ztg8}
\exo7id{3483}
\auteur{quercia}
\organisation{exo7}
\datecreate{2010-03-10}
\isIndication{false}
\isCorrection{true}
\chapitre{Déterminant, système linéaire}
\sousChapitre{Système linéaire, rang}

\contenu{
\texte{

}
\begin{enumerate}
    \item \question{Soient deux matrices $B \in \mathcal{M}_{n,r}(\R)$ et $C \in \mathcal{M}_{r,n}(\R)$ de même rang $r$.
    Montrer que $A = BC$ est de rang $r$.}
    \item \question{Réciproquement, soit $A \in \mathcal{M}_n(\R)$ de rang $r \ge 1$.
    Montrer qu'il existe des matrices $B$ et $C$ comme
    précédemment telles que $A = BC$.}
    \item \question{Déterminer explicitement une telle décomposition pour
    $A = \begin{pmatrix}1 & 2 & 3 & 4 \cr 2 & 3 & 4 & 5 \cr
    3 & 4 & 5 & 6 \cr 4 & 5 & 6 & a\cr\end{pmatrix}$, $a \in \R$.}
    \item \question{Supposons de plus $A$ symétrique. Montrer que $CB$ est aussi de rang $r$.}
\reponse{
$B$ admet $r$ lignes indépendantes d'indices $i_1,\dots,i_r$
    et $C$ admet $r$ colonnes indépendantes d'indices $j_1,\dots,j_r$.
    Soient $B'$ et $C'$ les sous matrices carrées associées dans $B$ et $C$.
    Alors la sous-matrice de~$A$ d'indices $i_1,\dots,i_r$ pour les lignes
    et $j_1,\dots,j_r$ pour les colonnes est $B'C'$, de rang~$r$.
    Donc $\mathrm{rg}(A)\ge r$ et l'inégalité inverse est bien connue.
Soient $i_1,\dots,i_r$ $r$ indices tels que les
    lignes associées dans~$A$ sont linéairement indépendantes,
    et $B\in\mathcal{M}_{r,n}(\R)$ la sous-matrice correpondante. Par construction, $\mathrm{rg}(B)=r$.
    Chaque ligne de~$A$
    étant combinaison linéaire des lignes de~$B$, il existe $C\in\mathcal{M}_{r,n}(\R)$
    telle que $A=BC$. Et on a $r=\mathrm{nb.lignes}(C)\ge\mathrm{rg}(C)\ge \mathrm{rg}(A)=r$.
Comprendre dans cette question que $B,C$ ne sont pas
    forcément les matrices construites en {\bf 2}.
    Notons $\mathrm{vect}(X)$ l'espace vectoriel engendré par les colonnes
    d'une matrice~$X$. De $A=BC={}^tC{}^tB$ on tire
    $\mathrm{vect}(A)\subset\mathrm{vect}(B)$ et $\mathrm{vect}(A)\subset\mathrm{vect}({}^tC)$,
    et tous ces espaces sont de dimension~$r$, donc ils sont égaux.
    On en déduit qu'il existe une matrice~$P\in GL_r(\R)$ telle
    que $B={}^tCP$ d'où $CB=C{}^tCP$. $\mathrm{rg}(C{}^tC)=\mathrm{rg}(C)=r$ et $P$
    est inversible donc $\mathrm{rg}(CB)=r$.
}
\end{enumerate}
}
