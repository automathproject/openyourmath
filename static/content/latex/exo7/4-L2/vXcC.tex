\uuid{vXcC}
\exo7id{3588}
\auteur{quercia}
\organisation{exo7}
\datecreate{2010-03-10}
\isIndication{false}
\isCorrection{true}
\chapitre{Réduction d'endomorphisme, polynôme annulateur}
\sousChapitre{Applications}

\contenu{
\texte{
Soit $f$, endomorphisme d'un espace vectoriel $E$ de dimension~$n$.
}
\begin{enumerate}
    \item \question{On suppose que pour tout sous-ev $D$ de dimension~1 il existe $x\in D$
    tel que $E = \text{vect}(x,f(x),f^2(x),\dots)$. Que dire de $E$ et $f$~?}
    \item \question{On suppose qu'il existe  $x\in E$
    tel que $E = \text{vect}(x,f(x),f^2(x),\dots)$.
    Montrer que si $f$ est diagonalisable alors ses valeurs propres sont
    toutes distinctes. Montrer que si $f$ est nilpotente alors $f^{n-1}\ne 0$.}
\reponse{
Le polynôme minimal de $f$ est de degré supérieur ou égal à~$n$ et n'a
pas de diviseurs non triviaux. Donc $\dim E = 1$ et $f$ est une homothétie
si $ K=\C$. Si $ K=\R$ on peut aussi avoir $\dim E = 2$ et $f$ n'a pas
de valeurs propres réelles.
}
\end{enumerate}
}
