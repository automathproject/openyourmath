\uuid{pto0}
\exo7id{3424}
\auteur{quercia}
\organisation{exo7}
\datecreate{2010-03-10}
\video{wKgDXlqxneE}
\isIndication{true}
\isCorrection{true}
\chapitre{Déterminant, système linéaire}
\sousChapitre{Système linéaire, rang}

\contenu{
\texte{
Trouver trois réels $\alpha, \beta, \gamma$ tels que pour tout polynôme
de degré $\le 3$ on ait :
$$  \int_{2}^4 P(x)\; dx = \alpha P(2) + \beta P(3) + \gamma P(4).$$
}
\indication{\'Ecrire les polynômes sous la forme 
$P(x)=ax^3+bx^2+cx+d$.
Calculer $  \int_{2}^4 P(x)\; dx$ d'une part 
et $\alpha P(2) + \beta P(3) + \gamma P(4)$ d'autre part.
L'identification conduit à un système linéaire à quatre équations,
 d'inconnues $\alpha,\beta,\gamma$.}
\reponse{
Tout d'abord calculons l'intégrale :
$$\int_{2}^4 P(x)\; dx = \left[a\frac{x^4}{4} + b\frac{x^3}{3}+c\frac{x^2}{2} +dx \right]_2^4
= 60a+\frac{56}{3}b+6c+2d.$$
D'autre part
$$\alpha P(2) + \beta P(3) + \gamma P(4)
=  \alpha\big(8a+4b+2c+d\big)+\beta\big(27a+9b+3c+d\big)+\gamma\big(64a+16b+4c+d\big).$$
Donc 
$$\alpha P(2) + \beta P(3) + \gamma P(4)
= (8\alpha+27\beta+64\gamma)a + (4\alpha+9\beta+16\gamma)b+(2\alpha+3\beta+4\gamma)c+(\alpha+\beta+\gamma)d.$$
Pour avoir l'égalité $\int_{2}^4 P(x)\; dx =\alpha P(2) + \beta P(3) + \gamma P(4)$
quelque soit les coefficients $a,b,c,d$ il faut et il suffit que 
$$(8\alpha+27\beta+64\gamma)a + (4\alpha+9\beta+16\gamma)b+(2\alpha+3\beta+4\gamma)c+(\alpha+\beta+\gamma)d =
60a+\frac{56}{3}b+6c+2d$$ 
ce qui équivaut à 
$$\left\{\begin{array}{rcrcrcl}
 \alpha & + &  \beta & + &  \gamma & = & 2 \\
 2\alpha & + & 3\beta & + & 4\gamma & = & 6 \\   
 4\alpha & + & 9\beta & + &  16\gamma & = & \frac{56}{3} \\ 
 8\alpha & + &  27\beta & + &  64\gamma & = & 60 \\     
         \end{array}
 \right.$$

De façon surprenante ce système à $3$ inconnues et $4$ équations a une solution unique :
$$\alpha = \frac 13,\quad \beta = \frac 43,\quad \gamma = \frac 13.$$
}
}
