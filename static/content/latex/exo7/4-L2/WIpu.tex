\uuid{WIpu}
\exo7id{2603}
\auteur{delaunay}
\organisation{exo7}
\datecreate{2009-05-19}
\isIndication{false}
\isCorrection{true}
\chapitre{Réduction d'endomorphisme, polynôme annulateur}
\sousChapitre{Diagonalisation}

\contenu{
\texte{
Soit $a\in\R$ et $A$ la matrice suivante
$$A=\begin{pmatrix}1&a&0 \\  a&0&1 \\  0&1&a\end{pmatrix}.$$
}
\begin{enumerate}
    \item \question{Calculer le d\'eterminant de $A$ et d\'eterminer 
pour quelles valeurs de $a$ la matrice est inversible.}
\reponse{{\it Calculons le d\'eterminant de $A$ et d\'eterminons 
pour quelles valeurs de $a$ la matrice est inversible.}

On développe le déterminant par rapport à la première colonne, on obtient
$$\det A=\begin{vmatrix}1&a&0 \\  a&0&1 \\  0&1&a\end{vmatrix}=\begin{vmatrix}0&1 \\  1&a\end{vmatrix}-a\begin{vmatrix}a&0 \\  1&a\end{vmatrix}=-1-a^3.$$
La matrice $A$ est inversible si et seulement si son déterminant est non nul.
$$\det A\neq 0\iff 1+a^3\neq 0\iff a\neq -1.$$}
    \item \question{Calculer $A^{-1}$ lorsque $A$ est inversible.}
\reponse{{\it Calculons $A^{-1}$ lorsque $A$ est inversible.}

On suppose $a\neq-1$, on a $\displaystyle A^{-1}={\frac{1}{\det A}}\,^t\!\!\tilde A$, où $\tilde A$ est la comatrice de $A$ et $^t\!\!\tilde A$ la transposée de $\tilde A$. On a 
$$\tilde A=\begin{pmatrix}-1&-a^2&a \\  -a^2&a&-1 \\  a&-1&-a^2\end{pmatrix}=^t\!\!\tilde A.$$
D'où $\displaystyle A^{-1}={\frac{1}{1+a^3}}\begin{pmatrix}1&a^2&-a \\  a^2&-a&1 \\  -a&1&a^2\end{pmatrix}.$}
\end{enumerate}
}
