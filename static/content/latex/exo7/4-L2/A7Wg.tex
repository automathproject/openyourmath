\uuid{A7Wg}
\exo7id{5629}
\auteur{rouget}
\organisation{exo7}
\datecreate{2010-10-16}
\isIndication{false}
\isCorrection{true}
\chapitre{Espace euclidien, espace normé}
\sousChapitre{Produit scalaire, norme}

\contenu{
\texte{

}
\begin{enumerate}
    \item \question{Soient $n\in\Nn^*$ et $E=\Cc_n[X]$. Pour $a\in\Cc$, on définit l'application $\varphi_a$ par : $\forall P\in E,\;\varphi_a(P)=P(a)$. Montrer que pour tout $a\in E$, $\varphi_a\in E^*$.}
\reponse{Soit $a\in\Cc$. Soient $(\lambda,\mu)\in\Cc^2$ et $(P,Q)\in E^2$.

\begin{center}
$\varphi_a(\lambda P+\mu Q)=(\lambda P+\mu Q)(a)=\lambda P(a)+\mu Q(a)=\lambda\varphi_a(P)+\mu\varphi_a(Q)$.
\end{center}

Donc, $\varphi_a$ est une forme linéaire sur $E$.}
    \item \question{Soient $a_0$, $a_1$,\ldots, $a_n$ $n+1$ nombres complexes deux à deux distincts. Montrer que la famille $\left(\varphi_{a_0},\ldots,\varphi_{a_n}\right)$ est une base de $E^*$ et déterminer sa préduale.}
\reponse{On a déjà $\text{card}\left(\varphi_{a_j}\right)_{0\leqslant j\leqslant n}=n+1=\text{dim}(E)=\text{dim}(E^*)<+\infty$. Il suffit donc de vérifier que la famille $\left(\varphi_{a_j}\right)_{0\leqslant j\leqslant n}$ est libre.

Pour $k\in\llbracket0,n\rrbracket$, on pose $P_k=\prod_{j\neq k}^{}\frac{X-a_j}{a_k-a_j}$. Chaque $P_k$ est un élément de $E$ et de plus 

\begin{center}
$\forall(j,k)\in\llbracket0,n\rrbracket^2$, $\varphi_{a_j}(P_k)=\delta_{j,k}=\left\{
\begin{array}{l}
1\;\text{si}\;j\neq k\\
0\;\text{si}\;j=k
\end{array}
\right.$\quad$(*)$.
\end{center}

Soit alors $(\lambda_0,\ldots,\lambda_n)\in\Cc^{n+1}$.

\begin{align*}\ensuremath
\sum_{j=0}^{n}\lambda_j\varphi_j=0&\Rightarrow\forall P\in E,\;\sum_{j=0}^{n}\lambda_j\varphi_j(P)=0\Rightarrow\forall k\in \llbracket0,n\rrbracket,\;\sum_{j=0}^{n}\lambda_j\varphi_j(P_k)=0\Rightarrow\forall k\in \llbracket0,n\rrbracket,\;\sum_{j=0}^{n}\lambda_j\delta_{j,k}=0\\
 &\Rightarrow\forall k\in \llbracket0,n\rrbracket,\;\lambda_k=0.
\end{align*}

Ceci montre que la famille $\left(\varphi_{a_j}\right)_{0\leqslant j\leqslant n}$ est libre et donc une base de $E^*$. Les égalités $(*)$ montrent alors que la préduale de la base $\left(\varphi_{a_j}\right)_{0\leqslant j\leqslant n}$ de $E^*$ est la famille $\left(P_k\right)_{0\leqslant k\leqslant n}$.}
    \item \question{Montrer qu'il existe $(\lambda_0,\ldots,\lambda_n)\in\Cc^{n+1}$ tel que $\forall P\in\Cc_n[X],\;\int_{0}^{1}P(t)\;dt=\lambda_0P(a_0)+\ldots+\lambda_nP(a_n)$ puis donner la valeur des $\lambda_i$ sous la forme d'une intégrale.}
\reponse{Pour $P\in E$, posons $\varphi(P)=\int_{0}^{1}P(t)\;dt$. $\varphi$ est une forme linéaire sur $E$ et donc, puisque la famille $\left(\varphi_{a_j}\right)_{0\leqslant j\leqslant n}$ est une base de $E^*$, il existe $(\lambda_0,\ldots,\lambda_n)\in\Cc^{n+1}$ tel que $\varphi=\sum_{j=0}^{n}\lambda_j\varphi_{a_j}$ ou encore il existe $(\lambda_0,\ldots,\lambda_n)\in\Cc^{n+1}$ tel que pour tout $P\in E$, $\int_{0}^{1}P(t)\;dt=\lambda_0P(a_0)+\ldots+\lambda_nP(a_n)$ (les $\lambda_j$ étant indépendants de $P$).

En appliquant cette dernière égalité au polynôme $P_k$, $0\leqslant k\leqslant n$, on obtient $\lambda_k=\int_{0}^{1}P_k(t)\;dt=\int_{0}^{1}\prod_{j\neq k}^{}\frac{t-a_j}{a_k-a_j}\;dt$.

\begin{center}
\shadowbox{
$\forall P\in\Cc_n[X],\;\int_{0}^{1}P(t)\;dt=\sum_{k=0}^{n}\lambda_kP(a_k)$ où $\forall k\in\llbracket0,n\rrbracket$, $\lambda_k=\int_{0}^{1}\prod_{j\neq k}^{}\frac{t-a_j}{a_k-a_j}\;dt$.
}
\end{center}}
\end{enumerate}
}
