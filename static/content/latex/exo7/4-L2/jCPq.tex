\uuid{jCPq}
\exo7id{5799}
\auteur{rouget}
\organisation{exo7}
\datecreate{2010-10-16}
\isIndication{false}
\isCorrection{true}
\chapitre{Espace euclidien, espace normé}
\sousChapitre{Problèmes matriciels}

\contenu{
\texte{
Montrer que le produit de deux matrices symétriques réelles positives est à valeurs propres réelles positives.
}
\reponse{
\textbf{Remarque.} Il faut prendre garde au fait que le produit de deux matrices symétriques n'est pas nécessairement symétrique. Plus précisément, si $A$ et $B$ sont deux matrices symétriques alors 

\begin{center}
$AB\in\mathcal{S}_n(\Rr)\Leftrightarrow{^t}(AB) = AB\Leftrightarrow{^t}B{^t}A = AB\Leftrightarrow BA = AB$
\end{center}

et le produit de deux matrices symétriques est symétrique si et seulement si ces deux matrices commutent. Donc au départ, rien n'impose que les valeurs propres de $AB$ soient toutes réelles .

Soient $A$ et $B$ deux matrices symétriques réelles positives. D'après l'exercice \ref{ex:rou2}, il existe deux matrices carrées $M$ et $N$ telles que $A ={^t}MM$ et $B={^t}NN$. On a alors $AB ={^t}MM{^t}NN$. La  matrice $AB$ a même polynôme caractéristique que la matrice $N({^t}MM{^t}N={^t}(M{^t}N)M{^t}N$. D'après l'exercice \ref{ex:rou2}, cette dernière matrice est symétrique positive et a donc des valeurs propres réelles positives. On a montré que les valeurs propres de la matrice $AB$ sont réelles et positives.

\begin{center}
\shadowbox{
$\forall(A,B)\in(\mathcal{S}_n^+(\Rr)$, $\text{Sp}(AB)\subset\Rr^+$.
}
\end{center}
}
}
