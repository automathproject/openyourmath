\uuid{mZos}
\exo7id{1400}
\auteur{hilion}
\organisation{exo7}
\datecreate{2003-10-01}
\isIndication{false}
\isCorrection{false}
\chapitre{Arithmétique}
\sousChapitre{Anneau Z/nZ, théorème chinois}

\contenu{
\texte{
On note $R_n$ la rotation du plan de centre $O$, d'angle $2\pi/n$, $S$ la symétrie par rapport à l'axe $(Ox)$.
}
\begin{enumerate}
    \item \question{Montrer que $S^2=id$, $(R_n)^n=id$ et $R_nS=SR_n^{-1}$.}
    \item \question{Montrer que le sous-groupe des isométries du plan engendré par $R_n$ et $S$ (ie le plus petit sous-groupe des isométries du plan qui contient $R_n$ et $S$) est de cardinal $2n$.
On le note $D_n$: c'est le groupe dihédral d'ordre $2n$.}
    \item \question{Montrer que $D_n$ préserve un polygone régulier à $n$ côtés, centré en $O$.}
    \item \question{En vous aidant de ce qui précède, construire un isomorphisme entre $D_3$ et $S_3$.}
\end{enumerate}
}
