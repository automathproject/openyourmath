\uuid{m5dm}
\exo7id{6930}
\auteur{ruette}
\organisation{exo7}
\datecreate{2013-01-24}
\isIndication{false}
\isCorrection{true}
\chapitre{Statistique}
\sousChapitre{Tests d'hypothèses, intervalle de confiance}

\contenu{
\texte{
On interroge 1000 électeurs, 521 d'entre eux ont déclaré avoir l'intention 
de voter pour le candidat A.
Indiquer avec une probabilité de  0,95  entre quelles limites se situe la 
proportion du corps électoral favorable à A au moment du sondage.
}
\reponse{
On note $p$ la proportion inconnue. Soit $X_i$ la variable qui vaut 1 si le 
$i$-ème électeur interrogé déclare avoir l'intention de voter pour A, 
0 sinon. Les $X_i$ sont indépendantes et suivent une loi de Bernoulli 
$\mathcal{B}(p)$, donc le nombre $Z=\sum_{i=1}^{1000}$ d'électeurs 
favorables à A dans un échantillon de 1000 électeurs suit une loi 
binomiale $\mathcal{B}(p,1000)$. Comme $p$ semble de l'ordre de 0,5, 
on peut approcher $\mathcal{B}(p,1000)$ par $\mathcal{N}(1000p,\sqrt{1000p(1-p)})$. 
De nouveau, comme $n=1000$ est grand, on peut estimer $p$ par la fréquence $f=0,521$
 observée dans l'échantillon, et supposer que l'écart-type de la variable 
fréquence d'échantillon $F=Z/1000$ vaut $\sqrt{p(1-p)}/\sqrt{1000}\simeq 0,0158$. 
Comme la variable $\displaystyle \frac{F-p}{\sigma(F)}$ est normale standard, 
la probabilité que l'intervalle $[f-1,96\times 0,0158,f+1,96\times 0,0158]$ 
ne contienne pas $p$ est inférieure à 0,05. On conclut que 
$[0,49, 0,55]$ (autrement dit $52\pm 3$\%) est intervalle de confiance relatif à $p$ au seuil de 95\%.
}
}
