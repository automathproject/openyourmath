\uuid{teJT}
\exo7id{6920}
\auteur{ruette}
\organisation{exo7}
\datecreate{2013-01-24}
\isIndication{false}
\isCorrection{true}
\chapitre{Probabilité discrète}
\sousChapitre{Espérance, variance}

\contenu{
\texte{
Soit $U$ et $V$ deux variables aléatoires de même loi, à
valeurs dans $\{1,\ldots, N\}$. On pose $X=U-V$ et $Y=U+V$. Déterminer la
covariance entre $X$ et $Y$.
}
\reponse{
$U$ et $V$ étant à valeurs dans $\{1, \dots ,N\}$, elles admettent des moments de tout ordre, donc $X$ et $Y$ aussi, les calculs qui suivent sont donc justifiés.
$\text{Cov}(X,Y)=E(XY)-E(X)E(Y)=E(XY)$ car $E(X)=E(U)-E(V)=0$ ($U, V$ ont
même loi).
$E(XY)=E((U-V)(U+V))=E(U^2)-E(V^2)=0$ ($U, V$ ont
même loi). D'où $\text{Cov}(X,Y)=0$.
}
}
