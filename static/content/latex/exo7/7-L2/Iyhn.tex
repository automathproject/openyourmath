\uuid{Iyhn}
\exo7id{6005}
\auteur{quinio}
\organisation{exo7}
\datecreate{2011-05-20}
\isIndication{false}
\isCorrection{true}
\chapitre{Probabilité discrète}
\sousChapitre{Variable aléatoire discrète}

\contenu{
\texte{
Une entreprise pharmaceutique décide de faire des économies sur les
tarifs d'affranchissements des courriers publicitaires à envoyer aux
clients.
Pour cela, elle décide d'affranchir, au hasard, une proportion de 3
lettres sur 5 au tarif urgent, les autres au tarif normal.
}
\begin{enumerate}
    \item \question{Quatre lettres sont envoyées dans un cabinet médical de quatre médecins:
quelle est la probabilité des événements:

A : <<Au moins l'un d'entre eux reçoit une lettre au tarif urgent>>.

B : <<Exactement 2 médecins sur les quatre reçoivent une lettre au tarif urgent>>.}
\reponse{On utilise une loi binomiale, loi de la variable aléatoire:
<<nombre de lettres affranchies au tarif urgent parmi 4 lettres>>
$n=5$, $p=\frac{3}{5}$. 
On obtient $P(A)=1-(\frac{2}{5})^{4}=0.9744$, 
$P(B)=\binom{4}{2}(\frac{2}{5})^{2}(\frac{3}{5})^{2}=0.3456$.}
    \item \question{Soit $X$ la variable aléatoire: <<nombre de lettres affranchies au tarif
urgent parmi 10 lettres>>:
Quelle est la loi de probabilité de $X$, quelle est son espérance,
quelle est sa variance?}
\reponse{La loi de probabilité de $X$ est une loi binomiale, loi de la variable
aléatoire: <<nombre de lettres affranchies au tarif
urgent parmi 10 lettres>>.
$n=10$, $p=\frac{3}{5}$, son espérance est $np=6$, sa variance est $np(1-p)=\frac{12}{5}$.}
\end{enumerate}
}
