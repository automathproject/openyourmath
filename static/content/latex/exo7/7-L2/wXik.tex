\uuid{wXik}
\exo7id{5984}
\auteur{quinio}
\organisation{exo7}
\datecreate{2011-05-18}
\isIndication{false}
\isCorrection{true}
\chapitre{Probabilité discrète}
\sousChapitre{Probabilité et dénombrement}

\contenu{
\texte{
Un étudiant s'habille très vite le matin et prend, au hasard
dans la pile d'habits, un pantalon, un tee-shirt, une paire de chaussettes;
il y a ce jour-là dans l'armoire $5$ pantalons dont $2$ noirs, 
$6$ tee-shirt dont $4$ noirs, $8$ paires de chaussettes, dont $5$ paires
noires. Combien y-a-t-il de façons de s'habiller? Quelles sont les
probabilités des événements suivants : il est tout en noir; une
seule pièce est noire sur les trois.
}
\reponse{
- Une tenue est un triplet $(P, T, C)$ : il y a $5\times 6\times 8=240$ tenues
différentes;

- <<Il est tout en noir>> : de combien de façons différentes ? 
Réponse : de $2\times 4\times 5=40$ façons.

La probabilité de l'événement <<Il est tout en
noir>> est donc : $\frac{40}{240}=\frac{1}{6}$.

- <<Une seule pièce est noire sur les trois >> : notons les événements :
$N_{1}$ la première pièce (pantalon) est noire, $N_{2}$ la deuxième pièce (tee-shirt)
 est noire, $N_{3}$ la troisième pièce
(chaussette) est noire: l'événement est représenté par :
$(N_{1}\cap \overline{N_{2}}\cap \overline{N_{3}})\cup (\overline{N_{1}}\cap
N_{2}\cap \overline{N_{3}})\cup (\overline{N_{1}}\cap \overline{N_{2}}\cap
N_{3})$.
Ces trois événements sont disjoints, leurs probabilités
s'ajoutent. La probabilité de l'événement <<une seule pièce est noire sur les trois>> est donc : 
$0.325$.
}
}
