\uuid{m6xz}
\exo7id{6031}
\auteur{quinio}
\organisation{exo7}
\datecreate{2011-05-20}
\isIndication{false}
\isCorrection{true}
\chapitre{Statistique}
\sousChapitre{Tests d'hypothèses, intervalle de confiance}

\contenu{
\texte{
Afin de mieux satisfaire leurs clients, une grande société
fournisseur d'accès internet fait ses statistiques sur le nombre
d'appels reçus en \emph{hotline}, elle pourra ainsi évaluer le temps
d'attente pour le client et le nombre d'employés à mettre au
standard; les résultats de l'enquête portent sur 200 séquences \
consécutives de une minute, durant lesquelles le nombre d'appels moyen a 
été de 3 appels par minute. On suppose que les appels sont répartis également dans le
temps: on partage un intervalle de temps en unités de une seconde;
alors dans chaque unité de temps, il y a au plus un appel.
}
\begin{enumerate}
    \item \question{Quelle est la loi de probabilité du nombre d'appels reçus
en 4 minutes?}
\reponse{L'intervalle de temps de $4$ minutes est la répétition de $240$
secondes, au cours desquelles les appels surviennent de façon indépendante, avec la probabilité d'appel de $\frac{1}{20}$; la loi de
probabilité du nombre d'appels reçus en $4$ minutes est donc une loi
binomiale, de paramètres $n=240$ et $p=\frac{1}{20}$.}
    \item \question{Montrer que l'on peut approcher cette loi par une loi de Poisson.}
\reponse{Comme $n\geq 30$ et $np\leq 15,$ il est possible d'approcher cette loi
par une loi de Poisson de paramètre $\lambda $ estimé par $np=12$.}
    \item \question{Donner un intervalle de confiance pour le nombre moyen d'appels en 4 minutes.}
\reponse{Un échantillon de taille $200$ a été réalisé pour
estimer le nombre moyen d'appels par minute; c'est un échantillon de
taille $50$ pour la variable précédente (nombre d'appels reçus
en 4 minutes) qui suit une loi de Poisson d'espérance et de variance $12$. 
Un intervalle de confiance au niveau 95\% pour la moyenne est $I_{\alpha}=[11;13]$.}
\end{enumerate}
}
