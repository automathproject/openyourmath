\uuid{tSVB}
\titre{Modélisation et rencontre de généraux}
\theme{variables aléatoires à densité}
\auteur{}
\datecreate{2022-11-28}
\organisation{AMSCC}
\contenu{



\texte{ Le général Contant doit échanger des documents top secrets avec le général Janty. Pour cela, ils ont rendez-vous au bar Le Torkel. On suppose que l'heure d'arrivée de chaque individu au point de rendez-vous est uniformément distribué entre 20h et 21h. En revanche, chacun a promis de ne pas attendre l'autre plus de 10 minutes sur place.  }
	
\begin{enumerate}
	\item \question{ 	Décrire la probabilité que ces deux personnes puissent effectuer leur transaction sous forme d'une intégrale double. }
	\item \question{ 	Quelle est la probabilité que ces deux personnes puissent effectuer leur transaction ? }
\end{enumerate}


\reponse{ Soit $X$ l'heure d'arrivée du général Contant et $Y$ l'heure d'arrivée du général Janty. L'énoncé suggère que $X$ et $Y$ sont des \vas indépendantes et nous dit qu'elles suivent toutes les deux une loi uniforme sur $[20;21]$.
	La probabilité cherchée est $\p(|X-Y|\leq \frac{1}{6})$.
	
	Comme $X$ et $Y$ sont indépendantes, on a
	\[ f_{(X,Y)}(x,y)=f_X(x)f_Y(y) =\mathbf{1}_{[20;21]^2} (x,y).\]
	Donc
	\[ p%=\int \int \mathbf{1}_{\{|x-y|\leq \frac{1}{6}\}}\mathbf{1}_{[20;21]^2} (x,y) dx\ dy
	= \int \int_D dx\dy,
	\]
	où
	$D=\{(x,y)\in\R^2|20\leq x \leq 21, \ 20\leq y \leq 21, \ \frac{-1}{6}\leq x-y \leq \frac{1}{6}\}$. 
	\vspace{1em}
	
	%%%%%%% représentation de D à faire
	On peut évaluer géométriquement cette aire mais nous allons faire le calcul de l'intégrale double. On commence par remplacer l'intervalle $[20;21]$ par l'intervalle $[0;1]$ en effectuant une translation. Ainsi, l'écriture est simplifiée.
	\begin{align*}
		p&= \int_0^{\frac{1}{6}} \int_0^{\frac{1}{6}+y} dx \dy
		+ \int_{\frac{1}{6}}^{\frac{5}{6}} \int_{y-\frac{1}{6}}^{y+\frac{1}{6}} dx \dy
		+ \int_{\frac{5}{6}}^1 \int_{y-\frac{1}{6}}^{1} dx \dy \\
		&= \int_0^{\frac{1}{6}} (\frac{1}{6}+y) \dy
		+ \int_{\frac{1}{6}}^{\frac{5}{6}} \frac{1}{3} \dy
		+ \int_{\frac{5}{6}}^1 (\frac{7}{6}-y) \dy \\
		&=\left[\frac{1}{6}y+\frac{1}{2}y^2 \right]_0^\frac{1}{6}
		+\left[\frac{1}{3}y\right]_\frac{1}{6}^\frac{5}{6}
		+\left[\frac{7}{6}y-\frac{1}{2}y^2 \right]_\frac{5}{6}^1 \\
		&=\frac{11}{36}.
	\end{align*}
	La probabilité que les deux généraux puissent effectuer leur transaction est donc de l'ordre de $0.31$.
}
}
