\uuid{S6PX}
\exo7id{6334}
\auteur{queffelec}
\organisation{exo7}
\datecreate{2011-10-16}
\isIndication{false}
\isCorrection{false}
\chapitre{Théorème de Cauchy-Lipschitz}
\sousChapitre{Théorème de Cauchy-Lipschitz}

\contenu{
\texte{
On considère l'équation  $x x'' = (x')^2 +1$ sur $\Rr$.
}
\begin{enumerate}
    \item \question{Montrer que, $x_0 \neq 0 $ et $x_0'$ étant donnés dans
$\Rr$, il existe une unique solution $\varphi$ définie au
voisinage de $0$, telle que $\varphi (0) =x_0$ et $\varphi '(0) =x'_0$.}
    \item \question{Si de plus $x'_0 \neq 0$, on peut supposer que $\varphi $ est
un $C^1$--difféomorphisme d'un voisinage  de $0$ sur un
voisinage de $x_0$ (pourquoi?); on note $\psi$ l'application
réciproque et on pose $z(x) = \varphi '( \psi (x))$. Calculer $z'(x)
$, trouver l'équation satisfaite par $z$ et expliciter $z$; en
déduire une expression de $\varphi$.}
    \item \question{Quelle est la solution  $\varphi$ de l'équation telle que
$\varphi (0) = x_0 \neq 0$ et $\varphi ' (0) =0$.}
\end{enumerate}
}
