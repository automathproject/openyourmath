\uuid{S3yI}
\exo7id{2562}
\auteur{tahani}
\organisation{exo7}
\datecreate{2009-04-01}
\isIndication{false}
\isCorrection{false}
\chapitre{Solution maximale}
\sousChapitre{Solution maximale}

\contenu{
\texte{
On consid\`ere l'\'equation
$xx''=(x')^2+1$ sur $\mathbb{R}$.
}
\begin{enumerate}
    \item \question{Montrer
que, $x_0\neq 0$ et $x_0'$ \'etant donn\'es dans $\mathbb{R}$, il
existe une unique solution $\varphi$ d\'efinie au voisinage de
$0$, telle que $\varphi(0)=x_0$ et $\varphi'(0)=x_0'$.}
    \item \question{Si de
plus $x_0' \neq 0$, on peut supposer que $\varphi$ est un
$C^1$-diff\'eomorphisme d'un voisinage de $0$ sur un voisinage de
$x_0$ (pourquoi ?); on note $\psi$ l'application r\'eciproque et
on pose $z(x)=\varphi'(\psi(x))$. Calculez $z'(x)$, trouver
l'\'equationb satisfaite par $z$ et expliciter $z$; en d\'eduire
une expression de $\varphi$.}
    \item \question{Quelle est la solution
$\varphi$ de l\'equation telle que $\varphi(0)=x_0\neq 0$ et
$\varphi'(0)=0$.}
\end{enumerate}
}
