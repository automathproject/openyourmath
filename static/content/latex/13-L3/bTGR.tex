\uuid{bTGR}
\exo7id{6318}
\auteur{queffelec}
\organisation{exo7}
\datecreate{2011-10-16}
\isIndication{false}
\isCorrection{false}
\chapitre{Solution maximale}
\sousChapitre{Solution maximale}

\contenu{
\texte{
On considère l'équation différentielle du second ordre
$$({\cal E})\quad\quad y''+ y=f(x)$$
où $f$ est une fonction continue sur ${\Rr}$ tout entier.
}
\begin{enumerate}
    \item \question{On suppose que $f$ est un polyn\^ome de degré $n$. 
On note $E$ l'espace vectoriel des polyn\^omes réels de degré $\leq n$. Montrer
que l'application
$u$, définie sur $E$ par $u(P) = P''+ P$, est une application injective de $E$
 dans lui-même. 

En déduire que l'équation ($\cal E$) a une et une seule solution polyn\^omiale $g$ qui est de même
degré que $f$.}
    \item \question{On suppose maintenant que $f$ est une fonction continue quelconque sur ${\Rr}$. 
Montrer que la fonction
$$\int_0^x f(t) \sin(x-t) \ dt $$ est une solution particulière de ($\cal E$); en déduire la
solution générale de l'équation.}
    \item \question{Montrer que si $f$ vérifie l'inégalité $f(t) \leq {1\over{1+t^2}}$ pour tout
$t\in {\Rr}$, toutes les solutions de ($\cal E$) sont des fonctions bornées.
A-t-on la même conclusion si la fonction $f$ est seulement bornée sur ${\Rr}$?}
\end{enumerate}
}
