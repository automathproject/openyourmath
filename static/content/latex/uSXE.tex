\uuid{uSXE}
\titre{Durées des contrôles fiscaux}
\theme{estimateurs, intervalle de confiance}
\auteur{}
\organisation{AMSCC}
\contenu{


\texte{ On s'intéresse à la méthode de contrôle fiscal d'une entreprise qui consiste à vérifier la comptabilité de l'entreprise.

On considère que le temps de contrôle d'une entreprise est une variable aléatoire de loi $\mathcal{N}\left(m, \sigma^2\right)$. On réalise 7 contrôles et on obtient les temps suivants (en jours) :
$$
\begin{array}{lllllll}
	57 & 61 & 42 & 53 & 45 & 65 & 58 .
\end{array}
$$
 }
\begin{enumerate}
	\item 
\question{ Donner une estimation de $m$ et de $\sigma^2$. On précisera les estimateurs utilisés en indiquant leurs propriétés.}
\reponse{ On estime la moyenne $m$ à l'aide de la moyenne empirique $\left(\bar{X}=\frac{1}{n} \sum_{i=1}^n X_i\right)$, qui est sans biais et convergente :
$$
\bar{x}=\frac{57+61+42+53+45+65+58}{7}=\frac{381}{7} \simeq 54.43
$$
 On estime la variance $\sigma^2$ à l'aide de la variance empirique $\left(S^2=\frac{1}{n-1} \sum_{i=1}^n\left(X_i-\bar{X}\right)^2\right)$, qui est sans biais et convergent :
$$
s^2=\frac{1}{6}\left((57-54.43)^2+(61-54.43)^2+\ldots+(58-54.43)^2\right) \simeq 69.95
$$ }
\item
\question{  Donner un intervalle de confiance de niveau $90 \%$ permettant d'estimer $m$. }
\reponse{ Il s'agit d'un intervalle de confiance d'une moyenne dans le cas où la variance est inconnue et la loi mère de l'échantillon est une loi Normale. On a donc:
$$
I C(\bar{X})=\left[\bar{x}-t \times \frac{s}{\sqrt{n}} ; \bar{x}+t \times \frac{s}{\sqrt{n}}\right]
$$
avec $\bar{x}=54.43, s=\sqrt{69.95} \simeq 8.36, n=7$ et $t$ est le réel tel que $\mathbb{P}(U \leq t)=1-\frac{0.10}{2}=0.95$, où $U \sim S t(6)$, c'est-à-dire $t=1.9432$. On obtient alors
$$
I C(\bar{X})=[48.29 ; 60.57]
$$ }
%\item
%\question{ En notant $X$ le nombre d'entreprises détectées en fraude par an en France, déterminer la loi de $X$, ainsi que son espérance et sa variance. }
%\reponse{ $$
%X \sim \mathcal{B}(1200,0.56), \mathbb{E}(X)=1200 * 0.56=672 \text { et } \mathbb{V}(X)=1200 * 0.56 *(1-0.56)=295.68
%$$ }
%\item 
%\question{ Par quelle loi peut-on approcher la loi de $X$ ? }
%\reponse{ Comme $n=1200$ est grand et que $p=0.56$ est proche de $0.5$, on peut approcher la loi de $X$ par la loi $\mathcal{N}\left(672, \sigma^2=295.68\right)$ }
%
%\item
%\question{  À l'aide de cette approximation, calculer la probabilité qu'en une année, il y ait entre 650 et 720 entreprises détectées en fraude fiscale. }
%\reponse{ On a :
%$$
%\begin{aligned}
%	\mathbb{P}(650 \leq X \leq 720) & \simeq \mathbb{P}(650 \leq Y \leq 720), \quad \text { où } Y \sim \mathcal{N}\left(672, \sigma^2=295.68\right) \\
%	& \simeq \mathbb{P}(-1.28 \leq Z \leq 2.79), \quad \text { où } Z=\frac{Y-672}{\sqrt{295.68}} \sim \mathcal{N}(0,1) \\
%	& \simeq \mathbb{P}(Z \leq 2.79)-\mathbb{P}(Z \leq-1.28) \\
%	& \simeq \mathbb{P}(Z \leq 2.79)-(1-\mathbb{P}(Z \leq 1.28)) \\
%	& \simeq 0.9974-(1-0.8997) \\
%	& \simeq 0.8971
%\end{aligned}
%$$
%soit une probabilité d'environ $89.71 \%$ d'avoir entre 650 et 720 entreprises détectées en fraude durant une année. }
\end{enumerate}}
