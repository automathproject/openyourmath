\uuid{GWu9}
\exo7id{2420}
\auteur{drutu}
\organisation{exo7}
\datecreate{2007-10-01}
\isIndication{false}
\isCorrection{true}
\chapitre{Espace topologique, espace métrique}
\sousChapitre{Espace topologique, espace métrique}

\contenu{
\texte{
On appelle \textit{base} d'une topologie $\mathcal{T}$ un sous-ensemble
$\mathcal{B}$ de $\mathcal{T}$ tel que tout ouvert $\mathcal{O} \in \mathcal{T}$ s'\'ecrit comme
$\mathcal{O}=\bigcup_{i\in I} B_i$, o\`u $B_i \in \mathcal{B}$ pour tout $i\in I$.
}
\begin{enumerate}
    \item \question{Montrer que $\mathcal{B}$ est une base de $\mathcal{T}$ si et
  seulement si pour tout ouvert $\mathcal{O}$ et tout point $x\in \mathcal{O}$ il
  existe un $B\in \mathcal{B}$ tel que $x\in B\subset \mathcal{O}$.}
    \item \question{Soit $\mathcal{T}_n$ la topologie sur $\R^n$ induite par la
  m\'etrique euclidienne
  $$\mathrm{dist} (\bar{x}, \bar{y})=\sqrt{(x_1-y_1)^2+...+(x_n-y_n)^2}\, .
  $$ Montrer que l'ensemble $\mathcal{B}$ de boules ouvertes ayant leur centre dans
  $\Q^n$et leur rayon dans $\Q$ est une base de $\mathcal{T}_n$.}
    \item \question{Soit $\mathcal{B}'$ l'ensemble de parallelipip\`edes ouverts dans
  $\R^n$ dont les ar\^etes sont parall\`eles aux axes de
  coordonn\'ees. Est-ce que $\mathcal{B}'$ est une base de $\mathcal{T}_n$ ?}
    \item \question{Est-ce que $\{ ]-\infty ,a[\; ;\; a\in \R  \}\cup \{ ]b, +\infty[\; ;\; b\in \R
  \}$ est une base pour $\mathcal{T}_1$ ?}
    \item \question{Pour tout $a\in \Q$ on note par $\delta_a$ la droite d'\'equation $y=ax$
   dans $\R^2$, et on note par $Y$ la r\'eunion des droites $\delta_a$. Soit $\mathcal{T}$ la
  topologie sur $Y$ induite par la topologie sur $\R^2$ et soit
  $\mathcal{T}'$ la topologie de base $\mathcal{B}'$ compos\'ee par tous les
  segments ouverts
   $]M,N[\subset \delta_a$, $O\not\in ]M,N[$, et par toutes les reunions $\bigcup_{a\in \Q, O\in ]M_a,N_a[} ]M_a,N_a[$. Les
  deux topologies $\mathcal{T}$ et $\mathcal{T}'$ sont-elles \'equivalentes ?}
\reponse{
Supposons que $\mathcal{B}$ est une base de $\mathcal{T}$, et soit $\mathcal{O}$ un
ouvert arbitraire dans $\mathcal{T}$ et $x$ un point de $\mathcal{O}$. L'ouvert
$\mathcal{O}$ s'\'ecrit comme $\mathcal{O}=\bigcup_{i\in I} B_i$, o\`u $B_i \in
\mathcal{B}$ pour tout $i\in I$. En particulier il existe un $i_0\in I$
tel que $x\in B_{i_0}$.
R\'eciproquement, si $\mathcal{O}$ est un ouvert arbitraire, pour tout point $x\in \mathcal{O}$ il
  existe un $B_x\in \mathcal{B}$ tel que $x\in B_x\subset \mathcal{O}$. Par
  cons\'equent $\mathcal{O}=\bigcup_{x\in \mathcal{O}}B_x$.
Il suffit de montrer la propri\'et\'e \'enonc\'ee dans (1).
Soit $\mathcal{O}\in \mathcal{T}_n$ et soit $x$ un point arbitraire de $\mathcal{O}$.
D'apr\`es le cours, il existe un $r>0$ tel que $B(x,r)\subset
\mathcal{O}$.
 \emph{Remarque.} Une autre manni\`ere de formuler ceci
est de dire que l'ensemble des boules ouvertes euclidiennes forme
une base de la topologie $\mathcal{T}_n$.



Puisque l'ensemble $\Q^n$ est dense dans $\R^n$, il s'ensuit que
$B\left( x, \frac{r}{2}\right)$ contient un vecteur $q\in \Q^n$.
En particulier $\mathrm{dist} (x,q)< \frac{r}{2}$, d'o\`u $B\left( q,
\frac{r}{2}\right)\subset B\left( x, r\right)\subset \mathcal{O}$.

L'intervalle $]\mathrm{dist} (x,q),\frac{r}{2}[ $ est non-vide, donc il
contient un nombre rationnel $R$. Ainsi $x\in B(q,R)\subset
B\left( q, \frac{r}{2}\right)\subset \mathcal{O}$.
Puisque $\mathcal{B}'\subset \mathcal{T}_n$, ce qu'il reste \`a d\'emontrer
est \`a nouveau la propri\'et\'e \'enonc\'ee dans (1). Soit $\mathcal{O}$
un ouvert et $x\in \mathcal{O}$. Il existe un $r>0$ tel que $B(x,r)\subset
\mathcal{O}$.

D'apr\`es le cours
$$
\mathrm{dist} (y,x)= \|y-x\|_2\leq \sqrt{n}\|y-x\|_\infty\, .
$$

Il s'ensuit que
\begin{equation}\label{bile}
B_\infty \left( x,\frac{r}{\sqrt{n}} \right) =\left\{y\; ;\;
\|y-x\|_\infty<\frac{r}{\sqrt{n}}\right\}\subset B(x,r) \subset
\mathcal{O}\, .
\end{equation}


Or $B_\infty \left( x,\frac{r}{\sqrt{n}} \right)$ n'est rien
d'autre que le cube de centre de sym\'etrie $x$ et de longueur des
ar\^etes $\frac{2r}{\sqrt{n}}$. En particulier $B_\infty \left(
x,\frac{r}{\sqrt{n}} \right)\in \mathcal{B}'$.

On conclut que $\mathcal{B}'$ est une base de $\mathcal{T}_n$.
Soit $]0,1[\in \mathcal{T}_1$. Il n'existe pas d'intervalle de la
forme $]-\infty ,a[\, ,\; a\in \R \, ,$ ou $ ]b, +\infty[\, ,\;
b\in \R \, ,$ contenu dans $]0,1[$. Donc $\mathcal{B}''$ n'est pas une
base pour $\mathcal{T}_1$.
Supposons que $\mathcal{T}'\subset \mathcal{T}$. En particulier $\mathcal{B}'\subset
\mathcal{T}$.

Pour tout $a=\frac{m}{n}\in \Q$, o\`u $m\in \Z^*\, ,\, n\in \N^*\,
,$ p.g.c.d. $(m,n)=1$, on choisit $M_a,N_a$ deux points sur la
droite $\delta_a$ tels que $O\in ]M_a,N_a[$ et $\mathrm{dist}
(O,M_a)=\mathrm{dist} (O,N_a)=\frac{1}{n}$. Pour $a=0$ on choisit
$M_0=(1,0)\, ,\, N_0=(-1,0)$. Soit
$$
\mathcal{C}=\bigcup_{a\in \Q} ]M_a,N_a[\, .
$$

Par hypoth\`ese $\mathcal{C}\in \mathcal{B}'\subset \mathcal{T}$. En particulier,
puisque $O$ est un point de $\mathcal{C}$, il existe $r>0$ tel que
$Y\cap B(O,r)\subset \mathcal{C}$. Pour tout $a\in \Q$ on a donc
$\delta_a\cap B(O,r)\subset ]M_a,N_a[$, d'o\`u $r<\mathrm{dist}
(O,M_a)=\frac{1}{n}$. Comme ceci est v\'erifi\'e pour tout $n\in
\N^*$, il s'ensuit que $r\leq 0$, ce qui contredit le choix de
$r$.

On a obtenu une contradiction. Donc on ne peut pas avoir
$\mathcal{T}'\subset \mathcal{T}$.
}
\end{enumerate}
}
