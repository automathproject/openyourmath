\uuid{NoDz}
\exo7id{2357}
\auteur{queffelec}
\datecreate{2003-10-01}
\isIndication{true}
\isCorrection{true}
\chapitre{Continuité, uniforme continuité}
\sousChapitre{Continuité, uniforme continuité}

\contenu{
\texte{

}
\begin{enumerate}
    \item \question{Montrer que $f$ est continue si et seulement si $f(\overline A)\subset
\overline{f(A)}$ pour tout $A$ dans $X$. Que peut-on dire alors de l'image par
$f$ d'un ensemble dense dans $X$?}
\reponse{$\Rightarrow$.
Soit $f$ continue et $y\in f(\bar A)$. Il existe $x\in \bar A$ tel que $y=f(x)$. Soit $x_n \in A$ tel que $(x_n)$ converge vers $x$. Alors
 $y_n=f(x_n)\in A$. Comme $f$ est continue alors 
$(y_n)$ converge vers $f(x)=y$. Donc $y$ est adhérent à $f(A)$.
Conclusion $f(\bar A) \subset \overline{f(A)}$.

$\Leftarrow$. Soit $f : X \rightarrow Y$ et soit $F$ un fermé de $Y$.
Notons $A = f^{-1}(F)$. Alors $f(A) \subset F$ donc l'équation 
$f(\bar A) \subset \overline{f(A)}$
devient $f(\bar A) \subset \bar F = F$ car $F$ est fermé.
Donc $\bar A \subset f^{-1}(F)=A$. Donc
$\bar A \subset A$, d'o\`u $\bar A=A$. Donc $A$ est fermé. Bilan l'image réciproque de tout fermé $F$ est un fermé, donc $f$ est continue.

Application : si $A$ est dense, alors $\bar A= X$, et sous les hypothèses précédentes alors $f(A)$ est dense dans l'image de $X$ par $f$ : en effet $\overline{f(A)}$ contient $f(\bar A) = f(X)$}
    \item \question{Montrer que $f$ est ferm\'ee si et seulement si $\overline{f(A)}\subset
f(\overline A)$, et que $f$ est ouverte si et seulement si ${f(\stackrel\circ
A)}\subset \stackrel\circ{f( A)}$.}
\reponse{$\Rightarrow$. Soit $f$ fermé et soit $A \subset X$.
Alors $A \subset \bar A$ donc $f(A) \subset f(\bar A)$, donc
comme $\bar A$ est un fermé et $f$ est fermée alors $f(\bar A)$ est un fermé contenant $f(A)$. Mais comme $\overline {f(A)}$ est le plus petit fermé contenant $f(A)$ alors $\overline {f(A)} \subset f(\bar A)$.

$\Leftarrow$. La relation pour un fermé $F$ donne 
$\overline {f(F)} \subset f(\bar F) = f(F)$. Donc $\overline {f(F)} = f(F)$.
Donc $f(F)$ est fermé. Donc $f$ est fermée.

Même type de raisonnement avec $f$ ouverte.}
\indication{\begin{enumerate}
\item Pour le sens direct utiliser la caractérisation de l'adhérence par les suites. Pour le sens réciproque, montrer que l'image réciproque d'un fermé est un fermé. 
   \end{enumerate}}
\end{enumerate}
}
