\uuid{s5G2}
\exo7id{6043}
\auteur{queffelec}
\organisation{exo7}
\datecreate{2011-10-16}
\isIndication{false}
\isCorrection{false}
\chapitre{Espace topologique, espace métrique}
\sousChapitre{Espace topologique, espace métrique}

\contenu{
\texte{
Si $A$ est une partie de l'espace topologique $X$, on pose
$\alpha(A)= \stackrel{\circ}{\overline{A}}$ et
$\beta(A)=\overline{ \stackrel{\circ}{A}}$.
}
\begin{enumerate}
    \item \question{Montrer que $\alpha$ et $\beta$ sont des applications croissantes pour
l'inclusion de ${\cal P}(X)$ dans ${\cal P}(X)$.}
    \item \question{Montrer que si $A$ est ouvert, $A\subset \alpha(A)$ et si $A$ est fermé,
$\beta(A)\subset A$. En déduire que $\alpha^2=\alpha$ et $\beta^2=\beta$.}
    \item \question{Construire $A\subset\Rr$ tel que les cinq ensembles  :

$A$, $\overline{A}$, $\stackrel{\circ}{A}$, $\alpha(A)$, $\beta(A)$
soient tous distincts.}
\end{enumerate}
}
