\uuid{qLLa}
\exo7id{2355}
\auteur{queffelec}
\organisation{exo7}
\datecreate{2003-10-01}
\isIndication{true}
\isCorrection{true}
\chapitre{Continuité, uniforme continuité}
\sousChapitre{Continuité, uniforme continuité}

\contenu{
\texte{
Soit $f,g$ deux applications continues de $X$ dans $Y$, espaces topologiques, $Y$ \'etant s\'epar\'e.
Montrer que $\{f=g\}$ est ferm\'e dans $X$; en d\'eduire que si $f$ et $g$
coïncident sur une partie dense de $X$, alors $f=g$.
}
\indication{Montrer que le complémentaire est un ouvert. Si vous le souhaitez, placez-vous dans des espaces métriques.}
\reponse{
Soit $A = \{ x \in X \mid f(x)=g(x) \}$. Alors soit 
$C = X\setminus A = \{ x \in X \mid f(x) \neq g(x) \}$.
Soit $x \in C$ comme $f(x)\neq g(x)$ et que $Y$ est séparé, il existe un voisinage ouvert $V_1$ de $f(x)$ et $V_2$ de $g(x)$ tel que 
$V_1 \cap V_2 = \varnothing$.
Notons $U = f^{-1}(V_1) \cap g^{-1}(V_2)$. Alors $U$ est un ouvert de $X$ contenant $x$. Maintenant pour 
$x' \in U$, alors $f(x') \in V_1$,
$g(x')\in V_2$ donc $f(x')\neq g(x')$, donc $x'\in C$. Bilan 
$U$ est inclus dans $C$. Donc $C$ est ouvert.

Application : si $A$ est dense dans $X$ alors $\bar A = X$, mais comme $A$ est fermé $A = \bar A$. Donc $A=X$, c'est-à-dire $f$ et $g$ sont égales partout.
}
}
