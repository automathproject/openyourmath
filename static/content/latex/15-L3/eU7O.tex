\uuid{eU7O}
\exo7id{2382}
\auteur{mayer}
\datecreate{2003-10-01}
\isIndication{false}
\isCorrection{true}
\chapitre{Compacité}
\sousChapitre{Compacité}

\contenu{
\texte{
On se donne une m\'etrique $d$ sur $X=[0,1]$ telle que
l'identit\'e $ i: (X, |.|) \to (X,d)$ soit continue (i.e. la
topologie d\'efinie par $d$ est moins fine que la topologie
usuelle de $X$).
}
\begin{enumerate}
    \item \question{Montrer que tout sous-ensemble de $X$ compact pour la
topologie usuelle est aussi compact pour la topologie d\'efinie
par $d$; puis montrer cette propri\'et\'e pour les ferm\'es.}
\reponse{Soit $K$ un compact pour $|.|$. Soit $U_i$, $i\in I$ tels
que $K \subset \bigcup_{i\in I} U_i$ et tels que $U_i$ soient des ouverts pour $d$. Alors les $U_i$ sont aussi des ouverts pour la topologie définie par $|.|$. 
Comme $K$ est compact pour $|.|$ alors on peut extraire un ensemble fini $J\subset I$ tel que $K \subset \bigcup_{i\in J} U_i$. Donc $K$ est aussi compact pour $d$. 

Si $F$ est un fermé pour $|.|$ alors $F\subset [0,1]$ est compact pour $|.|$
Donc compact pour $d$, donc fermé pour $d$.}
    \item \question{En d\'eduire que la topologie d\'efinie par $d$ est la topologie
usuelle.}
\reponse{Si $U$ est un ouvert pour $d$ alors $U$ est un ouvert pour $|.|$.
Car $i$ est continue. Réciproquement si $U$ est un ouvert pour $|.|$ alors
$F = X\setminus U$ est un fermé pour $|.|$ donc $F$ est un fermé pour $d$ par la question précédente, donc $U = X \setminus F$ est un ouvert pour $d$.
Conclusion les ouverts pour $|.|$ et $d$ sont les mêmes donc
$|.|$ et $d$ définissent la même topologie.}
\end{enumerate}
}
