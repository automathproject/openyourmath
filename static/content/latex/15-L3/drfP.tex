\uuid{drfP}
\exo7id{6037}
\auteur{queffelec}
\organisation{exo7}
\datecreate{2011-10-16}
\isIndication{false}
\isCorrection{false}
\chapitre{Espace topologique, espace métrique}
\sousChapitre{Espace topologique, espace métrique}

\contenu{
\texte{
\label{exn7}
Soit $X$ un ensemble non vide et $\Sigma$ une famille de parties de $X$ stable
par intersection finie et contenant $X$. Montrer que la plus petite topologie
$\cal T$ conte\-nant $\Sigma$ (la topologie engendrée par $\Sigma$) est
constituée des unions d'ensembles de $\Sigma$, ou, de fa\c con équivalente, 

$$A\in{\cal T}\Longleftrightarrow \forall x\in A\ \exists S\in \Sigma\ ;\ x\in
S\subset A.$$

Montrer que l'on peut affaiblir l'hypothèse de stabilité par intersection finie
en :

$(*)\qquad\forall S_1, S_2\in\Sigma,\ \forall x\in S_1\cap S_2,\ \exists
S_3\in\Sigma\ ;\ x\in S_3\subset S_1\cap S_2.$
}
}
