\uuid{2RHR}
\exo7id{6049}
\auteur{queffelec}
\datecreate{2011-10-16}
\isIndication{false}
\isCorrection{false}
\chapitre{Espace topologique, espace métrique}
\sousChapitre{Espace topologique, espace métrique}

\contenu{
\texte{
On rappelle la construction de l'ensemble triadique de Cantor : on part du
segment $[0,1]$ dont on supprime l'intervalle médian $]{1\over 3}, {2\over
3}[$; à la deuxième étape, on supprime les intervalles  $]{1\over 9},
{2\over 9}[$ et $]{7\over 9}, {8\over 9}[$ etc. On note $K_n$ la réunion des
intervalles restants à la $n$-ième étape, et $K=\bigcap K_n$. Quelle est
l'adhérence et l'intérieur de $K$?
}
}
