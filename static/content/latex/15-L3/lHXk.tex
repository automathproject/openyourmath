\uuid{lHXk}
\exo7id{6038}
\auteur{queffelec}
\organisation{exo7}
\datecreate{2011-10-16}
\isIndication{false}
\isCorrection{false}
\chapitre{Espace topologique, espace métrique}
\sousChapitre{Espace topologique, espace métrique}

\contenu{
\texte{
Soit $C$ l'ensemble des fonctions continues réelles sur $[0,1]$. Pour toute
$f\in C$ et $\varepsilon>0$ on définit
$$M(f,\varepsilon)=\{g/\int_0^1\vert f-g\vert<\varepsilon\}.$$
Montrer que la famille {\cal M} des ensembles $M(f,\varepsilon)$ lorsque $f\in
C$ et  $\varepsilon>0$ est une base de topologie. Même question avec la famille
$$U(f,\varepsilon)=\{g/\sup_x\vert f(x)-g(x)\vert<\varepsilon\}.$$
}
}
