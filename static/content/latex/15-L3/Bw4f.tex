\uuid{Bw4f}
\exo7id{2368}
\auteur{mayer}
\organisation{exo7}
\datecreate{2003-10-01}
\isIndication{false}
\isCorrection{true}
\chapitre{Application linéaire bornée}
\sousChapitre{Application linéaire bornée}

\contenu{
\texte{
Soit $X={\cal C}([0,1])$ avec la norme $\| f\| = \int _0^ 1 |f(t)|\, dt$.
Montrer que la forme lin\'eaire $f\in X \mapsto f(0)\in \Rr$ n'est pas continue.
Que peut-on en d\'eduire pour le sous-espace des fonctions de $X$ nulles en 0?
}
\reponse{
Notons $L : X \rightarrow \Rr$ l'application linéaire définie par
$L(f) = f(0)$. Prenons $f_n$ définie par
$f_n(t) = 2n(1-nt)$ pour $t\in[0,\frac 1n]$ et $f(t)=0$ si $t>\frac 1n$.
Alors $\|f_n\| = 1$ alors que $L(f_n) = 2n$. Donc le rapport
$\frac{|L(f_n)|}{\| f_n \|} = 2n$ n'est pas borné, donc $L$ n'est pas continue.
Si $H = \{ f \mid f(0)=0 \}$ alors $H= \mathrm{Ker}\, L = L^{-1}(0)$. Comme $L$ n'est pas continue alors $H$ n'est pas fermé (voir l'exercice \ref{exoferm}).
}
}
