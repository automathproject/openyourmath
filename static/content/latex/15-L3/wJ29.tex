\uuid{wJ29}
\exo7id{2341}
\auteur{queffelec}
\organisation{exo7}
\datecreate{2003-10-01}
\isIndication{true}
\isCorrection{true}
\chapitre{Espace topologique, espace métrique}
\sousChapitre{Espace topologique, espace métrique}

\contenu{
\texte{
Montrer que tout ouvert de $\Rr$ est union d\'enombrable d'intervalles ouverts
deux \`a deux disjoints. 
(\emph{Indication :} si $x\in O$ ouvert, consid\'erer $J_x$ qui est l'union des
intervalles ouverts inclus dans $O$ et  contenant $x$). \'Enoncer un résultat similaire pour
 les ouverts de $\Rr^n$.
}
\indication{Montrer que $J_x$ est un intervalle ouvert ; que $J_x=J_y$ ou $J_x\cap J_y=\varnothing$.
Et penser que $\Qq$ est dénombrable.}
\reponse{
$J_x$ est un ouvert non vide car c'est une union d'ouverts contenant $x$. De plus $J_x$ est un intervalle car c'est une union d'intervalles contenant tous le point $x$. Donc $J_x$ est un intervalle ouvert. 
On peut donc \'ecrire $\mathcal{O} = \cup_{x\in \mathcal{O}} J_x$. Mais cette union n'est pas n\'ecessairement d\'enombrable.

Tout d'abord si $z\in J_x$ alors $J_x=J_z$. En effet soit $I$ un intervalle inclus dans $\mathcal{O}$ contenant $x$ et $z$. Si $x'\in J_x$, soit $J$ un intervalle inclus dans $\mathcal{O}$ contenant $x$ et $x'$. Alors $I\cup J$ est un intervalle (car $x$ est dans les deux intervalles $I$ et $J$), $I\cup J$ est inclus dans $\mathcal{O}$ et contient $x'$ et $z$. Donc $x'\in J_z$. Donc $J_x \subset J_z$. Enfin comme $z\in J_x$ on a aussi $x\in J_z$, donc on montrerait de même $J_z\subset J_x$. Donc $J_x=J_z$.

Pour $x,y \in \mathcal{O}$ alors $J_x=J_y$ ou $J_x \cap J_y = \varnothing$. En effet supposons que $J_x\cap J_y \neq \varnothing$ et soit $z \in J_x \cap J_y$. Comme $z\in J_x$ alors $J_x=J_z$, comme 
 $z\in J_y$ alors $J_y=J_z$. Donc $J_x=J_y$.

Pour chaque intervalle ouvert $J_x$ il existe $q \in \Qq \cap J_x$, avec bien s\^ur $J_x = J_q$.
 Comme $\Qq$ est d\'enombrable $\mathcal{O}\cap \Qq$ l'est aussi. On a ainsi \'ecrit
$$\mathcal{O} = \bigcup_{q\in \mathcal{O}\cap \Qq} J_{q},$$
ce qui \'etait demand\'e.
Pour $\Rr^n$ on peut montrer le r\'esultat suivant : tout ouvert $\mathcal{O}$ de $\Rr^n$ s'\'ecrit comme l'union d\'enombrable de boules ouverte. On consid\'ere $J_x$ l'union des boules  ouvertes de rayon 
rationnel centr\'ees en $x$,  ensuite on regarde seulement les $x$ appartenant \`a $\mathcal{O} \cap \Qq^n$. Par contrer on autorise deux boules à s'intersecter.
}
}
