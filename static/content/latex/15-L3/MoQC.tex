\uuid{MoQC}
\exo7id{6102}
\auteur{queffelec}
\organisation{exo7}
\datecreate{2011-10-16}
\isIndication{false}
\isCorrection{false}
\chapitre{Espace topologique, espace métrique}
\sousChapitre{Espace topologique, espace métrique}

\contenu{
\texte{
On dit qu'une distance est \emph{ultramétrique} si elle vérifie l'inégalité
triangulaire renforcée :
$$d(x,z)\leq\max(d(x,y),d(y,z)).$$
Etablir les assertions suivantes :
}
\begin{enumerate}
    \item \question{Si $d(x,y)\neq d(y,z)$, alors $d(x,z)=\max(d(x,y),d(y,z)).$ En déduire que
tout triangle dans $E$ est isocèle.}
    \item \question{Toute boule ouverte $B(x,r)$ est un ensemble à la fois ouvert et fermé, et 
$$B(x,r)=B(y,r)\ \ \forall y\in B(x,r).$$}
    \item \question{Toute boule fermée $B'(x,r)$ est un ensemble à la fois ouvert et fermé, et 
$$B'(x,r)=B'(y,r)\ \ \forall y\in B'(x,r).$$}
    \item \question{Si deux boules ont un point commun, elles sont embo\^\i tées.}
\end{enumerate}
}
