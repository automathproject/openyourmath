\uuid{3GrS}
\exo7id{6184}
\auteur{queffelec}
\datecreate{2011-10-16}
\isIndication{false}
\isCorrection{false}
\chapitre{Compacité}
\sousChapitre{Compacité}

\contenu{
\texte{
Soit $f$ une surjection continue de $\Rr^2$ sur $\Rr$. On va montrer que
l'image réciproque de tout point est non bornée. On raisonne par l'absurde : 

 Sinon, il existe $a\in \Rr$ et un disque fermé $ D$ du plan tel que
$f^{-1}(\{a\})\subset  D$; en étudiant $f(D^c)$ et $f(D)$ montrer que
$f(\Rr^2)$ ne peut être égal à $\Rr$ tout entier.
}
}
