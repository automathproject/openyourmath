\uuid{SxwI}
\exo7id{6020}
\auteur{quinio}
\datecreate{2011-05-20}
\isIndication{false}
\isCorrection{true}
\chapitre{Probabilité discrète}
\sousChapitre{Lois de distributions}

\contenu{
\texte{
On effectue un contrôle de fabrication sur des pièces dont une
proportion $p=0.02$ est défectueuse.
}
\begin{enumerate}
    \item \question{On contrôle un lot de 1000 pièces :

Soit $X$ la variable aléatoire: <<nombre de pièces défectueuses
parmi 1000>>.
Quelle est la vraie loi de $X$ ? (on ne donnera que la forme générale);
quel est son espérance, son écart-type ?}
\reponse{La loi de $X$ est la loi binomiale $B(1000;0.02)$, d'espérance 20, 
d'écart-type $\sqrt{19.6}$.}
    \item \question{En approchant cette loi par celle d'une loi normale adaptée, calculez
la probabilité pour que $X$ soit compris entre 18 et 22 ($P[18 \leq X\leq 22]$) ; 
on fera les calculs avec et sans correction de continuité.
On fera également les calculs avec la vraie loi pour comparer.}
\reponse{En approchant cette loi par celle d'une loi normale de paramètre
$m=20$, écart-type $\sqrt{19.6}$.
$P[18\leq X\leq 22]=P[(17.5-20)/\sqrt{19.6}\leq (X-20)/\sqrt{19.6}\leq
(22.5-20)/\sqrt{19.6}]\simeq 0.428$.

Sans correction de continuité on trouve $P[(17-20)/\sqrt{19.6}\leq
(X-20)/\sqrt{19.6}\leq (22-20)/\sqrt{19.6}]\simeq 0.348$.

Approchée par la loi de Poisson de paramètres : espérance 20 et
variance 20, on trouve $P[18\leq X\leq 22]\simeq 0.423$.

Enfin par la vraie loi binomiale: on trouve $P[18\leq X\leq 22]\simeq 0.427$.}
\end{enumerate}
}
