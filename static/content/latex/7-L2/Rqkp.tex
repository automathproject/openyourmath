\uuid{Rqkp}
\exo7id{6024}
\auteur{quinio}
\datecreate{2011-05-20}
\isIndication{false}
\isCorrection{true}
\chapitre{Probabilité discrète}
\sousChapitre{Lois de distributions}

\contenu{
\texte{
On suppose qu'il y a une probabilité égale à $p$ d'être contrôlé lorsqu'on prend le tram. 
Monsieur A fait $n$ voyages par an sur cette ligne.
}
\begin{enumerate}
    \item \question{On suppose que $p=0.10$, $n=700$.
\begin{enumerate}}
\reponse{\begin{enumerate}}
    \item \question{Quelle est la probabilité que Monsieur A soit contrôlé entre
60 et 80 fois dans l'année ?}
\reponse{Pour calculer la probabilité que Monsieur A soit contrôlé
entre 60 et 80 fois dans l'année, posons le nombre de contrôles
comme une variable aléatoire. Elle obéit à une loi binomiale
$B(700;0.1)$.
On peut l'approcher par la loi normale $N(70;\sqrt{63})$.

$P[60\leq X\leq 80]=P[-10/\sqrt{63}\leq X\leq 10/\sqrt{63}]\simeq 2F(10.5/\sqrt{63})-1=0.814$.
La probabilité d'être contrôlé entre 60 et 80 fois dans l'année est $81.4$.}
    \item \question{Monsieur A voyage en fait toujours sans ticket. Afin de
prendre en compte la possibilité de faire plusieurs passages avec le même ticket, on suppose que le prix d'un ticket est de 1,12 euros.
Quelle amende minimale la compagnie doit-elle fixer pour que le fraudeur
ait, sur une période d'une année, une probabilité supérieure 
à $0.75$ d'être perdant ?}
\reponse{Calculons le prix que devrait payer le voyageur: $1,12 \times 700=
784$ euros. Il est perdant si l'amende dépasse ce prix.
Or l'amende est $aX$, si $a$ est l'amende fixée par la compagnie.

On cherche donc $a$ pour que : $P[aX\geq 784]\geq 0.75:$
Soit $P[aX\leq 784]\leq 0.25:$
Par lecture de table:
$a=784/64.642=12.128$
Il faut que l'amende dépasse 13 euros.}
\end{enumerate}
}
