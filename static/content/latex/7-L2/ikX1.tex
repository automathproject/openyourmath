\uuid{ikX1}
\exo7id{6032}
\auteur{quinio}
\organisation{exo7}
\datecreate{2011-05-20}
\isIndication{false}
\isCorrection{true}
\chapitre{Statistique}
\sousChapitre{Tests d'hypothèses, intervalle de confiance}

\contenu{
\texte{
On s'intéresse au problème des algues toxiques qui
atteignent certaines plages de France; après étude on constate que
10\% des plages sont atteintes par ce type d'algues et on veut tester
l'influence de rejets chimiques nouveaux sur l'apparition de ces algues.
Pour cela 50 plages proches de zones de rejet chimiques, sont observées; 
on compte alors le nombre de plages atteintes par l'algue nocive :
on constate que 10 plages sont atteintes par l'algue.
Pouvez-vous répondre à la question 
<<Les rejets chimiques ont-t-il modifié, de façon significative, 
avec le risque $\alpha =0.05$, le nombre de plages atteintes ?>>
}
\reponse{
Posons $H_{0}$ <<les rejets chimiques ne modifient pas le
nombre de plages atteintes par les algues>>.

Notons $p_{0}=0.1$ la proportion théorique de plages atteintes par
l'algue verte avant les rejets chimiques; $p$ la proportion théorique de
plages atteintes par l'algue verte après les rejets chimiques et $f$ la
fréquence observée dans l'échantillon.

Considérons alors la variable aléatoire $X_{i}$, $i\leq 50,$ qui a
deux modalités: $1$ si la plage est atteinte, $0$ sinon. C'est une
variable de Bernoulli, alors le nombre total de plages atteintes 
dans l'échantillon est une variable aléatoire qui, sous $H_{0}$, obéit à
une loi binomiale de paramètres $n=50$, $p_{0}=0.1$.

Sous $H_{0}$, <<$p=p_{0}=0.1$>>  la variable 
<<moyenne d'échantillon>> : 
\begin{equation*}
\overline{X}=\frac{\sum_{i=1}^{i=50}X_{i}}{n}
\end{equation*}
dont une réalisation est la fréquence observée, soit $\frac{10}{50}$, 
obéit à une loi que l'on peut approcher par une loi normale
de paramètres : moyenne $p_{0}$ et écart-type $\sqrt{\frac{p_{0}(1-p_{0})}{50}}$.

A l'aide de la formule de cours, on détermine l'intervalle de confiance
associé:
$I\simeq [0.017;0.183]$. On constate que la fréquence observée est dans la zone de rejet
(non chimique) : $0.2$ n'est pas dans
l'intervalle de confiance au seuil $95$\%.
On peut donc rejeter $H_{0}$ et conclure, au risque $0.05$, que les rejets
chimiques modifient de façon significative le nombre de plages atteintes par l'algue.
}
}
