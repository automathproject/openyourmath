\uuid{oNOP}
\exo7id{5995}
\auteur{quinio}
\datecreate{2011-05-20}
\isIndication{false}
\isCorrection{true}
\chapitre{Probabilité discrète}
\sousChapitre{Probabilité conditionnelle}

\contenu{
\texte{
Un professeur oublie fréquemment ses clés. Pour tout $n$, on note :
$E_n$ l'événement <<le jour $n$, le professeur oublie ses clés>>, 
$P_{n}=P(E_n)$, $Q_n=P(\overline{E_n})$.

On suppose que : $P_{1}=a$ est donné et que si le jour $n$ il oublie ses clés, 
le jour suivant il les oublie avec la probabilité $\frac{1}{10}$ ; 
si le jour $n$ il n'oublie pas ses clés, le jour suivant il les oublie
avec la probabilité $\frac{4}{10}$.

Montrer que $P_{n+1}=\frac{1}{10}P_{n}+\frac{4}{10}Q_{n}$.
En déduire une relation entre $P_{n+1}$ et $P_{n}$

Quelle est la probabilité de l'événement <<le jour $n$, le professeur oublie ses clés>> ?
}
\reponse{
$P_{n+1}=P(E_{n+1})=
P(E_{n+1}/E_n)P(E_n)+P(E_{n+1}/\overline{E_n})P(\overline{E_n})=\frac{1}{10}P_{n}+\frac{4}{10}Q_{n}$.
Donc
$P_{n+1}=\frac{1}{10}P_{n}+\frac{4}{10}(1-P_{n})=\frac{4}{10}-\frac{3}{10}P_{n}$.

La suite ($P_{n}-\ell)$ est géométrique, où $\ell$ est solution
de $\frac{4}{10}-\frac{3}{10}\ell=\ell$ soit $\ell=\frac{4}{13}$.
Donc $P_{n}=\frac{4}{13}+(-\frac{3}{10})^{n-1} \times (a-\frac{4}{13})$.
}
}
