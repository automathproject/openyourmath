\uuid{bfLr}
\exo7id{5986}
\auteur{quinio}
\organisation{exo7}
\datecreate{2011-05-18}
\isIndication{false}
\isCorrection{true}
\chapitre{Probabilité discrète}
\sousChapitre{Probabilité et dénombrement}

\contenu{
\texte{
Un QCM comporte $10$ questions, pour chacune desquelles $4$ réponses sont proposées, une seule est exacte. Combien y-a-t-il de
grilles-réponses possibles? Quelle est la probabilité de répondre au hasard au moins $6$ fois correctement?
}
\reponse{
Une grille-réponses est une suite ordonnée de $10$ réponses,
il y a $4$ choix possibles pour chacune. Il y a donc $4^{10}$ grilles-réponses possibles.
L'événement $E$ <<répondre au hasard au moins $6$ fois correctement>> est 
réalisé si le candidat répond bien à $6$ ou $7$ ou $8$ ou $9$ ou $10$ questions. Notons $A_{n}$ l'événement : <<répondre au hasard
exactement $n$ fois correctement>>. Alors, $A_{n}$ est réalisé si $n$ réponses sont correctes et $10-n$
sont incorrectes : $3$ choix sont possibles pour chacune de ces dernières.
Comme il y a $\binom{10}{n}$ choix de $n$ objets parmi $10$, et donc il y a :
$\binom{10}{n} \times 3^{10-n}$ façons de réaliser $A_{n}$ et :
\begin{equation*}
P(A_n)=\frac{\binom{10}{n}\cdot 3^{10-n}}{4^{10}}
\end{equation*}
pour $n= 6, 7, 8, 9, 10$.
$P(E)=\sum_{n=6}^{10}\frac{\binom{10}{n}\cdot 3^{10-n}}{4^{10}} \simeq 1.9728\times 10^{-2}$, soit environ $2$\%.
}
}
