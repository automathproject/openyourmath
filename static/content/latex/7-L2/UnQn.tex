\uuid{UnQn}
\exo7id{6929}
\auteur{ruette}
\datecreate{2013-01-24}
\isIndication{false}
\isCorrection{true}
\chapitre{Statistique}
\sousChapitre{Tests d'hypothèses, intervalle de confiance}

\contenu{
\texte{
On mesure une certaine grandeur physique $G$ avec un appareil dont la
précision est caractérisée par l'écart-type $\sigma$. On fait l'hypothèse que 
les mesures suivent une loi normale.
}
\begin{enumerate}
    \item \question{On effectue une seule mesure, on trouve $g_{1} = 1,364$.
On suppose connue la précision de l'appareil de mesure : 
$\sigma=4,3.10^{-3}$ (unité arbitraire).
%Donner la meilleure estimation possible de $G$.
Déterminer un intervalle de confiance contenant, avec une probabilité de 
90\%, la valeur $G$.}
\reponse{La meilleure estimation de $G$ est la valeur moyenne mesurée, $g_1 =1,364$. 
Pour donner un intervalle de confiance, on fait l'hypothèse que les mesures 
suivent une loi normale d'espérance $G$ et d'écart-type $\sigma$. 
La probabilité que l'intervalle $[g_1 -1,645\sigma_{pop},g_1 +1,645\sigma_{pop}]$ 
ne contienne pas $G_v$ est inférieure à $0,1$. On conclut que $[1,357;1,371]$ 
est un intervalle de confiance relatif à $G$ au seuil de 90\%.}
    \item \question{On ignore la précision de l'appareil de mesure.
On effectue 5 mesures. On trouve : 

\begin{center}
\begin{tabular}{|c|c|c|c|c|}
\hline
1,365&1,371&1,368&1,359&1,362\\\hline
%1.366	&1.365&1.367&1.363&1.364\\\hline
\end{tabular}
\end{center}
 
Donner des estimations  de $G$ et de $\sigma$.
Déterminer un intervalle de confiance contenant, avec une probabilité de 
90\%, la valeur $G$.}
\reponse{La meilleure estimation de $G$ est la valeur moyenne mesurée, $\bar{g}=\frac{1}{10}(g_1 +\cdots+g_{5})=1,365$. Soit 
$\bar\sigma^2=\frac{1}{4}\sum_{i=1}^5(g_i-\bar g)^2\simeq 2,25 .10^{-5}$.
La meilleure estimation de $\sigma$ est
$\bar\sigma\simeq 4,7 .10^{-3}$.

Comme $\frac{1}{5}(X_1 +\cdots+X_{5})$ suit une loi normale d'espérance $G$ 
et de variance $\frac{1}{25}(\text{Var}(X_1 )+\cdots+\text{Var}(X_{5}))=\frac{1}{5}\sigma^{2}$, 
la probabilité que l'intervalle 
$[\bar{g} -1,645\frac{\sigma}{\sqrt{5}},\bar{g} +1,645\frac{\sigma}{\sqrt{5}}]$ 
ne contienne par $G$ est inférieure à $0,1$. 
Si on estime $\sigma$ par $\bar\sigma$, on trouve que $[1,363;1,367]$ est un 
intervalle de confiance relatif à $G$ au seuil de 90\%.}
\end{enumerate}
}
