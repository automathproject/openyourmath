\uuid{t5SS}
\exo7id{5991}
\auteur{quinio}
\datecreate{2011-05-18}
\isIndication{false}
\isCorrection{true}
\chapitre{Probabilité discrète}
\sousChapitre{Probabilité et dénombrement}

\contenu{
\texte{
La famille Potter comporte $2$ enfants; les événements $A$ : 
<<il y a deux enfants de sexes différents chez les Potter>> 
et $B$ : <<la famille Potter a au
plus une fille>> sont-ils indépendants? 
Même question si la famille Potter comporte $3$ enfants. Généraliser.
}
\reponse{
Notons, pour le cas où la famille Potter comporte $2$
enfants, l'univers des possibles pour les enfants :
$\Omega =\{(G,G),(G,F),(F,G),(F,F)\}$, représente les cas possibles, 
équiprobables, d'avoir garçon-garçon, garçon-fille etc... :
Alors $P(A)=\frac{2}{4},$ $P(B)=\frac{3}{4}, P(A\cap B) = \frac{2}{4}$.
On en conclut que : $P(A\cap B)\neq P(A)P(B)$ et donc que les événements $A$ et $B$ ne sont pas indépendants.

Si maintenant la famille Potter comporte $3$ enfants :
Alors $\Omega' = \{ (a,b,c) \mid a \in \{ G,F \}, b\in \{G,F\}, c\in \{G,F\}\}$ 
représente les $2^{3}=8$ cas possibles, équiprobables.
Cette fois, $P(A)=1-P(\{(G,G,G),(F,F,F)\})=\frac{6}{8}$ ;
$P(B)=\frac{4}{8}, P(A\cap B)=P\{(F,G,G),(G,F,G),\{(G,G,F)\}=\frac{3}{8}$.
On a $P(A)P(B)=\frac{3}{8}=P(A\cap B),$ et les événements $A$ et $B$
sont indépendants

Avec $n$ enfants, on peut généraliser sans difficulté : 
$P(A)=1-\frac{2}{2^{n}},$ $P(B)=\frac{1+n}{2^{n}}$
$P(A\cap B)=\frac{n}{2^{n}}$
Un petit calcul montre que
$P(A)P(B)=P(A\cap B)$ si et seulement si $n=3$.
}
}
