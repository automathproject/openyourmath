\uuid{4iSJ}
\titre{ Qualité d'un algorithme }
\theme{ statistiques }
\auteur{}
\organisation{AMSCC}
\contenu{

\texte{ On souhaite quantifier la qualité d'un algorithme de classification d'images. %Dans une première partie, on s'intéresse aux erreurs de classification. Dans une seconde partie, 
%On voudrait quantifier la qualité de cet algorithme.

%\paragraph{Partie 1}
%On dispose d'un échantillon de 100 images type qui peuvent être classées en 2 catégories. On observe que l'algorithme a classé 84 images dans la bonne catégorie. 
%\begin{enumerate}
%	\item Donner une estimation par intervalle de confiance au niveau 95\% du taux de mauvaise classification de cet algorithme. 
%	\item Ce taux est-il significativement différent de $10\%$ ?
%\end{enumerate}

L'algorithme ne retourne pas une catégorie à partir d'une image $i$, mais seulement une grandeur $G_i$ qui permet de déterminer la classe la plus probable. On note $X_i$ l'erreur commise par l'algorithme pour la valeur $G_i$. 

On suppose que $X_i$ suit une loi normale centrée de variance $\sigma^2$. }
\begin{enumerate}
	\item \texte{ Sur un premier échantillon de 25 images, on obtient une erreur moyenne de $0{,}01$ calculée avec l'estimateur de moyenne empirique et une variance de 0{,}25 calculée avec l'estimateur $\tilde{S}^2=\frac{1}{n}\sum_{i=1}^n (X_i-\overline{X})^2$. }
	\begin{enumerate}
		\item \question{ Donner une estimation non biaisée de $\sigma^2$. }
		\item \question{ Avec un risque d'erreur de $10\%$, peut-on remettre en cause l'hypothèse que l'espérance de $X_i$ est nulle ? }
	\end{enumerate}
	\item \texte{ Pour étudier la stabilité de l'algorithme par rapport au bruitage des images, on étudie un second échantillon de $25$ images bruitées. On obtient avec les mêmes estimateurs que précédemment une erreur empirique de $0{,}012$ et une variance de $0{,}50$.  }
	\begin{enumerate}
		\item \question{ Donner une estimation non biaisée $\sigma^2$ à partir de cet échantillon. }
		\item \question{ Déterminer la loi de la variable $\frac{n}{\sigma^2}\tilde{S}^2$ puis de $\frac{n-1}{\sigma^2}{S}^2$ où $S^2$ est l'estimateur de variance non biaisé pour un échantillon de taille $n$.  }
		\item \question{ Déterminer la loi de la variable $\frac{S_2^2}{S_1^2}$ où $S_i^2$ est l'estimateur de variance non biaisé de l'échantillon $i \in \{1;2\}$. }
		\item \question{ En considérant le quotient de la plus grande variance par la plus petite, construire et mettre en oeuvre un test d'hypothèse permettant de dire si la variance obtenue dans l'échantillon 2 et significativement plus élevée que celle de l'échantillon 1. } %Pour information, un calcul sur tableur permet de dire que si une variable $F$ suit une loi de Fisher de paramètres $(24,24)$ alors $\PP(F<1{,}98) \approx 0{,}9495$. 
	\end{enumerate}
\end{enumerate}
}