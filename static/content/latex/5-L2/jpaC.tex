\uuid{jpaC}
\exo7id{4654}
\auteur{quercia}
\datecreate{2010-03-14}
\isIndication{false}
\isCorrection{true}
\chapitre{Série de Fourier}
\sousChapitre{Convergence, théorème de Dirichlet}

\contenu{
\texte{
Soit $R$ une fraction rationnelle à coefficients complexes, de degré strictement
négatif, n'ayant pas de pôle dans $\Z$.
On pose $f(x) = \sum_{n=-\infty}^\infty R(n)e^{inx}$.
}
\begin{enumerate}
    \item \question{\'Etudier l'existence et la continuité de $f$.}
\reponse{$R(n) = \frac an + S(n)$ avec $\deg S \le -2$.
             Donc $f(x) = R(0) + 2ia\sum_{n=1}^\infty \frac{\sin nx}n
                        + \sum_{n=1}^\infty (S(n)e^{inx} + S(-n)e^{-inx})$
             converge pour tout $x \in \R$.}
    \item \question{Montrer que $f$ est de classe $\mathcal{C}^\infty$ sur $\R\setminus 2\pi\Z$.}
\reponse{$R(n) = \frac {a_1}n + \dots + \frac{a_k}{n^k} + S(n)$
             avec $\deg S \le -k-1  \Rightarrow 
             f(x) = R(0) + a_1f_1(x) + \dots + a_kf_k(x) + g(x)$
             avec $f_p(x) = \sum_{n=1}^\infty
             \frac{e^{inx}+(-1)^pe^{-inx}}{n^p}$
             et $g$ de classe $\mathcal{C}^k$ sur $\R\setminus2\pi\Z$.
             $f_p' = if_{p-1}$ et $f_1$ est $\mathcal{C}^\infty$ sur $\R\setminus2\pi\Z$
             donc $f_p$ aussi.}
\end{enumerate}
}
