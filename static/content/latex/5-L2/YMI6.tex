\uuid{YMI6}
\exo7id{4104}
\auteur{quercia}
\datecreate{2010-03-11}
\isIndication{false}
\isCorrection{true}
\chapitre{Equation différentielle}
\sousChapitre{Equations différentielles linéaires}

\contenu{
\texte{
Soit $I=[a,b]\subset \R$. On suppose que la fonction $\Delta$ est strictement
positive sur $I$.

On pose $E=\{f\in \mathcal{C}^2(I) \, |\, f(a)=f(b)=0\}$. On
considère enfin l'opérateur $K:f\mapsto \frac{f''}{\Delta}\cdotp$
}
\begin{enumerate}
    \item \question{Montrer que Sp$(K)\subset {]-\infty,0[}$.}
    \item \question{Trouver un produit scalaire $(\ \mid\ )$ pour lequel deux vecteurs propres associés à des
    valeurs propres distinctes sont orthogonaux.}
    \item \question{On suppose que $I=\R^+$ et que $\Delta (x)\ge 1$ pour $x\ge 2$. Soit $\lambda <0$.
  \begin{enumerate}}
    \item \question{Montrer qu'il existe une unique $f_{\lambda}\in \mathcal{C}^1(\R^+,\R)$ telle 
    $\begin{cases}f''_{\lambda}=\lambda \Delta f_\lambda\cr f_{\lambda}(0)=0\cr f'_{\lambda}(0)=1.\cr\end{cases}$}
    \item \question{Montrer $f_{\lambda}$ a une infinité dénombrable de zéros $(x_0<x_1<\dots <x_n<\dots)$ et 
    que la suite $(x_n)$ tend vers~$+\infty$.}
\reponse{
Soit $f$ non identiquement nulle vérifiant $f''=\lambda\Delta f$ avec $\lambda > 0$~:
    sur tout intervalle où $f$ est strictement positive, $f$ est strictement convexe
    donc ne peut pas s'annuler aux deux bords~; idem quand $f$ est strictement
    négative, il y a contradiction. Le cas $\lambda=0$ est trivial.
$(f\mid g) =  \int_{t=a}^bf'(t)g'(t)\,d t = - \int_{t=a}^bf''(t)g(t)\,d t  = - \int_{t=a}^bf(t)g''(t)\,d t$.
\begin{enumerate}
Si $f_\lambda$ a un nombre fini de zéros, soit $x_n$ le
    dernier et $A=\max(x_n,2)$. Sur $[A,+\infty[$, $f$ est de signe constant, $\varepsilon$,
    et on a $f_\lambda''-\lambda f_\lambda = \lambda(\Delta-1)f_\lambda = \varphi$
    d'où $f_\lambda(x) =  \int_{t=A}^x\sin((x-t)\sqrt{-\lambda}\,)\varphi(t)\,d t + \alpha\cos(x\sqrt{-\lambda}\,) + \beta\sin(x\sqrt{-\lambda}\,)$.
    En particulier $f_\lambda(A)+f_\lambda\Bigl(A+\frac{\pi}{\sqrt{-\lambda}}\Bigr)
    =  \int_{t=A}^{A+\pi/\sqrt{-\lambda}}\sin((x-t)\sqrt{-\lambda}\,)\varphi(t)\,d t$ est du
    signe de $-\varepsilon$, absurde.
    
    Si l'ensemble des zéros de~$f$ admet un point d'accumulation~$x$ on
    a $f_\lambda(x) = f_\lambda'(x) = 0$ d'où $f_\lambda=0$, absurde.
}
\end{enumerate}
}
