\uuid{MLma}
\exo7id{855}
\auteur{gourio}
\datecreate{2001-09-01}
\isIndication{false}
\isCorrection{false}
\chapitre{Equation différentielle}
\sousChapitre{Résolution d'équation différentielle du premier ordre}

\contenu{
\texte{
Soit l'\'{e}quation diff\'{e}rentielle :
$$(E):\frac{dy(x)}{dx}+y(x)=x^{2}+2x $$
Int\'{e}grer $(E)$ et montrer que par un point donn\'{e} il passe une et une
seule courbe int\'{e}grale. Soit $H  $ l'ensemble des points $M  $ tels
que la courbe int\'{e}grale passant par $M$ a une tangente horizontale en ce
point, et $I   $ l'ensemble des points $M$ tels que la courbe
int\'{e}grale passant par ce point a un point d'inflexion en ce point.\
Tracer $H, I $ et la courbe int\'{e}grale passant par $O(0,0). $ En
d\'{e}duire un trac\'{e} g\'{e}om\'{e}trique des courbes int\'{e}grales.
}
}
