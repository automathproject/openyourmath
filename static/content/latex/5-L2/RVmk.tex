\uuid{RVmk}
\exo7id{4622}
\auteur{quercia}
\datecreate{2010-03-14}
\isIndication{false}
\isCorrection{true}
\chapitre{Série de Fourier}
\sousChapitre{Calcul de coefficients}

\contenu{
\texte{
Calculer le développement des fonctions $f$ $2\pi$-périodiques telles que :
}
\begin{enumerate}
    \item \question{$f(x) = \pi-|x|$ sur $]-\pi,\pi[$.}
\reponse{$a_0 = \pi$, $a_{2p} = 0$, $a_{2p+1} = \frac4{\pi(2p+1)^2}$,
             $b_n = 0$.}
    \item \question{$f(x) = \pi-x$ sur $]0,2\pi[$.}
\reponse{$a_n = 0$, $b_n = \frac2n$.}
    \item \question{$f(x) = x^2$ sur $]0,2\pi[$.}
\reponse{$a_0 = \frac{8\pi^2}3$, $a_n = \frac4{n^2}$,
             $b_n = -\frac{4\pi}n$.}
    \item \question{$f(x) = \max(0,\sin x)$.}
\reponse{$a_0 = \frac 2\pi$, $a_{2p} = \frac{-2}{\pi(4p^2-1)}$,
             $a_{2p+1} = 0$, $b_1 = \frac 12$, $b_p = 0$.}
    \item \question{$f(x) = |\sin x|^3$.}
\reponse{$a_{2p} = \frac{24}{\pi(4p^2-1)(4p^2-9)}$, $a_{2p+1} = 0$,
             $b_p = 0$.}
\end{enumerate}
}
