\uuid{wLse}
\exo7id{5868}
\auteur{rouget}
\organisation{exo7}
\datecreate{2010-10-16}
\isIndication{false}
\isCorrection{true}
\chapitre{Suite et série de fonctions}
\sousChapitre{Suite et série de matrices}

\contenu{
\texte{
Soit $A\in\mathcal{M}_n(\Cc)$. Montrer qu'il existe $p_0\in\Nn$ tel que $\forall p\geqslant p_0$, $\sum_{k=0}^{p} \frac{A^k}{k!}\in GL_n(\Rr)$.
}
\reponse{
Soit $A\in\mathcal{M}_n(\Cc)$. On sait que d'une part $\text{det}(\text{exp}(A))\neq0$ et d'autre part $\text{exp}(A)=\lim_{p \rightarrow +\infty}\left(\sum_{k=0}^{p} \frac{A^k}{k!}\right)$. Par continuité du déterminant, on a donc $\lim_{p \rightarrow +\infty}\text{det}\left(\sum_{k=0}^{p} \frac{A^k}{k!}\right)=\text{det}(\text{exp}(A))\neq0$. Par suite, il existe $p_0\in\Nn$ tel que $\forall p\geqslant p_0$, $\text{det}\left(\sum_{k=0}^{p} \frac{A^k}{k!}\right)\neq0$ et donc tel que $\sum_{k=0}^{p} \frac{A^k}{k!}\in GL_n(\Rr)$.
}
}
