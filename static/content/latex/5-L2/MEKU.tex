\uuid{MEKU}
\exo7id{4764}
\auteur{quercia}
\organisation{exo7}
\datecreate{2010-03-16}
\isIndication{false}
\isCorrection{true}
\chapitre{Topologie}
\sousChapitre{Topologie des espaces vectoriels normés}

\contenu{
\texte{
Soit $p$ une semi-norme sur $\mathcal{M}_n(\C)$ (ie. il manque juste l'axiome
$p(A)=0 \Rightarrow  A=0$). On suppose de plus que $\forall\ (A,B)\in(\mathcal{M}_n(\C))^2,\
p(AB)\le p(A)p(B)$. Montrer que $p=0$ ou $p$ est en fait une norme.
}
\reponse{
Si $A$ est une matrice de rang $r>0$ telle que $p(A) = 0$
alors pour toute matrice $M$ de rang $<r$ on peut trouver $P$ et $Q$
telles que $M=PAQ$ d'o{\`u} $P(M)=0$. Donc $p$ est nulle sur toute matrice
de rang~$1$ et par in{\'e}galit{\'e} triangulaire sur tout matrice.
}
}
