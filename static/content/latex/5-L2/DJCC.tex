\uuid{DJCC}
\exo7id{4170}
\auteur{quercia}
\organisation{exo7}
\datecreate{2010-03-11}
\isIndication{false}
\isCorrection{true}
\chapitre{Fonction de plusieurs variables}
\sousChapitre{Dérivée partielle}

\contenu{
\texte{
On considère la fonction de $\R^2$ sur lui-même définie par
$f(x,y)=(u,v)$, où
$$u(x,y)=x \sqrt{1+y^2} + y \sqrt{1+x^2}\quad \hbox{et}\quad v(x,y)=(x +
\sqrt{1+x^2})(y+\sqrt{1+y^2}).$$

Calculer sa matrice jacobienne. Est-elle inversible localement ?
Caractériser $f (\R^2)$.
}
\reponse{
$x=\sh(a),\ y=\sh(b)  \Rightarrow  u=\sh(a+b),\ v=\sh(a+b)+\ch(a+b)$.
donc $f(\R^2)$ est inclus dans l'hyperbole d'équation $v^2-2uv=1$
et on doit avoir $v\ge u$ ce qui donne la branche supérieure.
}
}
