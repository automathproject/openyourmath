\uuid{JIgi}
\exo7id{4600}
\auteur{quercia}
\organisation{exo7}
\datecreate{2010-03-14}
\isIndication{false}
\isCorrection{true}
\chapitre{Série entière}
\sousChapitre{Equations différentielles}

\contenu{
\texte{
On pose $f(x) = \sum_{n=0}^\infty \frac{x^n}{\strut C_{2n}^n}$.
}
\begin{enumerate}
    \item \question{Déterminer le rayon de convergence et montrer que $f$ vérifie l'équation :
    $x(4-x)y' - (x+2)y = -2$.}
\reponse{$R=4$.}
    \item \question{Résoudre l'équation précédente pour $x > 0$
    (utiliser le DL de $f$ en 0 à l'ordre 1 pour fixer la constante)
    et en déduire la somme de la série
    $\sum_{n=0}^\infty \frac 1{\strut C_{2n}^n}$.}
\reponse{$y = 4\sqrt{\frac x{(4-x)^3}}\left(\sqrt{\frac{4-x}x}
                  - \Arctan\sqrt{\frac{4-x}x} + c\right)$.
             $f(x) = 1 + \frac x2 + \text{o}(x)  \Rightarrow  c = \frac\pi2$.
             $ \Rightarrow \sum_{n=0}^\infty \frac 1{\strut C_{2n}^n}
              = \frac 43 + \frac{2\pi}{9\sqrt 3}$.}
\end{enumerate}
}
