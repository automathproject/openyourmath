\uuid{Baj9}
\exo7id{4210}
\auteur{quercia}
\organisation{exo7}
\datecreate{2010-03-11}
\isIndication{false}
\isCorrection{true}
\chapitre{Equation différentielle}
\sousChapitre{Equations aux dérivées partielles}

\contenu{
\texte{
Soient $a,b,c \in \R$ non tous nuls. On considère l'équation aux dérivées
partielles : $$(*) \Leftrightarrow a\frac{\partial^2 f}{\partial x^2} + b\frac{\partial^2 f}{\partial x \partial y} + c\frac{\partial^2 f}{\partial y^2} = 0$$ où $f$ est une
fonction inconnue : ${\R^2} \to \R$ de classe $\mathcal{C}^2$.
Soient $\alpha,\beta\in \R$ distincts, fixés. On fait le changement de
variable : $u = x+\alpha y$, $v = x+\beta y$.
}
\begin{enumerate}
    \item \question{\'Ecrire l'équation déduite de $(*)$ par ce changement de variable.}
    \item \question{En déduire que l'on peut ramener $(*)$ à l'une des trois formes réduites :\par
    $(1) : \frac{\partial^2 g}{\partial u \partial v}  = 0$,  \qquad
    $(2) : \frac{\partial^2 g}{\partial u^2}  = 0$,  \qquad
    $(3) : \frac{\partial^2 g}{\partial u^2} + \frac{\partial^2 g}{\partial v^2} = 0$.}
\reponse{
$  (a + b\alpha + c\alpha^2)\frac{\partial^2 g}{\partial u^2}
              + (2a + b(\alpha+\beta) + 2c\alpha\beta)\frac{\partial^2 g}{\partial u \partial v}
              + (a + b\beta + c\beta^2)\frac{\partial^2 g}{\partial v^2} = 0$.
}
\end{enumerate}
}
