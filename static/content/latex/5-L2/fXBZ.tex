\uuid{fXBZ}
\exo7id{4361}
\auteur{quercia}
\organisation{exo7}
\datecreate{2010-03-12}
\isIndication{false}
\isCorrection{true}
\chapitre{Intégration}
\sousChapitre{Intégrale de Riemann dépendant d'un paramètre}

\contenu{
\texte{
Domaine de définition de~$I(\alpha) =  \int_{x=0}^{+\infty}\frac{x\ln x}{(1+x^2)^\alpha}\,d x$.
Calculer $I(2)$ et~$I(3)$.
Déterminer la limite de $I(\alpha)$ en~$+\infty$.
}
\reponse{
$I(\alpha)$ est définie pour tout~$\alpha>1$.
\smallskip
$I(2) = (x=e^u) =  \int_{u=-\infty}^{+\infty} \frac{ue^{2u}}{(1+e^{2u})^2}\,d u
      =  \int_{u=-\infty}^{+\infty} \frac{u}{(e^u+e^{-u})^2}\,d u
      = 0$ (parité).
\smallskip
$I(3) =  \int_{u=-\infty}^{+\infty} \frac{ue^{-u}}{(e^u+e^{-u})^3}\,d u
      =  \int_{u=0}^{+\infty} \frac{-u(e^u-e^{-u})}{(e^u+e^{-u})^3}\,d u
      = \Bigl[\frac{u}{2(e^u+e^{-u})^2}\Bigr]_{u=0}^{+\infty}
      -  \int_{u=0}^{+\infty} \frac{d u}{2(e^u+e^{-u})^2}$
\smallskip
$\phantom{I(3)}
      = - \int_{u=0}^{+\infty} \frac{e^{2u}\,d u}{2(1+e^{2u})^2}
      = \Bigl[\frac{1}{4(1+e^{2u})}\Bigr]_{u=0}^{+\infty}
      = -\frac18$.
\smallskip
$I(\alpha)\to 0$ (lorsque $\alpha\to+\infty$) par convergence dominée.
}
}
