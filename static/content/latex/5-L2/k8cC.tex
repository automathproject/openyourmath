\uuid{k8cC}
\exo7id{5866}
\auteur{rouget}
\datecreate{2010-10-16}
\isIndication{false}
\isCorrection{true}
\chapitre{Suite et série de fonctions}
\sousChapitre{Suite et série de matrices}

\contenu{
\texte{
Soit $A=\left(
\begin{array}{cc}
4/3&-5/6\\
5/3&-7/6
\end{array}
\right)$. Convergence et somme de la série de terme général $A^n$, $n\in\Nn$.
}
\reponse{
$\chi_A=\left|
\begin{array}{cc}
4/3-X&-5/6\\
5/3&-7/6-X
\end{array}
\right|=X^2- \frac{1}{6}X- \frac{1}{6}=\left(X- \frac{1}{2}\right)\left(X+ \frac{1}{3}\right)$. Par suite, $A=PDP^{-1}$ où $D=\text{diag}\left( \frac{1}{2},- \frac{1}{3}\right)$, $P=\left(
\begin{array}{cc}
1&1\\
1&2
\end{array}
\right)$ et donc $P^{-1}=\left(
\begin{array}{cc}
2&-1\\
-1&1
\end{array}
\right)$.

Soit $n\in\Nn$.

\begin{center}
$\sum_{k=0}^{n}A^k=P\left(\sum_{k=0}^{n}D^k\right)P^{-1}=P\;\text{diag}\left(\sum_{k=0}^{n}\left( \frac{1}{2}\right)^k,\sum_{k=0}^{n}\left(- \frac{1}{3}\right)^k\right)P^{-1}$.
\end{center}

Puisque $ \frac{1}{2}$ et $- \frac{1}{3}$ sont dans $]-1,1[$, les séries numériques de termes généraux respectifs $\left( \frac{1}{2}\right)^k$ et $\left(- \frac{1}{3}\right)^k$ convergent. Il en est de même de la série de terme général $D^k$. Maintenant, l'application $M\mapsto PMP^{-1}$, converge est continue car linéaire sur un espace de dimension finie et on en déduit que la série de terme général $A^k$ converge. De plus,

\begin{align*}\ensuremath
\sum_{n=0}^{+\infty}A^n&=\sum_{n=0}^{+\infty}PD^nP^{-1}=P\left(\sum_{n=0}^{+\infty}D^n\right)P^{-1}\;(\text{par continuité de l'application}\;M\mapsto PMP^{-1})\\
  &=P\;\text{diag}\left(\sum_{n=0}^{+\infty}\left( \frac{1}{2}\right)^n,\sum_{n=0}^{+\infty}\left(- \frac{1}{3}\right)^n\right)P^{-1}=P\;\text{diag}\left( \frac{1}{1- \frac{1}{2}}, \frac{1}{1+ \frac{1}{3}}\right)P^{-1}\\
  &=\left(
\begin{array}{cc}
1&1\\
1&2
\end{array}
\right)\left(
\begin{array}{cc}
2&0\\
0& \frac{3}{4}
\end{array}
\right)\left(
\begin{array}{cc}
2&-1\\
-1&1
\end{array}
\right)=\left(
\begin{array}{cc}
\rule[-4mm]{0mm}{0mm}2& \frac{3}{4}\\
2& \frac{3}{2}
\end{array}
\right)\left(
\begin{array}{cc}
2&-1\\
-1&1
\end{array}
\right)=\left(
\begin{array}{cc}
\rule[-4mm]{0mm}{0mm} \frac{13}{4}&- \frac{5}{4}\\
 \frac{5}{2}&- \frac{1}{2}
\end{array}
\right).
\end{align*}

\begin{center}
\shadowbox{
$\sum_{n=0}^{+\infty}A^n=\left(
\begin{array}{cc}
\rule[-4mm]{0mm}{0mm} \frac{13}{4}&- \frac{5}{4}\\
 \frac{5}{2}&- \frac{1}{2}
\end{array}
\right)$.
}
\end{center}

\textbf{Remarque.} D'après l'exercice suivant, la matrice obtenue est $(I-A)^{-1}$.
}
}
