\uuid{Pmj5}
\exo7id{2616}
\auteur{debievre}
\datecreate{2009-05-19}
\isIndication{true}
\isCorrection{true}
\chapitre{Topologie}
\sousChapitre{Ouvert, fermé, intérieur, adhérence}

\contenu{
\texte{

}
\begin{enumerate}
    \item \question{Tracer le graphe de la fonction 
$f\colon \R^2 \longrightarrow\R$ d\'efinie par
$f(x,y)= x^2+y^2$ et tracer les lignes de niveau de cette fonction.}
\reponse{Le graphe est bien un parabolo\" \i de de r\'evolution
ayant l'origine pour sommet, d'axe de r\'evolution l'axe des $z$,
et dont la concavit\'e tourne vers le haut.
Les lignes de niveau sont les cercles
$x ^2+y^2=z_0,\ z_0 = c$, $c>0$ \'etant une constante;
pour $c=0$ c'est le somment, c.a.d. l'origine.}
    \item \question{Tracer les graphes des fonctions $f$ et $g$ d\'efinies 
par $f(x,y)=25-(x^2+y^2)$  et $g(x,y)=5-\sqrt{x^2+y^2}$ sur 
$D=\{(x,y)\in\R^2\mid x^2+y^2\leq 25\}$.}
\reponse{Le graphe de la fonction $f$ est  un parabolo\" \i de de r\'evolution
ayant le point $(0,0,25)$  pour sommet et plafonn\'e par
le plan des $x$ et $y$,
d'axe de r\'evolution l'axe des $z$, et
dont la concavit\'e tourne vers le bas. 
Les lignes des niveau sont les cercles
$x^2+y^2=25-z_0,\ z_0 = c$, $c<25$ \'etant une constante
qui d\'eg\'en\`erent en un point, le sommet,
pour $c=25$.

Le graphe de la fonction $g$ est un demi-c\^one
de r\'evolution
ayant le point $(0,0,5)$  pour sommet et plafonn\'e par
le plan des $x$ et $y$,
d'axe de r\'evolution l'axe des $z$, et
dont la concavit\'e tourne vers le bas. 
Les lignes des niveau sont les cercles
$x^2+y^2=(5-z_0)^2$, $z_0=c$ \'etant une constante
telle que $0 \leq c \leq 5$
qui d\'eg\'en\`erent en un point, le sommet,
pour $c=5$.}
    \item \question{Tracer le graphe de la courbe  param\'etr\'ee
$f\colon \R\longrightarrow \R^2$ d\'efinie par
$f(x)= (x\cos x,  x\sin x)$.}
\reponse{Dans $\R^3$ avec coordonn\'ees $(x,y,z)$,
avec $f(x)=(y,z)$,
le graphe en discussion est une h\'elice sur le c\^one de r\'evolution
$y2+z^2=x^2$.}
    \item \question{Peut-on repr\'esenter graphiquement 
%et  de fa\c con  diff\'erente 
l'application de la question (3.)? Comment?}
\reponse{Le support de cette courbe param\'etr\'ee
est une spirale planaire qui rencontre l'origine 
et dont la pente \`a l'origine vaut z\'ero.}
    \item \question{D\'ecrire les surfaces de niveau de la fonction 
$f\colon \R^3\longrightarrow  \R$ d\'efinie par
$f(x,y,z)= \exp (x+y^2-z^2)$.}
\reponse{Pour que $f(x,y,z)= \exp (x+y^2-z^2)$ soit constant 
il faut et il suffit que $x+y^2-z^2$ soit constant. Les surfaces de niveau
en discussion sont donc les surfaces
$x+y^2-z^2=c$. Ce sont des parabolo\"\i des hyperboliques.}
    \item \question{Pourquoi ne peut-t-on pas 
na\"\i vement
repr\'esenter le graphe
de l'application 
\[f\colon\R^2\longrightarrow \R^2,\  
f(x,y)= (-y, x),
\] 
sur une feuille de
papier. Comment peut-on graphiquement repr\'esenter cette application?}
\reponse{Le graphe de l'application $f$ 
en discussion est une surface dans $\R^4$, et la dimension
4 est trop grande pour
repr\'esenter, sur une feuille de papier, ce graphe plong\'e dans
$\R^4$.
L'application $f$ est un champs de vecteurs dans le plan cependant.
De fa\c con g\'en\'erale, on peut repr\'esenter graphiquement
le champ de vecteurs $X\colon U \to \R^2$ dans l'ouvert $U$ du plan en dessinant,
au point $(u_1,u_2)$ de $U$, le vecteur $X(u_1,u_2)= (x_1(u_1,u_2),x_2(u_1,u_2))$.

N.B. Quand on repr\'esente une surface dans l'espace de dimension 3 ordinaire
par un dessin sur une feuille de papier, en v\'erit\'e on ne dessine qu'une projection de
l'espace de dimension 3 sur un plan.}
\indication{Utiliser le langage de la g\'eom\'etrie \'el\'ementaire, y compris
les notions de surface de  r\'evolution, d'axe de
 r\'evolution, de sommet d'un parabolo\" \i de,
de sommet d'un c\^one, de concavit\'e  vers le haut ou vers le bas,
d'h\'elice, de spirale, etc.}
\end{enumerate}
}
