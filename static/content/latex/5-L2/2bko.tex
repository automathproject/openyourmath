\uuid{2bko}
\exo7id{4593}
\auteur{quercia}
\datecreate{2010-03-14}
\isIndication{false}
\isCorrection{true}
\chapitre{Série entière}
\sousChapitre{Etude au bord}

\contenu{
\texte{
Pour $x \in \R$ on pose $f(x) = \sum_{n=1}^\infty x^n\sin\frac1{\sqrt n}$.
}
\begin{enumerate}
    \item \question{Déterminer le rayon de convergence, $R$, de cette série.}
\reponse{$R=1$.}
    \item \question{\'Etudier la convergence de $f$ pour $x = \pm R$.}
\reponse{$x=-1  \Rightarrow $ cv (série alternée), $x=1  \Rightarrow $ dv.}
    \item \question{Déterminer $\lim_{x\to R^-} f(x)$.}
\reponse{$f$ est croissante sur $[0,1[$ donc $L$ existe dans $[0,+\infty]$.

             $L = \sup\limits_{[0,1[} f(x)
              \ge \sup\limits_{[0,1[} \sum_{n=1}^N x^n\sin\frac1{\sqrt n}
              \ge \sum_{n=1}^N \sin\frac1{\sqrt n}  \Rightarrow  L = +\infty$.}
\end{enumerate}
}
