\uuid{ZsaF}
\exo7id{1951}
\auteur{legall}
\organisation{exo7}
\datecreate{2003-10-01}
\isIndication{false}
\isCorrection{true}
\chapitre{Série de Fourier}
\sousChapitre{Calcul de coefficients}

\contenu{
\texte{
Soit $f$ la fonction $2\pi $-p\'eriodique sur $\Rr$ telle que 
$f(x)=\vert x \vert $ si $\vert x \vert \leq \pi .$
}
\begin{enumerate}
    \item \question{D\'eterminer la s\'erie de Fourier de $f$.}
\reponse{$a_0 = \pi$ et pour $n\ne 0$, $a_n = \frac{2}{\pi n^2} ((-1)^n - 1)$ (nul quand $n$ est pair), $b_n=0$.}
    \item \question{Calculer $\displaystyle{ \int _{-\pi}^\pi \vert x\vert ^2dx}$. 
En d\'eduire la valeur de $\displaystyle{ \sum _{p=0} ^{\infty } 
\frac{1}{(2p+1)^4}}.$}
\reponse{Par la formule de Parseval 
	\[
	\frac{2 \pi^2}{3} 
	= \frac{1}{\pi} \int_{-\pi}^\pi |x|^2 \, dx
	= \frac{\pi^2}{2} + \frac{16}{\pi^2} \sum_{p=0}^\infty \frac{1}{(2p+1)^4}
	\]
	donc $\sum_{p=0}^\infty \frac{1}{(2p+1)^2} = \frac{\pi^4}{96}$.}
    \item \question{Calculer $\displaystyle{ \sum _{p=1}^\infty \frac{1}{n^4}}$.}
\reponse{Comme
	\[
	\sum_{n=1}^\infty \frac{1}{n^4} = \sum_{p=1}^\infty \frac{1}{(2p)^2} + \sum_{p=0}^\infty \frac{1}{(2p+1)^2}
	= \frac{1}{16} \sum_{n=1}^\infty \frac{1}{n^4} + \frac{\pi^4}{96}
	\]
	il vient $\displaystyle \sum_{n=1}^\infty \frac{1}{n^4} = \frac{15}{16} \frac{\pi^4}{96}
	= \frac{\pi^4}{90}$.}
    \item \question{Montrer que $\vert x\vert = \displaystyle{\frac{\pi }{2} 
-\frac{4}{\pi} \sum _{p=0}^\infty \frac{ \cos (2p+1)x }{(2p+1)^2}}$.
En d\'eduire les valeurs de
$\displaystyle{ \sum _{p=0}^\infty \frac{1}{(2p+1)^2}}$
puis  $\displaystyle{ \sum _{p=1}^\infty \frac{1}{n^2}}$.}
\reponse{C'est le théorème de Dirichlet. Appliqué en $x=0$, cela implique que 
	$\displaystyle \sum_{p=0}^\infty \frac{1}{(2p+1)^2} = \frac{\pi}{4} \frac{\pi}{2} = \frac{\pi^2}{8}$. 
	On en déduit que $\displaystyle \sum_{p=0}^\infty \frac{1}{(2p+1)^2} = \frac{\pi^2}{6}$.}
\end{enumerate}
}
