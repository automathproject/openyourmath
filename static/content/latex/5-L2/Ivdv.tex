\uuid{Ivdv}
\exo7id{1788}
\auteur{drutu}
\organisation{exo7}
\datecreate{2003-10-01}
\isIndication{true}
\isCorrection{true}
\chapitre{Fonction de plusieurs variables}
\sousChapitre{Limite}

\contenu{
\texte{
Pour chacune des fonctions $f$ suivantes, \'etudier l'existence d'une
limite en $(0,0,0)$  :
}
\begin{enumerate}
    \item \question{$f(x,y,z)= \frac{xyz}{x+y+z}$ ;}
\reponse{Supposons $x+y+z \ne 0$. Alors
\[
\frac{xyz}{x+y+z} =\frac{xy(h(x,y)-x-y)}{h(x,y)}
=xy -\frac{xy(x+y)}{h(x,y)}
\]
 d'o\`u, avec
\[h(x,y)=(x+y)^4,
\]
nous obtenons
\[
\frac{xyz}{x+y+z} 
=xy -\frac{xy}{(x+y)^3} .
\]
Il s'ensuit que
\[
\mathrm{lim}_{\begin{smallmatrix}(x,y,z)\to (0,0,0)\\x+y+z=(x+y)^4\\
x\ne 0, y\ne  0, z \ne 0\end{smallmatrix}}\frac{xyz}{x+y+z} 
\]
n'existe pas, au moins non pas en tant que limite finie.
D'autre part,
\[
\mathrm{lim}_{\begin{smallmatrix}(x,y,z)\to (0,0,0)\\
x+z\ne 0,y= 0\end{smallmatrix}}\frac{xyz}{x+y+z} =0.
\]
Par cons\'equent,
\[
\mathrm{lim}_{\begin{smallmatrix}(x,y,z)\to (0,0,0)\\
x+y+z\ne 0\end{smallmatrix}}\frac{xyz}{x+y+z}
\]
ne peut pas exister.}
    \item \question{$f(x,y,z)= \frac{x+y}{x^2-y^2+z^2}$.}
\reponse{La limite
\[
\mathrm{lim}_{\begin{smallmatrix}(x,y,z) \to (0,0,0)\\ x\ne \pm y, z = 0\end{smallmatrix}}f(x,y,z)= 
\mathrm{lim}_{\begin{smallmatrix}(x,y) \to (0,0)\\
x \ne y\end{smallmatrix}}\frac 1{x-y}
\]
n'existe pas car $\mathrm{lim}_{\begin{smallmatrix}(x,y) \to (0,0)\\
y=x-x^2\end{smallmatrix}}\frac 1{x-y}$ n'existe pas.
Par cons\'equent,
\[
\mathrm{lim}_{\begin{smallmatrix}(x,y,z) \to (0,0,0)\\x^2-y^2+z^2\ne 0\end{smallmatrix}}f(x,y,z)
= 
\mathrm{lim}_{\begin{smallmatrix}(x,y,z) \to (0,0,0)\\x^2-y^2+z^2\ne 0\end{smallmatrix}}\frac{x+y}{x^2-y^2+z^2}
\]
ne peut pas exister.}
\indication{\begin{enumerate}
\item Raisonner \`a l'aide d'une fonction $h$ des variables $x$ et $y$
telle que $x+y+z=h(x,y)$ et $\lim_{(x,y) \to (0,0)} h(x,y)=0$.

\item Montrer que, d\'ej\`a sous la contrainte suppl\'ementaire $z=0$,
la limite ne peut pas exister.
\end{enumerate}}
\end{enumerate}
}
