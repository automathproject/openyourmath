\uuid{Xe8r}
\exo7id{5905}
\auteur{rouget}
\datecreate{2010-10-16}
\isIndication{false}
\isCorrection{true}
\chapitre{Fonction de plusieurs variables}
\sousChapitre{Différentielle seconde}

\contenu{
\texte{
Soit $f~:~\Rr^2\rightarrow\Rr^2$ de classe $C^2$ dont la différentielle en tout point est une rotation. Montrer que $f$ est une rotation affine.
}
\reponse{
Soit $(x,y)\in\Rr^2$. La matrice jacobienne de $f$ en $(x,y)$ s'écrit $\left(
\begin{array}{cc}
c(x,y)&-s(x,y)\\
s(x,y)&c(x,y)
\end{array}
\right)$ où $c$ et $s$ sont deux fonctions de classe $C^1$ sur $\Rr^2$ telle que $c^2+s^2=1$ $(*)$. Il s'agit dans un premier temps de vérifier que les fonctions $c$ et $s$ sont constantes sur $\Rr^2$.

Puisque $f$ est de classe $C^2$ sur $\Rr^2$, d'après le théorème de \textsc{Schwarz}, $ \frac{\partial^2f}{\partial x\partial y}= \frac{\partial^2f}{\partial x\partial y}$. Ceci s'écrit encore $ \frac{\partial}{\partial y}\left(
\begin{array}{c}
c\\
s
\end{array}
\right)= \frac{\partial}{\partial x}\left(
\begin{array}{c}
-s\\
c
\end{array}
\right)$ ou enfin

\begin{center}
$\forall(x,y)\in\Rr^2$, $\left(
\begin{array}{c}
 \frac{\partial c}{\partial y}(x,y)\\
\rule{0mm}{6mm} \frac{\partial s}{\partial y}(x,y)
\end{array}
\right)=\left(
\begin{array}{c}
- \frac{\partial s}{\partial x}(x,y)\\
\rule{0mm}{6mm} \frac{\partial c}{\partial x}(x,y)
\end{array}
\right)$ $(**)$.
\end{center}

En dérivant $(*)$ par rapport à $x$ ou à $y$, on obtient les égalités $c \frac{\partial c}{\partial x}+s \frac{\partial s}{\partial x}=0$ et $c \frac{\partial c}{\partial y}+s \frac{\partial s}{\partial y}=0$. Ceci montre que les deux vecteurs $\left(
\begin{array}{c}
 \frac{\partial c}{\partial x}\\
\rule{0mm}{6mm} \frac{\partial s}{\partial x}
\end{array}
\right)$ et $\left(
\begin{array}{c}
 \frac{\partial c}{\partial y}\\
\rule{0mm}{6mm} \frac{\partial s}{\partial y}
\end{array}
\right)$ sont orthogonaux au vecteur non nul $\left(
\begin{array}{c}
c\\
s
\end{array}
\right)$ et sont donc colinéaires. Mais l'égalité $(**)$ montre que les deux vecteurs $\left(
\begin{array}{c}
 \frac{\partial c}{\partial x}\\
\rule{0mm}{6mm} \frac{\partial s}{\partial x}
\end{array}
\right)$ et $\left(
\begin{array}{c}
 \frac{\partial c}{\partial y}\\
\rule{0mm}{6mm} \frac{\partial s}{\partial y}
\end{array}
\right)$ sont aussi orthogonaux l'un à l'autre. Finalement, pour tout $(x,y)\in\Rr^2$, les deux vecteurs $\left(
\begin{array}{c}
 \frac{\partial c}{\partial x}(x,y)\\
\rule{0mm}{6mm} \frac{\partial s}{\partial x}(x,y)
\end{array}
\right)$ et $\left(
\begin{array}{c}
 \frac{\partial c}{\partial y}(x,y)\\
\rule{0mm}{6mm} \frac{\partial s}{\partial y}(x,y)
\end{array}
\right)$ sont nuls. On en déduit que les deux applications $c$ et $s$ sont constantes sur $\Rr^2$ et donc, il existe $\theta$ dans $\Rr$ tel que pour tout $(x,y)\in\Rr^2$, la matrice jacobienne de $f$ en $(x,y)$ est $\left(
\begin{array}{cc}
\cos(\theta)&-\sin(\theta)\\
\sin(\theta)&\cos(\theta)
\end{array}
\right)$.

Soit $g$ la rotation d'angle $\theta$ prenant la même valeur que $f$ en $(0,0)$. $f$ et $g$ ont mêmes différentielles en tout point et coïncident en un point. Donc $f=g$ et $f$ est une rotation affine.

\begin{center}
\shadowbox{
\begin{tabular}{c}
Soit $f~:~\Rr^2\rightarrow\Rr^2$ de classe $C^2$ dont la différentielle en tout point est une rotation.\\
Alors $f$ est une rotation affine.
\end{tabular}
}
\end{center}
}
}
