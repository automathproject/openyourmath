\uuid{ERh6}
\exo7id{4800}
\auteur{quercia}
\organisation{exo7}
\datecreate{2010-03-16}
\isIndication{false}
\isCorrection{true}
\chapitre{Topologie}
\sousChapitre{Topologie des espaces vectoriels normés}

\contenu{
\texte{
Soit $f$ une fonction polynomiale sur~$\C$. Montrer que l'image par $f$ de tout
ferm{\'e} est un ferm{\'e}.
}
\reponse{
Si $f$ est constante c'est {\'e}vident. Sinon, on a facilement $|f(z)|\xrightarrow[|z|\to\infty]{}\infty$.
Consid{\'e}rons un ferm{\'e} $F$ et une suite $(f(z_n))$ d'{\'e}l{\'e}ments de~$f(F)$ convergeant
vers $Z\in\C$. D'apr{\`e}s la remarque, la suite $(z_n)$ est born{\'e}e, elle
admet une valeur d'adh{\'e}rence $z\in F$ et $Z=f(z)\in f(F)$.

Remarque~: ce r{\'e}sultat est faux pour une fonction polynomiale sur $\C^p$ avec $p\ge 2$,
prendre par exemple $f(x,y) = x$ sur $\C^2$ et $F = \{(x,y)\in\C^2$ tq $xy=1\}$.
}
}
