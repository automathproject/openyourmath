\uuid{je0v}
\exo7id{6998}
\auteur{blanc-centi}
\organisation{exo7}
\datecreate{2015-07-04}
\isIndication{true}
\isCorrection{true}
\chapitre{Equation différentielle}
\sousChapitre{Résolution d'équation différentielle du deuxième ordre}

\contenu{
\texte{
On considère $y''-4y'+4y=d(x)$. 
Résoudre l'équation homogène, puis trouver une solution particulière 
lorsque $d(x)=e^{-2x}$, puis $d(x)=e^{2x}$. 
Donner la forme générale des solutions quand $d(x)=\frac{1}{2}\ch(2x)$.
}
\indication{Pour la fin: principe de superposition.}
\reponse{
L'équation caractéristique associée à l'équation homogène est $r^2-4r+4=0$, 
pour laquelle $r=2$ est racine double. Les solutions de l'équation homogène 
sont donc les $(\lambda x+\mu)e^{2x}$. 

Lorsque $d(x)=e^{-2x}$, on cherche une solution particulière sous la forme $ae^{-2x}$, 
qui convient si $a=\frac{1}{16}$.

Lorsque $d(x)=e^{2x}$, comme 2 est la racine double de l'équation caractéristique, 
on cherche une solution comme le produit de $e^{2x}$ par un polyn\^ome de degré 2. 
Comme on sait déjà que $(\lambda x+\mu)e^{2x}$ est solution de l'équation homogène, 
il est inutile de faire intervenir des termes de degré $1$ et $0$: on cherche 
donc une solution de la forme $ax^2e^{2x}$, qui convient si et seulement si $a=\frac{1}{2}$. 

Puisque $\frac{1}{2}\ch(2x) =\frac{1}{4}(e^{2x}+e^{-2x})$, les solutions générales sont obtenues sous 
la forme $y(x)=\frac{1}{64}e^{-2x}+\frac{1}{8}x^2e^{2x}+(\lambda x+\mu)e^{2x}$.
}
}
