\uuid{jy2I}
\exo7id{7003}
\auteur{blanc-centi}
\organisation{exo7}
\datecreate{2015-07-04}
\isIndication{false}
\isCorrection{true}
\chapitre{Equation différentielle}
\sousChapitre{Résolution d'équation différentielle du deuxième ordre}

\contenu{
\texte{
\
}
\begin{enumerate}
    \item \question{Résoudre sur $]0;+\infty[$ l'équation différentielle $x^2y''+y=0$ (utiliser le changement de variable $x=e^t$).}
\reponse{En faisant le changement de variable $x=e^t$ (donc $t = \ln x$) et en posant 
$z(t)=y(e^t)$ (donc $y(x) = z(\ln x)$, l'équation $x^2y''+y=0$ devient $z''-z'+z=0$, 
dont les solutions sont les 
$z(t)=e^{t/2}\cdot\left( \lambda\cos(\frac{\sqrt{3}}{2}t)+\mu\sin(\frac{\sqrt{3}}{2}t) \right)$,
$\lambda,\mu\in\Rr$. 
Autrement dit, 
$$y(x)=\sqrt{x}\left(\lambda\cos(\frac{\sqrt{3}}{2}\ln x)+\mu\sin(\frac{\sqrt{3}}{2}\ln x)\right)$$}
    \item \question{Trouver toutes les fonctions de classe $\mathcal{C}^1$ sur $\R$ vérifiant
$$\forall x\not=0,\ f'(x)=f\left(\frac{1}{x}\right)$$}
\reponse{Supposons que $f$ convienne: par hypothèse, $f$ est de classe $\mathcal{C}^1$, 
donc $x\mapsto f(\frac{1}{x})$ est de classe $\mathcal{C}^1$ sur $\R^*$ et 
par conséquent $f'$ aussi. Ainsi $f$ est nécessairement de classe 
$\mathcal{C}^2$ sur $\R^*$ (en fait, en itérant le raisonnement, on montrerait 
facilement que $f$ est $\mathcal{C}^\infty$ sur $\R^*$). 


En dérivant l'équation $f'(x)=f\left(\frac{1}{x}\right)$, 
on obtient 
$$f''(x)=-\frac{1}{x^2}f'\left(\frac{1}{x}\right)$$
et en réutilisant l'équation :
$$f''(x)=-\frac{1}{x^2}f'\left(\frac{1}{x}\right)= -\frac{1}{x^2} f(x).$$
Ainsi on obtient que $f$ est solution de $x^2y''+y=0$ sur $\R^*$. 
Nécessairement, il existe $\lambda,\mu\in\Rr$ tels que 
$$\forall x>0,\ f(x)=\sqrt{x}\left(\lambda\cos(\frac{\sqrt{3}}{2}\ln x)
+\mu\sin(\frac{\sqrt{3}}{2}\ln x)\right)$$


Par hypothèse, $f$ est de classe $\mathcal{C}^1$ sur $\R$, en particulier 
elle se prolonge en $0$ de façon $\mathcal{C}^1$. Cherchons à quelle condition 
sur $\lambda,\mu$ cela est possible. Déjà, 
$f(x)=\sqrt{x}\left(\lambda\cos(\frac{\sqrt{3}}{2}\ln x)+\mu\sin(\frac{\sqrt{3}}{2}\ln x)\right)
\xrightarrow[x\to 0^+]{}0$ pour tous $\lambda,\mu$ ; donc $f(0)=0$. Mais
$$\frac{f(x)-f(0)}{x-0}= \frac1{\sqrt{x}}
\left(\lambda\cos(\frac{\sqrt{3}}{2}\ln x)+\mu\sin(\frac{\sqrt{3}}{2}\ln x)\right)$$ 
n'a pas de limite en $0$ si $\lambda\neq0$ ou $\mu\neq0$. En effet, pour 
$x_n=e^{\frac{2}{\sqrt{3}}(-2n\pi)}$, on a $x_n\to 0$ 
mais $\frac{f(x_n)-f(0)}{x_n-0}=\frac{\lambda}{\sqrt{x_n}}$ qui admet une limite finie seulement si $\lambda = 0$.
De même avec $x_n'=e^{\frac{2}{\sqrt{3}}(-2n\pi+\frac\pi2)}$ qui donne
$\frac{f(x_n')-f(0)}{x_n'-0}=\frac{\mu}{\sqrt{x_n'}}$ et implique donc $\mu=0$.


Par conséquent, la seule possibilité est $\lambda=\mu=0$. Ainsi $f$ est la fonction nulle, sur $[0,+\infty[$.
Le même raisonnement s'applique sur $]-\infty,0]$. La fonction est donc nécessairement nulle sur $\Rr$.
Réciproquement, la fonction constante nulle est bien solution du problème initial.}
\end{enumerate}
}
