\uuid{FLVw}
\exo7id{4337}
\auteur{quercia}
\organisation{exo7}
\datecreate{2010-03-12}
\isIndication{false}
\isCorrection{false}
\chapitre{Intégration}
\sousChapitre{Intégrale de Riemann dépendant d'un paramètre}

\contenu{
\texte{
Soient ${f,g} : \R \to \R$ continues et $a,b \in \R$.
On pose $\varphi(x) =  \int_{t=a}^b f(t)g(x-t)\,d t$.
}
\begin{enumerate}
    \item \question{Montrer que $\varphi$ est continue et que si $g$ est de classe $\mathcal{C}^k$, alors
    $\varphi$ l'est aussi.}
    \item \question{Montrer que si $f$ est de classe $\mathcal{C}^1$ (et $g$ continue), alors $\varphi$ est
    aussi de classe $\mathcal{C}^1$.}
\end{enumerate}
}
