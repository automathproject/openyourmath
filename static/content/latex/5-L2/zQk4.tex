\uuid{zQk4}
\exo7id{7568}
\auteur{mourougane}
\datecreate{2021-08-10}
\isIndication{false}
\isCorrection{false}
\chapitre{Série entière}
\sousChapitre{Développement en série entière}

\contenu{
\texte{
On définit les nombres de Bernouilli comme les nombres complexes $B_n$ tels que sur le domaine de convergence
$$\frac{z}{e^z-1}=\sum_{n=0}^{+\infty}\frac{B_n}{n!}z^n.$$
}
\begin{enumerate}
    \item \question{Montrer que le rayon de convergence de la série précédente est $2\pi$.}
    \item \question{A partir de la relation $\frac{z}{e^z-1}\frac{e^z-1}{z}=1$ déterminer une relation de récurrence entre les $B_n$.}
    \item \question{Exprimer $\frac{e^z-1}{z}$ en fonction de $\frac{\sinh(z/2)}{\cosh(z/2)}$
 et en déduire que pour $n$ impair plus grand que $3$, $B_n=0$.}
    \item \question{Calculer $B_0, B_1\cdots B_8$.}
    \item \question{Montrer que tous les $B_n$ sont rationnels.}
    \item \question{La suite des $B_n$ est-elle bornée ?}
\end{enumerate}
}
