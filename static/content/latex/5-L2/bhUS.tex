\uuid{bhUS}
\exo7id{4077}
\auteur{quercia}
\datecreate{2010-03-11}
\isIndication{false}
\isCorrection{true}
\chapitre{Equation différentielle}
\sousChapitre{Equations différentielles linéaires}

\contenu{
\texte{
Soit l'équation $(*) \Leftrightarrow y' + a(x)y = b(x)$ où $a,b$ sont des fonctions
continues, $T$-périodiques.
}
\begin{enumerate}
    \item \question{Montrer que si $y$ est solution de $(*)$, alors la fonction définie par
    $z(x) = y(x+T)$ est aussi solution.}
    \item \question{En déduire que si $ \int_{t=0}^T a(t)\,d t \ne 0$, alors $(*)$ admet une unique
    solution $T$-périodique.}
\reponse{
Soient $y_0$ une solution particulière et $y_1$ une solution
             non nulle de l'équation homogène : $y_1(x) = e^{-A(x)}$ avec
             $A'=a$. Alors $y_0(x+T) = y_0(x) + \alpha y_1(x)$, et pour
             une solution $y$ quelconque, $y = y_0 + \lambda y_1$ :
             $y(x+T)-y(x) = (\alpha + \lambda(e^{-I}-1))y_1(x)$ où
             $I =  \int_{t=0}^T a(t)\,d t$.
}
\end{enumerate}
}
