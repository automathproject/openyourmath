\uuid{hf9b}
\exo7id{4322}
\auteur{quercia}
\datecreate{2010-03-12}
\isIndication{false}
\isCorrection{true}
\chapitre{Intégration}
\sousChapitre{Intégrale de Riemann dépendant d'un paramètre}

\contenu{
\texte{
Soit $f : {[a,b]} \to {\R^+}$ continue. On pose
$\varphi(x) = \left({ \int_{t=a}^b (f(t))^x\,d t}\right)^{1/x}$.
}
\begin{enumerate}
    \item \question{Montrer que $\varphi(x) \to \max(f)$ lorsque $x\to+\infty$.}
    \item \question{On suppose $f > 0$ et $b-a = 1$.
    Montrer que $\varphi(x) \to \exp\left({ \int_{t=a}^b\ln(f(t))\,d t}\right)$ lorsque $x\to0^+$.}
\reponse{
Soit $\varepsilon > 0$ : Pour $x$ assez petit,
             $\bigl|f(t)^x-1-x\ln(f(t))\bigr| \le \varepsilon x$ car $\ln f$
             est borné sur $[a,b]$.\par
             Donc $\left| \int_{t=a}^b f(t)^xd t - 1 - x \int_{t=a}^b\ln(f(t))\,d t\right| \le \varepsilon x$,
             et   $\left|\ln\left({ \int_{t=a}^b f(t)^xd t}\right) - x \int_{t=a}^b\ln(f(t))\,d t\right| \le 2\varepsilon x$.
}
\end{enumerate}
}
