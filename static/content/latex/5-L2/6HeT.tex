\uuid{6HeT}
\exo7id{5914}
\auteur{rouget}
\datecreate{2010-10-16}
\isIndication{false}
\isCorrection{true}
\chapitre{Intégration}
\sousChapitre{Intégrale multiple}

\contenu{
\texte{
Calculer $I=\displaystyle\iint_{\frac{x^2}{a^2}+\frac{y^2}{b^2}\leqslant1}(x^2-y^2)\;dxdy$.
}
\reponse{
On pose déjà $x=ua$ et $y=vb$ de sorte que $ \frac{D(u,v)}{D(x,y)}=ab$. On obtient

\begin{center}
$I=\displaystyle\iint_{\frac{x^2}{a^2}+\frac{y^2}{b^2}\leqslant1}(x^2-y^2)\;dxdy=ab\displaystyle\iint_{u^2+v^2\leqslant1}(a^2u^2-b^2v^2)\;dudv$.
\end{center}

Ensuite,

\begin{align*}\ensuremath
\displaystyle\iint_{u^2+v^2\leqslant1}u^2\;dudv&=\displaystyle\iint_{u^2+v^2\leqslant1}v^2\;dudv= \frac{1}{2}\displaystyle\iint_{u^2+v^2\leqslant1}(u^2+v^2)\;dudv= \frac{1}{2}\int_{r=0}^{r=1}\int_{\theta=0}^{2\pi}r^2\times rdrd\theta\\
 &= \frac{1}{2}\int_{0}^{1}r^3dr\int_{0}^{2\pi}d\theta= \frac{1}{2}\times \frac{1}{4}\times2\pi= \frac{\pi}{4},
\end{align*}

et donc

\begin{center}
\shadowbox{
$I= \frac{\pi ab(a^2-b^2)}{4}$.
}
\end{center}
}
}
