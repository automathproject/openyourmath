\uuid{Oquu}
\exo7id{5860}
\auteur{rouget}
\datecreate{2010-10-16}
\isIndication{false}
\isCorrection{true}
\chapitre{Topologie}
\sousChapitre{Application linéaire continue, norme matricielle}

\contenu{
\texte{
Soit $N$ une norme sur $\mathcal{M}_n(\Rr)$.

Montrer qu'il existe $k>0$ tel que $\forall(A,B)\in(\mathcal{M}_n(\Rr))^2$, $N(AB)\leqslant k(A)N(B)$.
}
\reponse{
Soit $N$ une norme sur $\mathcal{M}_n(\Rr)$. D'après l'exercice \ref{ex:rou5}, $\|\;\|_1$ est une norme sous-multiplicative.

Puisque $\mathcal{M}_n(\Rr)$ est un espace vectoriel de dimension finie sur $\Rr$, $N$ et $\|\;\|_1$ sont des normes équivalentes. Par suite, il existe deux réels strictement positifs $\alpha$ et $\beta$ tels que $\alpha\|\;\|_1\leqslant N\leqslant\beta\|\;\|_1$.

Pour $(A,B)\in\left(\mathcal{M}_n(\Rr)\right)^2$,

\begin{center}
$N(AB)\leqslant\beta\|AB\|_1\leqslant\beta\|A\|_1\|B\|_1\leqslant \frac{\beta}{\alpha^2}N(A)N(B)$
\end{center}

et le réel $k= \frac{\beta}{\alpha^2}$ est un réel strictement positif tel que $\forall(A,B)\in(\mathcal{M}_n(\Rr))^2$, $N(AB)\leqslant kN(A)N(B)$.

\textbf{Remarque.} Le résultat précédent signifie que $N'= \frac{1}{K}N$ est une norme sous-multiplicative car pour $(A,B)\in\left(\mathcal{M}_n(\Rr)\right)^2$,

\begin{center}
$N'(AB)= \frac{1}{k^2}N(AB)\leqslant \frac{1}{k^2}N(A)N(B)= \frac{1}{k}N(A) \frac{1}{k}N(B)=N'(A)N'(B)$.
\end{center}
}
}
