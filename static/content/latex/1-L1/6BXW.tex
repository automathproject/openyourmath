\uuid{6BXW}
\exo7id{5081}
\auteur{rouget}
\datecreate{2010-06-30}
\isIndication{false}
\isCorrection{true}
\chapitre{Nombres complexes}
\sousChapitre{Trigonométrie}

\contenu{
\texte{
Résoudre le système $\left\{
\begin{array}{l}
\cos a+\cos b+\cos c=0\\
\sin a+\sin b+\sin c=0
\end{array}\right.$ où $a$, $b$ et $c$ sont trois réels.
}
\reponse{
\begin{align*}
\left\{
\begin{array}{l}
\cos a+\cos b+\cos c=0\\
\sin a+\sin b+\sin c=0
\end{array}
\right.&\Leftrightarrow(\cos a+\cos b+\cos c)+i(\sin a+\sin b+\sin c)=0\Leftrightarrow e^{ia}+e^{ib}+e^{ic}=0\\
 &\Rightarrow|e^{ia}+e^{ib}|=|-e^{ic}|=1\Leftrightarrow|e^{ia/2}e^{ib/2}(e^{i(a-b)/2}+e^{-i(a-b)/2})|=1\\
 &\Leftrightarrow|\cos\frac{a-b}{2}|=\frac{1}{2}\\
 &\Leftrightarrow\frac{a-b}{2}\in\left(\frac{\pi}{3}+\pi\Zz\right)\cup\left(-\frac{\pi}{3}+\pi\Zz\right)\Leftrightarrow
a-b\in\left(\frac{2\pi}{3}+2\pi\Zz\right)\cup\left(-\frac{2\pi}{3}+2\pi\Zz\right)\\
 &\Leftrightarrow\exists k\in\Zz,\;\exists\varepsilon\in\{-1,1\}/\;b=a+\varepsilon\frac{2\pi}{3}+2k\pi.
\end{align*}

Par suite, nécessairement, $e^{ib}=je^{ia}$ ou $e^{ib}=j^2e^{ia}$. Réciproquement, si $e^{ib}=je^{ia}$ ou
encore $b=a+\frac{2\pi}{3}+2k\pi$,

$$e^{ia}+e^{ib}+e^{ic}=0\Leftrightarrow e^{ic}=-(e^{ia}+e^{ib})=-(1+j)e^{ia}=j^2e^{ia}\Leftrightarrow\exists
k'\in\Zz/\;c=a-\frac{2\pi}{3}+2k'\pi,$$

et si $e^{ib}=j^2e^{ia}$ ou encore
$b=a-\frac{2\pi}{3}+2k\pi$,

$$e^{ia}+e^{ib}+e^{ic}=0\Leftrightarrow e^{ic}=-(e^{ia}+e^{ib})=-(1+j^2)e^{ia}=je^{ia}\Leftrightarrow\exists
k'\in\Zz/\;c=a+\frac{2\pi}{3}+2k'\pi.$$

\begin{center}
\shadowbox{
$\mathcal{S}=\{(a,a+\varepsilon\frac{2\pi}{3}+2k\pi,a-\varepsilon\frac{2\pi}{3}+2k'\pi),\;a\in\Rr,\;
\varepsilon\in\{-1,1\},\;(k,k')\in\Zz^2\}.$
}
\end{center}
}
}
