\uuid{BSG7}
\exo7id{5122}
\auteur{rouget}
\datecreate{2010-06-30}
\isIndication{false}
\isCorrection{true}
\chapitre{Nombres complexes}
\sousChapitre{Racine n-ieme}

\contenu{
\texte{
Soit $\alpha\in\left]-\frac{\pi}{2},\frac{\pi}{2}\right[$ donné. Résoudre dans $\Cc$ l'équation
$\left(\frac{1+iz}{1-iz}\right)^3=\frac{1+i\tan\alpha}{1-i\tan\alpha}$.
}
\reponse{
Soit $\alpha\in\left]\frac{\pi}{2},\frac{\pi}{2}\right[$.
$\frac{1+i\tan\alpha}{1-i\tan\alpha}=\frac{\cos\alpha+i\sin\alpha}{\cos\alpha-i\sin\alpha}=e^{2i\alpha}$. Donc,

$$\left(\frac{1+iz}{1-iz}\right)^3=\frac{1+i\tan\alpha}{1-i\tan\alpha}\Leftrightarrow\exists
k\in\{-1,0,1\}/\;\frac{1+iz}{1-iz}=e^{i(\frac{2\alpha}{3}+\frac{2k\pi}{3})}=\omega_k\Leftrightarrow\exists
k\in\{-1,0,1\}/\;i(\omega_k+1)z=\omega_k-1.$$
Maintenant, pour $k\in\{-1,0,1\}$,

$$\omega_k=-1\Leftrightarrow\frac{2\alpha}{3}+\frac{2k\pi}{3}\in\pi+2\pi\Zz\Leftrightarrow\alpha\in-k\pi+\frac{3\pi}{2}+3\pi\Zz,$$
ce qui est exclu pour $\alpha\in]-\frac{\pi}{2},\frac{\pi}{2}[$. Donc,

\begin{align*}
\left(\frac{1+iz}{1-iz}\right)^3=\frac{1+i\tan\alpha}{1-i\tan\alpha}&\Leftrightarrow\exists
k\in\{-1,0,1\}/\;z=\frac{\omega_k-1}{i(\omega_k+1)}\Leftrightarrow\exists
k\in\{-1,0,1\}/\;z=\frac{e^{i(\frac{\alpha}{3}+\frac{k\pi}{3})}}{e^{i(\frac{\alpha}{3}+\frac{k\pi}{3})}}
\frac{e^{i(\frac{\alpha}{3}+\frac{k\pi}{3})}-e^{-i(\frac{\alpha}{3}+\frac{k\pi}{3})}}{i(e^{i(\frac{\alpha}{3}
+\frac{k\pi}{3})}+e^{-i(\frac{\alpha}{3}+\frac{k\pi}{3})})}\\
 &\Leftrightarrow\exists
k\in\{-1,0,1\}/\;z=\frac{2i\sin(\frac{\alpha}{3}+\frac{k\pi}{3})}{i(2\cos(\frac{\alpha}{3}+\frac{k\pi}{3}))}\Leftrightarrow
\exists k\in\{-1,0,1\}/\;z=\tan(\frac{\alpha}{3}+\frac{k\pi}{3})
\end{align*}
}
}
