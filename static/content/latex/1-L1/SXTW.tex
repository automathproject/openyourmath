\uuid{SXTW}
\exo7id{2774}
\auteur{tumpach}
\organisation{exo7}
\datecreate{2009-10-25}
\isIndication{false}
\isCorrection{true}
\chapitre{Matrice}
\sousChapitre{Propriétés élémentaires, généralités}

\contenu{
\texte{
Soit $a $ et $b$ deux réels et $A$ la matrice
$$
A = \left(
\begin{array}{cccc}
a & 2 & -1 & b\\
3 & 0 & 1 & -4\\ 
5 & 4 & -1 & 2
\end{array}
\right)
$$
Montrer que $\textrm{rg}(A) \geq 2$. 
Pour quelles valeurs de $a$ et $b$ a-t-on $\textrm{rg}(A) = 2$~?
}
\reponse{
Montrons de différentes façons que $\textrm{rg}(A)\ge 2$.
  \begin{itemize}
\textbf{Première méthode : sous-déterminant non nul.}
On trouve une sous-matrice $2\times 2$ dont le déterminant est non nul.
Par exemple la sous-matrice extraite du coin en bas à gauche vérifie
$\begin{vmatrix}3 & 0\\ 5 & 4\end{vmatrix}= 12 \neq 0$ donc $\textrm{rg}(A)\ge 2$.
\textbf{Deuxième méthode : espace vectoriel engendré par les colonnes.}
On sait que l'image de l'application linéaire associée à la matrice $A$
est engendrée par les vecteurs colonnes. Et le rang est la dimension de cette image.
On trouve facilement deux colonnes linéairement indépendantes : 
la deuxième $\begin{pmatrix}2\\0\\4\end{pmatrix}$
et la troisième $\begin{pmatrix}-1\\1\\-1\end{pmatrix}$ colonne.
Donc $\textrm{rg}(A)\ge 2$.
\textbf{Troisième méthode : espaces vectoriel engendré par les lignes.}
Il se trouve que la dimension de l'espace vectoriel engendré par les lignes
égal la dimension de l'espace vectoriel engendré par les colonnes (car $\textrm{rg}(A)=\textrm{rg}({}^tA)$).
Comme les deuxième et troisième lignes sont linéairement indépendantes alors
$\textrm{rg}(A)\ge 2$. 

Attention : les dimensions des espaces vectoriels engendrés sont égales mais les espaces sont différents !
  \end{itemize}
En utilisant la dernière méthode : le rang est exactement $2$ si la première ligne est dans le sous-espace engendré
par les deux autres.
Donc 
\begin{align*}
\textrm{rg}(A) = 2
 & \iff (a,2,-1,b) \in \textrm{Vect} \big\{ (3,0,1,-4), (5,4,-1,2) \big\} \\
 & \iff \exists \lambda,\mu \in \Rr \quad (a,2,-1,b) = \lambda (3,0,1,-4) + \mu(5,4,-1,2) \\
 & \iff \exists \lambda,\mu \in \Rr \quad 
\left\{
\begin{array}{rcl}
3\lambda+5\mu &=& a \\
4\mu &=& 2 \\
\lambda-\mu &=& -1 \\
-4\lambda+2\mu &=& b \\
\end{array}
\right. 
 \iff  \left\{
\begin{array}{rcl}
\lambda &=& -\frac12 \\
\mu &=& \frac12 \\
a &=& 1 \\
b &=& 3 \\
\end{array} 
\right.\\  
\end{align*}
Conclusion la rang de $A$ est $2$ si $(a,b)=(1,3)$. Sinon le rang de $A$ est $3$.
}
}
