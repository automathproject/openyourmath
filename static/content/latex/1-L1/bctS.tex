\uuid{bctS}
\exo7id{371}
\auteur{cousquer}
\organisation{exo7}
\datecreate{2003-10-01}
\isIndication{false}
\isCorrection{true}
\chapitre{Polynôme, fraction rationnelle}
\sousChapitre{Division euclidienne}

\contenu{
\texte{
Effectuer la division de $A=X^6-2X^4+X^3+1$ par $B=X^3+X^2+1$~:
}
\begin{enumerate}
    \item \question{Suivant les puissances d\'ecroissantes.}
\reponse{Quotient $Q=X^3-X^2-X+1$, reste $R=X$.}
    \item \question{\`A l'ordre~$4$ (c'est-\`a-dire tel que le reste soit divisible par 
$X^5$) suivant les puissances croissantes.}
\reponse{Quotient $Q=1-X^2-X^4$, reste $R=X^5(1+2X+X^2)$.}
\end{enumerate}
}
