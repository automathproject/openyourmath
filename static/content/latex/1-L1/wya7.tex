\uuid{wya7}
\exo7id{251}
\auteur{bodin}
\datecreate{1998-09-01}
\isIndication{true}
\isCorrection{true}
\chapitre{Arithmétique dans Z}
\sousChapitre{Divisibilité, division euclidienne}

\contenu{
\texte{
Sachant que l'on a $96842=256\times 375 + 842$, d\'eterminer, sans faire
la division, le reste de la division du nombre $96842$ par chacun des nombres
$256$ et $375$.
}
\indication{Attention le reste d'une division euclidienne est plus petit que le quotient !}
\reponse{
La seule chose \`a voir est que pour une division euclidienne le reste doit \^etre plus petit que le quotient.
Donc les divisions euclidiennes s'\'ecrivent :
$96842 = 256 \times 378 + 74$ et $96842 = 258 \times 375 + 92$.
}
}
