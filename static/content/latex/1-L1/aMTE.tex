\uuid{aMTE}
\exo7id{985}
\auteur{cousquer}
\datecreate{2003-10-01}
\isIndication{true}
\isCorrection{true}
\chapitre{Espace vectoriel}
\sousChapitre{Base}

\contenu{
\texte{
Vrai ou faux ?
On d\'esigne par $E$ un $\mathbb{R}$-espace vectoriel de dimension finie.
}
\begin{enumerate}
    \item \question{Si les vecteurs $x,y,z$ sont deux \`a deux non colin\'eaires, alors la
famille $x,y,z$ est libre.}
\reponse{Faux. Par exemple dans $\Rr^3$, $x=(1,0,0)$, $y=(0,1,0)$, $z=(1,1,0)$.}
    \item \question{Soit $x_1,x_2, \dots,x_p$ une famille de vecteurs. Si aucun n'est une
combinaison lin\'eaire des autres, la famille est libre.}
\reponse{Vrai. Soit une combinaison lin\'eaire  nulle   
$\lambda_1x_1+\cdots \lambda_p x_p =0.$
Supposons qu'un des coefficient est non nul: par exemple $\lambda_1 \neq 0$.
Alors on \'ecrit $x_1 = -\frac{\lambda_2}{\lambda_1} x_2-\cdots - \frac{\lambda_p}{\lambda_1}x_p.$
Donc $x_1$ est une combinaison lin\'eaire de $\{x_2,\ldots,x_p\}$. Ce qui contredit l'hypoth\`ese de l'\'enonc\'e, donc  tous les coefficients sont nuls. Donc $\{x_1,\ldots,x_p\}$ est une famille libre.}
\indication{\begin{enumerate}
  \item Faux.
  \item Vrai.
\end{enumerate}}
\end{enumerate}
}
