\uuid{MIMV}
\exo7id{5352}
\auteur{rouget}
\datecreate{2010-07-04}
\isIndication{false}
\isCorrection{true}
\chapitre{Polynôme, fraction rationnelle}
\sousChapitre{Autre}

\contenu{
\texte{
Résoudre dans $\Cc$ l'équation $z^4-21z+8=0$ sachant qu'il existe deux des solutions sont inverses l'une de l'autre.
}
\reponse{
L'équation proposée admet deux solutions inverses l'une de l'autre si et seulement si il existe deux complexes $a$ et $b$ tels que

$$X^4-21X+8=(X^2+aX+1)(X^2+bX+8)=X^4+(a+b)X^3+(9+ab)X^2+(8a+b)X+8\;(*)$$

$(*)\Leftrightarrow b=-a$ et $ab=-9$ et $8a+b=-21\Leftrightarrow a=3$ et $b=-3$. Ainsi,

$$X^4-21X+8=(X^2+3X+1)(X^2-3X+8)=(X-\frac{-3+\sqrt{5}}{2})(X-\frac{-3-\sqrt{5}}{2})(X-\frac{3+i\sqrt{15}}{2})(X-\frac{3-i\sqrt{15}}{2}).$$
}
}
