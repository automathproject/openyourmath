\uuid{p8r0}
\exo7id{5350}
\auteur{rouget}
\datecreate{2010-07-04}
\isIndication{false}
\isCorrection{true}
\chapitre{Polynôme, fraction rationnelle}
\sousChapitre{Racine, décomposition en facteurs irréductibles}

\contenu{
\texte{
Soient $x_1$, $x_2$, $x_3$ les zéros de $X^3+2X-1$. Calculer $x_1^4+x_2^4+x_3^4$.
}
\reponse{
Pour $i\in\{1,2,3\}$, on a $x_i^3+2x_i-1=0$ et donc $x_i^4+2x_i^2-x_i=0$. En additionnant ces trois égalités, on obtient $S_4+2S_2-S_1=0$ et donc

$$S_4=-2((\sigma_1^2-2\sigma_2)+\sigma_1=(-2)(-2.2)=8.$$
}
}
