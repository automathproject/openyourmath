\uuid{wEsQ}
\exo7id{7204}
\auteur{megy}
\organisation{exo7}
\datecreate{2019-07-23}
\isIndication{false}
\isCorrection{false}
\chapitre{Logique, ensemble, raisonnement}
\sousChapitre{Relation d'équivalence, relation d'ordre}

\contenu{
\texte{
(Somme amalgammée d'ensembles. Cet exercice utilise la notion de coégalisateur.) 
Soient $A$,  $B$ et $C$  des ensembles et $f :  C\to A$, $g : C\to B$ des applications.

Soit $A\coprod B$ l'union disjointe de $A$ et $B$ et $i_A$ et $i_B$ les injections canoniques de $A$ et $B$ dans $A\coprod B$.

Les deux applications $i_A \circ f$ et $i_B \circ g$ vont toutes deux de $C$ dans $A\coprod B$.  Leur coégalisateur est appelé \emph{la somme amalgamée de $A$ et $B$ sous $C$}, est noté $A\coprod_C B$. La surjection canonique  $A\coprod B \to A\coprod_C B$ est notée $\pi$ et on note $j_A = \pi \circ i_A$ et $j_B = \pi \circ i_B$.

Montrer que $A\coprod_C B$ vérifie la propriété universelle suivante:

Pour tout ensemble  $D$ muni d'applications $\phi : A\to D$ et $\psi : B\to D$, il existe une unique application $h : A\coprod_C B \to D$ telle que $\phi = h\circ j_A$ et $\psi = h\circ j_B$.
}
}
