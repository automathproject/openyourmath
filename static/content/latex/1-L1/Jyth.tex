\uuid{Jyth}
\exo7id{150}
\auteur{bodin}
\organisation{exo7}
\datecreate{1998-09-01}
\isIndication{true}
\isCorrection{true}
\chapitre{Logique, ensemble, raisonnement}
\sousChapitre{Absurde et contraposée}

\contenu{
\texte{
Soit $(f_n)_{n\in\Nn}$ une suite d'applications de
l'ensemble $\Nn$ dans lui-m\^eme. On d\'efinit une application $f$
de $\Nn$ dans $\Nn$ en posant $f(n)=f_n(n)+1$. D\'emontrer qu'il
n'existe aucun $p\in \Nn$ tel que $f=f_p$.
}
\indication{Par l'absurde, supposer qu'il existe $p\in \Nn$ tel que $f=f_p$.
Puis pour un tel $p$, \'evaluer $f$ et $f_p$ en une valeur bien choisie.}
\reponse{
Par l'absurde, supposons qu'il existe $p\in \Nn$ tel que $f=f_p$.
Deux applications sont \'egales si et seulement si elles prennent
les m\^emes valeurs.
$$\forall n\in \Nn\ \ f(n) = f_p(n).$$
En particulier pour $n=p$, $f(p)=f_p(p)$. D'autre part la
d\'efinition de $f$ nous donne $f(p) = f_p(p)+1$. Nous obtenons
une contradiction car $f(p)$ ne peut prendre deux valeurs
distinctes. En conclusion, quelque soit $p\in \Nn$, $f\not= f_p$.
}
}
