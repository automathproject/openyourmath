\uuid{e8jL}
\exo7id{7413}
\auteur{mourougane}
\datecreate{2021-08-10}
\isIndication{false}
\isCorrection{true}
\chapitre{Matrice}
\sousChapitre{Inverse, méthode de Gauss}

\contenu{
\texte{
Pour un entier $n\geq 2$ et $x$ dans $\Rr$, considérons la matrice d'ordre $n$:
\[
D_n =
\left ( {\begin{array}{cccc}
 x & 1 & \ldots & 1 \\
 1 & x & \ddots & \vdots \\
 \vdots & \ddots & \ddots & 1 \\
 1 & \ldots & 1 & x
 \end{array} } \right ).
\]
}
\begin{enumerate}
    \item \question{Calculer $det(D_2)$ et $det(D_3)$.}
\reponse{$det(D_2)=x^2-1=(x+1)(x-1)$, $det(D_3)=x^3-3x+2=(x+2)(x-1)^2$}
    \item \question{Montrer d'abord que
 \[
 det(D_n)= \left | {\begin{array}{ccccc}
 x+n-1 & x+n-1 & \ldots & \ldots & x+n-1 \\
 1 & x & 1& \ldots & 1 \\
 \vdots & \ddots & \ddots &\ddots& \vdots \\
 \vdots &\ldots & \ddots & \ddots & 1 \\
 1 & \ldots & \ldots & 1 & x
 \end{array} } \right |,
\]
 
 et ensuite
 
 \[
 det(D_n)= (x+n-1)\left | {\begin{array}{cccc}
 1 & 1 & \ldots & 1 \\
 1 & x & \ddots & \vdots \\
 \vdots & \ddots & \ddots & 1 \\
 1 & \ldots & 1 & x
 \end{array} } \right |.
\]}
\reponse{Sommer toutes les lignes à la première ligne, on a :

\[
 det(D_n)= \left | {\begin{array}{ccccc}
 x+n-1 & x+n-1 & \ldots & \ldots & x+n-1 \\
 1 & x & 1& \ldots & 1 \\
 \vdots & \ddots & \ddots &\ddots& \vdots \\
 \vdots &\ldots & \ddots & \ddots & 1 \\
 1 & \ldots & \ldots & 1 & x
 \end{array} } \right |,
\]
et comme tous les éléments sur la première ligne sont $x+n-1$, donc 
\[
 det(D_n)= (x+n-1)\left | {\begin{array}{cccc}
 1 & 1 & \ldots & 1 \\
 1 & x & \ddots & \vdots \\
 \vdots & \ddots & \ddots & 1 \\
 1 & \ldots & 1 & x
 \end{array} } \right |
\]}
    \item \question{En utilisant la méthode du pivot de Gauss, calculer $det(D_n)$ pour tout $n$.}
\reponse{Par la méthode du pivot de Gauss, $L_2-L_1,L_3-L_1 \ldots, L_n-L_1,$, on obtient un détermiant de la forme triangulaire 
\[
 det(D_n)= (x+n-1)\left | {\begin{array}{cccc}
 1 & 1 & \ldots & 1 \\
 0 & x-1 & \ddots & \vdots \\
 \vdots & \ddots & \ddots & 0 \\
 0 & \ldots & 0 & x-1
 \end{array} } \right |.
\]
Donc $det(D_n)= (x+n-1)(x-1)^{n-1}$.}
    \item \question{Pour chaque $n$, pour quelles valeurs de $x$ la matrice $D_n$ est-elle inversible?}
\reponse{$D_n$ est inversible si et seulement si $det(D_n) \neq 0$, donc $x \neq 1-n$ et $x \neq 1$.}
\end{enumerate}
}
