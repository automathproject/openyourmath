\uuid{PsAY}
\exo7id{7210}
\auteur{megy}
\organisation{exo7}
\datecreate{2019-12-27}
\isIndication{false}
\isCorrection{false}
\chapitre{Nombres complexes}
\sousChapitre{Forme cartésienne, forme polaire}

\contenu{
\texte{
Résoudre sur $\C$ l'équation $4z^2+8\lvert z\rvert -3=0$.
}
\indication{Remarquer que si $z\in\C$ est solution, alors $z^2$ est forcément un nombre réel.}
\reponse{
Comme $z^2$ doit être réel, on en déduit que $z$ est soit réel, soit imaginaire pur. Dans chacun de ces deux cas, on obtient une équation d'inconnue réelle qui fait intervenir la valeur absolue. En séparant suivant le signe de l'inconnue, on obtient dinalement quatre cas qui correspondent à quatre trinômes réels.
}
}
