\uuid{KZaC}
\exo7id{3231}
\auteur{quercia}
\datecreate{2010-03-08}
\isIndication{false}
\isCorrection{true}
\chapitre{Polynôme, fraction rationnelle}
\sousChapitre{Racine, décomposition en facteurs irréductibles}

\contenu{
\texte{
Trouver tous les polyn{\^o}mes $P \in {\C[X]}$ tels que \dots
}
\begin{enumerate}
    \item \question{$P(X^2) = P(X-1)P(X+1)$.}
\reponse{si $P(x) = 0$, alors $P\bigl((x-1)^2\bigr) = P\bigl((x+1)^2\bigr) = 0$.
\\
On a toujours $|x| < \max\{ |x-1|,|x+1| \}$ donc, s'il y a une racine
de module $> 1$, il n'y a pas de racine de module maximal $ \Rightarrow  P = 0$.
\\
Or $\max\{ |x-1|,|x+1| \} \ge 1$ avec {\'e}galit{\'e} ssi $x = 0$.
Donc $P = 0$ ou $P = 1$.}
    \item \question{$P(X^2) = P(X)P(X-1)$.}
\reponse{Si $x$ est racine, alors $x^2$ et $(x+1)^2$ le sont aussi.
\\
$ \Rightarrow  |x| = 0$ ou $1  \Rightarrow  |x+1| = 0$ ou $1  \Rightarrow  x \in \{0,-1,j,j^2 \}$.
\\
$x = 0$ ou $x = -1  \Rightarrow  P(1) = 0$ : exclus.
\\
Donc $P = a(X-j)^\alpha (X-j^2)^\beta$. On remplace
$ \Rightarrow  P = (X^2 + X + 1)^\alpha$.}
    \item \question{$P(X)P(X+2) + P(X^2) = 0$.}
\reponse{Seule racine possible : 1 $ \Rightarrow  P=-(X-1)^k$.}
\end{enumerate}
}
