\uuid{ejZ5}
\exo7id{60}
\auteur{bodin}
\datecreate{1998-09-01}
\isIndication{true}
\isCorrection{true}
\chapitre{Nombres complexes}
\sousChapitre{Géométrie}

\contenu{
\texte{
D\'eterminer l'ensemble des nombres complexes $z$
tels que :
}
\begin{enumerate}
    \item \question{$\displaystyle{\left|\frac{z-3}{z-5}\right|=1},$}
    \item \question{$\displaystyle{\left|\frac{z-3}{z-5}\right|= \frac{\sqrt{2}}{2}}.$}
\reponse{
Nous identifions $\Cc$ au plan affine et $z=x+iy$ \`a  $(x,y) \in
\Rr\times \Rr$.

Remarquons que pour les deux ensembles $z=5$ n'est pas solution,
donc
$$\left| \frac{z-3}{z-5} \right| = 1
\Leftrightarrow |z-3| = |z-5|.$$ Ce qui signifie pr\'eci\'sement que les
points d'affixe $z$ sont situ\'es \`a \'egale distance des points
$A,B$ d'affixes respectives $3 = (3,0)$ et $5=(5,0)$. L'ensemble
solution est la m\'ediatrice du segment $[A,B]$.

\bigskip

Ensuite pour
\begin{align*}
\left| \frac{z-3}{z-5} \right| = \frac{\sqrt{2}}{2}
&\Leftrightarrow |z-3|^2 = \frac{1}{2}|z-5|^2 \\
&\Leftrightarrow (z-3)\overline{(z-3)} = \frac{1}{2}(z-5)\overline{(z-5)}\\
&\Leftrightarrow z\overline{z}-(z+\overline{z})=7\\
&\Leftrightarrow |z-1|^2=8\\
&\Leftrightarrow |z-1|=2\sqrt{2}\\
\end{align*}
L'ensemble solution est donc le cercle de centre le point d'affixe
$1 = (1,0)$ et de rayon $2\sqrt{2}$.
}
\indication{Le premier ensemble est une droite le second est un cercle.}
\end{enumerate}
}
