\uuid{5CN0}
\exo7id{3320}
\auteur{quercia}
\datecreate{2010-03-09}
\isIndication{false}
\isCorrection{true}
\chapitre{Espace vectoriel}
\sousChapitre{Base}

\contenu{
\texte{
Dans $ K^3$, on donne les sous espaces :
$\begin{cases} H = \{\smash{\overrightarrow X} = (x,y,z) \text{ tq } x + y + z = 0 \} \cr
         K = \text{vect}(\overrightarrow U = (1,1,2)).\hfill    \cr \end{cases}$
}
\begin{enumerate}
    \item \question{Déterminer $\dim H$ et en donner une base.}
    \item \question{Démontrer que $H \oplus K =  K^3$.}
    \item \question{Donner les expressions analytiques des projection et symétrie associées :
    $\pi_H$ et $s_H$.}
\reponse{
3. $\pi_H$ : $$\left\{
  \begin{array}{lllllll}
  4x' &=& 3x &-&  y &-& z  \cr
                     4y' &=& -x &+& 3y &-& z  \cr
                     4z' &=&-2x &-& 2y &+&2z, \cr
                     \end{array}\right.$$  \qquad
    $s_H$   : $$\left\{
    \begin{array}{lllllll}
    		2x' &=& x  &-&  y &-& z  \cr
                     2y' &=&-x  &+&  y &-& z  \cr
                     2z' &=&-2x &-& 2y.       \cr
                     \end{array}\right.$$
}
\end{enumerate}
}
