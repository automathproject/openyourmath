\uuid{8e3t}
\exo7id{237}
\auteur{bodin}
\organisation{exo7}
\datecreate{1998-09-01}
\isIndication{true}
\isCorrection{true}
\chapitre{Dénombrement}
\sousChapitre{Cardinal}

\contenu{
\texte{
Soit $E$ un ensemble \`a $n$ \'el\'ements, et
$A\subset E$ un sous-ensemble \`a $p$ \'el\'ements. Quel est le
nombre de parties de $E$ qui contiennent un et un seul \'el\'ement
de $A$ ?
}
\indication{Combien y-a-t'il de choix pour l'\'el\'ement de $A$ ?
Combien y-a-t'il de choix pour le sous-ensemble de $E\setminus A$ ?}
\reponse{
Fixons un \'el\'ement de $A$ ; dans $E\setminus A$ (de cardinal
$n-p$), nous pouvons choisir $C_{n-p}^k$ ensembles \`a $k$
\'el\'ements ($k = 0,1,\ldots,n$). Le nombre d'ensembles dans le
compl\'ementaire de $A$ est donc
$$ \sum_{k=0}^{n-p} C_{n-p}^{k} = 2^{n-p}.$$

Pour le choix d'un \'el\'ement de $A$ nous avons $p$ choix, donc
le nombre total d'ensembles qui v\'erifie la condition est :
$$p2^{n-p}.$$
}
}
