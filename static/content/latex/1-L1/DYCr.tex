\uuid{DYCr}
\exo7id{2693}
\auteur{matexo1}
\datecreate{2002-02-01}
\isIndication{false}
\isCorrection{false}
\chapitre{Matrice}
\sousChapitre{Autre}

\contenu{
\texte{
Un train qui ralentit avec une d{\'e}c{\'e}l{\'e}ration constante met 20s pour
parcourir le premier km et 30s pour parcourir le deuxi{\`e}me km. On veut calculer
la distance qu'il devra parcourir pour parvenir {\`a} l'arr{\^e}t.
\begin{itemize}
\item

En prenant pour origine la position initiale du train, {\'e}crire l'{\'e}quation
g{\'e}n{\'e}rale d'un mouvement uniform{\'e}ment   d{\'e}c{\'e}l{\'e}r{\'e}.

\item

En d{\'e}duire un syst{\`e}me de deux {\'e}quations dont les inconnues sont la
d{\'e}c{\'e}l{\'e}ration et la vitesse initiale du train, et r{\'e}soudre ce syst{\`e}me.
\item

Conclure.
\end{itemize}
}
}
