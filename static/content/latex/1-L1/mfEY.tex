\uuid{mfEY}
\exo7id{3356}
\auteur{quercia}
\organisation{exo7}
\datecreate{2010-03-09}
\isIndication{false}
\isCorrection{true}
\chapitre{Application linéaire}
\sousChapitre{Autre}

\contenu{
\texte{
Soit $f$ un endomorphisme  donné de $E$ de dimension $n$ et
$F=\{g\in \mathcal{L}(E)\mid g\circ f=f\circ g=0\}$. Trouver la dimension
de~$F$.
}
\reponse{
On veut $\Im g \subset\mathrm{Ker} f$ et $\mathrm{Ker} g\supset\Im f$
donc $g$ est entièrement définie par sa restriction à un supplémentaire
de~$\Im f$, application linéaire à valeurs dans~$\mathrm{Ker} f$.
On en déduit $\dim F = (\mathrm{codim}\Im f)(\dim\mathrm{Ker} f) = (\dim\mathrm{Ker} f)^2$.
}
}
