\uuid{yzMb}
\exo7id{6964}
\auteur{blanc-centi}
\organisation{exo7}
\datecreate{2014-04-08}
\isIndication{true}
\isCorrection{true}
\chapitre{Polynôme, fraction rationnelle}
\sousChapitre{Fraction rationnelle}

\contenu{
\texte{
Existe-t-il une fraction rationnelle $F$ telle que 
$$\big(F(X)\big)^2=(X^2+1)^3 \ \ ?$$
}
\indication{\'Ecrire $F=\frac{P}{Q}$ sous forme irréductible.}
\reponse{
\'Ecrivons $F(X)=\frac{P(X)}{Q(X)}$ avec $P$ et $Q$ deux polynômes premiers entre eux, avec $Q$ unitaire.
La condition $\big(F(X)\big)^2=(X^2+1)^3$ devient $P^2=(X^2+1)^3Q^2$. 
Ainsi $Q^2$ divise $P^2$. D'où $Q^2=1$, puisque $P^2$ et $Q^2$ sont premiers entre eux. 
Donc $Q=1$ (ou $-1$). Ainsi $F=P$ est un polynôme et $P^2=(X^2+1)^3$. 

En particulier $P^2$ est de degré $6$, donc $P$ doit être de degré 3. 
\'Ecrivons $P=aX^3+bX^2+cX+d$,
on développe l'identité $P^2=(X^2+1)^3$ :

{\small
$$\begin{array}{c}
X^6+3X^4+3X^2+1 = \\
a^2X^6 + 2abX^5 + (2ac+b^2)X^4 + (2ad+2bc)X^3 + (2bd+c^2)X^2 + 2cdX+d^2
 \end{array}
$$
}

On identifie les coefficients :
pour le coefficient de $X^6$, on a $a=\pm1$,
puis pour le coefficient de $X^5$, on a $b =0$ ;
pour le coefficient de $1$, on a $d=\pm 1$, 
puis pour le coefficient de $X$, on a $c=0$.
Mais alors le coefficient de $X^3$ doit vérifier $2ad+2bc=0$, ce qui est faux.

Ainsi aucun polynôme ne vérifie l'équation $P^2=(X^2+1)^3$, et 
par le raisonnement du début, aucune fraction non plus.
}
}
