\uuid{OyPd}
\exo7id{3392}
\auteur{quercia}
\organisation{exo7}
\datecreate{2010-03-09}
\isIndication{false}
\isCorrection{true}
\chapitre{Matrice}
\sousChapitre{Autre}

\contenu{
\texte{
Que dire des morphismes de groupe $\varphi : {GL_n(\R)} \to {\Z/p\Z}$~?
}
\reponse{
Soit $\varphi$ un tel morphisme. Alors pour toute matrice $M\in GL_n(\R)$
on a $\dot 0 = p\varphi(M) = \varphi(M^p)$, donc $\varphi$ s'annule
sur toute matrice qui est une puissance $p$-ème.
Notons $P(i,j,\alpha)$ la matrice de l'opération élémentaire $L_i \leftarrow L_i + \alpha L_j$,
qui est aussi la matrice de l'opération élémentaire $C_j \leftarrow C_j + \alpha C_i$.
Toute matrice $M\in GL_n(\R)$ peut être transformée, à l'aide de ces seules
opérations élémentaires, en une matrice $M' = \mathrm{diag}(1,\dots,1,\det(M))$
par une adaptation de l'algorithme de Gauss. Comme $P(i,j,\alpha) = P(i,j,\alpha/p)^p$
et $\det(M) = \pm(|\det(M)|^{1/p})^p$, on obtient~:
$\varphi(M) = \dot 0$ si $\det(M) > 0$ et $\varphi(M) = \varphi(\mathrm{diag}(1,\dots,1,-1)) = x$
si $\det(M) < 0$. Réciproquement, la fonction $\varphi$ ainsi définie est
effectivement un morphisme de groupe si et seulement si $2x = \dot 0$, soit
$x=\dot 0$ pour $p$ impair, et $x\in\{\dot 0,\dot q\}$ pour $p=2q$.
}
}
