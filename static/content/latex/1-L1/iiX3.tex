\uuid{iiX3}
\exo7id{122}
\auteur{bodin}
\organisation{exo7}
\datecreate{1998-09-01}
\isIndication{false}
\isCorrection{true}
\chapitre{Logique, ensemble, raisonnement}
\sousChapitre{Ensemble}

\contenu{
\texte{
Montrer par contraposition les assertions suivantes, $E$ \'etant
un ensemble :
}
\begin{enumerate}
    \item \question{$\forall A,B \in \mathcal{P}(E) \quad (A\cap B=A\cup B)\Rightarrow A=B$,}
\reponse{Tout d'abord de fa\c{c}on ``directe". Nous supposons
que $A$ et $B$ sont tels que $A\cap B = A \cup B$. Nous devons montrer 
que $A=B$. 

Pour cela \'etant donn\'e $x \in A$ montrons qu'il est aussi dans $B$.
Comme $x\in A$ alors $x \in A\cup B$ donc $x \in A \cap  B$ (car $A\cup B= A \cap B$). Ainsi $x \in B$. 

Maintenant nous prenons $x\in B$ et le m\^eme raisonnement implique $x \in A$.
Donc tout \'el\'ement de $A$ est dans $B$ et tout \'el\'ement de $B$ est dans $A$.
Cela veut dire $A=B$.}
    \item \question{$\forall A,B,C \in \mathcal{P}(E) \quad
(A\cap B=A\cap C \text{ et } A\cup B=A\cup C)\Rightarrow B=C$.}
\reponse{Ensuite, comme demand\'e, nous le montrons par contraposition.
Nous supposons que $A\not= B$ et nous devons montrer que 
$A\cap B \not= A\cup B$.

Si $A\not= B$ cela veut dire qu'il existe un \'el\'ement $x \in A\setminus B$
ou alors un \'el\'ement $x\in B \setminus A$. Quitte \`a \'echanger $A$ et $B$, nous supposons qu'il existe $x\in A \setminus B$. Alors $x \in A \cup B$ mais
$x \notin A\cap B$. Donc $A\cap B \not= A\cup B$.}
\end{enumerate}
}
