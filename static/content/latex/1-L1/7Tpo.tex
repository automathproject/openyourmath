\uuid{7Tpo}
\exo7id{3166}
\auteur{quercia}
\datecreate{2010-03-08}
\isIndication{false}
\isCorrection{false}
\chapitre{Polynôme, fraction rationnelle}
\sousChapitre{Autre}

\contenu{
\texte{
Soient $U,V \in  K[X]$ non constants. On pose $P_k = U^kV^{n-k}$.
Montrer que $(P_0,\dots,P_n)$ est libre \dots
}
\begin{enumerate}
    \item \question{lorsque $U \wedge V = 1$.}
    \item \question{lorsque $(U,V)$ est libre.}
\end{enumerate}
}
