\uuid{2gKg}
\exo7id{3385}
\auteur{quercia}
\datecreate{2010-03-09}
\isIndication{false}
\isCorrection{true}
\chapitre{Matrice}
\sousChapitre{Autre}

\contenu{
\texte{
Soient $A,B \in \mathcal{M}_n(K)$. Exprimer en fonction de $k$ le terme général de la
suite $(M_k)$ de matrices de $\mathcal{M}_n(K)$ définie par :
$\begin{cases} M_0 \text{ est donnée,}\cr M_{k+1} = AM_k + B. \cr\end{cases}$
}
\reponse{
$M_k = A^kM_0 + S_kB$ avec
$S_k = I + A + \dots + A^{k-1} = (I-A^k)(I-A)^{-1}$ si $I-A$ est inversible.
}
}
