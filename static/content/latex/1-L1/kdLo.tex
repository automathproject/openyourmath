\uuid{kdLo}
\exo7id{3122}
\auteur{quercia}
\organisation{exo7}
\datecreate{2010-03-08}
\isIndication{false}
\isCorrection{true}
\chapitre{Arithmétique dans Z}
\sousChapitre{Pgcd, ppcm, algorithme d'Euclide}

\contenu{
\texte{
Soient $a,m,n\ \in \N^{*}$, $a\ge 2$, et $d = (a^n - 1) \wedge (a^m - 1)$.
}
\begin{enumerate}
    \item \question{Soit $n = qm + r$ la division euclidienne de $n$ par $m$.
      D{\'e}montrer que $a^n \equiv a^r (\mathrm{mod}\, {a^m - 1})$.}
\reponse{$a^m - 1\mid (a^{qm} - 1)a^r = a^n - a^r$.}
    \item \question{En d{\'e}duire que $d = (a^r - 1) \wedge (a^m - 1)$,
      puis $d = a^{(n \wedge m)} - 1$.}
\reponse{$A\wedge(AQ+R) = A\wedge R$.
      Algorithme d'Euclide sur les exposants de $a$.}
    \item \question{A quelle condition $a^m - 1$ divise-t-il $a^n-1$ ?}
\reponse{ssi $m\mid n$.}
\end{enumerate}
}
