\uuid{Q67o}
\exo7id{3372}
\auteur{quercia}
\organisation{exo7}
\datecreate{2010-03-09}
\isIndication{false}
\isCorrection{true}
\chapitre{Matrice}
\sousChapitre{Autre}

\contenu{
\texte{
Soit $\alpha \in  K$, et $A,B \in \mathcal{M}_n(K)$. \'Etudier l'équation d'inconnue
$X \in \mathcal{M}_n(K)$ : $\alpha X + (\mathrm{tr} X)A = B$.
}
\reponse{
$(\alpha + \mathrm{tr} A)\mathrm{tr} X = \mathrm{tr} B$.

\leavevmode\vbox{
\halign{Si $#$ : \hfil &#\hfil\cr
\alpha(\alpha + \mathrm{tr} A) \ne 0 &solution unique :
  $X = \frac 1\alpha\left(B - \frac {\mathrm{tr} B}{\alpha + \mathrm{tr} A}A\right)$. \cr
\alpha = 0                   &solutions ssi $A$ et $B$ sont proportionnelles.                               \cr
\alpha + \mathrm{tr} A = 0           &solutions ssi $\mathrm{tr} B = 0$ :
  $X = \frac 1\alpha B + \lambda A$.               \cr}}
}
}
