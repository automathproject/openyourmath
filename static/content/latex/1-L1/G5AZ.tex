\uuid{G5AZ}
\exo7id{981}
\auteur{liousse}
\datecreate{2003-10-01}
\isIndication{true}
\isCorrection{true}
\chapitre{Espace vectoriel}
\sousChapitre{Base}

\contenu{
\texte{

}
\begin{enumerate}
    \item \question{Montrer que les vecteurs 
$v_1=(0,1,1)$, $v_2=(1,0,1)$ et $v_3=(1,1,0)$ forment une base de 
$\Rr^3$. Trouver les composantes du vecteur $w=(1,1,1)$ dans cette base $(v_1,v_2,v_3)$.}
    \item \question{Montrer que les vecteurs 
$v_1=(1,1,1)$, $v_2=(-1,1,0)$ et $v_3=(1,0,-1)$ forment une base de 
$\Rr^3$. Trouver les composantes du vecteur $e_1=(1,0,0)$,
$e_2=(0,1,0)$, $e_3=(0,0,1)$ et $w=(1,2,-3)$ dans cette base $(v_1,v_2,v_3)$.}
    \item \question{Dans $\Rr^3$, donner un exemple de famille libre qui n'est pas
g\'en\'eratrice.}
    \item \question{Dans $\Rr^3$, donner un exemple de famille g\'en\'eratrice qui n'est pas libre.}
\reponse{
Pour montrer que la famille $\{ v_1, v_2, v_3\}$ est une base nous allons 
montrer que cette famille est libre et génératrice.


  \begin{enumerate}
Montrons que la famille $\{ v_1, v_2, v_3\}$ est libre.
Soit une combinaison linéaire nulle $a v_1+b v_2 + c v_3 = 0$, nous devons montrer qu'alors
les coefficients $a,b,c$ sont nuls. Ici le vecteur nul est $0=(0,0,0)$

\begin{align*}
  &  a v_1+b v_2 + c v_3 = (0,0,0) \\
\iff & a (0,1,1) + b(1,0,1) + c(1,1,0) = (0,0,0) \\
\iff & (b+c,a+c,a+b)=(0,0,0) \\
\iff & \begin{cases}
       b+c = 0 \\
       a+c = 0 \\
       a+b = 0  \\
       \end{cases} 
\iff \begin{cases}
       a = 0 \\
       b = 0 \\
       c = 0  \\
       \end{cases} \\
\end{align*}
Ainsi les coefficients vérifient $a=b=c=0$, cela prouve que la famille est libre.
Montrons que la famille $\{ v_1, v_2, v_3\}$ est génératrice. Pour n'importe quel
vecteur $v=(x,y,z)$ de $\Rr^3$ on doit trouver $a,b,c\in\Rr$ tels que $a v_1+b v_2 + c v_3 = v$.
\begin{align*}
  &  a v_1+b v_2 + c v_3 = v \\
\iff & a (0,1,1) + b(1,0,1) + c(1,1,0) = (x,y,z) \\
\iff & (b+c,a+c,a+b)=(x,y,z) \\
\iff & \begin{cases}
       b+c = x \\
       a+c = y \quad (L_2) \\
       a+b = z \quad (L_3) \\
       \end{cases} 
\iff  \begin{cases}
       b+c = x \quad (L_1')\\
       a+c = y \\
       b-c = z-y  \quad(L_3')=(L_3-L_2)\\
       \end{cases} \\
\iff & \begin{cases}
       2b = x+z-y \quad (L_1'+L_3') \\
       a+c = y \\
       2c = x-(z-y) \quad  (L_1'-L_3')  \\
       \end{cases} 
\iff  \begin{cases}
       a = \frac12 (-x+y+z) \\
       b = \frac12 (x-y+z) \\
       c = \frac12 (x+y-z) \\
       \end{cases} \\
\end{align*}


Pour $a= \frac12 (-x+y+z)$, $b= \frac12 (x-y+z)$, $c = \frac12 (x+y-z)$ nous avons donc
la relation $av_1+bv_2+cv_3=(x,y,z)=v$. Donc la famille $\{ v_1, v_2, v_3\}$ est génératrice.
La famille est libre et génératrice  donc c'est une base.
Pour écrire  $w=(1,1,1)$ dans la base $(v_1,v_2,v_3)$  on peut résoudre le système correspondant
à la relation $a v_1+b v_2 + c v_3 = w$.
Mais en fait nous l'avons déjà résolu pour tout vecteur $(x,y,z)$, en particulier pour le vecteur
$(1,1,1)$ la solution est $a=\frac 12$, $b=\frac 12$, $c=\frac 12$.
Autrement dit $\frac 12 v_1+\frac 12 v_2 + \frac 12 v_3 = w$. Les coordonnées de $w$ dans la base $(v_1,v_2,v_3)$
sont donc $(\frac12,\frac12,\frac12)$.
}
\indication{\^Etre une base, c'est être libre et génératrice.
Chacune de ces conditions se vérifie par un système linéaire.}
\end{enumerate}
}
