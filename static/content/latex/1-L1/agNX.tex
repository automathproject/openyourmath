\uuid{agNX}
\exo7id{3234}
\auteur{quercia}
\organisation{exo7}
\datecreate{2010-03-08}
\isIndication{false}
\isCorrection{true}
\chapitre{Polynôme, fraction rationnelle}
\sousChapitre{Racine, décomposition en facteurs irréductibles}

\contenu{
\texte{
Soient $z_0,z_1,\dots,z_n \in \C$ tels que :
$\forall\ P \in \C_{n-1}[X]$, on a $P(z_0) = \frac {P(z_1) +\dots+ P(z_n)}n$.
\\
On note $\Phi(X) = \prod_{i=1}^n (X-z_i)$.
}
\begin{enumerate}
    \item \question{Calculer $\frac {\Phi(z_0)}{z_0-z_k}$.}
\reponse{$P = \frac \Phi{X-z_k}  \Rightarrow 
     \frac {\Phi(z_0)}{z_0-z_k}  = \frac {\Phi'(z_k)}n$.}
    \item \question{En d{\'e}duire que $\Phi(X) = \frac {(X-z_0)\Phi'(X)}n + \Phi(z_0)$.}
\reponse{Les deux membres sont {\'e}gaux en $z_0,\dots,z_n$.}
    \item \question{D{\'e}montrer que $z_1,\dots,z_n$ sont les sommets d'un polygone r{\'e}gulier de centre
     $z_0$.}
\reponse{D{\'e}composer $\Phi$ sur la base $\bigl((X-z_0)^k\bigr)$.}
    \item \question{R{\'e}ciproque ?}
\reponse{$\sum_k e^{2ikp/n} = 0$ pour $p < n  \Rightarrow $ OK.}
\end{enumerate}
}
