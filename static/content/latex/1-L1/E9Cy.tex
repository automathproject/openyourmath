\uuid{E9Cy}
\exo7id{336}
\auteur{bodin}
\datecreate{1998-09-01}
\isIndication{true}
\isCorrection{true}
\chapitre{Arithmétique dans Z}
\sousChapitre{Nombres premiers, nombres premiers entre eux}

\contenu{
\texte{
Soient $a,b$ des entiers sup\'erieurs ou \'egaux
\`a $1$. Montrer :
}
\begin{enumerate}
    \item \question{$(2^a-1) | (2^{ab}-1)$ ;}
\reponse{Nous savons que
$$x^b-1 = (x-1)(x^{b-1}+\cdots+x+1),$$
pour $x=2^a$ nous obtenons :
$$2^{ab}-1 = {(2^{a})}^b -1
     = (2^a-1)\left( 2^{a(b-1)}+\cdots+2^{a}+1 \right).$$
Donc $(2^a-1) | (2^{ab}-1)$.}
    \item \question{$2^p-1 \text{ premier}\ \   \Rightarrow \ \  p \text { premier }$ ;}
\reponse{Montrons la contraposée.
Supposons que $p$ ne soit pas premier. Donc $p=ab$ avec $1<p,q<a$.
Par la question précédente $2^a-1$ divise $2^p-1$
(et $1 < 2^a-1 < 2^p-1)$. Donc $2^p - 1$ n'est pas un nombre premier.}
    \item \question{$\pgcd(2^a-1,2^b-1) =  2^{\pgcd(a,b)}-1$.}
\reponse{Nous supposons $a \ge b$.
Nous allons montrer que faire l'algorithme d'Euclide pour le couple $(2^a-1,2^b-1)$
revient à faire l'algorithme d'Euclide pour $(a,b)$.
Tout d'abord rappellons la formule qui est à la base de l'algorithme d'Euclide :
$\pgcd(a,b) = \pgcd(a-b,b)$. 
Appliqué à $2^a-1$ et $2^b-1$ cela donne directement $\pgcd(2^a-1,2^b-1) = \pgcd(2^a-2^b,2^b-1)$.
Mais $2^a-2^b = 2^b(2^{a-b}-1)$ d'où
 $\pgcd(2^a-1,2^b-1) = \pgcd(2^b(2^{a-b}-1),2^b-1) = \pgcd(2^{a-b}-1,2^b-1)$.
La dernière égalité vient du fait $2^b$ et $2^b-1$ sont premiers entre eux (deux entiers consécutifs
sont toujours premiers entre eux).

Nous avons montrer : $\pgcd(2^a-1,2^b-1) =\pgcd(2^{a-b}-1,2^b-1)$.
Cette formule est à mettre en parallèle de $\pgcd(a,b) = \pgcd(a-b,b)$.
En itérant cette formule nous obtenons que si $a=bq+r$ alors :
 $\pgcd(2^a-1,2^b-1) = \pgcd(2^{a-bq}-1,2^b-1) = \pgcd(2^r-1,2^b-1)$
à comparer avec 
$\pgcd(a,b)=\pgcd(a-bq,b)=\pgcd(r,b)$.
Nous avons notre première étape de l'algorithme d'Euclide.
En itérant l'algorithme d'Euclide pour $(a,b)$,
nous nous arêtons au dernier reste non nul:
$\pgcd(a,b) = \pgcd(b,r) = \cdots = \pgcd(r_n,0)=r_n$.
Ce qui va donner pour nous
$\pgcd(2^a-1,2^b-1) = \pgcd(2^b-1,2^r-1) = \cdots = \pgcd(2^{r_n}-1,2^0-1) = 2^{r_n}-1$.

Bilan : $\pgcd(2^a-1,2^b-1) =  2^{\pgcd(a,b)}-1$.}
\indication{Pour 1. utiliser l'\'egalit\'e 
$$x^b-1 = (x-1)(x^{b-1}+\cdots+x+1).$$

Pour 2. raisonner par contraposition et utiliser la question 1.

La question 3. est difficile ! Supposer $a\ge b$.
Commencer par 
montrer que $\pgcd(2^a-1,2^b-1) = \pgcd(2^a-2^b,2^b-1) = \pgcd(2^{a-b}-1,2^b - 1)$.
Cela vour permettra de comparer l'agorithme d'Euclide pour le calcul de $\pgcd(a,b)$ avec
l'algorithme d'Euclide pour le calcul de  $\pgcd(2^a-1,2^b-1)$.}
\end{enumerate}
}
