\uuid{tqAb}
\exo7id{446}
\auteur{cousquer}
\organisation{exo7}
\datecreate{2003-10-01}
\isIndication{true}
\isCorrection{true}
\chapitre{Polynôme, fraction rationnelle}
\sousChapitre{Fraction rationnelle}

\contenu{
\texte{
D\'ecomposition en \'el\'ements simples
 $\displaystyle\Phi={2x^5-8x^3+8x^2-4 x+1\over x^3(x-1)^2}.$
}
\indication{Il y a une partie enti\`ere qui vaut $2$.}
\reponse{
La
division suivant les puissances d\'ecroissantes\linebreak %%%%
donne~: $\Phi=2+\Phi_1$ avec
$$
\Phi_1={4x^4-10x^3+8x^2-4x+1\over x^3(x-1)^2}=
{A\over x^3}+{B\over x^2}+{C\over x}+{D\over(x-1)^2}+{E\over x-1}.
$$
Faire remarquer que la m\'ethode de l'exercice pr\'ec\'edent permettrait
d'obtenir facilement $A$ et~$D$ par multiplication par $x^3$ et par $(x-1)^2$,
mais qu'il resterait encore 3 coefficients \`a d\'eterminer. \\Il y a ici une
m\'ethode plus efficace~: effectuer la division suivant les puissances
croissantes, \`a l'ordre~3 
(qui est l'exposant du facteur $x$) du num\'erateur
$1-4x+8x ^2-10x^3+4x^4$ par $(x-1)^2$, ou plut\^ot par $1-2x+x^2$~:
\begin{equation}
\label{eq22}
1-4x+8x ^2-10x^3+4x^4=(1-2x+x^2)\times(1-2x+3x^2)+(-2x^3+x^4).
\end{equation}
En divisant les deux membres de~(\ref{eq22}) par $x^3(x-1)^2$, 
on obtient $A$, $B$
et~$C$ d'un seul coup~:
$$
\Phi_1={1\over x^3}-{2\over x^2}+{3\over x}+{x-2\over(x-1)^2}.
$$
Le calcul de $D$ et $E$ est alors imm\'ediat par d\'ecomposition de
${x-2\over(x-1)^2}$~: m\'ethode de l'exercice pr\'ec\'edent, ou division
suivant les puissances d\'ecroissantes de $x-2$ par $x-1$~: $x-2=(x-1)-1$.
$${2x^5-8x^3+8x^2-4 x+1\over x^3(x-1)^2}=2+
{1\over x^3}-{2\over x^2}+{3\over x}-{1\over(x-1)^2}+{1\over x-1}.$$

\smallskip
Remarque~: cette m\'ethode est efficace pour un exposant assez grand (en gros
\`a partir de~3). Elle peut \^etre utilis\'ee pour une fraction du type 
${P(x)\over(x-a)^nQ(x)}$, mais il faut commencer par le changement de variable
$u=x-a$ avant de faire la division, puis bien entendu revenir ensuite \`a la
variable $x$.
}
}
