\uuid{2XBn}
\exo7id{397}
\auteur{legall}
\organisation{exo7}
\datecreate{2003-10-01}
\isIndication{false}
\isCorrection{false}
\chapitre{Polynôme, fraction rationnelle}
\sousChapitre{Racine, décomposition en facteurs irréductibles}

\contenu{
\texte{
Soit $P(X)= a_nX^n+\cdots + a_0$ un polyn\^ome \`a 
coefficients entiers premiers entre eux (c'est \`a dire tels que les 
seuls diviseurs
communs \`a tous les $a_i$ soient $-1$ et $1$). Montrer que si 
$r=\dfrac{p}{q}$ avec $p$ et $q$ premiers entre eux est une racine 
rationnelle de $P$
alors $p$ divise $a_0$ et $q$ divise $a_n.$
}
}
