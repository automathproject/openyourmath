\uuid{SjTZ}
\exo7id{7186}
\auteur{megy}
\organisation{exo7}
\datecreate{2019-07-17}
\isIndication{false}
\isCorrection{false}
\chapitre{Logique, ensemble, raisonnement}
\sousChapitre{Ensemble}

\contenu{
\texte{
Soit $E$ un ensemble et  $\mathcal O$ une partie de $\mathcal P(E)$. On dit que $\mathcal O$ est une \emph{topologie sur $E$} si les conditions suivantes sont vérifiées
\begin{itemize}
\item $\mathcal O$ est stable par intersection finie, autrement dit : pour tout $n\in \N^*$ et toute famille $U_1, \cdots U_n$ d'éléments de $\mathcal O$, on a $\bigcap_{i=1}^n U_i\in \mathcal O$.
\item $\mathcal O$ est stable par union quelconque, autrement dit : pour tout ensemble $I$ et toute famille $(U_i)_{i\in I}$ d'éléments de $\mathcal O$, $\bigcup_{i\in I}U_i \in \mathcal O$.
\item Les parties $\emptyset$ et $E$ sont des éléments de $\mathcal O$.
\end{itemize}
}
\begin{enumerate}
    \item \question{Montrer que $\mathcal O_1=\{\emptyset, E\}$ et $\mathcal O_2=\mathcal P(E)$ sont des topologies sur $E$.}
    \item \question{Montrer que 
\[ \mathcal O_3 = \left\{U\in \mathcal P(E)\:\middle|\: U=\emptyset \text{ ou }{}^cU\text{ est fini}\right\}
\]
est une topologie sur $E$.}
    \item \question{Combien de topologies différentes y a-t-il si $E$ est l'ensemble vide ? S'il n'a qu'un seul élément ? Deux éléments ? Trois éléments ?}
\end{enumerate}
}
