\uuid{lfw6}
\exo7id{7010}
\auteur{megy}
\datecreate{2016-04-29}
\isIndication{false}
\isCorrection{true}
\chapitre{Nombres complexes}
\sousChapitre{Géométrie}

\contenu{
\texte{
Soient $A$ et $B$ deux points distincts d'affixes $a$ et $b$, et $\theta \in \R$. Déterminer l'ensemble des points $M$ d'affixe $z$ tels que 
\[ \mathrm{Arg} \frac{z-b}{z-a} \equiv \theta [\pi].\]
}
\reponse{
Si $\theta \equiv 0 [\pi]$, c'est la droite $(AB)$. Sinon, c'est un cercle dont $[AB]$ est une corde, par le théorème de l'angle inscrit.

Si $\theta \equiv \pi/2 [\pi]$, c'est le cercle de diamètre $[AB]$.

Sinon, par le théorème de l'angle au centre, le centre de ce cercle a pour affixe $\frac{a+b}{2} \pm i \frac{b-a}{2 \tan \theta}$, selon si $\theta$ est aigu ou obtus.
}
}
