\uuid{LIN2}
\exo7id{5590}
\auteur{rouget}
\organisation{exo7}
\datecreate{2010-10-16}
\isIndication{false}
\isCorrection{true}
\chapitre{Application linéaire}
\sousChapitre{Morphismes particuliers}

\contenu{
\texte{
\label{ex:rou28}
Soit $E$ un espace de dimension finie. Montrer que la trace d'un projecteur est son rang.
}
\reponse{
Soit $p$ un projecteur de $E$. Si $p=0$, $\text{Tr}(p)=\text{rg}(p)=0$ et si $p=Id_E$, $\text{Tr}(p)=\text{rg}(p)=n$.

Dorénavant, $p$ est un projecteur de rang $r\in\llbracket1,n-1\rrbracket$. On choisit une base de $E$ $\mathcal{B}$ adaptée à la décomposition $E=\text{Im}(p)\oplus\text{Ker}(p)$. Dans cette base, la matrice de $p$ s'écrit $\left(
\begin{array}{cccccc}
1&0&\ldots& &\ldots&0\\
0&\ddots&\ddots& & &\vdots\\
\vdots&\ddots&1& & & \\
 & & &$0$&\ddots&\vdots\\
\vdots& & &\ddots&\ddots&0\\
0&\ldots& &\ldots&0&0
\end{array}
\right)$ où le nombre de $1$ est $\text{dim}(\text{Im}(p))=r$. Mais alors $\text{Tr}(p)=r$.

\begin{center}
\shadowbox{
En dimension finie, la trace d'un projecteur est son rang.
}
\end{center}
}
}
