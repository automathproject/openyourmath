\uuid{OPyT}
\exo7id{3042}
\auteur{quercia}
\datecreate{2010-03-08}
\isIndication{false}
\isCorrection{true}
\chapitre{Logique, ensemble, raisonnement}
\sousChapitre{Relation d'équivalence, relation d'ordre}

\contenu{
\texte{
Soit $E$ ordonn{\'e} tel que toute partie non vide et major{\'e}e admet une borne sup{\'e}rieure.
Montrer que toute partie non vide et minor{\'e}e admet une borne inf{\'e}rieure.
}
\reponse{
Soit $A$ non vide et minor{\'e}e, et $B = \{$minorants de $A \}$.\par
$B$ n'est pas vide et est major{\'e}e par $A$ donc $\beta = \sup(B)$ existe.\par
Soit $a \in A$ : $\forall\ b \in B,\ b \le a$ donc $\beta \le a$.\par
Par cons{\'e}quent, $\beta$ minore $A$, donc $\beta = \max(B)$.
}
}
