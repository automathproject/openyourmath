\uuid{h3fz}
\exo7id{1064}
\auteur{ridde}
\organisation{exo7}
\datecreate{1999-11-01}
\isIndication{true}
\isCorrection{true}
\chapitre{Matrice}
\sousChapitre{Propriétés élémentaires, généralités}

\contenu{
\texte{
Que peut-on dire d'une matrice $A \in \mathcal{M}_n (\Rr)$ qui vérifie $\text{tr} (A \ {}^{t}\!{A}) = 0$ ?
}
\indication{Appliquer la formule du produit pour calculer les coefficients diagonaux de $A\ {}^{t}\!{A}$}
\reponse{
Notons $A=(a_{ij})$, notons $B = {}^{t}\!{A}$ si les coefficients sont $B=(b_{ij})$ 
alors par définition de la transposée on a $b_{ij}= a_{ji}$.

Ensuite notons $C = A \times B$ alors par définition du produit de matrices 
le coefficients $c_{ij}$ de $C$ s'obtient par la formule :
$$c_{ij} = \sum_{k=1}^n a_{ik}b_{kj}.$$

Appliquons ceci avec $B = {}^{t}\!{A}$
$$c_{ij} = \sum_{k=1}^n a_{ik}b_{kj} = \sum_{k=1}^n a_{ik}a_{jk}.$$
Et pour un coefficient de la diagonale on a $i=j$ donc 
$$c_{ii} =  \sum_{k=1}^n a_{ik}^2.$$

La trace étant la somme des coefficients sur la diagonale on a :
$$\text{tr} (A \ {}^{t}\!{A}) = \text{tr} (C) 
= \sum_{i=1}^n c_{ii} =  \sum_{i=1}^n \sum_{k=1}^n a_{ik}^2 = \sum_{1\le i,k \le n} a_{ik}^2.$$

Si on change l'indice $k$ en $j$ on obtient 
$$\text{tr} (A \ {}^{t}\!{A})  = \sum_{1\le i,j \le n} a_{ij}^2.$$

Donc cette trace vaut la somme des carrés de tous les coefficients.

Conséquence : si $\text{tr} (A \ {}^{t}\!{A}) = 0$ alors la somme des carrés 
$\sum_{1\le i,j \le n} a_{ij}^2$ est nulle 
donc chaque carré $a_{ij}^2$ est nul. Ainsi $a_{ij}=0$ (pour tout $i,j$) autrement dit $A$ est la matrice nulle.
}
}
