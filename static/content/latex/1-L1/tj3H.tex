\uuid{tj3H}
\exo7id{156}
\auteur{bodin}
\organisation{exo7}
\datecreate{1998-09-01}
\isIndication{false}
\isCorrection{true}
\chapitre{Logique, ensemble, raisonnement}
\sousChapitre{Récurrence}

\contenu{
\texte{

}
\begin{enumerate}
    \item \question{Dans le plan, on consid\`ere trois droites $\Delta_{1},\Delta_{2},\Delta_{3}$ formant un
``vrai'' triangle : elles ne sont pas concourantes, et il n'y en a pas deux parall\`eles.
Donner le nombre $R_{3}$ de r\'egions (zones blanches) d\'ecoup\'ees par ces trois droites.}
    \item \question{On consid\`ere quatre droites $\Delta_{1},\ldots,\Delta_{4}$, telles qu'il n'en existe pas
trois concourantes, ni deux parall\`eles. Donner le nombre $R_{4}$ de r\'egions d\'ecoup\'ees par
ces quatre droites.}
    \item \question{On consid\`ere $n$ droites $\Delta_{1},\ldots,\Delta_{n}$, telles qu'il n'en existe pas
trois concourantes, ni deux parall\`eles. Soit $R_{n}$ le nombre de r\'egions d\'elimit\'ees par
$\Delta_{1}\ldots\Delta_{n}$, et $R_{n-1}$ le nombre de r\'egions d\'elimit\'ees par
$\Delta_{1}\ldots\Delta_{n-1}$. Montrer que $R_{n}=R_{n-1}+n$.}
    \item \question{Calculer par r\'ecurrence le nombre de r\'egions d\'elimit\'ees par $n$ droites en position
g\'en\'erale, c'est-\`a-dire telles qu'il n'en existe pas trois concourantes ni deux parall\`eles.}
\reponse{
Montrons par r\'ecurrence sur $n \geqslant 1$ la proposition suivante :
$$\mathcal{H}_n :  \quad n \text{\  droites en position g\'en\'erale d\'ecoupent le plan en \ } R_n = \frac{n(n+1)}{2}+1
\text{\  r\'egions.}$$

\begin{itemize}
\item[$\bullet$] pour $n=1$ alors une droite divise le plan en deux r\'egions. $\mathcal{H}_1$ est vraie.

\item[$\bullet$] Soit $n\geqslant 2$ et supposons que $\mathcal{H}_{n-1}$ soit vraie, et montrons $\mathcal{H}_n$.
Soient $\Delta_1,\ldots,\Delta_n$ $n$ droites en position
g\'en\'erale, la droite $\Delta_n$ rencontre les droites
$\Delta_1,\ldots,\Delta_{n-1}$ en $n-1$ points, donc $\Delta_n$
traverse (et d\'ecoupe en deux) $n$ r\'egions du d\'ecoupage
$\Delta_1,\ldots,\Delta_{n-1}$. Le d\'ecoupage par $\Delta_n$
donne donc la relation $R_n=R_{n-1}+n$.

Or par hypoth\`ese de r\'ecurrence $\mathcal{H}_{n-1}$ : $R_{n-1}
= \frac{(n-1)n}{2}+1$ donc
$$  R_n = R_{n-1}+n =  \frac{(n-1)n}{2}+1+n=\frac{n(n+1)}{2}+1 $$
Et $\mathcal{H}_n$ est vraie.\\
Ainsi $\forall n\in\Nn^* \quad \mathcal{H}_{n-1}\Rightarrow
\mathcal{H}_{n}$.

\item[$\bullet$] Conclusion :  par r\'ecurrence on a montr\'e que $\mathcal{H}_n$ est vraie quelque soit $n \geqslant 1$.

\end{itemize}
}
\end{enumerate}
}
