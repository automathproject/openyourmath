\uuid{pvVs}
\exo7id{5110}
\auteur{rouget}
\organisation{exo7}
\datecreate{2010-06-30}
\isIndication{false}
\isCorrection{true}
\chapitre{Injection, surjection, bijection}
\sousChapitre{Injection, surjection}

\contenu{
\texte{
\label{exo:suprou8}
Montrer que~:~($g\circ f\;\mbox{injective}\Rightarrow f\;\mbox{injective}$) et ($g\circ
f\;\mbox{surjective}\Rightarrow g\;\mbox{surjective}$).
}
\reponse{
Soit $(x_1,x_2)\in E^2$.

\begin{align*}
f(x_1)=f(x_2)&\Rightarrow g(f(x_1))=g(f(x_2))\;(\mbox{car}\;g\;\mbox{est une application})\\
 &\Rightarrow x_1=x_2\;(\mbox{car}\;g\circ f\;\mbox{est injective}).
\end{align*}
On a montré que $\forall(x_1,x_2)\in E^2,\;f(x_1)=f(x_2)\Rightarrow x_1=x_2$, et donc $f$ est injective.
Soit $y\in H$. Puisque $g\circ f$ est surjective, il existe un élément $x$ dans $E$ tel que $g(f(x))=y$. En
posant $z=f(x)\in G$, on a trouvé $z$ dans $G$ tel que $g(z)=y$. On a montré~:~$\forall y\in H,\;\exists z\in G/\;g(z)=
y$, et donc $g$ est surjective.
}
}
