\uuid{uvTx}
\exo7id{3393}
\auteur{quercia}
\organisation{exo7}
\datecreate{2010-03-10}
\isIndication{false}
\isCorrection{true}
\chapitre{Matrice}
\sousChapitre{Propriétés élémentaires, généralités}

\contenu{
\texte{
Soit $A = \begin{pmatrix} 1&2&3\cr 2&3&4\cr 3&4&5 \cr\end{pmatrix}$.
}
\begin{enumerate}
    \item \question{Montrer que l'équation en $X$ : $AX = B$, $X,B \in \mathcal{M}_{3,n}(K)$,
    a des solutions si et seulement si
   les colonnes de $B$ sont des progressions arithmétiques
   (traiter d'abord le cas $n=1$).}
    \item \question{Résoudre $AX = \begin{pmatrix}3 &3 \cr 4 &5 \cr 5 &7 \cr\end{pmatrix}$.}
\reponse{
$X = \begin{pmatrix} \alpha & 1 + \beta \cr -2\alpha &1 - 2\beta \cr
                            1 + \alpha & \beta \cr\end{pmatrix}$.
}
\end{enumerate}
}
