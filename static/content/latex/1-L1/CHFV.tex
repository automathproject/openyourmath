\uuid{CHFV}
\exo7id{7411}
\auteur{mourougane}
\datecreate{2021-08-10}
\isIndication{false}
\isCorrection{true}
\chapitre{Matrice}
\sousChapitre{Matrice et application linéaire}

\contenu{
\texte{
Soit $f : \Rr^3 \rightarrow \Rr^3$ l'application linéaire définie par
$$f(x,y,z) = (4x + y + z, 4x + 7y +2z, -6x -6y-z).$$
}
\begin{enumerate}
    \item \question{\'Ecrire la matrice $A$ de $f$ dans la base canonique $\mathcal{B}$ de $\Rr^3$.}
\reponse{$$A = [f]_{\mathcal{B}}^{\mathcal{B}} = \left( {\begin{array}{ccc}
 4 & 1 & 1 \\
 4 & 7 & 2 \\
 -6 & -6 & -1 \\
 \end{array} } \right)$$}
    \item \question{Montrer que les vecteurs $v_1 = (1,0,-2)$, $v_2 = (1,-1,0)$ et $v_3 = (0,-1,1)$ forment une base de $\Rr^3$ que l'on note $\mathcal{B}'$.}
\reponse{Pour montrer que les vecteurs $v_1$, $v_2$ et $v_3$ forment un système libre de $\Rr^3$, on considère $\lambda_1,~\lambda_2,~\lambda_3 \in \Rr$ tels que 
	$$\lambda_1 v_1 + \lambda_2 v_2 + \lambda_3 v_3 = 0$$
et on veut montrer qu'alors nécessairement $\lambda_1 = \lambda_2 = \lambda_3 = 0$. Cela revient à montrer que le système
	$$\left( {\begin{array}{ccc}
 1 & 1 & 0 \\
 0 & -1 & -1 \\
 -2 & 0 & 1 \\
 \end{array} } \right) 
\left( {\begin{array}{c}
\lambda_1 \\
\lambda_2 \\
\lambda_3
 \end{array} } \right) = 
 \left( {\begin{array}{c}
0 \\
0 \\
0
 \end{array} } \right)$$ 
possède pour unique solution $\lambda_1 = \lambda_2 = \lambda_3 = 0$.
On applique alors l'algorithme de Gauss à la matrice 
$$\left( {\begin{array}{ccc}
 1 & 1 & 0 \\
 0 & -1 & -1 \\
 -2 & 0 & 1 \\
 \end{array} } \right) $$
 Après avoir échelonné complètement cette matrice, on obtient le système équivalent suivant
 $$\left( {\begin{array}{ccc}
 1 & 0 & 0 \\
 0 & 1 & 0 \\
 0 & 0 & 1 \\
 \end{array} } \right)
\left( {\begin{array}{c}
\lambda_1 \\
\lambda_2 \\
\lambda_3
 \end{array} } \right) = 
 \left( {\begin{array}{c}
0 \\
0 \\
0
 \end{array} } \right)$$ 
 qui a pour unique solution $\lambda_1 = \lambda_2 = \lambda_3 = 0$. Les vecteurs $v_1$, $v_2$ et $v_3$ forment donc un système libre de $3$ éléments dans l'espace vectoriel $\Rr^3$ qui est de dimension $3$, donc ils forment un système également générateur de $\Rr^3$ et c'est une base de $\Rr^3$.}
    \item \question{Déterminer la matrice de passage $[Id_{R^3}]_{\mathcal{B}'}^{\mathcal{B}}$ de $\mathcal{B}$ à $\mathcal{B}'$, et la matrice de passage $[Id_{R^3}]_{\mathcal{B}}^{\mathcal{B}'}$ de $\mathcal{B}'$ à $\mathcal{B}$.}
\reponse{La matrice de passage $P := [Id_{R^3}]_{\mathcal{B}'}^{\mathcal{B}}$ de $\mathcal{B}$ à $\mathcal{B}'$ est 
	$$P = 
\left( {\begin{array}{ccc}
 1 & 1 & 0 \\
 0 & -1 & -1 \\
 -2 & 0 & 1 \\
 \end{array} } \right)$$
 La matrice de passage $[Id_{\Rr^3}]_{\mathcal{B}}^{\mathcal{B}'}$ de $\mathcal{B}'$ à $\mathcal{B}$ est l'inverse de la matrice de passage $[Id_{\Rr^3}]_{\mathcal{B}'}^{\mathcal{B}}$ de $\mathcal{B}$ à $\mathcal{B}'$ i.e. $[Id_{R^3}]_{\mathcal{B}}^{\mathcal{B}'} = P^{-1}$. Pour calculer l'inverse de $P$, on applique l'algorithme de Gauss à la matrice 
$$\left( {\begin{array}{ccc | ccc}
 1 & 1 & 0 & 1 & 0 & 0 \\
 0 & -1 & -1 & 0 & 1 & 0 \\
 -2 & 0 & 1 & 0 & 0 & 1 \\
 \end{array} } \right)$$
 Pour ce faire, on reprend les opérations effectuées sur la matrice $P$ lors de la question précédente pour obtenir 
$$\left( {\begin{array}{ccc | ccc}
 1 & 0 & 0 & -1 & -1 & -1 \\
 0 & 1 & 0 & 2 & 1 & 1 \\
 0 & 0 & 1 & -2 & -2 & -1 \\
 \end{array} } \right)$$ 
et ainsi 
$$[Id_{R^3}]_{\mathcal{B}}^{\mathcal{B}'} = P^{-1} = 
\left( {\begin{array}{ccc}
  -1 & -1 & -1 \\
 2 & 1 & 1 \\
 -2 & -2 & -1 \\
 \end{array} } \right)$$}
    \item \question{Déterminer la matrice $M$ de $f$ dans la base $\mathcal{B}'$.}
\reponse{On utilise la formule 
 $$[f]_{\mathcal{B}'}^{\mathcal{B}'} = [Id_{\Rr^3}]_{\mathcal{B}}^{\mathcal{B}'} [f]_{\mathcal{B}}^{\mathcal{B}} [Id_{\Rr^3}]_{\mathcal{B}'}^{\mathcal{B}}$$
 et on a alors 
\begin{equation*}
M = P^{-1}A P = 
 \left( {\begin{array}{ccc}
 2 & 0 & 0 \\
 0 & 3 & 0 \\
 0 & 0 & 5 \\
 \end{array} } \right)\end{equation*}}
    \item \question{En déduire $A^n$ pour tout $n \in \mathbb{N}^{*}$.}
\reponse{On a
 $$A = P M P^{-1}$$
 donc pour tout $n \in \mathbb{N}^*$
 $$A^n = P M^n P^{-1} = 
 \left( {\begin{array}{ccc}
 1 & 1 & 0 \\
 0 & -1 & -1 \\
 -2 & 0 & 1 \\
 \end{array} } \right) 
 \left( {\begin{array}{ccc}
  2^n & 0 & 0 \\
 0 & 3^n & 0 \\
 0 & 0 & 5^n \\
 \end{array} } \right) 
 \left( {\begin{array}{ccc}
  -1 & -1 & -1 \\
 2 & 1 & 1 \\
 -2 & -2 & -1 \\
 \end{array} } \right) $$
 $$= 
 \left( {\begin{array}{ccc}
  -2^n + 2.3^n & -2^n + 3^n & -2^n + 3^n \\
 -2.3^n + 2.5^n & -3^n + 2.5^n & -3^n + 5^n \\
 2.2^n - 2.5^n & 2.2^n - 2.5^n & 2.2^n - 5^n \\
 \end{array} } \right)$$}
\end{enumerate}
}
