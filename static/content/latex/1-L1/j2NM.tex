\uuid{j2NM}
\exo7id{366}
\auteur{bodin}
\datecreate{1998-09-01}
\isIndication{false}
\isCorrection{true}
\chapitre{Polynôme, fraction rationnelle}
\sousChapitre{Division euclidienne}

\contenu{
\texte{
Effectuer la division selon les puissances
croissantes de :
$$X^{4}+X^{3}-2X+1\text{ par }X^{2}+X+1 \text{ \`a l'ordre }2.$$
}
\reponse{
$X^{4}+X^{3}-2X+1 = (X^{2}+X+1)(2X^{2}-3X+1)+X^{3}(2-X)$.
}
}
