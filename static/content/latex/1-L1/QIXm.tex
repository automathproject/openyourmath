\uuid{QIXm}
\exo7id{954}
\auteur{cousquer}
\organisation{exo7}
\datecreate{2003-10-01}
\isIndication{true}
\isCorrection{true}
\chapitre{Application linéaire}
\sousChapitre{Image et noyau, théorème du rang}

\contenu{
\texte{
Soit $E$ un espace vectoriel de dimension $3$, $\{e_1,e_2,e_3\}$ une base
de $E$, et $t$ un param\`etre r\'eel. \\
D\'emontrer que la donn\'ee de
$\left\{
\begin{array}{rcl}
    \phi(e_1) & = & e_1+e_2  \\
    \phi(e_2) & = & e_1-e_2  \\
    \phi(e_3) & = & e_1+t e_3
\end{array}\right.$
d\'efinit une application lin\'eaire
$\phi$ de $E$ dans $E$. \'Ecrire le transform\'e du vecteur 
$x=\alpha_1e_1+\alpha_2e_2+\alpha_3e_3$. Comment choisir $t$ pour que 
$\phi$ soit injective ? surjective ?
}
\indication{$t=0$ est un cas à part.}
\reponse{
Comment est d\'efinie $\phi$ \`a partir de la d\'efinition sur les \'el\'ements de la base ?
Pour  $x\in E$ alors $x$ s'\'ecrit dans la base  $\{e_1,e_2,e_3\}$, $x=\alpha_1e_1+\alpha_2e_2+\alpha_3e_3$. 
Et $\phi$ est d\'efinie sur $E$ par la formule
$$\phi(x)=\alpha_1 \phi(e_1) + \alpha_2 \phi(e_2) + \alpha_3 \phi(e_3).$$
Soit ici :
$$\phi(x) = (\alpha_1+\alpha_2+\alpha_3) e_1 + (\alpha_1-\alpha_2)e_2 + t\alpha_3e_3.$$

Cette d\'efinition rend automatiquement $\phi$ lin\'eaire (v\'erifiez-le si vous n'\^etes pas convaincu !).
On cherche \`a savoir si $\phi$ est injective.
Soit $x\in E$ tel que $\phi(x)=0$ donc
$(\alpha_1+\alpha_2+\alpha_3) e_1 + (\alpha_1-\alpha_2)e_2 + t\alpha_3e_3=0$. Comme $\{e_1,e_2,e_3\}$ est une base alors tous les coefficients sont nuls :
$$\alpha_1+\alpha_2+\alpha_3=0, \quad \alpha_1-\alpha_2=0, \quad  t\alpha_3 = 0.$$
Si $t \neq 0$ alors en résolvant le syst\`eme on obtient $\alpha_1=0$, $\alpha_2=0$,
$\alpha_3=0$. Donc $x=0$ et $\phi$ est injective.

Si $t =0$, alors $\phi$ n'est pas injective, en résolvant le m\^eme syst\`eme on obtient
des solutions non triviales, par exemple $\alpha_1=1$, $\alpha_2=1$, $\alpha_3=-2$.
Donc pour $x= e_1+e_2-2e_3$ on obtient $\phi(x)=0$.
Pour la surjectivité on peut soit faire des calculs, soit appliquer la formule du rang. 
Examinons cette deuxi\`eme m\'ethode. $\phi$ est surjective si et seulement si 
la dimension de $\Im \phi$ est \'egale
\`a la dimension de l'espace d'arriv\'ee (ici $E$ de dimension $3$).
Or on a une formule pour $\dim \Im \phi$ :
$$\dim \ker \phi + \dim \Im \phi = \dim E.$$
Si $t \neq 0$, $\phi$ est injective donc $\ker \phi = \{0\}$ est de dimension $0$.
Donc $\dim \Im \phi =3$ et $\phi$ est surjective.

Si $t = 0$ alors $\phi$ n'est pas injective donc $\ker \phi$ est de dimension au moins $1$
(en fait $1$ exactement), donc  $\dim \Im \phi \leqslant 2$. Donc $\phi$ n'est pas surjective.


On remarque que $\phi$ est injective si et seulement si elle est surjective.
Ce qui est un r\'esultat du cours pour les applications ayant l'espace 
de d\'epart et d'arriv\'ee de m\^eme dimension (finie).
}
}
