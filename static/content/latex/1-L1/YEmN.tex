\uuid{YEmN}
\exo7id{3140}
\auteur{quercia}
\datecreate{2010-03-08}
\isIndication{false}
\isCorrection{true}
\chapitre{Arithmétique dans Z}
\sousChapitre{Nombres premiers, nombres premiers entre eux}

\contenu{
\texte{
On rappelle que si $p$ est premier et $n\wedge p = 1$, alors
$n^{p-1} \equiv 1(\mathrm{mod}\, p)$.
}
\begin{enumerate}
    \item \question{Soit $n\in \N$ et $p\ge 3$ un diviseur premier de $n^2+1$.
    Montrer que $p\equiv 1(\mathrm{mod}\, 4)$.}
    \item \question{En d{\'e}duire qu'il y a une infinit{\'e} de nombres premiers de la forme $4k+1$.}
\reponse{
$(-1)^{(p-1)/2}\equiv 1(\mathrm{mod}\, p)$.
}
\end{enumerate}
}
