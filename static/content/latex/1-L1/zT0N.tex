\uuid{zT0N}
\exo7id{3044}
\auteur{quercia}
\organisation{exo7}
\datecreate{2010-03-08}
\isIndication{false}
\isCorrection{false}
\chapitre{Logique, ensemble, raisonnement}
\sousChapitre{Relation d'équivalence, relation d'ordre}

\contenu{
\texte{
Un treillis est un ensemble ordonn{\'e} $E$ dans lequel
pour tous $x,y \in E,\ \sup(x,y)$ et $\inf(x,y)$ existent.
Soit $E$ un treillis.
}
\begin{enumerate}
    \item \question{Montrer que $\sup$ et $\inf$ sont des op{\'e}rations associatives.}
    \item \question{A quelle condition ont-elles des {\'e}l{\'e}ments neutres ?}
    \item \question{Montrer que :
        \begin{align*}
          &\forall\ x,y \in E,& &\sup\bigl(x,\inf(x,y)\bigr) =
          \inf\bigl(x,\sup(x,y)\bigr) = x,\\
          &\forall\ x,y,z \in E,& &x \le z \Rightarrow
          \sup\bigl(x,\inf(y,z)\bigr) \le
          \inf\bigl(\sup(x,y),z\bigr),\\
          &\forall\ x,y,z \in E,& &\inf\bigl(x,\sup(y,z)\bigr) \ge
          \sup\bigl(\inf(x,y),\inf(x,z)\bigr).
    \end{align*}}
\end{enumerate}
}
