\uuid{rCp0}
\exo7id{4924}
\auteur{quercia}
\datecreate{2010-03-17}
\isIndication{false}
\isCorrection{true}
\chapitre{Conique}
\sousChapitre{Hyperbole}

\contenu{
\texte{
Soit $\cal H$ une hyperbole équilatère et $ABC$ un triangle dont les
    sommets appartiennent à $\cal H$. Montrer que l'orthocentre, $H$, du triangle
    appartient aussi à $\cal H$. Comparer $H$ et le point $Q$ où le cercle
    circonscrit à $ABC$ recoupe $\cal H$.
}
\reponse{
On se ramène à une hyperbole d'équation $xy = 1$.
	     Soient $A=\Bigl(a,\frac1a\Bigr)$, $B=\Bigl(b,\frac1b\Bigr)$, $C=\Bigl(c,\frac1c\Bigr)$.
	     Alors $H=\Bigl(-\frac1{abc},-abc\Bigr) \in \cal H$.\par
	     L'équation du cercle circonscrit à $ABC$ est :
	     $x^2+y^2+\alpha x + \beta y + \gamma = 0$ et les points communs
	     au cercle et à $\cal H$ vérifient donc :
	     $x^4 + \alpha x^3 + \gamma x^2 + \beta x + 1 = 0$.
	     On conna\^\i t 3 racines : $x=a,b,c$ donc la quatrième est
	     $q=\frac1{abc}$ ce qui prouve que $Q$ et $H$ sont symétriques par
	     rapport à $O$.
}
}
