\uuid{wQvD}
\exo7id{7113}
\auteur{megy}
\datecreate{2017-01-21}
\isIndication{true}
\isCorrection{false}
\chapitre{Géométrie affine euclidienne}
\sousChapitre{Géométrie affine euclidienne du plan}

\contenu{
\texte{
% similitudes, cercles circonscrit, triangle rectangle
Soit $ABC$ un triangle et $D$ le projeté orthogonal de $A$ sur $(BC)$. On considère des points $E$ et $F$ appartenant à une droite passant par $D$ tels que $(AE)$ et $(BE)$ sont perpendiculaires ainsi que $(AF)$ et $(CF)$. Enfin, on note $M$ et $N$ sont les milieux respectifs de $[BC]$ et $[EF]$.
Montrer que $(AN)$ et $(NM)$ sont perpendiculaires.
}
\indication{Considérer les cercles de diamètre $[AB]$ et $[AC]$, le second point d'intersection, et une similitude de centre $A$.}
}
