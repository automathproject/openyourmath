\uuid{GXTc}
\exo7id{4957}
\auteur{quercia}
\organisation{exo7}
\datecreate{2010-03-17}
\isIndication{false}
\isCorrection{true}
\chapitre{Géométrie affine euclidienne}
\sousChapitre{Géométrie affine euclidienne du plan}

\contenu{
\texte{
Soit~$E$ un plan affine euclidien muni d'un repère orthonormé
d'origine~$O$. Soit~$A$ le point de coordonnées~$(a,0)$.
Pour tout point~$M$, on définit $M'=f(M)$ de la manière suivante~:
$A,M,M'$ sont alignés et $(MO)$ est orthogonale à~$(M'O)$.
Expliciter~$f$ en fonction des coordonnées~$(x,y)$ de~$M$.
Donner son domaine de définition. Montrer que~$f$ réalise une bijection
entre le demi-disque supérieur de diamètre~$[AO]$ et le quart de plan d'équations $x<0, y>0$.
}
\reponse{
$x' = \frac{ay^2}{\strut x^2+y^2-ax}$, $y' = \frac{-axy}{\strut x^2+y^2-ax}$,
$M'$ est bien défini ssi $M$ n'appartient pas au cercle de diamètre $[AO]$.

Soit $D$ le demi-disque supérieur de diamètre $[AO]$, $D$ est
caractérisé par les inégalités $x^2+y^2-ax < 0$, $y>0$ d'où $x'<0$ et
$y'>0$. La réciproque se traite (péniblement) en remarquant que seuls les points
de~$D$ ont une image dans ce quart de plan et que $f$ est quasi-involutive.
}
}
