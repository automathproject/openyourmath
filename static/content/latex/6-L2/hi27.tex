\uuid{hi27}
\exo7id{7047}
\auteur{megy}
\organisation{exo7}
\datecreate{2017-01-08}
\isIndication{true}
\isCorrection{false}
\chapitre{Géométrie affine euclidienne}
\sousChapitre{Géométrie affine euclidienne du plan}

\contenu{
\texte{
% Source :
% Tags :  
On donne une droite $\mathcal D$ et un point $P \not\in \mathcal D$.  Tracer la parallèle ainsi que la perpendiculaire à $\mathcal D$ passant par $P$.
}
\indication{Pour la parallèle, considérer deux points $A$ et $B$ sur la droite et construire un parallélogramme $ABPQ$.

Pour la perpendiculaire, construire un cerf-volant $APBQ$.}
}
