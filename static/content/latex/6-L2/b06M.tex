\uuid{b06M}
\exo7id{5202}
\auteur{rouget}
\datecreate{2010-06-30}
\isIndication{false}
\isCorrection{true}
\chapitre{Géométrie affine dans le plan et dans l'espace}
\sousChapitre{Géométrie affine dans le plan et dans l'espace}

\contenu{
\texte{
\label{exo:suprou8bis}
Soient $n$ un entier supérieur ou égal à $2$, puis $A_1$, $A_2$,..., $A_n$ $n$ points du plan. Existe-t-il $n$ points $B_1$, $B_2$,..., $B_n$ tels que, pour $i\in\{1,...,n\}$, $A_i$ soit le milieu de $[B_i,B_{i+1}]$ (avec la convention $B_{n+1}=B_1$)~? (Utiliser l'exercice précédent.)
}
\reponse{
Pour $1\leq i\leq n$, notons $s_i$ la symétrie centrale de centre $A_i$. Le problème revient à trouver $n$ points $B_1$,..., $B_n$ tels que $B_2=s_1(B_1)$, $B_3=s_2(B_2)$,...,$B_n=s_{n-1}(B_{n-1})$, $B_1=s_n(B_n)$. Ceci équivaut à

$$\forall i\in\{2,...,n\},\;B_i=s_{i-1}\circ s_{i-2}\circ...\circ s_1(B_1)\;\mbox{et}\;B_1=s_n\circ s_{n-1}\circ...\circ s_1(B_1)\;(*).$$

Posons alors $f=s_n\circ s_{n-1}\circ...\circ s_1$. $f$ est une composée de symétries centrales. Il y a donc deux cas. Si $n$ est pair, on peut regrouper les symétries deux par deux. $f$ est alors (d'après l'exercice \ref{exo:suprou7bis}) une composée de translations et donc $f$ est une translation. Si $n$ est impair, $n-1$ est pair et donc la composée des $n-1$ premières symétries est une translation. Par suite, $f$ est la composée d'une translation et d'une symétrie centrale et est donc une symétrie centrale (d'après l'exercice \ref{exo:suprou7bis}).

Maintenant, $(*)$ a une solution si et seulement si $f$ a un point invariant.

\begin{itemize}
\item[\textbf{1er cas.}] Si $n$ est impair, $f$ étant une symétrie centrale, $f$ a un et un seul point invariant~:~son centre. Il existe donc un et un seul point $B_1$ vérfiant $B_1=s_n\circ s_{n-1}\circ...\circ s_1(B_1)$ et finalement, un et un seul $n$-uplet $(B_1,...,B_n)$ solution du problème posé.
\item[\textbf{2ème cas.}] Si $n$ est pair, $f$ est une translation. Si son vecteur est non nul, $f$ n'a pas de point invariant et le problème n'a pas de solution. Si son vecteur est nul, $f$ est l'identité et tout point est invariant par $f$.

Déterminons le vecteur de $f$. On pose $n=2p$. On a alors

$$f=s_{2p}\circ s_{2p-1}\circ...s_2\circ s_1=t_{2\overrightarrow{A_{2p-1}A_{2p}}}\circ...\circ t_{2\overrightarrow{A_{1}A_{2}}}=t_{2(\overrightarrow{A_1A_2}+...+\overrightarrow{A_{2p-1}A_{2p}})}.$$

Quand $n=2p$ est pair, le problème posé a des solutions si et seulement si $\overrightarrow{A_1A_2}+...+\overrightarrow{A_{2p-1}A_{2p}}=\overrightarrow{0}$.
\end{itemize}
}
}
