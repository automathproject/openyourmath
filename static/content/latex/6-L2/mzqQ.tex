\uuid{mzqQ}
\exo7id{7425}
\auteur{mourougane}
\organisation{exo7}
\datecreate{2021-08-10}
\isIndication{false}
\isCorrection{false}
\chapitre{Géométrie affine dans le plan et dans l'espace}
\sousChapitre{Géométrie affine dans le plan et dans l'espace}

\contenu{
\texte{
Dans le plan affine $\Rr^2$ muni d'un repère $A_0,A_1,A_2$,
représenter l'enveloppe convexe des points donnés en coordonnées cartésiennes
$\displaystyle A\left(\begin{array}{c}
-2\\ -3\end{array}\right)$, $\displaystyle B\left(\begin{array}{c}
0\\0\end{array}\right)$, $\displaystyle C\left(\begin{array}{c}
0\\4\end{array}\right)$, $\displaystyle D\left(\begin{array}{c}
4\\3\end{array}\right)$, $\displaystyle E\left(\begin{array}{c}
5\\0\end{array}\right)$.
}
}
