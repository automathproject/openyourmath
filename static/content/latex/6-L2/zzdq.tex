\uuid{zzdq}
\exo7id{1967}
\auteur{gourio}
\organisation{exo7}
\datecreate{2001-09-01}
\isIndication{false}
\isCorrection{false}
\chapitre{Géométrie affine dans le plan et dans l'espace}
\sousChapitre{Géométrie affine dans le plan et dans l'espace}

\contenu{
\texte{
On appelle enveloppe convexe $co(A) $ d'une partie non vide $A$ d'un espace
affine $E $  l'intersection des ensembles convexes contenant $A$ ;
c'est le plus petit ensemble convexe contenant $A.$ Montrer que c'est aussi
l'ensemble des barycentres \`{a} coefficients positifs de points de $A.$ Que
sont $co(\{A,B\}), co(\{A,B,C\}) $ ?
}
}
