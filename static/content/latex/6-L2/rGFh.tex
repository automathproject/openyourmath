\uuid{rGFh}
\exo7id{7094}
\auteur{megy}
\datecreate{2017-01-21}
\isIndication{true}
\isCorrection{true}
\chapitre{Géométrie affine euclidienne}
\sousChapitre{Géométrie affine euclidienne du plan}

\contenu{
\texte{
% Source :  Audin par exemple
Soient $D$ et $D'$ deux droites. Soient $A, B, C$ trois points sur $D$, et $A'$, $B'$ et $C'$ trois points sur $D'$. Si $(AB') // (BC')$ et $(BA') // (CB')$, alors $(AA') // (CC')$.
}
\indication{Des homothéties de même centre commutent, de même que des translations.}
\reponse{
Soit $O$ le pt d'intersection. On note $\phi$ l'homothétie qui envoie $A$ sur $B$, et $\psi$ celle qui envoie $B$ sur $C$. Alors $\phi\psi = \psi\phi$. L'image de $A$ est $C$ et l'image de $A'$ est $C'$, d'où le  parallélisme demandé.

Si les droites sont parallèles, on remplace les homothéties par des translations.
}
}
