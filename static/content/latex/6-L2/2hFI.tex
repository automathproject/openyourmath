\uuid{2hFI}
\exo7id{7253}
\auteur{mourougane}
\organisation{exo7}
\datecreate{2021-08-10}
\isIndication{false}
\isCorrection{false}
\chapitre{Géométrie affine euclidienne}
\sousChapitre{Géométrie affine euclidienne du plan}

\contenu{
\texte{
On dit qu'une partie \(X\) du plan (ou de l'espace) est \emph{convexe} 
si elle vérifie:
$$
 \forall (P,Q) \in X^2 \quad [PQ] \subset X,
$$
autrement dit, pour tout couple \((P,Q)\) de points de \(X\), 
le segment \([PQ]\) tout entier est contenu dans \(X\).
}
\begin{enumerate}
    \item \question{Montrer qu'une intersection de parties convexes du plan est 
convexe.}
    \item \question{En déduire que les cellules d'un diagramme de Voronoï sont 
convexes.}
\end{enumerate}
}
