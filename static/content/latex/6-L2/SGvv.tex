\uuid{SGvv}
\exo7id{7459}
\auteur{mourougane}
\organisation{exo7}
\datecreate{2021-08-10}
\isIndication{false}
\isCorrection{false}
\chapitre{Géométrie affine dans le plan et dans l'espace}
\sousChapitre{Géométrie affine dans le plan et dans l'espace}

\contenu{
\texte{

}
\begin{enumerate}
    \item \question{Que peut-on dire d'une isométrie plane qui a trois points fixes non alignés ?}
    \item \question{Soit $\phi$ une isométrie plane qui a deux points fixes distincts $A$ et $B$.
Soit $C$ un point hors de la droite $(AB)$ et $C'$ son image par $\phi$.
Si $C'$ est différent de $C$, déterminer la médiatrice du segment $[CC']$
et montrer que $s_{(AB)}\circ \phi$ est l'identité.
En déduire la nature de $\phi$.}
    \item \question{Soit $\phi$ une isométrie différente de l'identité qui a un point fixe $A$.
Soit $B$ un autre point du plan et $B'$ son image.
Si $B'$ est différent de $B$, soit $d$ la médiatrice du segment $[BB']$
Montrer que $s_d\circ \phi$ a deux points fixes.
Montrer que $\phi$ est composée de réflexions.}
    \item \question{Montrer qu'une isométrie plane qui n'a pas de point fixe
peut s'écrire composée de moins de trois réflexions.}
\end{enumerate}
}
