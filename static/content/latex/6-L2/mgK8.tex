\uuid{mgK8}
\exo7id{6873}
\auteur{gammella}
\datecreate{2012-05-29}
\isIndication{false}
\isCorrection{true}
\chapitre{Analyse vectorielle}
\sousChapitre{Forme différentielle, champ de vecteurs, circulation}

\contenu{
\texte{
Déterminer si les formes différentielles suivantes sont exactes et dans ce cas, les intégrer :
}
\begin{enumerate}
    \item \question{$\omega_1=2xy dx +x^2dy $}
    \item \question{$\omega_2=xy dx - z dy +xz dz$}
    \item \question{$\omega_3=2xe^{x^2-y} dx -2e^{x^2-y}dy$}
    \item \question{$\omega_4=yz^2 dx + (xz^2+z) dy + (2xyz+2z+y) dz.$}
\reponse{
Pour $\omega_1$, on pose $P(x,y)=2xy$ et $Q(x,y)=x^2$. Comme $\omega_1$ est définie
sur l'ouvert étoilé $\Rr^2$
et que $ \frac{\partial P}{\partial y}=\frac{\partial Q}{\partial x}=2x$, le théorème de Poincaré
permet de dire que $\omega_1$ est exacte. On cherche $f$ tel que $df=\omega_1$. Ceci équivaut à résoudre le système 
$$  \left\{ \begin{array}{lll}
\frac{\partial f} {\partial x}& = &2xy \\
\frac{\partial f}{\partial y} &= & x^2\\
\end{array} \right .$$
En intégrant la première ligne par rapport à $x$, on trouve
$f(x,y)= x^2y+c(y)$. En dérivant l'expression que l'on vient d'obtenir par
rapport à $y$ et en identifiant avec la deuxième ligne
du système, on trouve $$ \frac{\partial f}{\partial y}= x^2+c'(y)=x^2.$$ Il s'ensuit que $c'(y)=0$ et 
donc que $c(y)=c \in \Rr$. Par suite, la fonction 
$f$ cherchée est :
$$f(x,y)=x^2y+c$$
o\`u $c$ est une constante réelle.
Pour $\omega_2$, on pose $P(x,y,z)=xy$, $Q(x,y,z)=-z$ et $R(x,y,z)=xz.$ 
On constate que 
$ \frac{\partial P}{\partial y}=x$ alors que $ \frac{\partial Q}{\partial x}=0$. La forme
$\omega_2$ n'est donc pas exacte.
Pour $\omega_3$, on pose $P(x,y)=2xe^{x^2-y}$ et $Q(x,y)=-2e^{x^2-y}$.
Là aussi, $ \frac{\partial P}{\partial y}\not= \frac{\partial Q}{\partial x} $
puisque $ \frac{\partial P}{\partial y}=-2xe^{x^2-y}$ alors
que $ \frac{\partial Q}{\partial x}=-4xe^{x^2-y}$ ; $\omega_3$ n'est donc pas exacte.
Pour $\omega_4$, posons
$P(x,y,z)=yz^2$, $Q(x,y,z)=xz^2+z$, $R(x,y,z)=2xyz+2z+y$. On constate
que 
\begin{enumerate}
$ \frac{\partial P}{\partial y}= \frac{\partial Q}{\partial x}= z^2$
$ \frac{\partial P}{\partial z}= \frac{\partial R}{\partial x}=2zy$
$ \frac{\partial Q}{\partial z}= \frac{\partial R}{\partial y}= 2xz+1$.
}
\end{enumerate}
}
