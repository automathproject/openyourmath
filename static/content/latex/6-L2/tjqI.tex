\uuid{tjqI}
\exo7id{6984}
\auteur{blanc-centi}
\datecreate{2015-07-04}
\isIndication{false}
\isCorrection{true}
\chapitre{Courbes planes}
\sousChapitre{Courbes paramétrées}

\contenu{
\texte{
Soit $\mathcal{C}$ la courbe plane paramétrée par 
$$\left\{\begin{array}{l}x(t)=t\ln t\\ y(t)=\frac{\ln t}{t}\end{array}\right.\ (t\in]0;+\infty[)$$
}
\begin{enumerate}
    \item \question{Comparer les points de paramètres $t$ et $1/t$, en déduire un domaine d'étude de $\mathcal{C}$.}
    \item \question{Représenter $\mathcal{C}$.}
\reponse{
Soit $t>0$:
$$\left\{\begin{array}{l}
x(\frac{1}{t})=\frac{1}{t}\ln(\frac{1}{t})=-y(t)\\
\ \\
y(\frac{1}{t})=t\ln(\frac{1}{t})=-x(t)
\end{array}\right.$$
et par conséquent, le point $M(\frac{1}{t})$ est le symétrique 
de $M(t)$ par rapport à la droite d'équation $y=-x$. 
\fbox{On restreint l'étude à l'intervalle $]0;1]$}, 
puis on obtiendra l'intégralité de la courbe par symétrie 
par rapport à la seconde bissectrice.
Les fonctions $x$ et $y$ sont de classe $\mathcal{C}^1$ sur $]0;1]$.
\begin{itemize}
Tableau de variations conjointes\\
Pour $t\in]0;1]$: 
$$\begin{array}{lcl}
x(t)=t\ln t&\ &y(t)=\frac{\ln t}{t}\\
x'(t)=1+\ln t&\ &y'(t)=\frac{1-\ln t}{t^2}\\
x'(t)>0\Longleftrightarrow t>1/e&\ &y'(t)>0 \\
x'(t)=0\Longleftrightarrow t=1/e &\ &y'(t)\not=0
\end{array}$$
puisque $\frac{1}{e}<1<e$. On obtient donc le tableau suivant:
$$\begin{array}{c|lcccr}
t&0&\ &1/e&\ &1\\\hline
x'(t)&-\infty &-&0&+&1 \\\hline
\ &0  &\ & & & 0\\
x&\ &\searrow &\ &\nearrow & \\
\ & & &-1/e &\ & \\\hline
\ & &\ & & &0 \\
&\ & &\ &\nearrow& \\
&\ & &-e & & \\
y&\ &\nearrow &\ &\ & \\
\ &-\infty & & &\ & \\\hline
y'(t)&+\infty & + &2e^2 & +&1 \\
\end{array}$$
Il n'y a pas de point singulier.
\'Etude des branches infinies\\
Comme $x(t)\xrightarrow[t\to 0^+]{}0$ et $y(t)\xrightarrow[t\to 0^+]{}-\infty$, l'axe des ordonnées est asymptote à $\mathcal{C}$.
\end{itemize}
}
\end{enumerate}
}
