\uuid{XRNS}
\exo7id{7119}
\auteur{megy}
\datecreate{2017-02-08}
\isIndication{false}
\isCorrection{true}
\chapitre{Géométrie affine euclidienne}
\sousChapitre{Géométrie affine euclidienne du plan}

\contenu{
\texte{
% application directe
On donne un segment $[AB]$ et un réel $\alpha \in ]-\pi,\pi[$. On suppose que l'on dispose également d'un triangle auxiliaire $XYZ$ avec $\widehat{(\overrightarrow{XY}, \overrightarrow{XZ})}=\alpha$, de sorte que les angles de mesure $\alpha$ sont constructibles.

 Construire le lieu des points $M$ tels que $\widehat{(\overrightarrow{MA}, \overrightarrow{MB})}=\alpha$.
}
\reponse{
Par le théorème de l'angle inscrit, c'est un arc de cercle, dont le centre est sur la médiatrice de $[AB]$. 

Par le cas limite du théorème de l'angle inscrit, on sait aussi que si $\mathcal T$ est la tangente à ce cercle en $A$, alors $(\mathcal T,AB)=\alpha$.

On trace donc la droite $\mathcal T$ faisant un angle $\alpha$ avec $(AB)$ en $A$, puis la perpendiculaire à $\mathcal T$ passant par $A$. Cette droite coupe la médiatrice en un point $O$ qui est donc le centre du cercle recherché.
}
}
