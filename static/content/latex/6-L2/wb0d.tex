\uuid{wb0d}
\exo7id{4958}
\auteur{quercia}
\datecreate{2010-03-17}
\isIndication{false}
\isCorrection{true}
\chapitre{Géométrie affine euclidienne}
\sousChapitre{Géométrie affine euclidienne de l'espace}

\contenu{
\texte{
Soient $A$, $B$, $C$ trois points distincts de l'espace.

Déterminer le lieu des points $M$ tels que
$\vec{MA}\wedge\vec{MB} + \vec{MB}\wedge\vec{MC} = 2\vec{MC}\wedge\vec{MA}$.
}
\reponse{
Droite parallèle à $(AC)$ passant par
         $D = \text{Bar}(A:1, B:-\frac13)$.
}
}
