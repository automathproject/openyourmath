\uuid{c38Z}
\exo7id{7503}
\auteur{mourougane}
\datecreate{2021-08-10}
\isIndication{false}
\isCorrection{false}
\chapitre{Géométrie affine euclidienne}
\sousChapitre{Géométrie affine euclidienne du plan}

\contenu{
\texte{

}
\begin{enumerate}
    \item \question{On considère dans le plan euclidien orienté un point $A$ et 
    la rotation $r$ de centre $A$ d'angle $+\pi/2$.
    Soit $M$ un point et $M'=r(M)$ son image par $r$. Soit $d$ une droite passant par $M$.
    Décrire un point et la direction de l'image $r(d)$ de la droite $d$.}
    \item \question{Soit $d_1$ et $d_2$ deux droites et $B\in d_1$ et $C\in d_2$ tel que 
    $ABC$ soit un triangle rectangle isocèle en $A$ (avec $mes \widehat{(\vec{AB},\vec{AC})}=+\pi/2$).
    Démontrer que $C$ appartient à l'image de la droite $d_1$ par $r$.}
    \item \question{Soit $\delta_1$ et $\delta_2$ deux droites non perpendiculaires et $E$ un point du plan. 
    Construire un triangle $EFG$ rectangle isocèle en $E$ et 
    tel que $F\in \delta_1$ et $G\in \delta_2$.}
\end{enumerate}
}
