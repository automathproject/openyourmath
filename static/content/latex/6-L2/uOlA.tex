\uuid{uOlA}
\exo7id{5053}
\auteur{quercia}
\organisation{exo7}
\datecreate{2010-03-17}
\isIndication{false}
\isCorrection{true}
\chapitre{Surfaces}
\sousChapitre{Surfaces paramétrées}

\contenu{
\texte{
Soit ${\cal S}$ la surface d'équation cartésienne $z^2-x^2-y^2 = 1$.
}
\begin{enumerate}
    \item \question{Reconnaître ${\cal S}$.}
\reponse{Hyperboloïde de révolution à deux nappes.}
    \item \question{Soit $D$ la droite d'équations : $2x+y = 0$, $z=0$. Déterminer les points
    $M$ de ${\cal S}$ tels que le plan tangent à ${\cal S}$ en $M$ est
    parallèle à $D$.
    (Contour apparent de ${\cal S}$ dans la direction de $D$)}
\reponse{$x = 2y$, $z^2 = 1 + 5y^2$.}
\end{enumerate}
}
