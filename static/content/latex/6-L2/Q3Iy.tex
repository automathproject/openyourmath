\uuid{Q3Iy}
\exo7id{7152}
\auteur{megy}
\datecreate{2017-05-13}
\isIndication{false}
\isCorrection{true}
\chapitre{Géométrie affine euclidienne}
\sousChapitre{Géométrie affine euclidienne du plan}

\contenu{
\texte{
Soit $ABCD$ un carré direct de côté $1$, et soient $E$  et $F$ deux points tels que $AEFD$ soit un rectangle direct, avec $AE > 1$. Dans la suite on note $l=AE$.


Montrer que qu'il existe une similitude directe $s$ envoyant $A$ (resp. $E$, $F$ et $D$) sur $C$ (resp.  $B$, $E$ et $F$) ssi $l$ est égal au nombre d'or $(1+\sqrt 5)/2$.
}
\reponse{
Il existe une unique similitude directe $s$ qui envoie $[AE]$ sur $[CB]$. Son rapport est $1/l$ et son angle $-\pi/2$.

Soit $D' = s(D)$. Alors $(\overrightarrow{AD},\overrightarrow{CD'})=(\overrightarrow{AD},\overrightarrow{s(A)s(D)})=-\pi/2$ et $CD' = \frac1lAD = \frac1l$.

Cette similitude envoie donc $D$ et $F$ ssi $CF=CD'$ c'est-à-dire ssi $1/l = l-1$. Comme $l\neq 0$, cette équation est équivalente à $1=l^2-l$ . Elle admet une unique solution positive : $\frac{1+\sqrt 5}{2}$.
}
}
