\uuid{c3Zx}
\exo7id{7106}
\auteur{megy}
\organisation{exo7}
\datecreate{2017-01-21}
\isIndication{true}
\isCorrection{true}
\chapitre{Géométrie affine euclidienne}
\sousChapitre{Géométrie affine euclidienne du plan}

\contenu{
\texte{
% source : Debart rotations au collège
% tags : rotations, collège
Soit $ABCD$ un carré de centre $O$, et $OPQR$ un second carré de même taille. Calculer l'aire de l'intersection de ces deux carrés en fonction de l'aire de $ABCD$.
}
\indication{L'aire de l'intersection vaut le quart de l'aire du carré $ABCD$.}
\reponse{
Tracer une figure en prolongeant les segments $[OP]$ et $[OR]$, et considérer une rotation de centre $O$ et d'angle $\pi/2$.
}
}
