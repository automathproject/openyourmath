\uuid{Q5iJ}
\exo7id{6436}
\auteur{potyag}
\organisation{exo7}
\datecreate{2011-10-16}
\isIndication{false}
\isCorrection{false}
\chapitre{Géométrie et trigonométrie hyperbolique}
\sousChapitre{Géométrie et trigonométrie hyperbolique}

\contenu{
\texte{
Soit $g\in M$ une homographie. Montrer que
}
\begin{enumerate}
    \item \question{Si $g$ est parabolique ${\rm fix}(g)=\{x\}$ alors
  $$\displaystyle\forall z\ \in \overline{\Cc}\ :\ \lim_{n\to \pm\infty} g^n(z)=x.$$}
    \item \question{Si $g$ est loxodromique et    ${\rm fix}(g)=\{x, y\}$ alors
 $$\displaystyle\forall z\ \in \overline{\Cc}\setminus\{y\}\  :\  \lim_{n\to +\infty} g^n(z)=x,$$
et le point $x$ est dit {\it point fixe attractif} de $g.$

 $$\displaystyle\forall z\ \in \overline{\Cc}\setminus\{x\}\ :\  \lim_{n\to +\infty} g^{-n}(z)=y,$$
et le point $y$ est dit {\it point fixe répulsif} de $g.$}
\end{enumerate}
}
