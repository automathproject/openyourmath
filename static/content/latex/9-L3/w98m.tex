\uuid{w98m}
\exo7id{6402}
\auteur{potyag}
\organisation{exo7}
\datecreate{2011-10-16}
\isIndication{false}
\isCorrection{false}
\chapitre{Géométrie différentielle élémentaire de R^n}
\sousChapitre{Géométrie différentielle élémentaire de Rn}

\contenu{
\texte{

}
\begin{enumerate}
    \item \question{Montrer que $3$ points $x, y$ et $z$ sont colinéraires
dans $\Rr^n$ avec $y$ entre $x$ et $z$ (rappelons que \c{c}a
signifie que $y=x+t(z-x)\ t\in[0,1])$ ssi

$$\vert\vert x-y\vert\vert + \vert\vert y-z\vert\vert = \vert\vert
x-z\vert\vert.$$}
    \item \question{Montrer que si $\gamma :[a,b]\mapsto {\Rr}^n$ est une courbe alors
$\vert\gamma([a,b])\vert\geq\vert\vert\gamma(a)-\gamma(b)\vert\vert$ et que
l'égalité a lieu ssi $\gamma$ est une géodésique (où $\vert\cdot\vert$
désigne la longueur d'une courbe).}
\end{enumerate}
}
