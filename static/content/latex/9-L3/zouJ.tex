\uuid{zouJ}
\exo7id{7749}
\auteur{mourougane}
\organisation{exo7}
\datecreate{2021-08-11}
\isIndication{false}
\isCorrection{false}
\chapitre{Géométrie projective}
\sousChapitre{Géométrie projective}

\contenu{
\texte{
Le but de l'exercice est de 
démontrer le théorème de Pappus affine~:
\textit{Soit $d$ et $d'$ deux droites d'un plan affine $E$. Soit $A, B, C$ (resp. $A', B', C'$) trois points sur $d$ (resp. sur $d'$.)
Si les droites $(AB')$ et $(BA')$ sont parallèles ainsi que les droites $(BC')$ et $(CB')$, alors les droites $(CA')$ et $(AC')$ le sont aussi.}

Dans le cas où $d$ et $d'$ sont sécantes en $I$
}
\begin{enumerate}
    \item \question{On considère l'homothétie $h$ de centre $I$ qui envoie $A$ sur $B$. Déterminer l'image de $B'$ par $h$.}
    \item \question{On considère l'homothétie $H$ de centre $I$ qui envoie $B$ sur $C$. Déterminer l'image de $C'$ par $H$.}
    \item \question{Déterminer l'image de $A$ et celle de $C'$ par $H\circ h$.}
    \item \question{Conclure.}
\end{enumerate}
}
