\uuid{qOFO}
\titre{Rayon de convergence de séries sentières}
\theme{séries entières}
\auteur{}
\datecreate{2024-06-17}
\organisation{AMSCC}	

\contenu{

		 Déterminer le rayon de convergence des séries entières réelles suivantes :
\begin{enumerate}
	\item  \question{ $\displaystyle \sum_{n \geq 1}  \left(\frac{n+1}{n}\right)^nx^{n}$ }
	\reponse{On pose $u_n(x) = \left(\frac{n+1}{n}\right)^nx^{n}$ et on utilise le théorème de Cauchy : 
		\begin{align*}
			\left(|u_n(x)|\right)^{\frac{1}{n}} &= \left( \left(\frac{n+1}{n}\right)^n|x|^{n}\right)^{\frac{1}{n}} \\
			&= \frac{n+1}{n} |x| \xrightarrow[n\to+\infty]{} |x|
		\end{align*}	
		donc le rayon de convergence est $R=1$.
	}
	\item \question{ $\displaystyle \sum_{n \geq 0} \frac{n^2+1}{3^{2n+1}\sqrt{n^2+n+1}} x^{2n}$ }
	\reponse{On pose $u_n(x) = \frac{n^2+1}{3^{2n+1}\sqrt{n^2+n+1}} x^{2n}$ et on utilise le théorème de d'Alembert : 
		\begin{align*}
			\frac{|u_{n+1}(x)|}{|u_n(x)|} &= \frac{ ((n+1)^2+1)  \times 3^{2n+1}\sqrt{n^2+n+1} }{(n^2+1)3^{2n+3}\sqrt{(n+1)^2+n+1+1}}\frac{|x^{2n+2}|}{|x^{2n}|} \\
			& \sim  \frac{1}{9} \frac{n^3}{n^3} |x|^2 \\
			&\xrightarrow[n\to+\infty]{} \frac{1}{9} |x|^2
		\end{align*}	
		Or $ \frac{1}{9}|x|^2 <1 \iff |x| < 3$ donc le rayon de convergence est \fbox{$R=3$}.
	}
	
\end{enumerate}
}