\uuid{qZkI}
\titre{Dilemme}
\theme{probabilités}
\auteur{ bibmath }
\organisation{AMSCC}


\contenu{
	\texte{ On considère deux avions $A$ et $B$ ayant respectivement 4 et 2 moteurs. Les moteurs fonctionnent de manière indépendante et chacun a une probabilité $p \in ]0;1[$ de tomber en panne. On admet qu'un vol se termine bien si moins de la moitié des moteurs tombe en panne.
	}

	\question{ Quel avion est-il préférable de choisir ?  }

\reponse{ Soit $X$ le nombre de moteurs tombant en panne sur l'avion $A$ et $Y$  le nombre de moteurs tombant en panne sur l'avion $B$. Alors $X$ suit une loi binomiale $\mathcal{B}(4,p)$ et $Y$ suit une loi binomiale $\mathcal{B}(2,p)$. 

On en déduit que $\prob(X=0)+\prob(X=1) = (1-p)^4 + 4p(1-p)^3$ et $\prob(Y=0) = (1-p)^2$.

Il est préférable de prendre l'avion $A$ si et seulement si $\prob(X=0)+\prob(X=1) \geq \prob(Y=0)$, c'est-à-dire :
$$p(1-p)^2(2-3p) \geq 0 \iff 2-3p \geq 0$$

En conclusion, si $p < \frac{2}{3}$, il est préférable de prendre l'avion $A$. 
 }
	
}
